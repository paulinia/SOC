\documentclass{book}

\usepackage[slovak]{babel}
\usepackage[utf8]{inputenc}
\usepackage{microtype}
\usepackage{fancyhdr}
\usepackage{amsmath, amssymb}
\usepackage{latexsym}
\usepackage{subfig}
\usepackage[hidelinks]{hyperref}

\usepackage{graphicx}
\graphicspath{ {/} }
\DeclareGraphicsExtensions{.pdf,.png,.jpg}
\author{Paulina Smolarova}
\title{Rod Tlogena}
\frenchspacing

\pagestyle{fancy}
\fancyhf{}
\fancyhead[LE,RO]{Paulína Smolárová}
\fancyhead[RE,LO]{Rod Tlogena}
\fancyfoot[CE,CO]{\leftmark}
\fancyfoot[LE,RO]{\thepage}
\usepackage{caption}
%\setbeamertemplate{caption}[numbered]

\begin{document}

\maketitle

\newpage

\tableofcontents

\chapter*{Prológ}
\addcontentsline{toc}{chapter}{Prológ}

Meškal. Ako len neznášala to, keď meškal. Pocit strachu, trochu hnevu a rezignácie z toho, že nikdy nepríde načas. Áno, stáva sa to, každému sa to stáva, ale stále?

Hľadela na zatvorené dvere, unavená ako po misii, klopkala rukou na stôl, ale len potichu, žiadnu plač v ten den nechcela. Nech boli kedysi deti akékoľvek zlaté, už mala chuť ísť preč, konečne sa vrátiť do zamestnania. Po pol roku.

Rolliusovi povedala, že si chce odýchnuť. Ej odýchnuť, to by radšej niekde bojovala o život, pomyslela si.

Keďže dieťa, dieťa o nej sa nemal nik dozvedieť, nemohli len tak niekoho najať! To by boli reči, to by boli reči. Akokeby ich nebolo dostatok.

Pamätala si dobre ten deň, keď sa zjavili na titulke nejakého bulváru. Morja a Jegrigsen, každý to počul, každý to videl. Aké prekliatie byť Tlogenom. Iste, čakali to, ale preboha, kto mal čakať že im to nechvália! Tak... každý? Pomyslela si unavene a trochu zúfalo.

Čo bude ďalej? Napadlo ju, ale zatlačila to do úzadia jej hlavy, nie nie, na toto netreba myslieť. Potom... ale kedy bude potom? Čo to spravili...

Také okamihy, keď premýšľala mávala v poslednej dobe často. Keď jej spoločnosť nerobil Jeg, vtedy... Cítila sa sama, tak sama... Ale spoločnosť zo spoločenstva? Keďže boli práve na blackliste ľudí, ktorí sa práve vo veľkom ohovárali o tom nemohlo byť ani počutia. Čo by spravili, keby...

Mala pocit, že jej šibe, ach toto nie je dobré. Asi požiada Jega, už teraz, lebo sa z toho načisto zblázni. Ale čo preboha bude, keď sa vráti do Agentúry? Bože, to budú reči, pomyslela si a zachvátila ju silná túžba nebyť nikým a môcť si nepozorovane zmeniť meno, priezvisko, prácu, sfalšovať preukaz a všetko a začať nový život. Ale nie, ona mala to nešťastné priezvisko. Spoznali by ju všade, všade.

A Jegrigsen nechodil. Nielenže mohol chodiť do práce, nielenže mohol mať život, mohol sa objaviť na čerstvom vzduchu, ale aj meškal! Mala v sebe zlosť, frustráciu. Chýbala jej sloboda, aspoň ten odtieň slobody ktorý mala predtým než... Ako to preboha mohli dopustiť...

Nech mala svoju dcéru akokoľvek rada, zároveň ju rovnako neznášala ako osobu, ktorá jej zničila život, ktorá jej zobrala spoločnosť a čerstvý vzduch. Majúc v sebe toľko protichodných ale hlavne negatívnych myšlienok, jej zlú náladu ešte katalyzoval fakt, že Jeg meškal. Mala ho chuť hľadať, vybehnúť za ním do búrky, nechať otvorené dvere a...

Potrebujem prestávku, fakt potrebujem prestávku, pomyslela si. Nezvládam to. Naozaj to nezvládam. A kde je, kde je?!

Prepínala medzi magickým a klasickým pohľadom, hľadela do stropu a premýsľala aký je jej život mizerný.

Magický pohľad nebol ničím zvláštny, chuchvalce pobehujúce okolo, žijúce si svojím vlastným životom. Von z Jegrigsenovej vily, mágia bola ako všade, nič nenasvedčovalo ničomu podozrivému. Mágia plynula ako v normálnej blízkosti magického príbytku, žiadne abnormálnosti. A jej v mozgu vŕtala myšlienka. Kde je Jegrigsen?

Pohľad jej sa zvrtol na telepatión na stole. Ach jasné, prečo si na to nespomenula skôr? Priletel jej do ruky a už volal Jegovi. A Jegrigsen Goon, jediná normálna spojka s ľuďmi jej partner a priateľ, jej nedvíhal. Mala pocit, že ho nenávidí. Iste, ľúbila ho, predsa, prečo by nemala, ale niekedy ju vytáčal do nepríčetnosti. Ale zachovávala kľud. Niečo, čo ju naučila jej práca v agentúre, vždy zachovať kľud, strach, hnev a stres sú nepriatelia, rovnako ako D. Jej izolácie jej jej návyky kazila.

Nedvíhal. Nuž čo, pomyslela si, a prešla do menej príjemnej časti mágie a sama začala prehľadávať telepatické pole.

Po Jegovi nebolo ani chýru, ani slychu, nebolo ho vidieť, akokeby sa mu niečo stalo, zomrel? A vlastne, celé pole bolo akési prázdne. Čo sa stalo? A čo je s Jegom?! Chytil ju des, bol to jej strach, ktorý sa postupne dostával na povrch.

Nie, to sa nemohlo stať. Arabela by jej povedala, aspoň niekto... Arabela sa proti jej bratovi neobrátila ako zvyšok rodiny.

Morja kľud, hovorila si.

"Arabel?" Zdalo sa jej, že z telepatiónu sa ozval familiárny zvuk pripojenia, ale bolo to len zdanie. Po niekoľkých dlhých minútach, sekundách? Netušila ako dlho čakala na ten mizerný zvuk, sa z telepatiónu ozval hlas. Ale Arabela to nebola.

"Ak chcete dosiahnuť telepatické spojenie, prosím uvoľnite mágiu." Ozval sa hlas operátora. Morja zvraštila čelo, čo sa dialo? Veď v magickom pohľade sa nič nedialo... A pri Jegovi to nehlásilo... Žeby Arabel...

Ponorila sa do telepatického poľa hľadajúc ju. Ale znova, ako predtým, čo to... Akokeby ju niekto blokoval...

Žeby? Ale kto, preboha?

Nebuď paranoidná, pomyslela si. Paranoja je zlá.

Na druhý pohľad do magického poľa jej naozaj mágie pripadalo málo. Akokeby... Teória, hypotéza, ale tie treba preskúmať, isteže, pomyslela si. Hľadala niečo netypické, niečo zvláštne, niečo čo by nasvedčovalo tomu, že mágiu pred ňou niekto ukryl.

"Preboha." Ticho preglgla, uvedomujúc si, že práve našla tú stratenú mágiu okolo nej. Niekto spravil vysoko sofistikovanú magickú slučku. Že si to nevšimla, ako jej nenápadne kradne mágiu pred nosom.

Zanadávala. Ona, veď pracovala pre Rolliusa a nebola tak zlá! Pomyslela si, že stráca cvik.

Kto, kto to spravil? Nikto iný okrem Jega tu nebol nemal kto... lietalo jej hlavou. Jeg? Preboha. Preboha.

Zrazu sa jej v hlave vírila teória tak desivá, že to priam považovala za namožné, len aby na to nemusela myslieť.

Slučku, potrebovala zničiť slučku. Znova sa na ňu pozrela, zmietala sa tá papľuha ako o život (aj keď to bol len stoh mágie). Niekto ju držal mimo kontaktu.


\newpage
\chapter{Po rokoch}

"Mária, mohla by si mi podať ten list? Mária, čo si zas neprítomná!?"$ $ Mária sa strhla.

"Prepáčte, tu je.“

"Mária, čo ti je?“

"Nič, dnes... nič, fakt nič...“

"Čo sa ti stalo? Niečo v rodine?"$ $ Zas ju akoby zamrazilo.

"Nie... nemám rodinu. Len... Ale nič."$ $ Niečo si zamrmlala.

"Však také depresívne počasie?"$ $ Snažila sa jej kolegyňa nadviazať rozhovor. "Stále prší a sme uprostred júla. Však Morjo... teda Mária."$ $ Mária sa zas strhla. "Prepáč... pomýlila som si tvoje priezvisko, veď máš podobné meno – Maria Morjová. Nie si unavená?“

$ $Asi áno, dám si týždeň dovolenky, aj tak mi ostávajú štyri týždne. Idem to vybaviť... A už sa mi aj tak končí pracovná doba...“

"Rozumiem. Veď už máme všetko hotové...“

Mária si nadávala aké meno si vybrala, lebo ona nebola Mária, ale Morja Tlogenová, dcéra Morny Tlogenovej a utečenkyňa. Bolo to jej šieste meno a už jej chýbala fantázia. Chcela meno, čo sa jej bude podobať na jej bývalé meno. Meno, ktoré mala vyše dvadsať rokov. Nemôže... nemôže, aby ju spoločenstvo vyhľadalo. Svoju matku dostatočne poznala. To stačilo. Premýšľala o tom, že si znásobí peniaze a ostane nezamestnaná... nechcela sa stretávať s ľuďmi... radšej s nikým... stále, stále nezabudla...

Husto pršalo. Zas. Ako pred štrnástimi rokmi. Tlačili sa jej do očí slzy. Bolo to tak dávno... Morja v ruke držala peňaženku a triasla sa. Štrnásť rokov. Vzdychla si. Kúpila si hodinový lístok na MHD a čakala. Keď konečne prišiel autobus, ani nebol preplnený. Našla jedinú prázdnu sedačku.

"Môžem si sadnúť?“

$ $Ach, samozrejme, smerujete na zastávku číslo 19?"$ $ Morja sa striasla. Toto číslovanie používali len oni. Spoločenstvo. Toto musel byť niekto, kto patril k nim. Nie k nej. Pomyslela si. Nesmie sa odhaliť. Hovorila si Morja. Ale zas... prečo by hovorili o číslovaní ich zastávok človeku zo Zeme? Prebehol jej mráz po chrbte. Niečo jej hovorilo, že ju tá osoba poznala.

"Veď sadnite si, Morja."$ $ A bolo to tu.

"Vy... vy ma poznáte?“

"Samozrejme Morja. A predpokladám, že ma poznáš aj ty."$ $ Zahľadela sa na sediacu osobu a rozmýšľala. Najskôr si nevedela spomenúť a vtom si spomenula na ten jej špecifický prízvuk a spomenula si, kto to je. Niela. Zahnala myšlienku na návrat s tým, čo by sa mohlo stať, a že rodina ju už iste vydedila a pokúšala sa hrať svoju doterajšiu rolu.

"Nie."$ $ Pokrútila hlavou.

"Neklam samu seba, Morja. Poznáš ma."$ $ Morja nevydržala.

"Pre mňa neexistujete. Ja som odišla. Nesledujte ma!“

"Morja, ty si neodišla dobrovoľne...“

"Kto si...?"$ $ spýtala sa, i keď vedela, o koho ide.

"Poznáš ma Morja.“

"Nepoznám, ja som na vás zabudla. Na všetko. Ja som sa toho vzdala, tak ma nechajte tak."$ $ Bránila sa.

"Morja, potrebujeme ťa. Ty si pre nás dôležitá. Musíš, teda môžeš."$ $ Opravila sa.

"Neprikazujte čo mám robiť! Ja som sa vás vzdala, ja vás nepoznám! A aj tak! Kto som? Čo spravili Tlogenovci po mojom odchode?!"$ $ Bola príliš hlučná. Pomaly sa začali na nich ostatní cestujúci otáčať.
"Morja, vyjasníme si to potom, tu nie je vhodné miesto.“

"Nikde nie je vhodné miesto! Vystupujem!“

"Nie, ideš so mnou na 22. Ideš,"$ $ chytila ju.

"Nie nejdem, stojí autobus, pusti ma,“

"Nie,"$ $ autobus už zatváral dvere. Morja sa snažila vystúpiť, ale už sa nedalo. Cestovala na zastávku 22. Niela ju donútila... späť... Späť do života, ktorý zavrhla. Späť.

Vystúpili. Morja chcela ostať v autobuse, nedalo sa. Zas. Od zastávky bolo blízko k bytu.

"Morja, nespoznávaš ma, stále?"$ $ Povedala keď vošli do domu. Morja si už bola osobou istá, keď si zložila kapucňu. Morja rezignovala.

"Ty si Niela. Detektívka Izabety Tlogenovej."$ $ Už sa neskrývala za nevedomie.“

"Bingo! Ja som to vedela Morja, že ty sa chceš vrátiť. Možno podvedome... ale...“

"Nie nechcem, Niela, ja som na vás sa snažila zabudnúť! A vy? Nie! Kto ma dal vyhľadať, Izabeta!? Chcete ma nájsť, súdiť a zas vyhnať?! Ďakujem, to radšej odídem sama!“

"Mýliš sa Morja, ja už vyše päť rokov nepracujem pre Izabetu a Spoločenstvo. Pohádali sme sa. O tom, že som ťa vyhľadala, nikto nevie. Ani Iza. Chcem len jednu vec. Či to urobíš... to sa rozhodneš sama.“

$ $A ako ti mám veriť, že tu nevtrhnú agenti a nezatknú ma? Kto mi to ručí? Dobre viem o tom, čo by sa stalo Messii, keby sa vrátila! Bola som na tej rade! A myslíš si, že ku mne by boli akí? Som ešte horšia ako Messia! Ona odišla za človekom... ja pre... služobníka D..."$ $ Na chvíľu sa odmlčala a pokračovala. "Ja vám neverím. Počas prvých piatich rokov som musela šesťkrát meniť identitu. Prečo ste ma našli, čo so mnou chcete?!“

"Nič, prišla som sa s tebou len porozprávať. Môžem ti kľudne ukázať originál mojej písomnej výpovede. Inak Morja, celkom fajn skrývanie, objavila som ťa len teraz. Pred mesiacom, ale Morja, to si si fakt mohla na skrývanie vybrať inú krajinu ako Slovensko. To je celkom malé, poradím ti – ak sa chceš skrývať radšej zájdi do USA. Alebo inde do Európy.“

$ $Ostala som tu, pretože to tu poznám. Prežila som tu takmer celý život... Síce okrem toho keď sme boli na misii v Sao Paole a Pekingu. A pár pobytov, ale... Ale to predsa bolo dávno.“

"Ty si bola celý čas tu? Celých dvanásť rokov?“

"Samozrejme, okrem toho času, keď som žila pol roka v Rajke, rok na Českom pohraničí. Kým ste sa tam podozrivo nezačali sťahovať.“

"Morja, nie si trochu paranoidná?“

"To vieš Niela, ak si dvanásť rokov na úteku tak si myslíš, že každý tvoj známy ťa môže zradiť, že každý tvoj bývalý priateľ je detektív s poverením vyhľadať ťa.“

$ $Ale prečo si odišla aj od nás? Pre Jega? Kým si bola preč sa to zmenilo na verejné tajomstvo. A nikto ťa za to neodsudzuje."$ $ Morja sa zľakla.

$ $Oni vedia, že som porušila zákon? Všetci? Zákon 7?"$ $ Teraz to vyviedlo z miery aj Nielu. Šokovane sa na ňu zahľadela a Morja pochopila, že toho povedala až príliš veľa.

"Ty si porušila zákon 7, z prvej zbierky zákonov, platiacej od začiatkov Spoločenstvo M?"$ $ Morja mimoriadne zahanbene sklopila zrak.

"Žiaľ áno a nechcem sa ospravedlňovať Jegrigsenom. Stala som sa človekom."$ $ Niela pokrútila hlavou a snažila sa Morju upokojiť, i keď i ona považovala porušenie zákona 7, za niečo opovrhnutiahodné.

"Nie, nestratila si mágiu Morja, len si porušila zákon, to je rozdiel.“

"Ty ma za to neodsudzuješ?"$ $ Zničene sa jej spýtala.

"Teraz, aspoň pri tejto situácii, je jedno, kto čo urobil. Možno keby to nebolo také naliehavé...“

"Takže ty ma chceš využiť?!“

"Ja ti len chcem vysvetliť vážnosť situácie, rozhodnúť sa môžeš sa. A čo sa týka zákona... síce nepoznám nikoho od nás, kto by to porušil... ale vieš, že zakázané ovocie najviac chutí.“

"$ $Ale to nie je všetko Niela."$ $ Sklamane priznala Morja.

"$ $Ešte niečo?“

"My sme zákon 7 porušovali neustále, i zákon 95. Bola som hlúpa. Jegrigsen ma zradil.“

"To?“

"Nie len... teraz som si to uvedomila. Je to pre proroctvo vo veľkej knihe. Tlogen a Goon.“

"Toto?

\begin{center}
Prastarý rod Goonových, I starých Tlogenových,
Preslávi žiara. Tá čo zničí prameň temnoty
Traja bratia, traja otcovia Jedna
matka, jedna sestra? Zhora i obnova,
ruka v ruke, Príslušníka dcéry
Tlogena, So synom Goona. Čo
chcel svetlo ochrániť. I pretrhol
vyhnaný.
Dcéra vzíde z toho
zväzku nešťastného, Čo zachráni krajinu, ju a jeho.“
\end{center}

"Toto, teraz ti to je asi jasné. Domysli si to prečo chcel D Jega. A nakoniec sa mu to podarilo."$ $ Niela sa zamyslela.

"Preto, D vedel čo je medzi vami, napriek zákonu 95. Vedel, že to nezastaví inak, ako pritiahnutím jedného z vás na jeho stranu. A Jeg bol dobrým cieľom. Alebo ty. Ale Jeg mu, ako vidno, prišiel do rúk prvý.“

"$ $A potom... porušili sme zákon číslo sedem. Neustále.“

"D vedel, že tomu, aby sa naplnili podmienky na proroctvo nejde zabrániť, a tak vedel, že jeho posledná šanca je, aby dieťa nebolo v spoločenstve. Aby bola človekom. Vtedy si prestala chodiť na verejnosť a po roku si odišla. Každý vedel, že si niečo tajila ale... toto mi nenapadlo.“

"Dcéra Tlogena a Syn Goona. Proroctvo."$ $ Ticho stáli. Každá zaťažená odhalením. Morja svojím a Niela Morjiným. "$ $A čo chceš? Alebo už nič? Zľakla si sa ma?“

"Nie, počúvaj. Nevieš čo sa tu diale keď si bola preč. Arabela Tlogenová je mŕtva, aj z manželom. Ale ich duch je našťastie stále pri nás. Už pred tvojím odchodom zomrela naša spojka Čeria Lietavá, a Tanery je stále nezvestný a Cecília s nami stále nekomunikuje. Aspoň v tomto sa nič nezmenilo."$ $ Ironicky podčiarkla poslednú vetu a pokračovala vážne. "Z Wymyslenska prichádzajú ročne stovky odídencov, medi nimi aj Lietavý. Mokrí – Plavčíkoví tam ostali. O Messii žiadne stopy. Po tom ako s nami pred tridsiatimi rokmi prerušila kontakt nikto nevie kde je."$ $ Morja si pri Nielinom rozprávaní spomínala.

Na Arabelu Tlogenovú, jednu z rodu Tlogenových, bojovníčku proti D, a asi jedinej osoby z rodiny, ktorá by ju mohla neodsúdiť, hoci si to zaslúžila. Na Čierku Lietavú, ich spojku medzi Wymyslenskom a nimi, ktorá mierne patrila do rodiny. Na Taneryho, chemika Wymyslenského ústavu pre vedu, ktorý naplno podporoval Spoločenstvo, objaviteľa chemických prvkov s protónovým číslom 115 a 113 dávno predtým ako zem. Na Messiu, ktorú síce poznala možno v prvých rokoch svojho života, ale spomienka v nej bola. Ambiciózna filozofka so záľubou v ľuďoch. Až kým im prepadla až fanaticky. Či skôr jednému. Jedného dňa sa nevrátila. Nikto o nej viac nepočul. Na Cecíliu Žblnkotaničkovú, prezidentku (po wymyslensky corlovne) ktorá fanaticky nenávidela Izabetu Tlogenovú, za to, že odvádza wymyslenčanov z Wymyslenska, a rovnako spoločenstvo a Zem, tiež Spoločenstvo M, ktoré Izabeta založila. Wymyslenčania si ju zvolili preto, že malou náhodou pomohla k zvrhnutiu diktátorskej prezidentky Solemy Krutej a jej najväčším snom je mať v krajine čo najviac občanov, ako prezidentka najväčšieho štátu. Jednu diktatúru nahradila druhá. To bolo Wymyslensko a jeho občania. A všetko to bolo pod zámienkou demokracie.

"$ $A? Čo teda?“

"Duchovia. Strácajú sa duchovia. A nikto nevie, kde. Akoby... duchovia nás, prisahajúcich pod dýkou majú v niečom väčšiu silu ako my. Akoby... neviem. Aj Thixov duch, aj duchovia Čerie. A rovnako nevieme, kde je Thinix. Ten o tom niečo vedel, ale zmizol.“

"$ $A načo chceš mňa?“

"Teraz sa musíme spájať. Akoby laná tmy sa čoraz viac spletali, Morja. Môžeš sa ďalej skrývať, ale tomuto sa nevyhneš. D napadne i ľudí a je to aj naša úloha. Ale môžeš sa vybrať, dvere sú otvorené.“

"Nevysvetlila si mi to?“

"Poznáš Jega...“

"To som si len myslela.“

"$ $A bola i jedna z najlepších. Nezavrhuj nás pre D a Jega. Tak...prijímaš môj návrh?"$ $ Morja premýšľala.

"Prijímam... Ale mám podmienku. Nájdeme moju dcéru."$ $ Povedala po dlhšej úvahe.

"$ $Okej Morja, veď kto by nám mohol viac pomôcť, ako tá ktorú na to predurčujú proroctvá?“

\newpage

\chapter{Späť do sveta}

"Tulienka Deľa... Rozsvietiš alebo to mám urobiť ja?“

"Počkaj Tarny. Vieš dobre, že nie som v zmyslovej mágii taká dobrá ako ty. Vo Wymyslensku sa to nevyučuje tak do hĺbky. Cecília si to necháva maximálne pre agentov. Viem len to, čo si na naučil ty a pár vecí z kníh. A ticho sústredím sa – ty pekne riaď. Vzhľadom na to, že to robíme nelegálne, nechci vyvolať ďalší hurikán. Ten minulý im hádam stačil."$ $ Tarny sa usmial.

"Dobre viem, že je to nelegálne. Ale nemáš mi čo vyčítať. To čo bolo minulé prázdniny to hádam nelegálne nebolo?“

"No... máš pravdu. Veď inak by som nenašla tú nádhernú odrodu piktopísma...“

"$ $A Sylvia sa mohla ďalej sťažovať, že nebola ešte v žiadnom životnom nebezpečenstve...“

"$ $Ale.. nepamätáš na to, na Vianoce, keď sme si šli prezrieť v severnom mori stopy minulosti?“

"No... to bolo super... okrem toho, že sa na to takmer prišlo...“

"Nám vo výletoch šťastie moc nepraje...“

"Veď to... A si si istý týmto? Nerada by som sa do niečoho zaplietla s tým, že Sylvia bude v škole na tej prehliadke Cecíliinho štátu sama. Nevedela by som zaručiť, čo by povedala... Hlavne po tom, aký blog písala minule...“

"Hm.. to som nečítal. Čo napísala?“

"É... Rozohnila sa tam o analýze úspešností a neúspešností revolúcií. A zaplietla tam pôvodnú verziu Fentenzíjskej filozofie, takže tak.“

"Sylvia, ako inak..."$ $ Usmial sa Tarny.

"Tak preto by som chcela byť už doma, aspoň večer.“

"Rozumiem.“

"$ $A vlastne... ku komu ma to berieš?“

"Minule som ju našiel. Nie je v spoločenstve, ale má v sebe mágiu. Zaujímavá osoba. A spoznal som ju vcelku náhodne, a ani sa mi nechcelo veriť, že nepozná spoločenstvo.“

"To bolo kedy?“

"No... asi týždeň dozadu.“

"Veríš jej?“

"Nemám dôvod neveriť.“

"Len aby.“

"Preveril som to. Neboj...“

"Ja sa nebojím... všetko je dobre, nik nás neodhalí...“

"$ $Ako má byť."$ $ Ironicky poznamenala Tulienka Deľa. Tarny sa zas venoval riadeniu.

"Prosím ťa ešte to svetlo zaostri – svieti mi do očí. Ako mám nájsť ten vchod do podzemnej vody do jazera pri Chicagu.“

"Pardon Tarny. Posnažím sa."$ $ Vyhlásila a nasmerovala lúč svetla na úzky priechod v podzemnej vode. Mohli si síce vybrať pohodlnejšiu cestu vzduchom, ale keďže cestovali nelegálne, respektíve na "požičanom"$ $ mačičkoese a žiadny z nich nemal vodičský preukaz radšej išli po komplikovanejšej ceste podzemím, ale nebola tam polícia, a to bolo hlavné.

"$ $A sme tu Tulienka Deľa.“

"Chicago... tak predsa sme tu... kde necháš mačičkoes, alebo ho berieš zo sebou? Máš ovládač?“

"Mám. Vylez Tulienka Deľa a použi na nás kúzlo. Vieš ho už dobre.“

"Nepodceňuj ma! To, že nemáme moc zmyslovej mágiu v škole, to neznamená, že knihy a ty nie ste.“

"Ja ti nič nevyčítam. Zas ty vieš lepšie jazyky... a zas k jazde... daj to kúzlo.“

"Dobre Tarny, len zaparkuj najskôr niekde, kde nebude pod obomi dvermi voda. 
Nechceš, aby sme za ňou prišli celý mokrí.“

"Tu budeš len ty Mokrá, ale aj Plavčíková!"$ $ Zasmial sa Tarny v narážke na jej priezvisko.

"Vieš čo Tarny, prečo sme neleteli, keď si lietavý?"$ $ Vrátila mu to.

"Ha, ha, ha, Tulienka Deľa. Radšej mi pomôž, prezri terén hore.“

"Samozrejme Tarny, kamera je zapnutá. Sme pod parkom. Momentálne v časti bez návštevníkov. Si premenený? Ja som.“

"Sme Okej. Vystúp a vyber lano. Bez neho sa nepohneme na vrch.“

"Podaj mi to pozorovadlo. Nemalo by tu byť veľa ľudí. Vzhľadom na to, že je pracovný týždeň...“

"Veľa ľudí tu je aj tak, aj tak. A tu máš pozorovadlo."$ $ Tarny vyslal nanokameru vyššie cez vzduch a pozrel sa na palubnú dosku kde prebiehal obraz.

"Sú tu ešte dve návštevníčky čo sú v dosahu nás. Zapni skrývač.“

"$ $A ty sa nejako dostaň z tohto bahna kde si pristál. Kúzlo funguje.“

"$ $Okej, Tulienka Deľa,"$ $ Tarny zapol cez zmyslovú mágiu motor a pomaly sa začal spolu s mačičkoesom vyhrabávať. Už mali okná nad zemou. Tarny vyhrabával a Tulienka Deľa zahrabávala stopy.

"$ $A sme tu Tarny. Dúfam, že si nepristál v nejakom prameni."$ $ Zasmiala sa. Spomenula si na príhodu keď sa spolu boli pozrieť na pyramídy a skončili v Níle.

"Neboj, sme na tráve. Môžeš vystúpiť. Ja zoberiem krídla. Tri páry.“

"Tri? Aha, jasné aj pre ňu."$ $ Tulienka Deľa vystúpila rýchlo prešla na chodník, aby nebola podozrivá. Tarny zatiaľ zmenšil cez ovládač mačičkoes a dal si ho do vrecka.

"Nie že ho stratíš. Je požičaný a okrem toho na krídlach sa nedostaneme domov. Máme batériu len na 5 hodín letu a nad Atlantikom by sme ju nemohli vymeniť.“

"Pravda, Tulienka Deľa. Radšej si ho vezmi ty.“

"Kde vzlietneme?“

"Niekde kde nie je veľa ľudí. Nesmie nás nikto vidieť.“

"Samozrejme Tarny. Poďme do centra.“

"Do centra?! Tam je málo ľudí?!“

"Nikto si nás nevšimne. Vyjdeme na nejakú veľkú budovu a odtiaľ odletíme.“

"$ $Ako to myslíš?“

"Keď je veľa ľudí, stratíme sa v dave.“

"$ $A nie, že nás uvidia. Dobre vieš, že sme tu ilegálne.“

"Poď."$ $ Chytila ho a spolu vyšli z parku. Tarny mal ešte niekoľko dolárových bankoviek a tie použili na metro. Prešli rýchlo na jednu avenue. Bola hustá dopravná špička. Keď Tarny vystúpil z metra, Tulienka Deľa si prezerala vysoké mrakodrapy.

"Z ktorého?“

"Tarny, nehovor po wymyslensky, budeme podozriví. Ale zas sme neviditeľní..."$ $ Povedala bezchybnou angličtinou.

"Na kamery to neplatí... na to je T...“

"Viem. A teraz nie. Táto budova finančníctva. Nejako sa len hore dostaneme."$ $ Tulienka Deľa spolu s Tarnym vošli do budovy. Tulienka predtým pomocou zmyslovej mágie ich oboch spriehľadnila a spolu sa viezli výťahom na strechu.

"Tu býva?“

"Samozrejme. Aspoň si myslím, že sme na správnej adrese.“

"$ $Určite tam nie je niekto iný?“

"Telepatizoval som jej. O dve minúty sa otvoria dvere a vyjde. Mám to načasované.“

"Kto? Ten, ktorý tam je alebo ona?“

"$ $Ona, Tulienka Deľa. Ale aj tak zazvoň.“

O dve minúty a trinásť sekúnd sa otvorili dvere. Stálo tam dievča, a keď uvidelo 
Tarnyho usmiala sa. Mala tmavohnedé vlasy, ktoré jej lietali po celej tvári a zdalo sa, že ich už dlhšie čakala.

"$ $Ahoj Tarny! Ja som sa bála, že príde otec. Inak prišiel si presne na čas.“

"$ $A ty meškáš trinásť sekúnd."$ $ Oznámila jej Tulienka Deľa.

"Skrývala som si tých pár kníh, ktoré si mi minule dal. Inak... Nabudúce nepríď...“

"Prečo?"$ $ Nechápal. Ona sa len usmiala.

"Lebo teraz odchádzam. Ja som sa rozhodla včera. Odchádzam s vami.“

"Si si istá? Podstúpiš to riziko?“

"Sám si vravel, že som jedna z vás, tak by som mala žiť medzi svojimi.“

"$ $A čo keď som sa mýlil? Predsa, bol to len pocit...“

"Ja som jedna z vás. Aspoň čiastočne viem to, čo si mi ukazoval vtedy. Pozri."$ $ Uprene sa pozrela na pero pohodené na zemi. Nehovorila nič. Pero sa akoby strhlo a začalo sa otáčať. Nakoniec sa zdvihlo asi pol metra od zeme a vzápätí spadlo. Dievča sa rozkašľalo. Oči mala celé červené.

"Si úžasná. Použila si mágiu bez odborného vedenia...“

"Tarny tým myslí seba.“

"Teba som jej ešte nepredstavil. Paulina, toto je Tulienka Deľa, Tulienka Deľa, to je Pauline. A prosím zatvor dvere."$ $ Keď boli dvere zavreté, obrátil sa zas na Pauline.

"Pauline, teraz posledné tvoje slovo. Je to v mojich silách, našich silách teda, ťa odtiaľto dostať.“

"Ja idem. Môžeme odísť. Zbalila som si veci. Môžeme ísť.“

"Skutočne?“

"Áno.“

"V tom prípade vyjdime na balkón, pripevni si krídla.“

"Krídla? My ideme lietať. Nasaď si ich a uvidíš. Tu máš.“

"$ $Aby si nespadla."$ $ Poučila ju Tulienka a zapla si ich.

"$ $Ukáž, zapnem ti to."$ $ Naťukala na ovládač pár čísel a dodala. "Môžeme. Neboj sa. Nič sa ti nestane.“

"Radšej nie... predsa len...“

"Rýchlo choď na balkón! Za chvíľu sa načíta plná rýchlosť a mohlo by sa všeličo stať. Za dvadsať sekúnd... poď!“

"Ja..."$ $ Tulienka Deľa ju schmatla za ruku a vybehla za Tarnym na balkón. "Bojím sa. Sme na trinástom poschodí!“

"Neboj sa. Budem ťa držať."$ $ Tarny sa zľahka prehupol cez zábradlie.

"Tarny!"$ $ Skríkla Pauline a naklonila sa dole.

"Ja som tu."$ $ Ozval sa Tarny z hora.

"$ $Idem,"$ $ Tulienka Deľa vyletela k Tarnymu, ale vzápätí rýchlo a prudko zletela zas dolu.

"Pozor!"$ $ Pauline zrazu prudko zamáli krídla, dostali vysoký stupeň mávania a ona sa náhle vysokou rýchlosťou rútila hore dospodu balkóna nad nimi. Tulienka Deľa ju schytila presne pred tým ako narazila do balkóna časťou krídla.

"Nabudúce si daj pozor.“

"$ $A dokelu."$ $ Skonštatoval Tarny.

"Musíme ujsť.“

"Prečo?“

"Pozri sa hore. Krídlo rozbilo balkón. Jegrigsen sa dozvie kde si. Že si s nami.“

"$ $Ona je...?"$ $ Tulienke Deli to Tarny zjavne ešte neoznámil.

"Potom... vysvetlím to potom. Nie je s D..."$ $ Tulienka Deľa mu dôverovala.

"$ $A nemôžeme to opraviť?“

"$ $Asi nie Tulienka Deľa. Vystopoval by nás. Môžeme urobiť jediné, ujsť. Sme v nebezpečenstve a keby sme prišli niekde do našej komunity tak by sme vážne ohrozili ich.“

"Jegrigsen?“

"Tvoj otec.“

"$ $Ale veď... on sa volá inak.“

"Jeho meno v našom svete je Jegrigsen. A je nebezpečný. Veľmi."$ $ Pauline stále držala za ruku Tulienka Deľa a až teraz pochopila na čom sú.

"Kde pôjdeme?“

"Do nejakého iného väčšieho mesta, to bude najlepšie. Neboj sa, nenecháme ťa samú. Len budeme musieť pravidelne meniť našu polohu, aby nás nenašli. Z tohto pohľadu je najbezpečnejšie miesto Wymyslensko, ale skutočne tam nechci ísť. Asi by si sa už nikdy nedostala späť.“

"Prečo Tarny?“

"Pauline, to je základ Wymyslenskej politiky. Nenávisť voči nám. Má to historické súvislosti a Cecília si na tom stavia kampaň.“

"Kto je Cecília?“

"No myslím si, že vzduch nie je najlepšie miesto na vysvetľovanie. Ak by som mohol navrhnúť, poďme do jednej z našich reštaurácii,“

"My sme vo vzduchu?!"$ $ Pauline sa pozrela dole a zvreskla.

"Neboj sa! Nepútaj na nás pozornosť!“

"$ $Ako to, že nespadneme?“

"Normálne. Buď kľudná, prosím ťa. Tieto krídla sú vyrobené z jedného z najpevnejších, ale najľahších materiálov s dobrým odporom vzduchu. Sú na batériu. Tebe ešte ostáva 5 hodím. Takže nespadneš. Stačí sa to naučiť ovládať. Tak na začiatok, nepozeraj sa dole. Už ťa môžem pustiť?“

"Radšej nie... inak do akej reštaurácie to chcete ísť, myslím, že jednu poznám. Bola tam dosť dlho, ale vždy som ju videla len ja. Proste nikto iný.“

"Tak to je naša. Kde je?“

"$ $Uptown."$ $ 

"Tam to poznám. Môžeme ísť. Ale... ...Pauline čo to robíš!?“
Objavili sa nad reštauráciou v Uptown.

"$ $Ako sme sa tu dostali?“

"Netuším, absolútne netuším. Len som si proste predstavila toto miesto a boli sme tu.“

"Tak toto je len jedno vysvetlenie."$ $ Zamračene oznámil Tarny. \textit{"Že mi toto nenapadlo."} Zatelepatizoval Tulienke-Deli. "Si démon, respektíve polodémon. Len oni sa môžu premiestňovať."$ $ Precedil Tarny.

"Démon? Ja?“

"$ $Ak sa stalo to, čo hovoríš, tak z najväčšou pravdepodobnosťou áno.“

"Nie...“

"$ $Ale démon nie je len to, čo sa o nich hovorí na zemi. Naopak, ľudia majú poväčšine zlé informácie. Za posledné desiatky rokov sa takmer všetci démoni zhromažďovali pri pánovi z veľkým D. Ale to neboli prirodzení démoni. On asi našiel spôsob ako z obyčajného človeka, či jedného z nás urobiť démona. Preto sa ich ľudia boja. Ale sú aj démoni, čo sa narodili ako démoni. Démoni sú daní krvou. Dedí sa to. Ako ty. Jegrigsen je démonom od D, ale ty si jeho dcéra teda polodémon,“

"$ $A aký je rozdiel medzi démonom a polodémonom?"$ $ Zaujímalo Pauline.

"Podľa legiend sú iba dvaja praví démoni, ostatní sú polodémoni. Jedným z démonov je D. A o druhom..."$ $ Tulienka Deľa mu venovala škaredý pohľad. Tarny to zaregistroval a pokračoval. "$ $O druhom sa veľa nevie. A rozdiel je ten, že démoni majú nejaké schopnosti, ale polodémoni ich nemajú všetky, ale podstatnú väčšinu hej. Obaja sa vedia premiestňovať, obaja vedia meniť seba na niekoho iného... Ale len démoni nezomierajú. Ak ich aj zabijú zas ožijú, ale nie ako duchovia.“

"Preto nejde D zabiť."$ $ Zašepkala Tulienka Deľa.

"Kto je D?“

"Démon, vodca zla, je tajomný a neznámy. Hovorí sa, a je to podľa mňa aj pravda, že sa pokúša vytvoriť niečo, čo porazí Wymyslensko aj Spoločenstvo M. Niečo... Aby bol on vládcom. Súvisí to s proroctvami. A... A keď tak rozmýšľam, viem kde teraz pôjdeme. V Londýne žije jedna veštica, Deoque, mohla by ťa naučiť využívať svoje schopnosti. Nebolo by totiž veľmi praktické, keby si sa premiestnila vždy tam, na aké miesto si pomyslíš. Tak by si mohla za deň preletieť celý vesmír."$ $ Prerušila ho Tulienka Deľa.

"Tarny si si istý s Deoque? Vieš aká je?“

"$ $Ako to myslíš?“

"Keď mala naposledy návštevu nedopadlo to príliš šťastne.“

"Viem, ako to dopadlo. Ale to bol D.“

"Deoque?“

"Predpovedala osud Napoleona, vznik nášho sveta na zemi aj porážku Solemy Krutej vo Wymyslensku. Tvrdí sa, že ona sama je démonkou, ale to je blbosť, lebo... no, nič. Za posledných pár storočí sa nepohla z Londýna.“

"Pár stoviek? Napoleon? Ako to...? Ja ničomu nerozumiem.“

"Poďme dole. A prosím ťa. Snaž sa nás tam nepreniesť.“

"To sa nedá, keď si pomyslím silnejšie na nejaké miesto, som tam. Už sa mi to párkrát stalo.“

"Prečo si to nepovedala?“

"Nechápala som tomu a nejako mi to vypadlo z pamäte.“

"Tak sa zatiaľ snaž to nerobiť. Asi chápeš, prečo nie pred ľuďmi, ale ani tí od nás nie sú k démonom zrovna priateľský. Deoque by ti mala pomôcť.“

"$ $Ak nás rovno nezabije, Tarny."$ $ Podotkla Tulienka a zamračila sa. Boli vo vzduchu nad ružovou svetielkujúcou budovou reštaurácie.

"Zabiť?“

"Tarny zabudol na pár, ehm, povahových čŕt Deoque.“

"Viem, že je tak trochu prchká, ale na druhej strane...“

"Nevyhováraj sa Tarny. Inak, zlož si krídla, sme len meter nad zemou.“

"Sorry.“

"Nemáš sa za čo ospravedlňovať... poď Pauline,"$ $ Tulienka Deľa šťukla gombík na krídlach, tie sa okamžite deaktivovali a zložili do jedného súvislého balíka. Stáli na parkovisku a vtedy sa odneviditeľnili. Samozrejme, Tarny aj Tulienka Deľa ostali v ľudskej podobe.

\begin{center}
*
\end{center}

\textit{Neviem, kto som. Bojím sa seba. Ja... ja som len tak zdvíhala predmety, písala som bez dotyku a neviem ako som to dokázala... keď sa na to sústredím, prestanem vidieť tento svet, ale vidím nejaké pračudesné pole... neviem, čo si mám o tom myslieť. Ostatní to nedokážu... Veď hádam by som to tom počula... Bojím sa toho... kto som?}

\textit{Moja matka (teda adoptívna) nechce, aby som šla na školu, akú chcem ja... ale prečo? Nechápu ma! Ja... čo vlastne chcem? A kým som...?}

\begin{center}
*
\end{center}

Sadli si k bočnému stolu, Tarny objednal Buble banker pre každého, pre seba kostičkovú, pre Tulienku Deľu chlebovú a Pauline si váhavo objednala Topinambúrovú. Medzitým, ako im doniesli jedlo Pauline sa usadila na pohodlnom kresle a spýtavo sa na nich zahľadela.

"Tak čo? Už mi to vysvetlíte?“

"Na Zemi je spoločenstvo M. Toto spoločenstvo založila Izabeta Tlogenová odštiepením sa od štátu na planéte Fanasa, ktorá je zo zemou spojená červou dierou. Tak na polhodinu cesty. Ale to nie je podstatné. V tomto štáte Wymyslensko bol vždy demokraticko – autoritárny režim. Keď Izabeta zakladala ríšu, bol tam dosť tvrdý policajný režim. Najväčší problém toho štátu je, že ak si niekoho ľudia zvolia, zvolia ho aj desaťkrát za sebou. Niekto iný sa na to miesto dostane zväčša vďaka prevratu. Terajšia corlovne je Cecília ktorá rapídne zhoršila vzťahy so Spoločenstvom M, i keď v porovnaní s predchádzajúcou corlovne... Ale nechaj tak... to je až príliš zložité. Cecíliina kampaň je vždy postavená na nenávisti k nám, vyťahuje tabuľky s číslami o počtoch obyvateľov, ako ich raz prevážime a zaútočíme. Vzťahy medzi nami nikdy neboli ideálne, ale Cecília s predchádzajúcou corlovne to zhoršila. Represie, zákazy chodiť na zem, vízové povinnosti. Najnovšie Cecília vyhľadáva mladých od nás a unáša ich. Berie ich do prevýchovy a učí ich nenávisti a láske ku Wymyslensku. Hádajú sa z Izabetou o veľkú knihu proroctiev. Cecília nás obviňuje zo spolupráce s D. D to využíva.“

"Počkaj, pár nepresností. Po prvé, kto je v spoločenstve M a tom Wymyslensku vlastne Jegrigsen? Po druhé, ako to, že Izabeta založila spoločenstvo M a stále žije, keď sa tu hovorí o Jegrigsenovi... A po ďalšie, čo s týmto mám spoločné ja?“

"Toto,"$ $ povedal Tarny, zmenil sa na svoju pravú podobu a pohľadom zdvihol pohár, čo čašník doniesol a naklonil ho k Pauline.

"Ja nevyzerám ako vy."$ $ Tarnyho výzor ju nevyviedol z miery – už ho tak videla. Nevyzeral ako človek, ale ako nejaký iný druh inteligencie. Uši mal ako pes, tri ruky... Bol vysoký ako normálny človek. Tulienku Deľu si Pauline prezerala pozornejšie. Tú ešte nevidela. Ruky jej vyzerali ako niečo medzi normálnou, ľudskou rukou a plutvou, ktorá bola trochu osekaná. Normálne bolo vidieť prsty, ale neboli zaguľatené ako u človeka, ale zostrené. Mala podlhovastejší nos, a uši jej takmer vidieť nebolo. Jej koža bola sivá, ako keď bola Pauline ako dieťa v zoo a pozorovala delfíny a tulene.

"Ja viem,"$ $ odpovedal Tarny na jej námietku. "Tu nejde o výzor, u nás vyzerá každý inak. Tu ide o schopnosti. Veď toto isté si ukázala v byte. Mágia. Je to druh energie, veľmi tajomný, ale nedokáže ho zachytiť každý, len človek s magickými schopnosťami. Wymyslensko nám, obdareným mágiou, v dobe prenasledovania čarodejníc poskytovalo úkryt. Vtedy bol vo Wymyslensku aký – taký právny štát. Wymyslenčania žijú veľmi dlho, priemerne tristo rokov. Po nastúpení autoritárskych režimov Izabeta odišla na pár rokov preč, niekde to Tramtárie, ale to každopádne nik okrem nej o tom nevie. Vrátila sa z dvoma svojimi deťmi. Vtedy začala so svojím priateľom, čo jej odpustil aj neveru, spájať wymyslenčanov aj veľa tých, čo boli zo Zeme a mali mágiu, a pripravovať odchod. Nechceli prevrat, lebo videli, ako to chodí. Prisahali na Izabetinu dýku a svoju krv, že vytvárajú novú krajinu a budú dodržovať jej pravidlá. Na to odišli na zem. Touto prísahou dosiahli zastavenie starnutia... a odolnosť voči chorobám. Len ak sa im niečo stane a zomrú, vrátia sa ako duchovia. Priehľadné bytosti z hlavami a rukami, ale vedia používať mágiu. A toto je dedičné. Chápeš?“

"$ $Asi hej. Napadla mi teraz ešte jedna otázka, teda dve. Čo je to tá veľká kniha proroctiev? A ako ste ma našli? A ešte moja druhá otázka.“

"Jegrigsen?“

"Áno.“

"$ $O tom nech hovorí Tulienka Deľa, to je sčítaná osoba do akejkoľvek literatúry."$ $ Tulienka a mierne zapýrila, i keď pocítila náznak sarkazmu.

"$ $Okej. Jegrigsen je démon, polodémon. Spolupracuje s D a vraj tvoj otec."$ $ Pri tvoj otec stíšila. "Veľká kniha proroctiev ukrýva predpovede. Každá živá bytosť má svoju časť proroctva, či rovno celé proroctvo. Wymyslensko si tú knihu privlastňuje a Izabeta si na ňu robí nárok tiež. Nadôvažok vydala hlavná amazonka vílej ríše uznesenie, že kniha patrí im. D sa chce knihy zmocniť a dosiahnuť, aby sa proroctvá splnili.“

"Prečo? A čo to má spoločné s mojou druhou otázkou?“

"Počkaj... Kniha predpovedá jeho zánik. Pomaly, ale iste. A k otázke. Kniha je mimoriadne zaujímavá v tom, že keď ju vieš používať nájde každého kto v nej má proroctvo. Tarny sa...“

"Tichšie, niekto nás môže začuť. Už to, že sme tu nelegálne by nám mohlo poriadne zavariť, pretože nemám vodičák a mačičkoes som si bez vedomia otca požičal. A keď sa ešte nemôžeme vrátiť...“

"Pokračuj prosím ťa Tulienka Deľa. A Tarny, nabudúce nekradni.“

"$ $Ale...“

"Nechaj tak."$ $ Tulienka Deľa sa poobzerala, či niekto nepočúva. Počkala kým im čašník doniesol nápoje a polievky a pokračovala chlípajúc polievku. "Tarny je vzdialený príbuzný Izabety Tlogenovej. Keď mala raz u seba knihu, potajomky s ňou experimentoval. A s jedným dôležitým proroctvom mu vyšla tvoja pozícia. Tak sa tam raz vybral a našiel ťa. Tarny..."$ $ Naraz prerušila rozprávanie a sledovala dvojicu v uniforme.

"Do kelu, ehm... Pauline bola si už niekedy v Londýne?“

"Nie.“

"Kde si bola odtiaľto najďalej?“

"V New Yorku.“

"Prenes nás tam je tu polícia, a ty formálne k nám nepatríš. Nemáš preukaz, a keby videli môj, môžu sa dovtípiť, že sme tu nelegálne.“

"Mám to urobiť? Aby sme sa niekde neocitli...“

"Neboj sa. Ja ti dôverujem."$ $ Tarny a Tulienka Deľa sa chytili Pauline za rukáv a ona pred očami prekvapených ľudí zo spoločenstva M zmizla, a následne sa všetci traja, Tulienka Deľa v ruke s nápojmi ocitli pred sochou slobody.


\newpage
\chapter{Nevermore}

Rok predtým:

Keď Jean prišiel späť do laboratória, privítal ho starší muž a žena, očividne mala predkov s Wymyslenska, ktorá mala spustené, podlhovasté uši, obaja v plášťoch, muž držal v ruke ampulku a prázdnu, novú injekčnú striekačku. Chvíľu sa niečo rozprávali po francúzsky, až kým žena, očividne im nerozumela, prehovorila wymyslenčinou.

"Ste si tým úplne istý Jean?"$ $ Jean prikývol.

"Pán to chce.“

"My sme neutrálny, Jean, a dodržujem zákony.“

"Tak prečo spolupracujete doktorka?“

"Klient je klient. A toto je biznis."$ $ Pousmial sa.

"Začneme doktorka?"$ $ Opýtal sa muž.

"Rozprávam sa z naším klientom François, Jean? Chcete začať alebo...?"$ $ Jean prikývol.

"Veď na to som tu.“

"V poriadku. Nasledujte nás prosím. Inak, v objednávke ste neurčili vek klonu. Chcete, aby sa zhodoval z vaším, aby bol vek dieťaťa, či z polovice zrýchlený vývin?“

"To prvé pani...?“

"Meno neuvádzam. V záujme diskrétnosti. Prvá forma trvá rok.“

"V poriadku. Zaplatím tu v hotovosti. O rok prídem. A nie, že niečo pokazíte.“

"Nemajte obavy. Možno to bude trochu bolieť..."$ $ Injekciou mu odobrala krv, Jean jej dal sľubované peniaze, a akonáhle odišiel doktorka dala krv do prístroja, aby o rok bol produkt hotový.

\begin{center}
*
\end{center}

\textit{Podpálila som stôl. Neviem ako sa mi to podarilo, ale on vzbĺkol. Pohádali sme sa s našimi a ja som si náš dom predstavovala v plameňoch, až som ho... podpálila... bojím sa.. bojím sa, veľmi... Ja chcem, aby bol niekto ako ja...}

\begin{center}
*
\end{center}

"Je možné, že piktopísmo v knihe je staršie ako sama kniha, a keďže žiadny adekvátny preklad nie je dostupný, treba nájsť ten čo je vraj v Stonehenge."$ $ Morja sa zamračila a rukou vyčarila mapu Anglicka.

"Stonehenge?“

"Je pár legiend. A všetky ukazujú na Stonehenge. Prvá vec – vek. Stonehenge polo postavené, respektíve začalo sa stavať asi 1900 pred založením Wymyslenska alebo ako sa vraví medzi ľuďmi, pred Kristom. Vtedy podľa výskumov vznikla aj kniha. Ako vraví legenda, ako druhá z troch veľkých kníh. Druhá vec – jeden z starších zachovaných rukopisov z doby keď vznikla kniha, tvrdí, citujem: ‚I mocní čo zoslali knihu nám nedali nápis ako lúštiť písmo. Lenže abeceda je na inej planéte, kde mágia je prenasledovaná a na ostrove, čo bol spojený s pevninou a veľký kamenný kruh postavili.‘ Takže, kniha je veľká kniha proroctiev. Planéta kde mágia je prenasledovaná je asi zem, prenasledovanie čarodejníc. Ostrov spojený s pevninou, je ich síce viac, ale o kamennom kruhu dopĺňa informáciu na Anglicko. A po tretie. Je ti myslím jasné, že Stonehenge nepostavili ľudia. Myslím klasický. Najnovšie naše výskumy dokazujú, že Stonehenge bolo postavené cez druidov, ktorí mali schopnosť mágie. Je tam nezanedbateľná magická stopa. A tá pomaly vyprcháva. Čo nasvedčuje tomu, že to nebolo prirodzené množstvo, ale vyvolané niekým.“

"Rozumiem. Ideme?"$ $ Niela prikývla a začala Morji zverovať plán.

"Z Bratislavy ideme do Londýna. Predpokladám, že máš falošný pas.“

"Mám. Asi päť.“

"Úžasné. V Londýne pôjdeme taxíkom na Stonehenge. Bude pre nás bezpečnejšie neisť mačičkoesom. Mohli by nás zastaviť. A predpokladám, že tvoj preukaz už neplatí, ak ho vôbec máš..."$ $ Morja prikývla. "$ $A vieš čo robia hrozby vojny s D a už začatá studená vojna s Cecíliou, teda Wymyslenskom.“

"Viem. Máš letenky?“

\begin{center}
*
\end{center}

\textit{Som nažive. Vidím ho. Musím ho zabiť. Cítim to. Nenávidím to. Nikdy som to necítil, ale viem, že ho nenávidím. Musím ho zabiť. Musím ho zničiť. A toto všetko s ním...}

"Doktor Vorien? Prosím vás odpojte umelé dýchanie a prebuďte klon. Potom ho zas uspite. Nemôžeme si dovoliť neposlušnosť. Viete, že pre náš biznis to môže mnoho stáť.“

"$ $Odpájam.“

Doktorka sledovala mozgovú aktivitu klonu a zrazu skríkla. "Doktor Vorien! Okamžite pripojte dýchanie a zapnite narkózu. Neplánované prebudenie!“

"Čo sa deje doktorka?“

"Klon sa prebúdza, okamžite mu dajte narkózu! Nemôžeme dopustiť prebudenie!"$ $ 

Zmätený doktor sa obzeral po narkóze, ale klon sa medzitým prebudil, a miestom, kde stál pred malou chvíľou komplex, sa ozval výbuch a ostali z neho len ruiny.

V ich strede stál muž a kričal. Z rúk mu prúdila červená skaza a metala kamene, tehly a ostatné zbytky ruín do vzduchu kde sa rozpadali na atómy, ktoré sa vzápätí spájali na bralá a padali na zem. A muža obchádzali akoby bol zabalený do pevnej priehľadnej kukly. Skríkol, nie, zreval, jediné slovo zo slov, ktoré počul, ktoré si pamätal, ale jeho spomienky to neboli.

"Nevermore!“

\begin{center}
*
\end{center}

Loviisa sedela a prehadzovala si ohnivú guľu z jednej ruky do druhej. Mala si robiť úlohu do školy, ale to sa jej nechcelo.

"Loviisa! Je takmer večer! Choď si napísať tú esej na tému riadenie krajín zeme a rozdiely medzi ním a riadením spoločenstva M.“

"Mami, prosím ťa, nechaj ma. Idem na dvadsať päťku do reštaurácie písať. Mám tablet a napíšem si to na ňom.“

"Dobre Loviisa, ideš MHD?“

"Hej, mami,“

"Máš peniaze na večeru?“

"Dobre vieš, že mám kartu do reštaurácie Pri Jazere.“

"Prepáč.“

"Môžem ísť?“

"Áno, samozrejme. Ale daj si pozor. Dobre vieš... Loviisa."$ $ Obzrela sa za matkou a odvetila.

"Samozrejme..."$ $ Loviisa odkráčala k najbližšej zastávke MHD a kúpila si lístok na 25. Metro jej išlo o päť minút. Loviisa potlačila nutkanie vytvoriť zas ohnivú guľu, pre možných ľudských cestovateľov MHD. Tých päť minút sa jej zdalo ako večnosť. Naraz počula ako niekto kráča dole po schodoch. Tým niekto bola žena. Nevyzerala ako človek. Zovero.

"Kto ste?"$ $ Žena sa jej milo spýtala. Loviise sa zdalo, že ju už niekde videla.

"Ja,"$ $ odvetila a vtedy Loviise niečo napadlo. "Prečo nemáš ľudskú podobu, mohli by sme sa prezradiť. Je to proti naším zákonom."$ $ Povedala jej Loviisa o jednom z mála zákonoch o ktorých vedela.

"Ja vaše zákony neuznávam. Poď.“

"Ja nejdem, nemám chodiť s cudzími.“

"$ $Oklamali ťa Loviisa, poď.“

"Nie.“

"Tak dobre!"$ $ Žene vytryskli z rúk svetlé kruhy a Loviisa padala na zem. Žena bleskurýchle vytiahla z rúk niečo ako malé autíčko a hodila ho na zem. To sa zrazu zväčšilo a otvorilo dvere, takže Loviisa spadla priamo na mäkkú pohovku. Žena vošla do toho čudného auta s poznávacou značkou AA1114KR a zatvorila dvere. Ľudia v metre čo tam práve zastalo sa nestíhali čudovať. Auto zrazu malo krídla a odletelo. Pretože to nebolo auto, ale nový model mačičkoesu typu AA.

\begin{center}
*
\end{center}

Deoque sedela na kresle pred svojím počítačom a sledovala prenos kamery ktorú namontovala na konci aj na začiatku ulice na ktorej práve bývala. Coniston Gardens bola jej tridsiata adresa za posledných trinásť rokov, ale odvtedy čo sa Jegrigsen Goon pridal k D mohlo byť každé miesto na zemi nebezpečné. Ale pre ňu nie. Jej sa D bojí, a ona ho napokon zabije. Veď to si aj zaslúži. Chvíľu premýšľala aj že sa odsťahuje z Londýna, ale nakoniec vždy Deoque dospela uvažovaním k tomu, že v Londýne žije od napoleonských vojen kedy odišla z Uhorska. Pri pohľade na ulicu sa jej najskôr nezdalo nič zvláštne. Ale z jej každodennej rutiny ju prebrala trojica ľudí, zrejme zo spoločenstva M, ktorí sa naraz len tak zjavili na ulici. Nemohli byť neviditeľní, voči zmyslovým kúzlam bola Deoque imúnna od istých experimentálnych kúziel. Musel to byť démon, respektíve polodémon. A na tých bola Deoque expert. Naposledy sa stretla dobrovoľne s D, a to, čo sa Deoque vtedy podarilo sa nikomu zo spoločenstva alebo Wymyslenska nepodarilo, alebo nechcelo zatiaľ zopakovať. Aspoň bez ujmy v podobe zdravia alebo pridania sa k D. Ani jeden z tých ľudí nevyzerá ako D. Pomyslela si Deoque. Ale D by nechodil po ulici len tak. Jej paranoja jej našepkávala, že sú to ľudia D, ale ten by predsa mohol vedieť čoho je schopná. Automaticky chytila do ruky jej Laserový meč s dosahom na dva metre, skontrolovala batériu a pre istotu ju nahradila novou. V druhej ruke mala zosilňovač Solan kruhov, slnečnej energie ktorá dokázala ľudí (a Wymyslenčanov a démonov) omráčiť a pri väčšej sile aj spáliť. Asi ľudia, sa približovali čím ďalej, tým viac. Deoque vyšla pred dom.

\begin{center}
*
\end{center}

Policajt Jason Newman sa pozrel na hodinky a zistil, že sa jeho služba končí. Oproti nemu vykročil muž. Jason sa naňho pozrel a hneď sa mu zdalo, že on nie je človek. Mal predtuchu. Muž ho desil. Toto sa mu ešte nestalo. Zrazu, vzápätí, v sekunde, sa Jasonovi zdalo, že zo vzduchu sa bez dotyku sa zdvihol prach, sformoval sa do palice, ktorá mu v zlomku sekundy tlačila na krk.

"Kto ste? Čo chcete?"$ $ Sťažka sa spýtal. Muž ostal chladný. Priblížil sa.

"Kde je Jean?"$ $ Desivo chladne sa spýtal. Jason cítil, že by to mohli byť jeho posledné sekundy. V hlave si namáhavo triedil myšlienky, no žiadny Jean v nich nebol.

"Netuším. Kto ste, a čo odo mňa chcete?"$ $ Muž to znova zopakoval.

"Kde je Jean?“

"Hovorím, že neviem, kto ste a čo chcete?! Ja nič neviem, neviem!"$ $ Muž desivo chladne zúžil zreničky. Vyzeral, že vôbec nevedel čo mu bolo hovorené.

"Nevermore!"$ $ V tom okamihu o kúsok pohol rukou a celá ulica, okrem muža explodovala. Ten zas zreval ako keď prvý raz okúsil svoju silu. 

"Nevermore!"$ $ Mesto sublimovalo, vzduch zas desublimoval a ako ruiny padal na bývalé obydlia. Len Nevermore, ako sa sám nazval, ostal nedotknutý.

\begin{center}
*
\end{center}

Morja otvorila notebook a pozrela na správy so spoločenstva M. hneď prvá správa na serveri \url{www.newnews.ym} ju zarazila.

"$ $Och, môj bože,“

"Čo sa stalo Morja?"$ $ Spýtala sa Niela.

"Pozri."$ $ Morja klikla na správu \textit{Mesto pri Londýne záhadne zmizlo}.

\begin{center}
\textit{\textbf{Mesto pri Londýne záhadne zmizlo}}

\textit{Mesto v našom jazyku Lowwer, ktoré leží pri Londýne zmizlo. Podľa našich zdrojov pozemská polícia si myslí, že mesto bolo zničené explóziou, no naši fyzici, chemici a odborníci na mágiu svorne tvrdili. že mesto bolo zničené pri veľkej dávke mágie po ktorej sú tam nepochybné stopy. Posledné výpočty mágie, ktoré boli robené v meste pred dvoma mesiacmi zaznamenali o 99\% nižšiu dávku, ktorá aj tak nebola prirodzená, pretože v meste žila aj komunita Spoločenstva M. Musel tam byť nejaký zvlášť silný mág, čo dokázal túto dávku vyvolať. Je isté podozrenie, že útok zincesoval D. Satelitné snímky ukazujú, že v zlomku sekundy mesto zmizlo. V našom satelite z vysokým rozlíšením sú, ale aj snímky, ktoré ukazujú, že mesto pravdepodobne sublimovalo. Cez snímky sa nám podarilo zistiť, že v meste bol predsa len niekto, kto katastrofu prežil. S najväčšou pravdepodobnosťou to bol útočník. V meste pravdepodobne zahynulo všetkých 7 000 obyvateľov z toho asi sto občanov spoločenstva M. Pripravujeme podrobnosti.}
\end{center}

"Mám podozrenie."$ $ Hlesla po dlhej odmlke Niela.

"Áno?“

"Toto nebol D.“

"Prečo? Kto by takéto dačo urobil?“

"$ $Osamelý vlk. D podľa mojich informácií nemá nikoho z takouto veľkou mocou, ak samozrejme nerátame Jeana Parvasîeho, ale ten bol tento týždeň sledovaný, nepohol sa od Uluru v Austrálii. To sa nedá len tak urobiť. Musel by to byť niekto bez emócií. Alebo s vygumovaným mozgom. V hneve. Hnev ti dá väčšiu silu ako si myslíš Morja.“

"$ $Alebo na príkaz.“

"$ $A Jean je pri D z vlastnej vôle, ale on by to bol schopný...“

"$ $Ale nebol to on, to bude niečo s tým hnevom... afektom... alebo niekto úplne chladný...“

"$ $A nerobí to D svojím služobníkom?“

"Hm... nie, podľa mňa to musel byť niekto s upraveným mozgom. A myslím, že toto súvisí so zmiznutím nemocnice v severnom Francúzsku.“

"Čo sa tam stalo? Severné Francúzsko? Tam bola tá kontroverzná klonovacia klinika, nie?“

"Veď práve to. Stalo sa to, čo v tom meste. A ja mám podozrenie, že tam bol po klon aj jeden z najnebezpečnejších služobníkov D. Jean. A aj to, že to urobil jeho klon.“

\begin{center}
*
\end{center}

"Tarny?“

"Áno?“

"Si si istý, že Deoque nás prijme? A pomôže?“

"Prečo? Je istá šanca že...?“

"Deoque je tak trochu... no, nevyspytateľná, je nažive už stovky rokov, ale od celého spoločenstva M si vždy držala odstup. Nikdy nechcela byť našou súčasťou. Napriek prenasledovaniu čarodejníc dokázala prežiť. Žila po celej Európe. Ovplyvnila názory Márie Terézie, jedinej žene na Rakúsko – Uhorskom tróne. Potom žila vo Francúzsku a následne utiekla cez prvú svetovú vojnu do Londýna. Je prchká a strašne paranoidná. Pre svoju bezpečnosť by urobila čokoľvek. A nestojí na žiadnej strane, je nebezpečná a vôbec nechápem prečo sa Tarny rozhodol ísť sem.“

"Skutočne Tarny?“

"Bola to jediná možnosť.“

"Tarny!“

"Prepáč Tulienka Deľa, ale sú to okolnosti, Pauline je dcérou Jegrigsena, jedného z polodémonov D, je to polodémonka a D násobí svoju silu. A Pauline je tá z proroctiev.“

"Chceš zo mňa urobiť zbraň?!“

"Nie, ja ti chcem pomôcť, a ako hovorím...“

"Prestaň Tarny!"$ $ Povedala Pauline nahnevane.

"Deoque..."$ $ Vydýchla pomedzi ich spor Tulienka Deľa. Vtedy spozorneli.

\begin{center}
*
\end{center}

"Tí okoloidúci sú čudní, vyzerajú ako deti, myslím, že nebudú od D, ale nikdy netreba strácať ostražitosť,"$ $ pomyslela si Deoque. Prezerala si ich s diaľky. Chlapec sa o niečom hádal s dievčaťom, čo by naň najmenej tipovala príslušnosť k spoločenstvu M, či k D. Zrazu spozorneli. Zbadali ju. Deoque sa snažila zapojiť čo najväčšiu časť svojho mozgu do premýšľania. Keď bola vzdialenosť medzi nimi len okolo sto metrov Deoque okolo seba, automaticky vyčarila obrannú stenu. Dobre poznala vojenské stratégie, sama niekoľko rokov strávila zisťovaním stratégií oboch strán v oboch svetových vojnách. Ale D žil ešte dlhšie ako ona sama. Deoque nebojovala, ani nesympatizovala so žiadnou s dvoch, respektíve troch strán. Celý svoj život venovala štúdiu a predpovediam. A samozrejme sama sebe. Síce robila zodpovednú vedu, ale jednou z jej stránok osobnosti bolo to, že bola vešticou. Vedela čo sa stane, keď prijala štatút veštice v piatich chrámoch Fanasy a Zeme. Vedela, že to čo predpovie nebude môcť nikomu prezradiť. V tom boli proroctvá iné. Skutočné pravdy ukrývala ona. Vtedy keď v poslednom chráme, v chráme kde je uložená schránka s dýkou slova. Odvtedy... bolo to takmer pred pol tisícročím.

\begin{center}
*
\end{center}

Všetky svetové médiá mali v ten deň na titulnej stránke len jednu správu. Mesto pri Londýne záhadne zmizlo. Nikto netušil kto to mohol byť. Svetové vlády sa začali obviňovať. Objavovali sa špekulácie o možnom jadrovom výbuchu no žiadne neobvyklé množstvá rádioaktivity sa nenamerali. Objavili sa teórie o konci sveta, božom treste, iní sa to pokúšali vysvetliť vedecky. Bezvýsledne. Rovnako, ako médiá spoločenstva, ktoré mali aspoň nejaké stopy.

Valná väčšina ľudí a zovero v spoločenstve sa domnievalo, že sa jednalo o útok D, ale agenti, ktorí mali na starosť pátranie, sa skôr domnievali, že išlo o nejakého nového nepriateľa. Jedným s tých, ktorí to, čo sa tam odohralo vedeli, bol aj líder druhej strany vo studenej vojne medzi D a spoločenstvom M a tiež Fanasou. D.

"Jean.“

"Áno pane?“

"Vysvetlite mi to Jean.“

"Netuším čo sa stalo v nemocnici. Isté je len to, že klon zmizol.“

"Vy ste hlupák, Jean, keby ste nemali takú cenu, tak by som vás okamžite zabil. Nie je vám jasné, že to spôsobil klon?! Jean, vy pôjdete a ten klon zabijete. Rozumiete?!"$ $ Jean sa sklonil.

"Rozumiem pane,“

"Tak to okamžite splň, máš čas, kým sa o klone nedozvie spoločenstvo M.“

\begin{center}
*
\end{center}

"Deoque..."$ $ vydýchla Tulienka Deľa. Pauline videla okolo nej jej ochranné pole. Ešte nikdy ho predtým nevidela, tak zastala.

"Hej! Čo to je okolo Deoque?!“

"Ja nič nevidím..."$ $ Začal Tarny, ale spozornel a uvedomil si zakrývanú obranu okolo Deoque.

"Vie že sme tu. Inak by asi nevyšla. Ako... Máme aktivovanú zmyslovku...“
Deoque ich počula. Ľudské dievča bolo polodémonka. Ako vo veštbe. Konečne mohla prehovoriť, vydýchla si, o ďalšiu veštbu menej. To, kto sú dvaja zvyšní zovero nemala poňatia. Nič o nich nehovorilo. Nič, čo si držala v pamäti. Ale zaujímalo ju to.

"Kto ste?"$ $ Zvolala na nich, keď boli ku sebe bližšie. Mlčali. Nechceli to povedať, D a jeho... začínala prudko dýchať, ale pokúšala sa ukľudniť. Len žiadnu paniku... Vyhrážky, to pôjde. Len D je nesmrteľný. Ani on sa nemôže duplikovať. "$ $Ak neodpoviete,"$ $ otočila dlaň na nich a zašepkala "Laser,"$ $ a znova sa obrátila na nich. "Namierim ten laser na vás."$ $ Pozreli sa na seba.

"Tak, toto je Deoque, asi by sme to mali urobiť,“

"Tarny môže to byť D.“

"Múdra úvaha, ale nesprávna,"$ $ podotkla Deoque. "Nemáte veľa času. Poviete alebo nie?“

"Radšej by sme mali,"$ $ pripojila sa Pauline.

"Dobre, aj tak nemáme čo stratiť.“

"$ $Ale pamätaj Tarny, že to, že pôjdeme za Deoque bol tvoj nápad.“

"$ $Idete?"$ $ Deoque si vymenila ruku čo jej vysielala Laser. Udržiavanie tohto kúzla vyžadovalo množstvo fyzickej energie.

"Áno,“

"Tak začni."$ $ Tarny sa snažil jej pozrieť do očí a vyslovil.

"Tarny Lietavý, syn Belly Lietavej, bývalej Kostičkovej, čo bojovala s D a Tarryho Lietavého jedného zo spolupracovníkov vedeckého ústavu Spoločenstva M."$ $ Deoque na chvíľu stíchla a vzápätí nevrlo dodala.

"Premeň sa do svojej pôvodnej podoby Lietavý,"$ $ Tarny zakýval rukami a za chvíľu tam stál pravý Tarny, nie ten v ľudskej podobe. Tulienka urobila to isté a prehovorila.

"Ja som Tulienka Deľa Mokrá-Plavčíková, z federatívnej republiky Wymyslensko, dcéra Tulienky Plavčíkovej a Delyho Mokrého z vodného výskumného ústavu."$ $ Deoque prikývla a pozrela sa na Pauline. Zhasla Laser.

"Som Pauline Goonová, vraj dcéra Jegrigsena Goona, a netuším, v absolútnosti netuším, čo sa tu deje, do tohto sveta som sa dostala len dnes. A podľa Tarnyho a asi aj vás som démon.“

"Polodémon."$ $ Zašomrala Deoque. "Čo chcete?“

"Keby ste nám pomohla. Jegrigsen s najväčšou pravdepodobnosťou vie, že Pauline odišla, proroctvá...“

"Viem čo hovoria proroctvá Lietavý, čo myslíte, kto som?!"$ $ Hlas Deoque bol hrozivý.

"Musíte prejsť skúškou. A pamätajte si, odtiaľto sa nedá vycúvať, ako sa to snažil urobiť D. Takmer som ho zabila, lenže on je takmer nesmrteľný.“

"Prijmem.“

"Ja sa nepýtam teba, ale jej. Goonová?“

"Prijmem, aj tak nemám čo stratiť.“

"Si zaujímavá osoba, len musím zistiť, či mi neklamete, poďte dovnútra!"$ $ Tarny sa prekvapene pozrel na Tulienku Deľu.

"Toto je Deoque? Musím priznať, že predstavoval som si ju inak.“

"Hm... Myslím si, že to najhoršie nás ešte čaká."$ $ Prispela do diskusie Pauline.
Dom Deoque nebol nejako honosný. Jediné čo zdobilo steny boli knihy bez poličiek, zoradené podľa abecedy a témy. Väčšina izieb malo zatvorené dvere. Aj tá kde smerovali. Všade bolo absolútne ticho. Deoque otvorila dvere a vzala do ruky fľašu a naliala zvláštnu priehľadnú tekutinu.

"Vypite to,"$ $ povedala chladne.

"Čo je to?"$ $ Spýtala sa Pauline Tarnyho. Deoque ju začula a nahnevane odsekla.

"Nápoj, nie je to jed. Vypite to."$ $ Pozreli sa na seba. "Rýchlo, lebo zas vytasím Laser!“

"Zas nemáme na výber, Tarny, kde si nás priviedol?!“

"Ja som ti to vysvetľoval Pauline..."$ $ Pokúšal sa nejako obhájiť.

"Tak?“

"Dobre,"$ $ povedal Tarny a s nedôverou chytil pohár na kraji na vypil ho. Chutil hrozne, ale nebol to jed, aspoň mu to jeho prvotný dojem našepkával.

"Teraz vy,"$ $ Pauline a Tulienka Deľa síce s nedôverou, ale povzbudené príkladom Tarnyho a vypili nápoj. Pauline sa zatočila hlava. V okamihu striedala podoby, v okamihu pár sekúnd sa premenila na všetkých ľudí s ktorými sa kedy stretla okrem Jegrigsena. Nakoniec sa jej podoba uistila na pôvodnej. Bola poriadne otrasená.

"Čo to bolo?“

"Nápoj, aby som sa uistila, že si v svojej pôvodnej podobe, a či ešte niektorý z vás nie je polodémon, alebo démon. Okrem teba nikto, takže je to v poriadku. Môžeme pokračovať. Čo odo ma chcete?"$ $ Spýtala sa, no odpoveď vedela. Ale keby jej to nepovedali oni, a ona by to naznačila, nebola si istá, či prísaha túto situáciu nezakazovala.

"Potrebovali by sme pomoc pre Pauline. Vzhľadom na to, že je démonka,“

"polodémonka,"$ $ zašomrala Deoque.

"Dobre tak polodémonka, no ako hovorím, a tiež vzhľadom na to, že jej otec je Jegrigsen, a to, že ušla by mohlo byť pre ňu nebezpečné a v neposlednom rade by to mala byť Goonová proroctiev."$ $ Deoque na malú chvíľu zatvorila oči. Niečo uvidela.

"Naozaj? Dcéra Tlogena a syn Goona? Zaujímavé. Ste si istý. To by platilo aj na Jegrigsena a Arabelu. Alebo?“

"Ste predsa veštica...“

"Ja viem, ale vy asi nepoznáte naše pravidlá. Študujte,"$ $ kývla rukou a z jednej z chodieb prileteli dve knihy. "$ $A nedotýkajte sa ničoho!"$ $ Zas mávla rukou a okolo stien sa zalesklo sklo. Pauline sa otočila k priateľom. "Počkaj, ty ideš so mnou,"$ $ povedala Deoque. Pauline prebehol mráz po chrbte. Z tohto nemala dobrý pocit.

\begin{center}
*
\end{center}

Nevermore letel. Všetka mágia ktorú využíval z ľahkosťou ruky ho obalila a on si z hlavy vyťahoval spomienky toho, koho bol klonom. Boli to útržkové spomienky, ale napriek tomu si niečo pamätal. Meno a strašný hnev. Mal po ňom však aj niečo iné ako len DNA a spomienky, mal jeho schopnosti. A magickú kapacitu, samozrejme.

\begin{center}
*
\end{center}

Miestnosť kde vstúpila Deoque s Pauline nemala lampu, a predsa sa v nej svietilo. V jedenej jej časti stáli fľašky a nádoby s rôznymi chemickými a inými vedeckými esenciami, v ďalšej časti stál notebook Deoque. V treťom rohu zase bol pracovný stôl a pri ňom kreslá.

"Sadnite si Goonová,"$ $ povedala a kreslá sa k nim priblížili.

"Čo chcete?“

"Ja, že niečo chcete vy,"$ $ odvetila jej na otázku Deoque.

"Ja som sa o celom Spoločenstve M dozvedela až dnes."$ $ Bránila sa Pauline.

"Medzi svet mágie nepatrí len Spoločenstvo M.“

"Tak o celom tomto, o mágii, som vedela od vtedy, čo mi to Tarny povedal a niečo mi ukazoval.. bolo to pred týždňom, ale sama som to dokázala až dnes."$ $ Pauline sa zahľadela na Deoqueine perá a mysľou jedno zdvihla do vzduchu a to vletelo Deoque do ruky.

"To je všetko?"$ $ Pauline prikývla.

"Tak to sa musíš ešte veľa učiť. Začneme od začiatku. Tvoj otec je Jegrigsen, a kto je tvoja matka?"$ $ Pýtala sa Deoque a zároveň si všetko zapisovala.

"Netuším."$ $ Deoque sa zamyslela.

"Dobre, si polodémonka, vieš sa premeniť na kohokoľvek, koho si videla, koho si stretla. Tak sa premeň.“

"$ $Ale ja neviem, nespomínam si,“

"$ $Ak si ju len počula, mohla by si sa premeniť, skús si spomenúť,“

"$ $Ale ja...“

"Pozri, rezala som mnohé mozgy a to doslova, viem, že v pamäti by si mala mať aspoň malú informáciu, skús sa premeniť na všetkých ľudí z detstva,“

"Ja sa neviem premeniť,“

"Vieš, každý démon to vie, aj polodémon. Podľa mojej štúdie, si treba predstaviť na sebe hlas alebo podobu toho človeka, respektíve Wymyslenčana a mala by si sa naňho zmeniť. Nejde sa premeniť na démonov či polodémonov,“

"Úžasná útecha, po svete nebehá moja kópia,“

"Tak...“

"Dobre..."$ $ Pauline zatvorila oči. Snažila si spomenúť na všetkých ľudí z detstva. 

"$ $Ak chceš môžem ti pomôcť elektródami.“

"Čo?“

"Neboj sa, mám v tom päťdesiatročnú prax na zemi a predtým som mozog sto rokov skúmala, hlavne tých čo majú schopnosť mágie.“

\begin{center}
*
\end{center}

"Morja?“

"Áno?“

"Nemali by sme sa teraz medzi ľuďmi nazývať našimi menami. Čo keď tam bude niekto kto spolupracuje s D? A teraz sme ľudia bez mágie... kým nebudeme na mieste A zmeň aj svoju podobu. Mali sme to spraviť, už dávno. Začínam strácať cvik. Obe nás môžu odhaliť. Máš falošnú identitu?“

"Tú s ktorou sme cestovali. Mária Morjová.“

\begin{center}
*
\end{center}

Deoque zapla počítač a pripojila Pauline na hlavu poslednú elektródu.

"Neboj sa, o nič nejde, budú sa ti vyhadzovať obrazy a ty sa skús premieňať, keď ti poviem stop, tak prestaň,"$ $ Deoque spustila. Pauline začalo nekontrolovane šklbať očnými viečkami, následne to prestalo. Začali sa jej vybavovať obrazy a hlasy, o ktorých si myslela, že ich v živote nepočula. Po chvíli bola zmenená na niekoho úplne iného, za chvíľu zas. Deoque trochu spomalila a sledovala premeny.

"Stop!"$ $ Pauline ostala premenená na mladú ženu.

"Toto je Morja, zatiaľ oficiálne posledná vo vetve rodu Tlogena. Hovorí sa, že predtým ako zmizla chodila s Jegrigsenom, hoci to mala de facto zakázané. Incest, chápeš? Odoberiem ti krv a potom sa premeň do svojej pôvodnej podoby.“

"Krv? Načo?“

"Nepýtaj sa. Potrebujem DNA Morje, a keďže, keď sa premení démon na rozdiel od obyčajného má jeho DNA, zistím rozdiel a potom môžem spoľahlivo povedať, či je tvoja matka.“

\begin{center}
*
\end{center}

Linka Bratislava – Londýn pristála v cieli svojej cesty. Morja a Niela nemali zo sebou veľa vecí. Len počítače. Morja sa obzrela.

"Je všetko OK?"$ $ Niela prikývla. "Tak, a môžeme ísť.“

\chapter{Tajomstvá Piktopísma}

Loviisa sa prebudila. Posledné na čo si pamätala bolo, že videla tú ženu. Nič viac. Muselo sa tam niečo, diať ďalej, ale mala okno, nespomínala si na nič viac. Striaslo ju. Nikdy predtým sa jej nič také... Ako mohla zabudnúť? A kde vôbec je? Otvorila oči, ale to čo videla jej nepripadalo ako niečo, kde by malo byť. Spomenula si na niečo. Keď si útržkovito čítala svetové správy, si spomenula na titulok o únosoch čo zosnovalo Wymyslensko. No neprikladala im význam. Nikdy by si neuvedomila, že sa to môže týkať aj jej.

"Čo sa stalo? Prečo ste to urobili?!"$ $ Prehovorila po fínsky. Žena, Wymyslenčanka v ktorej spoznala Cecíliu Žblnkotaničkovú, corlovne Wymyslenska. Tá, akoby ju neuniesla, s ňou priateľky začala konverzovať.

"Prosím, povedz to po anglicky. Povedala jej.

"Perkele! Vittu!"$ $ Skríkla Loviisa na Cecíliu. Tá sa iba falošne milo usmievala.

"Po anglicky. To je jediný zemský jazyk čo viem, okrem francúzštiny,"$ $ Loviisa sa zamračila, ale po anglicky jej neodvetila. Stala sa obeťou únosu a nikto ju neoklame.

"Vittu! Perkele! Haista Vittu!“

"Nerozumiem.“

"$ $Ani nebudete!"$ $ Povedala po fínsky a dodala po anglicky: "Nič odo mňa nedostanete!"$ $ Cecília sa zamračila a otočila sa na dvoch jej spolupracovníkov.

"Dávam vám povolenie,"$ $ pracovníci omráčili Loviisu takými istými kruhmi ako vtedy Cecília a pripevnili ju na prístroj. Zapli. "Vymažte pamäť.“

\begin{center}
*
\end{center}

"Morja je tvoja matka, cez DNA je to teraz preukázané. Si teda dcéra dcéry Tlogena a syna Goona. Pôvod je preukázaný. Ďalej, premiestňovanie. Vieš to?“

"Viem, a to jej jeden z problémov. Keď si silnejšie predstavím nejaké miesto, tak sa tam zjavím, a ešte horšie ak sa tam zjavím vo vzduchu. To nemôžem spomínať?“

"Musíš sa to naučiť. Podľa mojej štúdie sa démoni premiestňujú keď sa aktivuje isté centrum v mozgu, no to je prepojené s tvojou pamäťou. Môžem ti to pomôcť spraviť, no zas ti na hlavu pripevním elektródy."$ $ Pauline si vzdychla.

"Dobre."$ $ Deoque šikovne zdvíhala mysľou elektródy. Prax mala dlhú. Keby bola v spoločenstve a nebola by vešticou, zrejme by už bola v nejakom prominentnom výskumnom ústave.

\begin{center}
*
\end{center}

Loviisa sa nič nepamätala. Cecíliine technológie jej pamäť zavreli v zásuvke a ona si nespomínala na žiadny zážitok či fakt spojený s Wymyslenskom a Cecíliou, maximálne to, že Wymyslensko je rovnako štát s obyvateľmi obdarenými darom vyvolať mágiu a ovládať ju a Cecília je jeho Corlovne.

"Kto ste?"$ $ Spýtala sa po fínsky. Cecília jej nerozumela, a tak odvetila po anglicky.

"Nerozumiem ti, hovor prosím po anglicky.“

"$ $Okej, kto ste, kde to som?"$ $ Cecília jej odvetila vymysleným príbehom.

"Fínske spoločenstvo M napadol D, aspoň Helsinki, podľa dohody s Izabetou, sme neplnoletých zachránených evakuovali do Wymyslenska, kým sa všetko zariadi ostanete tam."$ $ Loviisa ju prerušila.

"Kedy sa to stalo? Ja som išla do reštaurácie na večeru a písať si úlohu. Čakala som v metre.“

"Mala si šťastie, útok bol na reštauráciu a na vaše domy. Teba len tlaková vlna omráčila.“

"$ $A čo moji rodičia. A Elisa? A brat Heikki?“

"Naša posádka našla len dve malé deti. Ešte ich musíme identifikovať.“

"$ $A... Prečo zachraňujete vy? A prečo by som nemohla ostať v spoločenstve M? Veď mám rodinu!?"$ $ Cecília bola zaskočená. S týmito otázkami nerátala. Nabudúce vymaže pamäť dôslednejšie. Tieto prípady sa v poslednej dobe objavovali.

"Tvoja rodina záhadne zmizla. A podľa našich zmlúv, kým sa situácia nevyrieši ostaneš pár dní vo Wymyslensku v Žblnkotaničkovskom megadome. A prečo ja zachraňujem? Veď pravá corlovne má slúžiť svojím občanom."$ $ Toto Loviisu upokojilo.

"$ $A koľko tam ostanem?“

"Kým sa nevyrieši situácia... Počkaj zvoní mi telepatión."$ $ Ľudia na planéte Fanasie a v spoločenstve M nemali mobily, ale telepatióny ktorými sa dorozumievalo na veľké diaľky telepaticky. Telepatia sa dala síce využívať aj bez telepatiónu, ale na veľké vzdialenosti bolo jej používanie nanajvýš problematické a bolestivé. Cecília si teda vybrala z vrecka telepatión a pozrela sa toho kto jej volal. To bolo dobré. Jej dcéra. Zdvihla. 

"Čo chceš Micanda?"$ $ Spýtala sa jej. Jej dcéra mala funkciu Biely správca, často nazývanú biela mačka. Dosadila ju tam, resp. odporučila ju voličom samotná Cecília. Wymyslenčania mali proste takú mentalitu, keď si už niekoho zvolili nechceli ho odvolávať. V tom bol aj vlastne zakliaty úspech Cecílie, totiž tvrdila, že ak ju nezvolia, odíde z politiky a Wymyslenčania ju tam proste nechali. A keďže nemohla zastávať dva posty, hneď ako mohla nahradila svoju spolupracovníčku svojou dcérou. A aj keď sa hovorí, že Wymyslensko je demokracia, už dvadsaťšesť rokov tak vládla Cecília.

"Počúvaj mama, máme problém,“

"Áno? Vieš, že sme zachraňovali tie úbohé deti zo spoločenstva M.“

"Viem mama, ale vo Fentenzii vyrástlo občianske združenie, akýsi priatelia spoločenstva.“

"Zakáž ho!“

"Mama, ty dobre vieš, že oficiálne je sloboda prejavu. Nesmiem to zakázať.“

"Prosím ťa! Vieš predsa postup! Proste ich obžaluj, cez našu tajnú políciu, niečo vážne, aby to prišlo o lídra,“

"Jasné prepáč, že ťa otravujem, potom... o hodinu je hlasovanie o zákone.“

"Jasné, a nezabudni."$ $ 

Cecília vypla hovor a Loviisa sa okamžite spýtala.

"Čo sa stalo?“

"Nič, len je hlasovanie o zákone o hodinu, našťastie o päť minút sme tam."$ $ Usmievala sa. Mačičkoes pristál na parkovisku pred megadomom. Loviisa predtým videla megadom len raz, preto ju na zem neobyčajná stavba očarila. Vošli. Žblnkotaničkovský megadom bol, dalo by sa povedať, preslávený. Wymyslenčania tu žili ako jedna rodina, a Cecília si tým medzi nimi získavala popularitu. Vnútro ju prekvapilo rovnako ako stavba.

"$ $Izby máme nejaké voľné, tridsiate poschodie, piata izba vľavo, je tam napísané tvoje meno,“

"$ $Ale ja tu neostávam bývať! Ja tu som predsa len dočasne!“

"$ $Ale keď si na dovolenke tiež máš izbu a si tam len dočasne, a meno je predsa pripevnené mágiou, Loviisa."$ $ Loviisa jej uverila a išla k výťahu a medzitým si prezerala chodbu. Bola lemovaná plagátmi, inzerátmi, pracovnými ponukami aj plagátmi z politickej kampane. Konečne dorazila k výťahu. Vo výťahu už bolo dievča, čo Loviisu prekvapilo, vzhľadom na to, že toto bolo prízemie. Vtom ju napadol suterén.

"Ty si tu nová?"$ $ Spýtala sa.

"Ja som tu prišla len na pár dní... kým nenájdu moju rodinu.“

"Takže Cecília už používa toto, klame ťa.“

"$ $Určite nie, veď spolupracujú...“

"$ $Ak Cecília a spoločenstvo M spolupracuje tak ja som nebola v suteréne.“

"Čo?“

"Predpokladám, že ti vymyla pamäť. To je bežná prax,"$ $ naklonila sa k nej.

"$ $Ale nikomu nehovor, že to viem. Aj to, že som bola v suteréne, je to zakázané.“

"$ $Ale ja ničomu nerozumiem, veď som bola v bezvedomí,“

"Tu to je nebezpečné, poď do mojej izby, tam ti to vysvetlím, nevie to nikto okrem Tulienky Deli,“

"Ty si sa tu narodila?"$ $ Chcela Loviisa odľahčiť situáciu.

"Nie, som z Nemecka, tu som sa premiestnila keď som mala osem."$ $ Vtedy si chcela zatvoriť úst... povedala niečo, čo nechcela.

"Z Nemecka? Ty si polodémon? Slúžiš D?"$ $ Spýtala sa zdesená Loviisa.

"Po prvé, démoni nie sú vždy pri D, je to mýtus, ale nie je pravdivý. A ja nie som polodémon, ale..."$ $ Zamračila sa a ticho, a tónom, akoby to bolo niečo, čo by chcela zničiť, roztrhať, utopiť a spáliť, dodala. "Som démonka, jedna z dvoch.“

"Čo? To je...“

"Ticho.."$ $ Prosila ju. "Nechcem aby...“

"Tak prečo mi to...“

"Lebo som idiot!"$ $ A strelila si facku.

"Nerob...“

"Musím! Ja som idiot! Absolútny idiot! Hanbím sa!“

"Hej... ale... ty vážne si...“

"Poď so mnou... potom...“

"$ $Ale prečo... ako...?“

"$ $Aj tak všetci zomrieme, sakra! Ale ja nie! Ja si počkám na D!“

"Čože?"$ $ Vydesene sa na ňu pozrela.

"Legenda, pravý démon zomrie len keď zomrie ten druhý, D nezomrie bezo mňa, a ľudia démonov nemajú práve v láske. Nespolupracujem z D, ale som proti Wymyslensku. Tento totalitný režim pod zásterkou demokracie... Ale nikomu nehovor, že som proti Wymyslensku, zničili by ma... a už sme tu, poď,"$ $ Loviisa bola zmätená, zdalo sa jej, že niektoré spomienky akoby nemala, v jej pamäti bola hrozná prázdnota, žeby utrpela amnéziu pri útoku D ako vraví Cecília, alebo má pravdu tá vraj démonka, že jej Cecília vymazala pamäť, zdalo sa jej, že o Cecílii predtým počula no nepamätala si. Zmätená nasledovala démonku do izby.

"$ $A ešte som sa ti nepredstavila. Som Sylvia.“

"Ja... som Loviisa, z Fínska, z Helsínk.“

"Ja myslím z Berlína, ale nie som si istá. No, aby bolo jasné,"$ $ Začala vážnejšie. "$ $Ak si zo Spoločenstva M, budú ťa brať ako chúďa, čo hľadalo pomoc v dobrom Wymyslensku,"$ $ pričom dobrom povedala v s nádychom irónie. "Pozri, Wymyslensko je totalita pod rúškom demokracie, ver mi, ak uveríš Cecílii, staneš sa ďalšou obeťou jej moci, som členka disidentskej skupiny Priatelia. Vieš čo je problém Wymyslenska? Oficiálne to je demokracia, ale skutočne totalitný režim ktorému vládne Cecília a pár jej poskokov, sloboda slova je iba oficiálne, vždy sa nájde zámienka na umlčanie, resp. Wymyslenská tajná služba sa ťa zbaví."$ $ Vysypala zo seba. Loviisa nevedela čo má povedať.

"Nerozumiem.“

"Nečítala si o totalitách? Tie najrafinovanejšie sa skrývali za demokraciu. Aspoň zo začiatku.“

"Ty myslíš...?“

"Pozri, ak niekomu chceš ublížiť povieš mu ‚Poď so mnou, chcem ťa zbiť...‘? Nie, povieš mu, ‚Poď so mnou, bude ti dobre...‘. nič iné. Chápeš?“

"Rozumiem,"$ $ povedala Loviisa i keď v skutočnosti nevedela komu veriť a nerozumela ničomu.

"Pozri, ja rozumiem, že mi nemusíš veriť, ak mi neveríš, pozri sa na toto, som napojená, nie na Wymyslenskú wi-fi, takže môžem sledovať aj stránky spoločenstva M, takže to nebude preplnené Cecíliinou nenávisťou, napríklad..."$ $ Zapla internet na jej notebooku a naťukala stránku www.newnews.ym, prezrela titulku a klikla na článok Wymyslensko obžalovalo Fentenzíske občianske združenie Priatelia spoločenstva z objednania vraždy disidentky. Loviisa prešla očami názov, zo strany Sylvie ju tlačili dôkazy zo strany Cecílie slová.

"$ $A čo keď...?"$ $ Sylvia si vzdychla a našla článok Zomrela prominentná Wymyslenská disidentka, podozrivá je tajná polícia Wymyslenska. Loviisu to presvedčilo o tom, že medzi Spoločenstvom M a Wymyslenskom nie je všetko v poriadku, jej logika jej tvrdila, že teda by určite nemali dohody...

"Ty si myslíš, že sa už nebudem môcť vrátiť?“

"Nie, aspoň nie legálnou cestou, vízum ti nedajú.“

"$ $A ako sa odtiaľto dostanem?“

"$ $Ak ti dajú peniaze, čo ti dajú môžeš si zaplatiť nelegálnu cestu, alebo, a to je zdĺhavejšie môžeme zosnovať plán a utiecť, a celému Wymyslensku povedať pravdu.“

"Si zaujatá. Prečo si odtiaľto nezmizla?“

"Mohla by som, lenže svoje schopnosti nevyužívam, a ..."$ $ Uškrnula sa, "To by nebolo také hrdinské,“

"Takže ty sa tu hráš na hrdinu.“

"$ $I tak by sa to dalo povedať.“

"Prečo to robíš? Môžeš odtiaľto zmiznúť okamžite, prečo to neurobíš?“

"To nie je príjemné mojej povahe...“

"$ $A čo je príjemné tvojej povahe?"$ $ Sylvia sa na chvíľu zamračila a následne vybehla do druhej časti miestnosti a Loviisu pridržala na mieste krátkym počkaj. Vytiahla z pod postele knihu a zavolala Loviisu k sebe.

"Sto zaručených spôsobov ako sa stať lupičom? To je čo za knihu?“

"Nevedela som ju nikde zohnať, nakoniec som ju našla v Ulove, u jedného obchodníka. A. A... Takže príjemné mojej povahe sú tieto veci, Dobrodružstvo, hrdinstvo a inteligencia. Teda DHI.“

"$ $A tebe nevadí, že je takmer isté zlyháš?“

"Nezlyhám, verím si,"$ $ Loviisa jej nerozumela.

"Môžem tú knihu?“

"Môžeš, len o nej nikomu nehovor, och ja som strašne dôverčivá, prečo?!"$ $ Loviisa ju ignorovala a opatrne si zobrala knihu a prezrela si obsah. \textit{1. Kapitola: Lupič a zlodej v histórii, 2. Kapitola Zámky a kľúče, ... nové technológie, ... rukojemníci... Ako spraviť revolúciu ... 100. Kapitola Sklá a kúzelné mreže.}

Nalistovala kapitolu ako spraviť revolúciu, pretože tam mala Sylvia záložku.

\textit{Ak chcete spraviť všeobecne prospešnú revolúciu (delenie pozri odsek Typy revolúcií), treba si na svoju stranu získať národ, pretože vás inak zvrhne. Cez súkromné médiá a správy (pozri 15. Kapitolu šifry) je potrebné si získať davy (ak robíte občiansku revolúciu) alebo armádu (štátny prevrat).}

Úryvok jej stačil.

"Ty robíš revolúciu?“

"Áno, aj tak by sa to dalo povedať.“

"Čože??!“

"Ticho prosím ťa, a... nikdy si predsa nespomenuli na toho, čo ušiel boja.“

"Počúvaj, nemôžeš ma proste premiestniť do spoločenstva M a vrátiť sa?“

"Nie, v živote už nie!"$ $ Vyzerala vydesene a zdalo sa, že a každú chvíľu zas udrie.

"Čože? Veď ťa to nič nestojí.“

"$ $Ale ja moje schopnosti nevyužívam, nechcem byť mocnejšia len preto, že som sa tak narodila. A nechcem byť už vôbec démonom! Je to... nechcem byť taká...“

"Hlúposť, keď už máš schopnosti, tak prečo ich nevyužiť.“

"Možno je na tom niečo pravdy, ale to by nebolo také dobrodružstvo, a tie schopnosti nie sú požehnaním, sú prekliatím. Som démon! Chápeš! Vieš čo to znamená? A vôbec, na čo by som to potom robila?"$ $ Loviisa mlčala.

\begin{center}
*
\end{center}

"$ $Operácia dokončená,"$ $ Skonštatovala Deoque. "Teraz si pomysli na vedľajšiu izbu."$ $ Pauline si ju predstavila. Nič.

"Ty si ma pripravila o...“

"Nie, na to nemám právo, len musíš silnejšie myslieť.“

"Premiestni sa a následne príď zas sem.“

"Dobre."$ $ Pauline si predstavila izbu kde čakali Tulienka Deľa a Tarny, všetko, čo tam zacítila, všetky obrazy. Zatvorila oči... keď ich otvorila bola pri nich.

"Dávaj pozor, si pár centimetrov od kníh Deoque."$ $ Povedala Tulienka Deľa.

"Podarilo sa, idem späť."$ $ Zas zatvorila oči a vzápätí bola zas späť.

"Dobre, Teraz tvoje znalosti, čo vieš o spoločenstve M, tvojej rodine, mágii a démonoch?“

"Dokopy nič. Maximálne to, že Wymyslensko a Spoločenstvo M nemajú práve dobré vzťahy.“

"Slabé slovo.“

"$ $A tiež že ich spoločným nepriateľom je D. Teda sú tri strany..."$ $ Deoque ju zastavila.

"Goonová, ja nie som na nikoho strane, lebo veriť sa nedá nikomu, takže ťa nejdem presviedčať o správnosti nejakej strany, nejdem ti hovoriť pozitíva, ale naopak negatíva, budeš sa musieť rozhodnúť, ktoré je to menšie zlo, ako z obľubou hovoria politici. Alebo ostať nestranná ako ja.“

"Začnite prosím,“

"Mágia, mágia je sila, vysvetlená Wymyslenskými mágofyzikmi a tými zo Spoločenstva M, ale i nami – neutrálnymi. V prírode na spontánne uvoľňuje mágia a niektorý jedinci majú v génoch schopnosť ju uvoľňovať aj s ňou nakladať. Tento gén je, ale recesívny, tak sa nerozvinul u každého podľa prírodného výberu. Odjakživa nás prenasledovali, a tak sme sa zhlukovali. Na planéte, čo je s nami prepojená červou dierou boli bytosti, rovnako inteligentné, len trochu inak vyzerajúce, rozprávajúce a tiež... gén, ktorý spôsobuje možnosť ovládať mágiu je tam ten silnejší a majú ho všetci, počas nášho prenasledovania nám pomohli a dostali sme sa na planétu Fanasie, kde sme sa usídlili. Ale demokracia sa odtiaľ, aspoň z Wymyslenska, pomaly vytrácala. Izabeta Tlogenová, náhodou aj tvoja prapraprastará mama. Ale na starú nevyzerá, hoci som ju naposledy stretla pred sto rokmi, ale to je vedľajšie. Tá sa stala disidentkou a odišla do azylu. Rollius Tlogen, jej neskorší manžel ostal vo Wymyslensku a pripravoval rozdelenie. Čo urobila Izabeta v Tramtárii to nikto netuší, každopádne sa odtiaľ vrátila staršia."$ $ Deoque sa nachvíľu odmlčala. Ona dobre vedela čo sa v Tramtárii stalo, ale nemohla to povedať. Veštectvo. "Každopádne sa odtiaľ vrátila s dvoma deťmi. Botesom, ktorý odišiel na Zem a Čeriou ktorá sa rozhodla ostať. Kto je otcom, to si nechala Izabeta pre seba. Rollius jej jasnú neveru odpustil. Izabeta doniesla tiež dýku, čo nazvala svojou, na tú prisahali, získali nesmrteľnosť, respektíve odolnosť voči starnutiu a chorobám. A ak boli zabitý, stali sa duchmi. Na Zemi vzniklo Spoločenstvo M, ultrakonzervatívna komunita čo si získavala čím viac tým viac priaznivcov pre svoje demokratické zriadenie. Popravde, nebola konzervatívna napríklad postojom ku ženám alebo homosexuálom, v tom je pokrokovejšia ako Anglicko, ale to mala na začiatku svojho založenia, nie že by bola neochotná k novým technológiám, ale tým, že sa od svojho vzniku nijako spoločensky nereformovala. Stále tu sú nútené sobáše, spoločenské odsúdenie pre porušenie tradície... Aj strach s démonov. Naproti tomu je tu diktatúra, ktorá odmieta všetko so Spoločenstvom M, ale príma pokrok. A ich spoločným nepriateľom je D, démon, ktorý raz prišiel a bojí sa svojho konca, a preto chce zničiť všetko. Vyber si stranu alebo ostaň nestranná, v každom prípade boj sa blíži, teraz môžem svoje druhé a posledné proroctvo. Takže..."$ $ Deoque zatvorila oči, z jej rúk vyšľahol laser a Solan narazili do seba a ovinuli Deoque striebornou kuklou.

\begin{center}
"Boj, posledný boj nás čaká,

Keď prídu tí čo vládnu,

A keď tí nedokázali

Spoznať tajomstvá,

Museli vytvoriť,

To čo museli,

Posledný boj nás čaká,

Keď heslo sa naplní,

Za chvíľu, boj, možno posledný...“

\end{center}

\begin{center}
*
\end{center}

"Kedy príde ten taxík?“

"$ $Už tu je.“

"Si si istá?“

"Áno, určite."$ $ Vošli do taxíka.

"Kam to bude?"$ $ Spýtal sa ich taxikár.

"Stonehenge.“

"To je takmer dve hodiny.“

"Ja viem.“

"Dobre...“

"Mohli by ste už ísť?"$ $ Spýtala sa podráždene.

"Ni..."$ $ Začala Morja, ale zháčila sa.

"Mária!"$ $ Šepla Niela.

"Prepáč.“

"Čo sa deje...?"$ $ Poobzeral sa a dokončil otázku. "Morja?"$ $ Morja nahlas, so 
strachom, vydýchla.

"Jegrigsen.“

"$ $A čo si si myslela?"$ $ Povedal takým tónom ako vtedy, keď sa videli naposledy. Stočil volant a odbočil na prázdnu, kedysi benzínovú pumpu.

"Dé..."$ $ Vydýchla.

"Niela... Ako vidím aj ty si tu. Služba Izabete ťa omrzela?"$ $ Morja ho nenávidela. Nenávidela viac ako čokoľvek na svete, a nedokázala pochopiť, že ho kedysi milovala. Keby mala príležitosť, zabila by ho. Jeho tón jej ho pripomínal, vtedy v tú noc kedy opustila spoločenstvo M. Nenávidela ho s odstupom času, za všetko to, čo musela prežiť, za to, že nevie kde je jej dcéra, a to, že vôbec existuje. Za všetko dávala vinu jemu. A on slúžil D.

Prepol na autopilota a pôvodne taxík sa zmenil na mačičkoes. Niela zaťala zuby a pomedzi ne precedila.

"Sklapni Jegrigsen... Prečo nás chceš zničiť? Sme bezcenné.“

"Nie, aspoň podľa môjho pána.“

"Dé ťa využíva, Jeg!“

"$ $Už si ma tak dávno neoslovila Morja, čo? Chceš obnoviť náš vzťah?"$ $ Iróniu z jeho hlasu sa len ťažko nedalo nevyčítať. Morja by ho najradšej zaškrtila.

"Nenávidím ťa... Jegrigsen!“

"Nenávisť..."$ $ Pohýbal prstami a Morji sa pred očami vyhotovili obrazy. "Tá je vzácna, nenáviď Morja, budeš pre môjho pána užitočná...“

"Nikdy!“

"Vieš koľkí to hovorili? Pamätaj, že máš šancu, zatiaľ si hosť... a to si ceň... nemáme veľa hostí...“

"$ $Ak chceš poznať definíciu slova hosť,"$ $ Zamumlala Niela. "Hosť je, podľa slovníkovej definície, dobrovoľný návštevník niekoho iného ktorého dobrovoľnosť je z oboch strán, toto je skôr únos..."$ $ Niela za svoju kariéru súkromnej detektívky a kriminalistky Izabety Tlogenovej zažila dva únosy. Z toho raz uniesli ju, ale nikdy to nebol D. Spomínala si na to, čo jej hovorili v Agentúre, nikdy sa nesmie únoscom nechať vyprovokovať a hľadať jeho slabinu. Jegrigsen jej myšlienky prerušil.

"$ $Ak si myslíš, že toto je únos, tak si únos nezažila."$ $ Nielu niečo napadlo, v arogantnom úsmeve Jega asi našla jeho slabinu. Vedela, že ak sa plán nepodarí všetko sa pokazí, no musela to skúsiť, pretože vedela, že od D by sa tak ľahko nedostali.

"Jegrigsen...“

"Začínaš rozhovor Niela? Dobré znamenie, začínate spolupracovať..."$ $ Nielou to myklo no snažila sa nič nedať najavo, Jeg sa ju snažil vyprovokovať a zlomiť, ale to sa mu nesmelo podariť.

"Kde je Dé? Myslím jeho sídlo.“

"$ $Ak to chceš vedieť, pridaj sa k nám.“

"Pozri Jeg, je to vec na ktorú treba závažné rozhodnutie, len sa pýtam... a... Prečo si nás proste neuniesol, si polodémon, predsa..."$ $ Morja sa snažila pochopiť čo to Niela robí. Sama študovala na Agentúre, má z nej predsa magisterský titul, ale z inej oblasti ako Niela, a mala menej praxe. Zas sa započúvala do rozhovoru Jegrigsena a Niely.

"Vieš čo je potrebné na prenesenie Niela...? Takže sú tri dôvody prečo som to neurobil. "Nechcel som vzbudiť pozornosť, chcel som sa pozhovárať a kvôli Morji... že?"$ $ Jeho arogantný tón sa niesol Morji celým mozgom. Ovládaj sa... ovládaj sa... Hovorila si v duchu.

"Fuck you!"$ $ Zanadávala na adresu Jegrigsena. Ten otočil pohovku na ktorej práve sedel a priblížil sa k Morji, tá ho z nechuťou odtláčala.

"Morja... ja som si myslel, že si slušná...“

"Pri tebe sa to nedá...!"$ $ Opľula ho.

"Ty si sa znížila na..."$ $ Morja sa odpútala vytiahla Solan. 

"$ $Ako chceš Morja!"$ $ Povedal Jegrigsen a vzápätí skríkol.

"Laser!"$ $ Z oboch jeho rúk mu vyšľahol červený prúd fotónov.

"Morja! Nechaj ho, aby laserom prepílil mačičkoes, ujdeme."$ $ Povedala jej Niela telepaticky a sama vytiahla laser. Mačičkoes bol rozdelený na dve časti. Morja využila Jegrigsenove prekvapenie a omráčila ho. Obe s Nielou vyskočili s polomačičkoesu a rozhliadli sa.

"Dé!"$ $ Vykríkla Morja a vzápätí jediné čo ju napadlo. "Teleport!“

"Ty sa chceš teleportovať Morja? Kde?“

"Stonehenge!“

"Ty vieš kúzlo?“

"Viem, ale neskúšala som to. Namier na seba laser a pritom rob čo ja.“

"Zbláznila si sa Morja!? To je samovražda!!!“

"Robila som z toho magisterský titul z mágie, nezabiješ sa."$ $ Dé sa blížil. A ona nechcela mu padnúť do zajatia. "Rob čo ja!"$ $ Sledovala Morju ako vyslovila Laser a prúd energie namierila na seba zatvorila oči a rozpadla sa, pričom Niele posielala pokyny. Keď jej o sekundu na to prišla telepatická správa od Morje a D sa blížil, zopakovala to.

V Stonehenge, na kameni sa zrazu, pred očami prekvapených turistov zjavili dve postavy, Morja a Niela. Obe boli veľmi unavené z množstva energie ktorú museli použiť.

"Funguje to,"$ $ Zašepkala Niela, keď si uvedomila, že žije.

\begin{center}
*
\end{center}

\textit{Ja som to nebola! A neviem im to vysvetliť, odmietam žiť takýto život. Matka ma nenávidí a ja ju tiež, odídem, nemám tu čo hľadať. Nemôžem za to, že kuchyňa vzbĺkla! Alebo áno?! Priznávam, že som na to myslela, ale ako som mala čakať, že sa to stane! Určite ma nenávidí! Nikdy ma nepočúva! Chcem odísť! Preč! Už nikdy viac... Moje zásoby na núdzové situácie majú zmysel. Keďže bývame na prízemí, nemal by byť problém s výškou. Je asi 2 metre nad zemou. Dnes v noci, dnes odídem. A je mi už jedno čo sa mi stane.}

\begin{center}
*
\end{center}

\begin{center}
\textit{\textbf{2. Kapitola Slávne veštice}}
\end{center}

\textit{...}

\textbf{\textit{Deoque. Narodená: Yucatán 1320 5. Decembra. }}

\textit{Vešticou sa stala po tom čo od bývalej veštice Oka vo Wymyslensku získala od nej prsteň Veštice z tým, že sa má stať novou Vešticou, takto sa prerušila päťstoročná tradícia Veštíc z Fentenzie. Svoj prvý sľub zložila v chráme Mairinej dýky slova. Vešticou sa stala v Tramtáríjskom chráme.
}
\textit{V roku 1752, opustila spolu s Izabetou Wymyslensko, ale nepridala sa k spoločenstvu. Žila v Uhorsku a tam ostala až do času keď sa Napoleon vydal do Ruska, vtedy predpovedala jeho koniec. Po uvoľnení protičarodejníckych zákonov odišla do Francúzska. Počas prvej svetovej vojny emigrovala do Anglicka a usadila sa v Londýne. Tam je doteraz. Je jednou z najväčších odborníkov na Démonov, mágiu a genetiku...}

"To písala Deoque?“

"Nie Tarny. I keď sa to tak zdá...“

"Ticho prosím ťa, môže nás počuť. Vieš, že je to takmer zázrak, že sme tu. Tak sa ju nepokúšaj uraziť.“

"Že to práve ty vravíš...“

\textit{*}

"Vidíš? Fungovalo to.“

"Stratila som priveľa energie, to je hnusné...!“

"Keby sme to nespravili zajal by nás D.“

"To mi je jasné... Otázkou je teraz ako sa odtiaľ dostaneme a ako dlho to potrvá kým sa on dostane sem.“

"Nemôžeme použiť kúzlo pred ľuďmi a na neviditeľnosť máme primálo energie. To je výhoda démonov – neunavia sa až tak. Musíme nejako zliezť.“

"$ $A ľudia? Budú sa nás pýtať, ako sme sa tu dostali.“

"To je pravda Morja,“

"Zmyslové kúzlo.“

"Morja, nemáme dosť síl!“

"Dopekla!“

"Máš iný nápad?“

"K peklu!"$ $ Zašepkala Morja. Obe boli síce agentky, resp. ona bývalá, ale napriek tomu...

"$ $Ako rýchlo sa dočerpáva mágia?“

"Dlho, alebo krátko, ak máš magický zdroj. Nikdy si nemala mágiu vyčerpanú? To nevrav.“

"Mala... ale vždy v boji a vždy som mala zbraň. Teraz sú to ľudia. Nemáš žiadnu zbraň?“

"Niečo by sa našlo...“

"Necítiš mágiu...?“

"Tak áno, je tu staré miesto, ale že by tu bol zdroj...?“

"Je..."$ $ Odvetila. "Cítim ho.“

"Máš už aspoň trochu mágie?“

"Áno. Už áno. Mám ju. Zmyslová mágia?"$ $ Povedala už telepaticky Morja Niele. "Dobrý nápad... lenže budú si nás pamätať. Pamäť jednoducho neviem vymazať.“

"Ja rovnako. Musíme sa odtiaľto aspoň nachvíľu vytratiť.“

"$ $A piktopísmo?“

"Mysli na D. Určite tu je za chvíľu, takže toto je plán B."$ $ Morja použila mágiu. Ľudia, návštevníci, nikoho na Stonehenge nevideli. Niektorí obraz dvoch žien a ich zjavenie považovali za halucináciu, iní nie. Morja spôsobila i to, že ich nepočuli. Zoskočili zo skaly, pevne na nohy. Niela sa pozrela na Morju.

"$ $Odchádzame, toto musíme vyriešiť nabudúce, ale..."$ $ Pozastavila sa a pozrela sa na oblohu. "Čo to je!?“

\begin{center}
*
\end{center}

Nevermore cítil hnev, v skutočnosti to bolo jediné, čo dokázal cítiť. Hnev... hnev... videl kamennú stavbu, kruh. Ľudia, Jean. Napadlo mu. Zamieril dole. Pristál v strede kruhu. Ľudia sa naňho prekvapene zadívali. Nevermore cítil zvláštne vyžarovanie mágie zo zeme. Kým zistí čo to je, spýta sa na Jeana. Návštevníci Stonehenge naňho nechápavo pozerali. Dnes toho zažili až dosť. Nevermore sa ich chladným hlasom spýtal jedinú otázku.

"Kde je Jean?"$ $ Nechápavo sa na seba pozreli, nerozumeli kto to je, ani čo chce. Morja s Nielou boli pri ňom neuveriteľne blízko, i keď ich nemohli zazrieť, ale Nevermore ich videl, cítili to. Obe tušili o ktorého Jeana sa jedná.

"Kde je Jean? Kde je Jean?!"$ $ Zúril. Medzi návštevníkmi to zašumelo, nenapadlo ich pýtať sa, koho Nevermore myslí. Vydesene naňho hľadeli.

"Kto to je...?"$ $ Zaznela tam tichá otázka. Nevermore to počul.

"Nevermore!"$ $ Skríkol a pokrútil rukami. Kamene Stonehenge vyleteli do vzduchu a spopod nich sa začala uvoľňovať mágia. Niela a Morja už nevládali držať kúzlo.

"Zadrž tú mágiu!"$ $ Skríkla Morja. "Ver mi! Nič sa ti nestane, Niela! Zadrž ju, inak zomrieme! Toto je ten, čo zničil to mesto!"$ $ Niela v tom zmätku prestávala rozmýšľať, ale Morjine pokyny boli jasné. Ona premýšľala rovnako. Prúdom mágie sa obalili práve vo chvíli, keď sa kamene rozpadli a uvoľňovalo sa ešte viac mágie. Neuveriteľná horúčava spôsobila, že väčšina vecí sublimovala. Len Morja, Niela a Nevermore boli obalení do striebornej kukly ktorá udržovala izbovú teplotu.

"Pozri Morja! Šifra!"$ $ V strede bývalého Stonehenge sa otvorila zem a stúpala z nej neuveriteľná horúčava, ešte väčšia ako tá, čo urobil Nevermore. Z nej sa začala vysúvať kamenná tabuľa s piktopísmom. Letela. Morja sa na ňu pozrela a vtedy Nevermore k nej vyslal obrovské množstvo magickej energie. Tabuľa to prežila a letela stále viac do výšin. Morja sa z energie, čo ju obklopovala, už dostatočne magicky naplnila, a tak mohla vysielať čoraz viac a viac mágie. Zamerala sa na tabuľu, aby ju privolala. Niela, ale pozorovala čosi iné.

"Morja! Jegrigsen a D!"$ $ Vykríkla a prúdiacu energiu použila na vyvolanie Lasera. Ruiny, čo Nevermore nechal padať na zem to rozťalo a Jegrigsen sa tomu len, len uhol. Tabuľa sa priťahovala k Morji a čoraz viac žiarila. Z ruky D vyšľahol Laser, ale Nevermore ho zastavil svojím silovým poľom. D pochopil, že ho dokáže poraziť.

"Jegrigsen! Okamžite sa premiestni po Jeana!"$ $ Nevermore to počul. Oči mu sčerveneli a kusy balvanov, ktoré ostali zo Stonehenge vyleteli do vzduchu a opäť sa rozdelili na tisíce skaliek čo padali na zem ako dážď. Morja akoby nevnímala realitu. Tabuľa bola pri nej. Chytila a posledné čo cítila, že mizne a kričí na Nielu. Tá už nevládala držať Laser. Všimla si, že Morja zmizla. Boli v prevahe. Jegrigsen sa vrátil a s ním aj Jean. Nevermore sa naňho zahľadel a po chvíli domnelého ticha skríkol znovu.

"Nevermore!"$ $ Z jeho rúk akoby vyšľahol plameň, plazma, horúci a hustý prúd mágie. Namieril ho na D, Jegrigsena a Jeana. Jean z ľahkosťou ruky zastavil plamene ľadovou stenou z tekutého dusíka a z druhej ruky mu vyžaroval Laser. Nevermore na to odpovedal silnou spŕškou Solanu a skalky, čo ostali zo Stonehenge namieril na nich. Niela stála pri nich. Cítila, že môže ujsť, ale jej fascináciu týmto, priam nadmagickým bojom nevedelo zastaviť ani jej analytické a racionálne zmýšľanie. Morja zmizla, D má nepriateľa čo však dokáže ničiť s ľahkosťou ruky a je tu Goonová proroctva. Pomaly odvrátila zrak a rozhodla sa, že to musí urobiť, síce dalo by sa povedať, zradí Morju, ale toto bolo nevyhnutné. Spoločenstvo má nového nepriateľa.

Medzitým Jean a Nevermore na seba vystreľovali striedavo lúče Laseru a rádioaktivitu vzniknutú z rozpadu Stonehenge na atómy a jednotlivé alfa a beta častice.

"Nikdy ma neporazíš!"$ $ Zakričal Jean a Nevermore naňho vyslal rádioaktívne beta častice. Jean sa obrnil plátom z častíc a vyslal na Nevermorea Laser. Nevermore sa uhol a laserom preťal zem a svojou silou urobil atóm uránu a rozbil ho. Obrnil sa pred obrovskou rádioaktivitou ktorá sa uvoľnila. D rýchlo, bez váhania v zlomku sekundy sa spolu s D a Jegrigsenom premiestnili späť. Nad bývalým Stonehenge sa objavil rádioaktívny hríbik.

\chapter{Prúd tyrkysovej}

Pauline bola trochu šokovaná, veď práve si vypočula proroctvo, posledné proroctvo Deoque. Tá sa už upokojila a niečo písala na počítači a medzitým sa Pauline vypytovala.

"$ $Čo vieš robiť mágiou?"

"$ $Dokopy nič, maximálne zdvihnúť niečo... to som ukazovala."

"$ $Dobre, tvoji priatelia ťa vyššiu mágiu môžu naučiť, ja ti ukážem základy. Ak sa chceš ochrániť pred tlakovou vlnou, rádioaktívnym žiarením, alebo proste energiou, musíš okolo seba urobiť silové pole, respektíve pole, ktoré ruší energiu. Ľudský vedci už vyvinuli niečo čo ruší magnetické pole lenže na to potrebujú tekutý dusík. To pole čo chceš vytvoriť má približne žltú farbu, urob ho a vyskúšame ho."

"$ $Ale ako?"

"$ $Jednoducho si v mozgu vytvor obraz toho a silou vôle to vytvor, skúšame!"

"$ $Ale..."

"$ $Vytvor to!"$ $ Povedala Deoque a vytvorila tyrkysový prúd a vyslala ho na Pauline, ale tá bola tak zmätená, že nič nevytvorila a lúč ju zasiahol. Zaťala zuby od bolesti.

"$ $Au, to čo bolo?"

"$ $Tyrkysová magická energia, spôsobuje popáleniny."

"$ $Skvelé,"$ $ Ironicky poznamenala.

"$ $Mohla by si to púšťať až vtedy, keď si vytvorím tú obranu?"

"$ $Nie, nepriateľ ťa v boji šetriť nebude."$ $ Pauline zaťala zuby a sústredila sa na silové pole. Deoque zas vyčarila lúč a vyslala ho na ňu. Asi spolovice sa lúč odrazil a spolovice ju zasiahol do ruky.

"$ $Ja budem spálená Deoque!"

"$ $Vyliečim ťa, teraz vytvor obranu, táto bola slabá a oberala ti veľa síl!"

"$ $Dobre!"$ $ Bola naštvaná. Deoque mávla rukou Pauline popáleniny zmizli. Z celých síl sa sústredila na pole z jej rúk sa rinuli žiarivé, ale studené iskry a ukladali sa okolo nej a cez toto už prúd tyrkysovej neprenikol.

"$ $Dobre, teraz si to dala len ti to ubralo veľa síl, naučím ťa už len Laser a Solan. Vystri ruku a povedz Solan, to sú jediné dva kúzla pri ktorej vlnenie, ktoré spôsobujú tvoje slová majú na výsledok vplyv. A prosím nemier na mňa ani na elektroniku, nechceš mi predsa niečo poškodiť."$ $ Povedala a namierila rukou na kreslo.

"$ $Z ďaleka."

"$ $Prepáč... Solan!"$ $ Roztiahla ruku a vyšiel z nej jeden malý kruh, následne vyčerpane klesla na zem.

\begin{center}
*
\end{center}

Morja nič necítila odvtedy ako sa chytila dosky s prekladom. Až teraz mala pocit, že sa vznáša. Otvorila oči. Najskôr sa jej zdalo, že nevidí nič, až potom si uvedomila, že je tam obrovská žiara vyžarujúca z dosky. Pomaly zdvihla hlavu a zas si uvedomila, že sa vznáša v stave beztiaže. Žiara ju oslepovala, a tak radšej zavrela oči. Hmatom sa snažila nájsť nejaký záchytný bod. Chytila sa tabule a tá ešte viac zažiarila, až tú žiaru videla Morja a očné viečka. Tabuľa sa otvorila a začal z nej vychádzať vír a vťahovať Morju dovnútra. Snažila sa okamžite niečoho pridržať, ale nedokázala to. Ocitla sa v doske. Keď otvorila oči už tá žiara nebola. Bola v zlatej miestnosti, v ktorej už platila gravitácia. Vstala a podišla k stene. Keď sa bližšie zapozerala pochopila, že tam sú znaky ako vo veľkej knihe. Dotkla sa jedného z nich a zdesene odskočila. Tabuľka s piktopísmom sa začala otáčať a na jej mieste sa zjavila obrazovka. Morja videla osobu, asi pôvodom z Fanasie. Chvíľu sa nehýbala čo vytvorilo ilúziu, že je to len obraz. Preto bolo šokom pre Morju, že osoba zrazu začala rozprávať wymyslenčinou.

"$ $Vitaj človek, vitaj vo víre, som corlovne Medizo, otočila si písmeno Sán, vo wymyslenčine podľa Stuvely prekladané ako Sl."$ $ Morja sa prekvapene zamyslela a spýtala sa.

"$ $Vy ma počujete?"

"$ $Samozrejme. Môžeš sa pýtať."

"$ $Kde som to? Ty si z Wymyslenska, a pokiaľ viem, corlovne je Cecília, posunula som sa v čase dopredu? Veď žiadna corlovne s menom Medizo nebola."

"$ $Si vo víre, prúde myšlienok a mimopriestorovej a časovej anomálii."$ $ Na chvíľu sa zamyslela. "$ $A nie som z Wymyslenska, respektíve narodila som sa pred tím, ako vôbec vzniklo. A corlovne bolo vo Wymyslensku zle pochopené. Corlovne je titul, ktorý po prenesení do piktopísma čítame ako Garrenzisie, teda múdry, múdra, starší, alebo staršia."$ $ Morja sa zamyslela. Ak vie táto osoba písmená piktopísma tak by mohla získať preklad, otázne je ako sa odtiaľ dostať.

"$ $Corlovne Medizo, alebo ako ťa mám volať, dá sa dostať z tohto víru?"

"$ $Neviem, popravde, myslím, že to vie Oko, ale ja nie, ja som tu len corlovne,"

"$ $A kto je Oko? A čo je to vôbec vír?"

"$ $Vír je..."$ $ zmĺkla a rozmýšľala. Pravú jeho podstatu nevedela ani corlovne Medizo.

"$ $Toto, časopriestorová anomália. Ak sa ti zdá plynutie času v skutočnosti tu čas nie je. Myslím, že sa odtiaľto dá dostať do ktoréhokoľvek času a priestoru. Stačia štyri súradnice, ale otázne je ako."

"$ $Super,"$ $ ironicky poznamenala Morja. "$ $A Oko? Táto odpoveď sa tu nevyskytla."

"$ $Oko?"$ $ Vzdychla si. "$ $Oko je oko. Ja ju neviem inak charakterizovať, je z Fentenzie, kde ju volali Ja'Sno\v{}vid, tá čo vidí, raz sa tu zjavila a odvtedy tu je. Myslím, že je to corlovne zo všetkých corlovne, ale stretla som sa s ňou len párkrát."

"$ $Okej, ak si si nevšimla, tak trochu odtiaľto, kde som nie je východ."

"$ $Je, len musíš povedať heslo."

"$ $Heslo? Viete ho?"

"$ $Nie, ale vie ho Oko, môžeš sa jej spýtať, ale na to potrebuješ pootáčať tabuľkami s piktopísmom do jeho mena,"

"$ $Ja'Sno\v{}vid či oko?"

"$ $Ja'Sno\v{}vid."

"$ $Ak som správne pochopila tak ak otočím tabuľku, ty, alebo niekto iný mi povie preklad."

"$ $Správne."

"$ $Je tu nejaké pero a papier?"

"$ $Nie, musíš si to zapamätať."

"$ $Dokelu, nemám takmer nijakú fotografickú či sluchovú pamäť..."$ $ Zamumlala pre seba, ale corlovne Medizo ju počula.

"$ $Odporúčam použiť pár kúziel na prepájanie neutrónov."

"$ $Vďaka."$ $ Morja podišla k písmenu vľavo a dotkla sa ho. Corlovne Medizo sa otočila a zas zjavila.

"$ $Písmeno Djá, vo wymyslenčine S."$ $ Morja sa pokúšala sústrediť a zapamätať si. Otočila ďalšiu tabuľku.

"$ $Vlieg, inak Ze,"$ $ ... "$ $Gu, inak Já."$ $ ... 

Morja mala takmer celé slovo Ja'Sno\v{}vid, ‚..... Furied eldaie'. Ja nevedela. Napadlo ju ako sa to píše. Ja dĺžeň. Teda to môže byť Já, a teda Djá. Teda... Morja okamžite si to dala dohromady a vyslovila.

"$ $Djáfuried eldaie."$ $ Vyslovila a rozhliadla sa. Najskôr sa nič nedialo, Zlé heslo. Pomyslela si Morja, ale vtom sa začali otáčať znaky z písmenami čo Morja vyslovila. Na obrazovke bola Oko, mala vlasy spletené do vrkoča práve sedela na kresle.

"$ $Oko?"$ $ Oslovila ju. 

"$ $Wymyslensky? Anglicky? Fínsky? Fentenzísky? Tramtáríjsky? Trestovsky? Nemecky? Povedzte jazyk prosím, posledné časy vo víre sa ich učím."$ $ Morja, zaskočená Okou, odpovedala.

"$ $Wymyslensky, prosím."

"$ $V poriadku, som Oko, resp. po Fentenzísky Ja'Sno\v{}vid. Kto ste vy, a čo by ste chceli?"

"$ $Som Morja, Tlogenová, potrebovala by som sa odtiaľto dostať."

"$ $Skvelé, ste v miestnosti piktopísma predpokladám. Ak mám dobré informácie, všetky písmená, čo si počula a videla máš v pamäti."

"$ $Odkiaľ...?"

"$ $To viem? Jednoduché, vo víre mám možnosť nahliadnuť do akéhokoľvek priestoru v akomkoľvek čase, teda som vedela čo sa stane."

"$ $Myslíte ako veštica?"

"$ $Nie, je to niečo iné, veštica to žije, ja to len vidím, veštica hovoriť nemôže ja môžem, je viac verzií budúcnosti, ja vidím verzie, ona to čo príde. Ak sa nemýlim vo vašej dobe je vešticou človek Deoque, ktorá veštecký prsteň prebrala od Oka Fentenzie."

"$ $Vy...?"

"$ $Nie, ja som bola dávno predtým ako a začal prsteň odovzdávať. Síce je pravda, že funkcia veštice je stará ako samotný démon."

"$ $Rozumiem, ale teraz by som vás chcela sa spýtať, viete aké je heslo od tejto miestnosti?"

"$ $To je jednoduché, bola som jedna z tých čo vytvárali šifru. Piktopísmo."

"$ $Čože?"

"$ $Piktopísmo."

"$ $To je heslo?"

"$ $Samozrejme."$ $ Morja si spomenula na písmená a vyslovila.

"$ $Parvdchenzrusme."$ $ Tabuľky sa po stlačení ich začali otáčať a pomaly sa vytrácali. Morja sa ocitla v chodbe na ktorej bolo plno dverí bez názvov.

\begin{center}
*
\end{center}

Izabete zazvonil telepatión.

"$ $Dobrý deň, tu je Izabeta Tlogenová, vedúca ministerstva etiky spoločenstva M pre svet."

"$ $Izabeta, prosím prestaň s tými formálnosťami."

"$ $Niela, ty u nás už nepracuješ."

"$ $Viem Izabeta, ale toto si vyžaduje spoluprácu, máš čas?"

"$ $O čo ide? Ak si si nevšimla, nie som na voľnej nohe ako ty."

"$ $Telepatión môže byť odpočúvaný. Povedz kedy, je to naliehavé."

"$ $Ak ide o bezpečnosť volaj Roliusovi."

"$ $Tu ide o teba, aj o teba a celé spoločenstvo. Vieš, že si skutočne hlavou spoločenstva, takže sa to nesnaž skrývať. Ja som u teba pracovala."

"$ $Dobre Niela ty nedáš pokoj. Kde a kedy?"

"$ $Ja som vo vlaku do Londýna, nemám mačičkoes."

"$ $Som práve v Helsinkách, zdá sa, že došlo k únosu zo strany Wymyslenska, ale môžem prísť, o hodinku."

"$ $Obávam sa, že toľko času nemáme Izabeta."

"$ $Otvorene."

"$ $Vieš, čo zmizlo mesto Lowwer? Viem kto. A nie je to Dé človek."

"$ $Čože?"

"$ $Takto to nepochopíš, príď čo najrýchlejšie do Londýna. Pôjdeme niekde, kde to nevzbudí pozornosť, tu ide o národnú bezpečnosť a proroctvá."

"$ $Bezpečnosť? Wymyslensko?"

"$ $Nie, myslím, že Nevermore nie je ani na jednej strane."

"$ $Kto?"

"$ $Uvidíš. Näkemiin."$ $ Akonáhle Niela zložila, vytiahla zo svojej tašky, čo sa jej zázrakom nezničila, tablet. Otvorila register a zadala Nevermore, i keď pochybovala, že o ňom bude čo i len zmienka keďže každého koho stretol nechal sublimovať. Pár sekúnd čakala na výsledky.

Register

Hľadané "$ $Nevermore"

Výsledky:

www.englishterms.ym

Správa o používaní rečí vo spoločenstve M

Magazín Zo života

SPRÁVA 1001

Prvé tri preskočím, štvrtá vyzerá ako správa od D. Povedala si Niela v duchu a vzápätí. Predpokladám, že to je zašifrované, našťastie som vyvinula ten program, keby D tušil, že môj filter vie nájsť jeho dokumenty. Klikla na SPRÁVA 1001. Správa 5, zaujímavé. Otvoril sa súbor, na ktorom, ale vyskočilo okno. Zadajte Heslo. Niela otvorila svoj program a po piatich minútach mala odheslovanú správu. Ako vidím, sa D poučil. Začal používať heslá v piktopísme. Niela sa ešte raz poobzerala, či ju niekto nesleduje a začala čítať. Všimla si, že aj správa je zašifrovaná. Skúsila všetky jazyky. Nič. Bola tam krátka podobnosť piktopísmu a japončine, ale ani ich obmeny to neboli. Len japonský prepis vyšiel ako čudná zlátanina. Vtedy to Nielu napadlo. Je možné, že D zašifroval text šifrou a tú šifrou a tú šifrou. Skúsim aj piktopísmo aj Japončinu. Výsledné texty sa podobali na všetko len nie na jazyk. Keď sa na to Niela, ale pozornejšie pozrela zistila, že v tom vidí značky chemické a fyzikálne. O H s v... keď skúsila použiť preklad veda vyšli jej len názvy prvkov a to moc šifru nepripomínalo. Dala to síce rozšifrovať, ale jej program nenašiel jedinú šifru čo by bola aplikovateľná na text. Rozmýšľala. Vedela, že D a jeho služobníci, aspoň niektorí sú vcelku inteligentní a boli by schopní vynájsť šifru čo by jej program neodhalil. Budem sa im musieť hacknúť do systému. Rozmýšľala. Práve dorazila do Londýna, vypla tablet a vložila ho do tašky. Vystupovala. Rozhodla sa zavolať Izabete.

"$ $Si v Londýne?"

"$ $Som, kde si?"

"$ $Za chvíľu vystupujem."

"$ $Poď do hotela Locom. Je pri Temži a ..."

"$ $Viem kde je, hneď som tam."

Na Rennie street prišla Niela o desať minút po príchode vlaku Izabetiným mačičkoesom, ktorý po ňu poslala ona. V hoteli Izabeta rýchlo rezervovala izbu.

"$ $Vitaj. O čo ide?"

"$ $Počkaj kým budeme v izbe."$ $ Rýchlo vyšli výťahom a zamkli.

"$ $Tak? Vyklop."

"$ $Poznáš meno Parvasîe Jean?"

"$ $Ten patrí k D, a to dobre viem."$ $ Povedala podráždene.

"$ $Počkaj, vieš čo zmizla súkromná klinika klonovania?"

"$ $Áno, ty myslíš...?"

"$ $Jean má klon, lenže ten sa im vymkol kontrole. Má Jeanove spomienky, má jeho moc a má hnev a chce ho zničiť, a jemu nezáleží koľko bude obetí."

"$ $On zničil to mesto."

"$ $Áno, Stonehenge je zničené a Morja zmizla."

"$ $Zmizla pred trinástimi rokmi."

"$ $Viem, ale celý čas žila v Bratislave a falšovala identitu, teraz sa vrátila."

"$ $Pošpinila meno rodu!"

"$ $A ty? Vzhľadom na to, že tvoje aféry sú verejným tajomstvom je priam čudné, že ty hovoríš toto. Nechala si svoje morálne zákony zmrazené stáročia v tekutom dusíku!"

"$ $Morja..."

"$ $Prestaň, porušila zákon rovnako ako ty."

"$ $Zákon?"

"$ $Nemôžem prezradiť, sľúbila som to."

"$ $Ešte niečo?"

"$ $Jednu vec by si mala vedieť. Morja má dcéru... Jegrigsen má dcéru."

"$ $Takže dva a dva sú štyri a dcéra je len jedna."

"$ $Správne. Morja sa preto vrátila, aby ju získala späť. Je to Goonová proroctiev."

"$ $Ako vieš? Potom by mohla byť Goonová proroctiev aj Arabela, aj Morna, aj Marone aj..."

"$ $Nie sú priamym potomkom. Okrem Arabely, ale teraz... teraz je väčšia šanca..."

"$ $Ale ani ona..."

"$ $Je, Morja Tlogenová a Jegrigsen Goon. Obaja majú v sebe krv oboch rodov."

"$ $Morja už nie..."

"$ $Tu nejde o rodinné rady! A aj keby, vtedy ešte bola!"

"$ $Máš pravdu. Musíme ju nájsť. Dcéru myslím."$ $ Neochotne priznala.

"$ $A Morja, neviem kde je, ani Morja ani dcéra, chceli sme zistiť šifru na knihu proroctiev, aby sme ju našli, lenže Morja sa po dotyku s tabuľou niekde stratila a Stonehenge je zničené."

"$ $Zničené?"

"$ $Vyzerá to tam, ako po výbuchu atómovej bomby."

"$ $Predpokladám, že Nevermore. Alebo ako..."

"$ $Nevermore. Vyhlásil, že je Nevermore."

"$ $Okej, vyšlem agentúru nech ho nájde."

"$ $Podceňuješ ho! Je silnejší ako si myslíš. Myslím, že nevie cítiť emócie, a to ho robí neovládateľným. Je schopný ničiť celé mestá!"

"$ $Viem, Niela! Máš pravdu, Morju a dcéru hľadaj ty, my začneme s varovaniami. Vyhlásime výnimočný stav."

"$ $Čo myslíš tou prvou vetou?"

"$ $Že som ťa zas zamestnala."

"$ $Ja som..."

"$ $Toto je výnimočný stav, si súkromná detektívka a vyhrala si jednočlennú súťaž."

"$ $Dohodnuté, daj mi mačičkoes a zbrane, no nie tie z KKKv, lebo tie zaostali sto rokov za reálnym vývojom."$ $ Izabete sa moc nepáčilo, že Niela kladie podmienky, ale prikývla.

"$ $Dohodnuté, ale vieš prečo to robím, len preto, že dcéra a piktopísmo môže byť užitočné."

"$ $Si strašná."$ $ Podotkla Niela.

"$ $To je moja vec."$ $ S pobavením odvetila Izabeta.

"$ $V poriadku."

"$ $Choď do národného inštitútu zbraní do Švajčiarska, máš tu tisíc zlatniakov a mačičkoes ťa čaká na ministerstve. Kľúče dostaneš."

"$ $Kde?"

"$ $Na ministerstve. A Niela, buď rýchla."

"$ $Je to moja práca Izabeta."

"$ $V poriadku."$ $ Niela rýchlo odomkla a vyšla z hotela. Izabeta ju mačičkoesom zobrala na ministerstvo. Tam si Niela vybrala praktický čierny mačičkoes triedy AA a s číslom generácie 5, pôvodne si chcela síce zobrať 6, ale to by bolo vcelku podozrivé keďže to bol jeden z najnovších mačičkoesov, ktorý mali len príslušníci vlád alebo prominentní občania. Naštartovala a zapla stav viditeľnosti na neviditeľné, to bola jedna z výhod áčok. Zapla autopilota a začala zas šifrovať šifru.

\begin{center}
*
\end{center}

Dokázala som to! Som preč! Mám niekoľko eur, takže môžem niekoľko dní ostať v relatívnom poriadku. Vydržia mi na týždeň. Bojím sa. Bojím sa toho kto som. Dokážem niečo čo ostatní nie. Je to čudné. Práve pred chvíľou mi z rúk vyšli zvláštne žlté kruhy. Všetko je čudné. Nie je nikto ako ja? Toho sa najviac bojím... som vcelku nerozvážne odišla... nie som plnoletá... Tak sa bojím...

\begin{center}
*
\end{center}

"$ $Je to čudné, máš veľa magickej sily, ale nevieš ju využívať, asi by sme ťa mali hodiť do vody a učiť plávať."

"$ $Náhodou plávať viem."

"$ $Metafora."$ $ Zahundrala Deoque a chodila po miestnosti. Pauline zničene hľadela pred seba.

"$ $Nepozeraj akoby ťa práve viedli na popravu! Vieš koľko veľa mágov začínalo ako ty?"

"$ $Ale neboli Goonovou proroctiev!"

"$ $Prosím ťa! Okamžite vstaň a pôjdeme do miestnosti na boj. Už som tam síce dávno nebola, ale aj tak je v dobrom stave. Za mnou!"$ $ Deoque ju viedla ďalšou chodbou, až do miestnosti v ktorej bolo veľa palíc, pištolí, mečov a laserov, že to vyvolávalo protikladné reakcie. V jednom kúte bola sieť, v ďalšom niečo ako prekážková dráha. Deoque ukázala na uzavretú komoru.

"$ $Tak je výcvik obrany čo som ťa učila, budú tam tieto tyrkysové prúdy, ty, ale budeš mať špeciálnu kombinézu, aby si sa zas nepopálila, zase tam,"$ $ ukázala na bludisko. "$ $Musíš pomocou Solanu a Laseru sa dostať von. Ktoré ideš skúsiť?"

"$ $Bludisko nevyzerá tak desivo..."

"$ $Vážne?"

\begin{center}
*
\end{center}

Jean, musíš zničiť tvoj klon. Výsledky dopadli inak ako sme očakávali. Odvolajte všetky pokusy. Ten Nevermore je mocný, ale dá sa poraziť.

Ak bol preklad správny tak sa Niela dostala k ďalšej správe od D. Pozrela sa na hodinky. Ešte minúta. Aspoň podľa jej autopilota. Nechala text správy uložiť pod heslom a rozmýšľala aké zbrane si vyberie. Dobre vedela, že spoločenstvo M má zbrane na úrovni väčšej ako ľudská špička, lenže stále sa predávali len meče a palice. Z jednoduchého dôvodu. Loby. Loby Marone Tlogenovej a jej siete obchodov. A to všetko pod tajným patronátom Izabety.

Pristála. Pomocou ovládača mačičkoes zmenšila a dala si ho do tašky. Pred budovou inštitútu stáli dvaja agenti SAACM, polície a tajnej služby spoločenstva. Niela ich poznala. Keď tam sama pracovala boli jej spolupracovníci.

"$ $Preukaz Niela,"$ $ povedala žena, Yoon.

"$ $Mám povolenie od Izabety, sem je."$ $ Vytiahla tablet a otvorila Izabetin príkaz spolu s prístupovým kódom.

"$ $Vzorka DNA."$ $ Povedal pre zmenu muž a vytiahol prístroj. Poriadne strážia ich víťazné ťahy. Pomyslela si. Bola dnu. Rázne kráčala ku dverám Nepozvaným vstup zakázaný. Dobre vedela čo v nich je. Síce v inštitúte bola raz, ako mladá agentka, pár týždňov po jej osvedčení.

"$ $Kto ste?"

"$ $Orbielová Niela, mám zvláštne povolenie od Izabety,"

"$ $Ukázať identitu, DNA aj povolenie spolu s dôvodovou zložkou."

"$ $Odoslať alebo ukázať?"

"$ $Ukázať!"

"$ $V poriadku."

Keď prešla Niela cez ešte ďalšie tri bezpečnostné skúšky mohla vojsť do skladu. Ale to nebol iba sklad. Ku každej zbrani bol pridaný manuál a každá sa mohla vyskúšať vo výcvikovom bludisku.

"$ $Vyberte si, dali ste zálohu tisíc, teda zbrane v hodnote päťsto zlatniakov. Môžete si ich vyskúšať."$ $ Niela okamžite zamierila k najnovším. Rok výroby 2014. Opatrne chytila pištoľ s laserovými nábojmi, nový prototyp, starý mal príliš krátku batériu a jeho hmotnosť bola priveľká na včasnú reakciu. Niela si bola istá, že si ich vyskúša v labyrinte, ale napriek tomu si chcela vziať ešte niečo. Vedela to z vlastnej skúsenosti. Bola v ňom len raz ako novopečená agentka... Vzala si ešte pištoľ z predchádzajúceho roku s uspávacím efektom a dve pištole s jednoduchými laserovými nábojmi.

Výcvikový labyrint bol taký ako si ho pamätala. Vošla a okamžite sa uzatvoril. Rozhliadla sa vytiahla zbraň, nebezpečenstvá v ňom boli čisto náhodné, aby sa predišlo naučeniu bludiska. Nič. Otočila a mieriac zbraňou pred seba sa snažila neprestať sa sústrediť. Zrazu si Niela všimla, že z jednej steny vyrazili ostne a posúvala sa k Niele. Tá rýchlo vytiahla Laserovú pištoľ a strelila. Jeden osteň odpadol. Strieľala náboje až kým v stene nebola diera a mohla cez ňu prejsť do druhej časti labyrintu.

\begin{center}
*
\end{center}

Ten útek bol myslím nepremyslený... môžu ma zavrieť... môžu zavrieť našich.. ale ja tak chcem ľudí, ako ja... Ja tak nechcem sa vrátiť domov... Som hladná... tých pár eur bolo podľa mňa dosť málo... ale musela som odísť...

\begin{center}
*
\end{center}

Dvere boli navlas rovnaké a ani chodba sa nijako neodlišovala. Akoby nekonečná.

"$ $Musím vojsť."$ $ Pomyslela si a otvorila dvere na pravo. Miestnosť bola tyrkysovo modrá. Neboli v nej ďalšie predmety a len jedny dvere, tie ktorými Morja vošla. Napriek tomu neodchádzala. Dotkla sa steny. Nič. Všimla si, že steny sú predsa len trochu odlišné. Jedna bola najviac tyrkysová, ďalšia mala modrý fľak práve vo všetkých rohoch. Podlaha bola tyrkysová a strop modrý. Nič. Morja zatvorila oči a začala hmatať až kým nepocítila dvere. Otvorila oči, ale dvere na tom mieste nevidela. Čudné. Pomyslela si. Zavrela oči a chytila kľučku. Stisla a cítila, že dvere sa pomaly otvárali a Morja pomaly vykročila. Stále mala ruku na kľučke keď ju začala tá ruka príšerne páliť. Ani nevedela, či od chladu, či od horúcej kľučky. Pustila počula buchnutie dverí. Otvorila oči. Ale zas bola v tyrkysovo modrej miestnosti. Zvláštne. Zas si pomyslela keď sa pozrela sa svoju ruku. Mala ju celú červenú, ako keby ju mala v ľade. Mohla odísť, ale nechcela. Chcela sa dozvedieť viac. Znova zavrela oči a pocítila kľučku pod svojou rukou. Tá ju ešte trochu pálila, ale napriek tomu sa rozhodla zájsť ďalej ako predtým. Otvorila ich, rýchlo vstúpila a okamžite pustila kľučku. Zas sa dvere zabuchli. Ako vtedy. Vedela, že nemôže otvoriť oči. Hmatala až kým sa dostala k iným dverám. Otvorila ich a uvidela krásne obrazy na stenách. Podišla k jednému, ale nedotkla sa ho. Cítila, že ju niekto ťahá. Obrátila sa, ale nikoho nevidela. Morja rozmýšľala. Nikto tam nebol, alebo bol? Rozmýšľala aj prečo vidí keď má zatvorené oči. Chcela ich otvoriť, ale zároveň nechať zatvorené, aby pokračovala. Prebehol jej mráz po chrbte, čo keď už nikdy nebude chcieť vidieť to čo predtým. Nevedela sa rozhodnúť. Ak neotvorí možno sa nikdy nevráti, ale ak ostane v tomto čudnom svete... niečo ju ťahalo. Počula matne svoje meno. Uvedomila si, že už nevie rozlišovať svety.

"$ $Kto?"$ $ Vyslovila a ťahanie stále zosilnelo. Zas nevedela v ktorom svete sa to odohráva. Bála sa, že sa už nikdy nevráti. Jedna jej ruka si chcela násilím otvoriť viečka a druhá sa jej v tom snažila zabrániť. Z obrazu pred ňou sa vynorila ruka a schmatla ju. Morja sa snažila vymaniť, ale stále bojovala sama zo sebou. Cítila zrazu veľkú bolesť v nohe a niekto jej otváral oči. Bola polovicou tela v obraze. Doslova. Chcela sa odtiaľ dostať, ale nedarilo sa. Zvreskla a tomu kto jej oči otváral sa jeho cieľ podaril. Morja uvidela tyrkysovú izbu a corlovne Medizo. Nohy jej tŕpli a Medizo si vydýchla. "$ $Žijete Morja,"$ $ Morja bola stále v štádiu šoku.

"$ $Čo to bolo?"

"$ $Prúd iného sveta. Ak by ste sa dostali za obraz nebola by cesta späť. Máte šťastie."

"$ $Corlovne Medizo, kde som to?"

"$ $Vo víre a teraz v izbe prúdu tyrkysovej. Ak máte pravé zmyslové vnímanie, v tomto prípade zrak tak sa doňho nedostanete, ale vtedy keď to strácate vás prúd tyrkysovej má a vy sa doňho definitívne dostávate."

"$ $Ak by som sa dostala cez ten obraz tak by som tam ostala?"

"$ $Áno, vo víre sú aj zákerné veci, v každom prípade, Oko by vás chcela vidieť."

"$ $Ale kde sme to, corlovne Medizo? Viem, že vo víre ale..?"

"$ $Vír má aj neprebádané zákutia. V tejto miestnosti som bola ale aj ja. Toto je časť víru, nikdy neviete kde tie dvere vedú. Ak tam nikto nie je i keď hovoriť vo víre o čase je... nerelevantné."$ $ Stále nevyšli z miestnosti. Morja položila otázku.

"$ $Dá sa odtiaľto dostať. Viem, že ste vraveli... ale..."

"$ $Oko vás očakáva,"$ $ povedala bez toho, aby si všimla Morjinu otázku. Morja sa napriek tomu, že sa, podľa nej, presvedčila o ich dôveryhodnosti, stále bola ostražitá. "$ $Tadiaľto,"$ $ corlovne Medizo otvorila dvere a vyšla. "$ $Čo čakáte? Kým vás zas stiahne vír?"$ $ Jej milý tón sa z jej hlasu už takmer úplne vytratil.

"$ $Prepáčte, corlovne Medizo, zamyslela som sa."

"$ $Vír vás môže stiahnuť aj keď si neuvedomujete presne vaše okolie."$ $ Povedala nevrlo. "$ $Doprava,"$ $ chodba bola stále rovnaká, ale to corlovne Medizo akoby vôbec nevadilo. Jednofarebné steny, rovnako prerušované dvermi sa zlievali do jedného bodu a nič iné nebolo vidieť. "$ $Sme tu."$ $ Povedala Medizo a otvorila dvere. Tie boli pokryté dvermi, pričom na každých bol iný znak. Corlovne Medizo začala vyrátavať a nakoniec otvorila dvere zo znakom v ktorom Morja spoznala znak piktopísma Ftsa, po wymyslensky len. V nej sedela na zemi Wymyslenčanka. Mala tmavé vlasy spletené do vrkoča, bola oblečená do jasno tyrkysového obleku. Na krku mala retiazku, ale Morja nevidela akú keďže bola otočená chrbtom. Vyzerala ako duchom neprítomná. Morja pochopila, že je to Oko. Čakala čo povie corlovne Medizo. Sama nikdy nebola pri ani jednej z Fentenzíjskej tradície Oka tak nepoznala ich zvyklosti. Corlovne Medizo mlčala. Morja ju ticho pozorovala. Pohla sa. Načiahla ľavú ruku pred, resp. vzad Oka a vystrela ju. Roztiahla prsty a pomaly ich uvoľnila, toto urobila niekoľko krát. Morja premýšľala, či to nie je nejaké kúzlo, až kým pochopila, že to je nejaké zvláštne mávanie. Medizo stiahla ruku a priložila si ju na pravé plece a vzápätí položila pravú na ľavé plece a uklonila sa. Oko sa stále neotáčala. Teraz corlovne Medizo vystrela obe ruky akoby jej chcela niečo dať, pričom v rukách nič nemala.

"$ $Oko,"$ $ šepla jemne. Oko sa teraz úplne potichu a pomaly otočila a usmiala sa.

"$ $Vitaj corlovne Medizo a aj ty Morja,"$ $ povedala úplne vľúdne. Corlovne Medizo čakala kým dohovorí a potichu podotkla.

"$ $Oko, vy ste vraveli..."

"$ $Medizo, Medizo, vy viete, že Morja sa musí rozhodnúť sama. Nechajte nás osamote prosím."$ $ Medizo mierne sfialovela, ale poklonila sa a odišla. Morja nevedela čo má robiť. Rozmýšľala, či nemá zopakovať krok corlovny Medizo, ale Oko ju prerušila.

"$ $Vitajte Morja, som Oko, ak môžete, resp. ak chcete tak ma prosím nasledujte."$ $ Hovorila veľmi ticho, ale nikto ju neprerušil. Morja čakala kým vstane a ukáže kam ísť. Nič.

"$ $Druhé dvere, z názvom gé, v piktopísme."$ $ Ticho zašepkala. Morja vyšla z miestnosti a zas bola pri dverách. Hľadala znak gé. Rozmýšľala odkiaľ druhé dvere keďže dvere boli rozmiestnené k presnom kruhu bez začiatku. Pamätala ako vyzerá znak gé. Začala hľadať od dverí z ktorých vyšla. Nič. Počula ich otvorenie a vošla Oko.

"$ $Na opačnej strane."$ $ Povedala a otvorila ich. "$ $Až po vás."$ $ Morja sa ocitla v miestnosti. Tvorilo ju sklo, ktoré bolo oddelené bielym mramorom. Za každým sklom bol iný obraz, ale niektoré boli zahmlené.

"$ $Morja, toto je okno do vášho sveta. Z víru sa dá dostať prakticky všade, na akékoľvek miesto v akomkoľvek čase."

"$ $Rozumiem. Ale prečo sú niektoré okná zahmlené?

 "$ $Budúcnosť Morja, budúcnosť. Tam nemáš pozorovateľa, nemôžeš vedieť čo sa tam deje. Ja som zo staršej doby ako vy, preto by som mala vidieť zahmlených viac okien ako vy a corlovne Medizo. Napriek tomu vidím to čo vy, lebo už mám pozorovateľa, ním ste vy."

"$ $Medizo tvrdila, že toto je mimo času a mimo priestoru."

"$ $Corlovne Medizo má v niečom pravdu a v niečom sa mýli, rovnako ako všetci. Vír nie je spojený takmer ničím s vaším priestorom a časom, ale priestorom je prepojený, lebo... uvážte ako by sme sa tu inak dostali? A čas tu plynie, inak ako u vás, u vás sa nedá vrátiť, tu vo víre je možné sa vrátiť na zem v inom čase ako tu prišli predtým a potom sa tu zas vrátiť a my si budeme myslieť, že tu bol, lebo minulosť sa zmení. Zamotané že?"

"$ $Ja rozumiem Oko. Len... čo chcela Medizo?"

"$ $Dostať sa do budúcnosti?"

"$ $Cez zahmlené okno sa dá prejsť?"

"$ $Nie, lebo nemáš pozorovateľa. Vy môžete vidieť čo tam je, ale bez pozorovateľa sa tam nedostanete, ak to je budúcnosť, hoci budúcnosť je veľmi relatívna téma. My ju nemôžeme poznať skôr ako sa tam dostaneme."

"$ $Ale proroctvá... a veštectvo..."

"$ $Proroctvá ukazujú verzie, vždy je viac verzií budúcnosti a ak sa nemýlim aj prítomnosti a dokonca minulosti. A to, že sa väčšinou plnia je vec toho, že ľudia sa ich snažia napĺňať. A Dé tiež, hoci sám ich odmieta, ale to len zo strachu. Bytosti majú radi istotu vedenia čo bude."

"$ $A veštectvo?"

"$ $To je pozoruhodný jav. Veštice vidia za časopriestor, oni sú pozorovatelia sami pre seba, Morja, tým, že to nemôžu nikomu povedať, dávajú tomu význam. Ja mám osobne názor, že je to ako predpoveď počasia, niekedy sa vydarí, niekedy nie, a keď sa nevydarí, tak to veštica nepovie."

"$ $Ďakujem za vysvetlenie. A ja..."

"$ $Pýtali ste sa ma ako sa dá odtiaľto dostať. Proste vyjdete cez určité okno. Je to tu, ak sa nemýlim zariadené Hlagenovou teóriou telepatie, takže pomyslíte si úsek a okno sa objaví a vy ním proste prejdete. Ale... chcela by som vám povedať ešte jednu vec, ak zasiahnete do minulosti zabudnete na všetko čo bolo medzi tým časovým úsekom keď ste sa vrátili a vtedy keď ste odišla, okrem víru, takže je dosť možné, že urobíte to isté."

"$ $Rozumiem Oko."

"$ $A... corlovne Medizo by sa chcela, ak budete súhlasiť premiestniť do vášho času."

"$ $Otázka?"

"$ $Áno?"

"$ $Dá s premiestniť aj do iného priestoru?"

"$ $Samozrejme."

"$ $Súhlasím zo zobratím corlovne Medizo."

"$ $Zavolám ju Morja. Zatiaľ..."

"$ $Počkám."$ $ Oko vyšla a Morja využila príležitosť, že sa nachádza v miestnosti ovládanou Hlagenovou teóriou telepatie. Goonová proroctiev, vtedy keď som sa stratila vo víre. Pomyslela si a okolo nej sa objavili okná z rôznymi výjavmi. Nemenili sa. Hnedovlasé dievča spolu s niekým z Wymyslenska alebo Spoločenstva, kto sa nápadne podobal na jej priateľku (odhliadnuc od toho, že sa posledných trinásť rokov nevideli) Bellu Lietavú, agentku v utajení, mala pocit, že bola nezvestná, aspoň tak to pred niekoľkými rokmi matne počula a tiež na jej manžela Tarryho Lietavého, výskumníka. Toto bude ich syn, alebo aspoň príbuzný. Vzhľadom na to, že Morja svoju dcéru videla tri mesiace, nevedela, či je to ona. Ďalšie okno ukazovalo rozbité okno. Obraz, ktorý tam nečakala, bolo inkské mesto, Machu Picchu. Keďže vedela, že sa môže zmýliť vybrala si to čo jej, jej heslo najmenej pripomínalo – mesto Inkov. Vošla Oko a corlovne Medizo.

"$ $Vieš ktoré okno?"

"$ $Samozrejme,"$ $ povedala aj ukázala na inkské mesto.

"$ $Čože?"

"$ $Kľud, corlovne Medizo, je to Morjina voľba."

"$ $Odpusťte oko,"

"$ $Viete to okno otvoriť?"

"$ $Nie,"

"$ $Áno,"$ $ Povedali naraz Morja a corlovne Medizo.

"$ $Tak teda,"$ $ opustila miestnosť.

"$ $Ako sa to otvára corlovne Medizo?"

"$ $Piktopísmo, povedz ‚otvor sa okno' v jazyku piktopísma a všetky sa otvoria. Takže musíme byť na jednom, a tom istom."

"$ $Jasné,"$ $ Morja sa nadýchla a vyslovila. "$ $Tvereji dej lenjed."$ $ Okna sa začali obracať a zhadzovať do sveta všetko čo na nich bolo. Morja zas, spolu s corlovne Medizo letela časopriestorom tam kde mierilo okno. Zatvorila oči. Keď sa ich odvážila otvoriť corlovne Medizo bola preč.

\chapter{Stretnutia}

Rosa sa pod rúškom noci vydala do centra Dublinu. Do parku už viac radšej nepôjdem, po tom čo sa stalo. Pomyslela si. Bola tam dva razy v živote. Síce naposledy pred štyrmi rokmi niektoré veci sa zmenili. Tak kde pred predtým bolo prázdne miesto teraz stála krikľavo zelená budova. Okolo nej bez povšimnutia prechádzali ľudia. našli sa dokonca ja taký čo sa snažili prejsť cez ňu. Akoby bola neviditeľná. To je, ale hlúposť, pomyslela si Rosa, veď ona ju predsa videla. Na budove svietil farby meniaci nápis. "$ $CMrestaurant Dublin, jedna zo siete pre ľudí neviditeľných reštaurácií."$ $ Rosa si nápis ešte raz prečítala. Premýšľala, čo znamená pre ľudí neviditeľných, veď ona ju videla a bola si istá, že človekom je. Napriek tom, že sa jej to zdalo zvláštne, rozhodla sa vojsť. Bola hladná, a tak sa rozhodla, že za tých pár eur sa ide navečerať, síce nevedela čo to za stravovacie zariadenie je, nikdy tam nebola a matka, Rosa odišla hlavou z týchto nebezpečných vôd. Ona vedela prečo sa tam chce ísť pozrieť. Najviac ju zaujímal nápis. Pre ľudí neviditeľný.

Keď vošla, najskôr bola v miernom šoku, pretože v nej neboli len ľudia. boli tam aj zvláštne tvory, podobné človeku, okrem toho, že ich výzor sa podobal zvieratám, ale vyzerali ako inteligentné bytosti. Rosa si dala facku, aby zistila, či náhodou iba nesníva. Nie. Naozaj tam bola a všetko okolo nej bolo pravdepodobne reálne. Keď sa otriasla z počiatočného šoku našla prázdny stôl a prezrela jedálny lístok. Bol písaný v troch jazykoch – írčine, angličtine a zvláštnom jazyku, mierne podobnom angličtine, označenom ako W. Všimla si, že to je rozdelená ako obed raňajky a večera. I jedlá boli čudné. Kostičková polievka, Stačí pridať vodu s šunkou, Mäsová kaša, Buble banker, To sú čo za jedlá? Pomyslela si Rosa. Na chvíľu sa odpútala od menu a sledovala ostatných. Prekvapená zistila, že robia to čo ona. Jedna žena, (Hlava a uši vyzerali ako zajacove, ale inak bola normálna) si bez dotyku pritiahla pohár a iná žena (z Číny) si nechala miešať jej modrý nápoj bez dotknutia sa lyžičky.

"$ $Dokážu to čo ja,"$ $ ticho šepla. Keď sa zas zahľadela so jedálneho lístka a uvidela ďalšiu zvláštnosť. Ceny neboli v eurách, ale v čudnej mene Zlatniak.

"$ $Čože? Aký zlatniak?"$ $ Nechápala, a pre jej zamyslenie zabudla, že je v reštaurácii, a okolo nej sú nejaké osoby, a povedala svoju myšlienku nahlas. Obrátila sa na ten smer kde bola niekoľko hláv s výrazom, ktorý hovoril, že nevie niečo tak obayčajné. Netušila, čo má povedať. Nakoniec si povedala, že keď to už vyslovila, tak sa spýta.

"$ $Čo je to ten zlatniak?“

"$ $Veď to je jasné nie?"$ $ Vyprsklo dievča v tmavo modrom plášti.

"$ $Musí to byť jasné, keď si tu? k by si bola človek nevidela by si túto reštauráciu.“

"$ $Čože? Ja som človek.“

"$ $Anatomicky sme to tiež, okrem tých z Wymyslenska, ale nie ľudia ako tí na zemi. Čo nie sú v spoločenstve M.“

"$ $Spoločenstvo M? Wymyslensko?“

"$ $Kto si, oh my god?"$ $ Povedala žena z Číny čo si bez dotyku miešala nápoj.

"$ $Ja som... no... ja."$ $ Zmohla sa len na to. Vtedy ženin pohľad trochu stvrdol a mierne sa zdesil.

"$ $Myslím kto si, keď nevieš o nič o Spoločenstve M, ani o Wymyslensku? Si z D?“

"$ $Nerozumiem vám,"$ $ hlesla Rosa a žena sa jej spýtala ešte raz.

"$ $Otvorene, si od nás, keď vidíš a si v tejto reštaurácii alebo naše kúzla prestávajú účinkovať?"$ $ Rosu to prekvapilo.

"$ $Počkať, neviem či to, čo viem, je to, čo označujete vy za kúzla, ale ľudia čo na ne veria, by to za ne označilo.“

"$ $Ty nie si v spoločenstve, alebo to len hráš?"$ $ Rosa sa začudovala.

"$ $Prečo? Nikdy som o spoločenstve nepočula ale..."$ $ Z jej stola sa do vzduchu vznieslo menu, následne sa samo skladalo so komplikovaného orgiami a pristálo zas na stole.

"$ $Toto sú čary,"$ $ skonštatovala žena. "$ $Nepoznám ťa? Odkiaľ si, vzhľadom na to, aká je doba, by bolo potrebné identifikačná karta.“

"$ $Identifikačná karta? Myslíte niečo ako občiansky preukaz alebo..."$ $ Žena vyprskla.

"$ $Občiansky preukaz? Tie ľudské ľahko falšovateľné kartičky? Odkiaľ si?“

"$ $Myslíte kde bývam?“

"$ $Nie, ale kto boli tvoji rodičia? Čarodejníci, čo pochybujem, len keby boli takí ako Messia Tlogenová. Alebo ľudia?“

"$ $Ľudia predpokladám, i keď som ich nikdy nevidela?“

"$ $Ako to myslíš?“

"$ $Som zo Švédska, ale žijem v Írsku kde som sa dostala cez medzinárodné osvojenie odkiaľ som ušla. Spokojná?"$ $ Žena sa zamračila.

"$ $Je možné, že by boli ľudia, predsa alela aa, ale... čarodejníci by toto spravili...?"$ $ Mrmlala si sama pre seba žena. "$ $Vieš ako sa volali? Potom môžem ťa lepšie identifikovať a budeš dôveryhodnejšia.“

"$ $Prečo to potrebujete?“

"$ $Žijeme v nebezpečných časoch. D zbrojí a nikto nemôže byť jasne na jednej strane, i keď D svojich stúpencov patrične poznačuje...“

"$ $Viem, že matka bola Rosa, ako ja a otec neviem, ale moje pôvodné priezvisko, ak sa nemýlim, bolo Tlogenová Goonová, ale neviem to presne."$ $ Pri jej priezvisku reštaurácia potichu zhíkla.

"$ $Si si istá, že Tlogenová Goonová?“

"$ $Na deväťdesiatdeväť percent áno, a čo s tým...?“

"$ $Ty nevieš alebo to len hráš? Veď každý vie kto sú Tlogenovci a Goonovci!"$ $ Rosu to mierne vyviedlo z miery. Tak kto je?

"$ $Hovorím, že so o vašom spoločenstve doteraz netušila, jediné čo som vedela, bolo to, že viem niečo viac, ako normálny ľudia."$ $ Neveriacky na ňu hľadeli.

"$ $Si si skutočne istá...? Lebo ak sa pozrieme do rodokmeňa...“

"$ $Veď na svete môže byť kopa Tlogenovcov a Goonovcov.“

"$ $Rod Tlogena a Goona si svoje meno vždy strážil, aj medzi ľuďmi, nedovolili aby... A podľa www.familytree.ym sú jediní Tlogenovci a Goonovci, vetva Od Izabety a Rolliusa. A pochybujem, že by niekto z nich... Ak nie Messia..."$ $ Poslednú vetu si povedala sama pre seba.

"$ $Messia?“

"$ $Ani to nevieš?“

"$ $Hovorím, že...“

"$ $Prepáč, no v každom prípade Messia je, inak povedané bola, keďže Tlogenovci ju zo svojej rodiny vylúčili, sestra Marone.“

"$ $Vylúčili?“

"$ $Odmietla mágiu a odišla za ľuďmi,"$ $ Povedala pohŕdavo.

"$ $Prečo?“

"$ $Odišla za človekom... ale odvtedy sa o nej nikde nepočulo.“

"$ $Ten www.familytree.ym je len o tých čo majú...“

"$ $Mágiu? Nie, inak by bol takpovediac nanič.“

"$ $Tak tam môžete nájsť mňa.“

"$ $Nie, je to tu len o tých čo sú už oficiálne plnoletý alebo niečo znamenajú, a čo sa týka Messie, existuje spôsob... Ale ona sa odtiaľ jednoducho stratila, posledné záznamy sa o nej objavujú predtým zmizla. Ale tak isto aj o Morji Tlogenovej...“

"$ $Morja Tlogenová?“

"$ $Ušla tiež,"$ $ Rosa na chvíľu zmĺkla a snažila sa to pochopiť. Bola tu možná jedna vec, ale s touto hypotézou sa nechcela deliť.

"$ $Kde je niečo kde sa dajú vymeniť eurá za vašu menu?“

"$ $Tam,"$ $ Žena ukázala na automat. Svietil na ňom kurz 1 Gc : 20 eur a pri ňom ďalší, už slabšie svietiaci, 1 Gc : 1200 mač. Rosa si ešte všimla ako sa menia peniaze.

"$ $Päťdesiat Strieborných je jeden Zlatniak a sto Meďnákov je jeden Strieborniak. Teda jeden strieborniak je štyridsať centov a jeden cent sú štyri meďnáky."$ $ Prepočítala Rosa. Zmenila si niečo nad polovicu svojich peňazí. Teraz mala v peňaženke trinásť Zlatniakov, osemnásť Strieborniakov a šesťdesiatdva Meďnákov.

"$ $Dá sa tu kúpiť elektronika?“

"$ $Tu nie, ale vo vedľajšej budove, môžem ti ju ukázať."$ $ Povedala žena z Číny.

"$ $V poriadku."$ $ Akonáhle vyšli z budovy žena sa jej spýtala.

"$ $Ty si ušla z domu navždy?“

"$ $Ja sa tam vrátiť neplánujem. Už nie som jediná... teraz už nie...“

"$ $A kde chceš byť?"$ $ Rosa vyzerala, že o tomto ešte nepremýšľala.

"$ $Váš svet je zaujímavý, ale neviem či by som v ňom mohla zostať.“

"$ $Mám pre teba ponuku.“

"$ $Akú?“

"$ $Ak súhlasíš, môžeš bývať u mňa, síce sa dosť často sťahujem, ale...“

"$ $Ako mám vedieť, že vám môžem dôverovať?“

"$ $Správny prístup, nie som s Dé, ani s Wymyslenskom.“

"$ $To ma zatiaľ nezaujíma.“

"$ $Som v spoločenstve M.“

"$ $To ma zatiaľ nezaujíma, tu ide o dôveryhodnosť, nie o príslušnosť.“

"$ $Ako to...“

"$ $Ja sa nepohybujem vo vašej politike tak, aby som mohla posudzovať podľa politickej príslušnosti, mne ide o to prečo, lebo ak vám ide najmä o vedľajšie záujmy...“

"$ $Môžem prisahať, ak...“

"$ $Prísahy, pcha,"$ $ pohŕdavo povedala Rosa.

"$ $Prísahy majú veľkú moc, nie ako u ľudí, je to jedno z najľahších programovacích kúziel...“

"$ $Čože?“

"$ $Veľmi komplikované odvetvie mágie. Ale to teraz môžeme nechať tak, som ochotná...“

"$ $A prečo? Nepoznáte ma ani hodinu? Čo chcete?“

"$ $Máš talent Rosa, Poznám len dvoch, okrem teba, čarodejníkov čo dokázali bez výučby a vedenia urobiť to čo ty. Izabetu Tlogenovú, zakladateľka Spoločenstva M a ... Jean Parvasîe. Máš na mágiu nadanie a myslím, že... by pre teba a pre všetkých nás bolo škoda, keby taký talent ostal nerozvinutý...“

"$ $Ak ste ochotná prisahať, že to je pravda a nič so mnou neplánujete zlé, vám verím, ale najskôr si niečo zistím. Ďalej dôjdem sama. Dám vám vedieť. A... len.. kto ste?“

"$ $Chen Chengová, a ako... ...telepaticky?"$ $ Posledné slovo si povedala iba sama pre seba.

"$ $Kľudne,"$ $ Rosa sa jej ozvala v hlave.

"$ $Ty vieš...“

"$ $Telepatiu, ako ste to nazvali.“

"$ $Ty si ma počula?“

"$ $Áno,“

"$ $Ty ovládaš aj zmyslovú mágiu časť dva?“

"$ $Netuším ako sa to nazýva, ale počula som to."$ $ Chen sa obrátila a bola presvedčená, že Rosa má v sebe ešte viac magickej sily ako Jean Parvasîe. A o nikom takom ešte nepočula...

Rosa si za šesť zlatniakov kúpila vcelku jednoduchý tablet. Bola prekvapená nízkou cenou elektroniky. Doplatila ešte dva a mala na ňom neobmedzený internet. Na stránke ktorú spomínala Chen napísala svoje meno. Nič. Ani slovo. Skúsila to ešte raz, ale tento raz napísala bez priezviska Goonová. Teraz našla meno svojej matky, aspoň to predpokladala. Akonáhle, ale na jediný výsledok klikla vyskočilo maximálne okno Zadajte Heslo. "$ $Dokelu!"$ $ Zadala ešte meno Chen Chengová. Našlo jej to minimálne tri výsledky. Zadajte heslo. Rosa aspoň vedela, že Chen skutočne existuje. Bola rozhodnutá, že po prísahe prijme jej ponuku. Z čisto osobných dôvodov. Rozhodla sa, z pocitom nezvyčajnej slobody zistiť niečo o tých prísahách. Všimla si, že v tablete je nejaká funkcia zvaná register. Pozrela sa do príručky kde bola jeho funkcia opísaná len troma slovami. Vyhľadávač podľa hesla. Otvorila ho a zjavilo sa okno.

Register

Hľadané -...(Zadajte Vyhľadávané)...-

Rosa napísala do toho okna jedno slovo. Prísahy. Chvíľu prehľadával, a napokon naskočili výsledky:

Register

Hľadané "$ $Prísahy“

Vows – Časopis o celebritách

www.magie.ym

Nebezpečenstvá Prísah Spoločenstvu – Rozhovor z Cecíliou Žblnkotaničkovou

www.loitsu-taika.ym

Prísahy – encyklopédia mágie

Ako spoznať nepriateľa – Najnovší bestseller

www.izabetasdagger.ym

Všetky nepotrebné veci čo potrebujete vedieť

www.knowthing.ym/en/magic/davy

Ďalšie výsledky:

Rosa zistila, že prvý, tretí, štvrtý (ukázal sa ako nepotrebný, lebo bol len po fínsky, teda, i keby v ňom niečo bolo, nerozumela by ani slovo), šiesty a siedmy sú absolútne nepotrebné. Posledný výsledok na strane, jej dal to, čo hľadala.

\begin{center}
*
\end{center}

Loviisa mala rozčítanú Sylviinu knihu. Tá niečo písala do notebooku a mračila sa.

"$ $Tulienka Deľa mešká. Tvrdila, že príde od štyri hodiny.“

"$ $Kto?“

"$ $Tulienka Deľa, jedna z nášho... ako by som to povala... hnutia odporu.“

"$ $Si si skutočne istí tým čo hovoríš?“

"$ $Samozrejme, nerozumiem prečo po tých dôvodoch mi neveríš, ...ale tebe Cecília vymazala pamäť.“

"$ $Počkať, ty si zo Spoločenstva?“

"$ $Áno, to som už hovorila...“

"$ $Ale prečo potom to neurobila aj tebe?“

"$ $Prišla som tu sama?“

"$ $Hovoríš protichodné tvrdenia, tvrdíš, že Wymyslensko nemáš rada.“

"$ $Nemám rada režim, nie Wymyslensko. Keď som tu prišla, bolo to po prvý a jediný,"$ $ pričom slovo ‚jediný‘ náležite podčiarkla, "$ $raz, keď využila svoje schopnosti.“

"$ $A.. prečo?“

"$ $Nerozumieš?! Som démonka! Ľudia nenávidia démonov! Keby to zistili zničili by ma. Ja som si to uvedomila, vtedy, keď som odišla. Moji rodičia ma nenávideli! Vieš aké to je?! Len preto... je to prekliatie! Vo Wymyslensku verejná inkvizícia démonov neexistovala a ja som si neuvedomovala totalitu, čo tu je, a nebola som taká hlúpa, ako keď som ešte bola na zemi a nerozhlasovala to. Ja nie som ani zo spoločenstvom, ktoré pre mňa za moje prekliatie spravilo príšeru a služobníka D a ani s Wymyslenským režimom, ktorý zatvára vlastných ľudí za to, že sa odvážia povedať, že Spoločenstvo M je viac demokratické ako Wymyslenská federálna republika! Keby som nevedela čo je Dé zač, asi by som sa k nemu pridala. A nechápem prečo ti to rozprávam,"$ $ Nahnevane penila Sylvia a mala čo robiť, aby zastavila slzy čo sa jej tisli do očí.

"$ $Si v poriadku? A... ty by si sa vážne pridala k D?"$ $ Tak trochu vystrašene, tak trochu v šoku povedala Loviisa.

"$ $Ale iba keby som nevedela aký je. Sú len tri strany, a D je k démonom najlepší, ale len z jediného dôvodu, je ním sám.“

"$ $Ešte ľudia,"$ $ podotkla Loviisa.

"$ $Ľudia? Tí robili inkvizíciu, sú rovnakí, len nemajú naše schopnosti, čo uznávam, preto by nás nemali radi.“

"$ $Prepáč..."$ $ Vyjachtala Loviisa. "$ $Ja som si to neuvedomila.“

"$ $Nevadí."$ $ Sklesnuto povedala Sylvia. "$ $Ja... nevadí mi to,“

"$ $Vďaka,“

"$ $Počúvaj,“

"$ $Áno?"$ $ Sylvii sa zdalo, že od tej nebezpečnej témy už odišli.

"$ $Ty tvrdíš, že mi Cecília vymazala pamäť.“

"$ $Nie celkom, len dôležité momenty o Wymyslensku.“

"$ $Neprerušuj prosím ťa.“

"$ $Prepáč,“

"$ $Keď ju vymazala... je ju možné vrátiť? Pamäť myslím.“

Sylvia premýšľala. Vzala do ruky notebook a rýchlo niečo sledovala.

"$ $Hm..."$ $ Hovorila a klikala a klikala. "$ $Cecília ich uskladňuje a skúma, aby mohla, aspoň podľa pár... ehm, priateľov zničiť Spoločenstvo M. Neničí ich, ale ukladá. A ak sa nemýlim..."$ $ Zas si niečo pozerala. "$ $Tak spomienky sa dajú vrátiť jednoducho, potrebuješ samozrejme ich, ako to mám nazvať, odraz, lebo ona ti vlastne zobrala len informáciu v neurónoch, ktorú má, ak sa zas nemýlim v počítačoch.“

"$ $Kde?"$ $ Pri Loviisinej otázke sa Sylvia rozžiarila. Ešte bola nádej...

"$ $Ty premýšľaš...“

"$ $Áno. Chcem to urobiť, aby každý vedel čo Cecília robí."$ $ Sylvia sa rozžiarila tak, ako sa to dalo nepozorovane.

"$ $Začínaš sa mi páčiť ešte viac.“

"$ $Ďakujem,"$ $ povedala prvú vec čo ju napadla, ale Loviisa zrejme netušila, čo si má o Sylviinej poznámke myslieť.

"$ $Aby som nezabudla odpovedať, spomienky sú podľa Mrany v mieste Cecíliiných kancelárií. Ak sa tam vlámeme, tak nás bude hľadať možno aj Cecíliina špeciálna jednotka,“

"$ $Myslela som si, že dobrodružstvo máš rada.“

"$ $S Cecíliinou špeciálnou jednotkou som sa ešte nemala česť zoznámiť. Oni... ...oni obviňujú a konštruujú procesy s disidentmi, a tiež... majú povolenie, či skôr príkaz, zabíjať. Mňa nemôžu, vieš prečo..."$ $ Mierne sa zachmúrila, ale pokračovala, "$ $ale ty... si si istá?“

"$ $Keby tam nebolo nebezpečenstvo, bolo by to na nič, niečo podobné si povedala ty.“

"$ $To je pravda.“

"$ $Ja to chcem, i za cenu takéhoto nebezpečenstva.“

"$ $V poriadku. Ja sa spojím s Mranou, ty prosím ťa nalistuj sedemdesiatu ôsmu kapitolu knihy o špeciálnych jednotkách.“

"$ $Jasné,"$ $ Loviisa prevracala stránky.

"$ $Budeme musieť toho urobiť veľa, zistiť, teda okuknúť kancelárie, podľa toho pomôcky a pripraviť bezpečný úkryt.“

"$ $Načo je posledný bod?“

"$ $Keď po nás pôjde jednotka je potrebné vedieť, kde ísť.“

"$ $No super,"$ $ Skonštatovala Loviisa.

"$ $Ty si to chcela,"$ $ upozornila ju.

"$ $Ja viem, ale... to riziko začína byť zaujímavé.“

"$ $Aspoň vidíš čo na tom vidím ja. Spojím sa s Mranou, nech mi pošle mapu. A aby bolo jasné, ani slovo, dobre? Cecília je schopná všetkého, ale aj my."$ $ Uškrnula sa.

\begin{center}
*
\end{center}

Existuje viac druhov prísah, od zakázaných (trpiace prísahy) až po tie čo sa skladajú viac, či menej zo zábavy (porušiteľná prísaha). Prísahy ako samotné musia byť "$ $uskladnené"$ $ v tom kto prisahal. Ak sa poruší podmienka (resp. ak sa neporuší pozri porušiteľná prísaha) tak sa udeje vec, na ktorú sa prisahalo, nezávisle od vonkajších podmienok. Síce sú prísahy jednou z najneprebádanejších odvetví mágie, fungujú. Vo všeobecnom zákone spoločenstva M, zbierka dva, zákon číslo osemnásť, je zakázané prisahať na prísahu Ló, ktorá spôsobuje nevysloviteľné bolesti, ale nezomrie. Druhá zakázaná prísaha je prísaha, pri ktorej porušení porušujúci zomiera. Za použitie týchto dvoch prísahách je väzenie, od troch do desiatich rokov. Prísahy je zakázané používať pri vydieraní a pri priatí do zamestnania a štúdia. Za tieto priestupky sa automaticky aplikuje zákon číslo 18 z druhej zbierky zákonov. V súčasnosti je najpoužívanejšou prísahou prísaha zabudnutia, teda pri jej porušení okamžite porušujúci na všetky udalosti súvisiace s prísahou. Prísaha sa aplikuje cez kúzlo Pentea, v jazyku prisahajúcich. Prísahy sa podľa najnovších výskumov nededia.

Rosa vedela čo urobí.

"$ $Chen Chengová, počuješ ma?“

"$ $Rosa? Rozhodli ste sa?“

"$ $Áno, ale ste ochotná porušiť zákon?“

"$ $Ktorý?“

"$ $Číslo osemnásť, zbierka dva, o prísahách,“

"$ $Poznám zákony, ale tieto... Ide o bezpečnosť a rozumiem vám. Tak, samozrejme,“

"$ $Ale, nikto na ten zákon...“

"$ $Na porušovanie tohto zákona nikto nedohliada.“

"$ $V poriadku. Kde?“

"$ $Pri stole.“

"$ $Vašom?“

"$ $Tykaj mi; pri mojom stole.“

"$ $Čakám.“

"$ $A ja prichádzam."

\begin{center}
*
\end{center}

"$ $Tarny, čo robíš?“

"$ $Ticho,"$ $ odsekol namiesto odpovede.

"$ $Nekradni Deoque knihy, je zázrak, že nás neprizabila,“

"$ $Kľud, čo si myslíš, že chcem čítať tú narcistickú literatúru o nej, čo nám nechala?!“

"$ $Nie ale...“

"$ $Tak, prosím ťa, ma neprerušuj. Zoberiem túto knihu D a boj proti nemu a tiež túto Čierna mágia v praxi, a na čo je dobrá.“

"$ $Tarny...“

"$ $Už to mám, použijem zmyslové kúzlo a bude si myslieť, že sme nič nezobrali.“

"$ $Neprekukne to?“

"$ $Ak nevieš, že tú mágiu použili nemáš šancu ju odhaliť. To by si musel...“

"$ $Poznáš Deoque. Je paranoidná. Neprestane byť opatrná.“

"$ $To som si istý.“

"$ $Len si to prečítam, nekradnem.“

"$ $Ale Deoque...“

"$ $Deoque je Deoque. Podľa mňa tam bude dostatočne dlho, na to, aby sme to prečítali.“

"$ $Nemôžeš to proste urobiť, že urobíš zmyslové kúzlo na nás?"$ $ Tarny povedal niečo ako nemáme pozorovateľa.

"$ $Fajn."$ $ Teda neochotne súhlasila. "$ $Ale ak sa niečo stane, je to tvoja vina.“

"$ $Je to v piktopísme Tulienka Deľa.“

"$ $A? Mám to preložiť?“

"$ $Ty si tu expert na jazyky.“

"$ $V poriadku.“

"$ $V piktopísme je čierna mágia. Ja idem čítať túto hrubšiu."$ $ Ukázal na knihu D a boj proti nemu.

"$ $Je v staršom piktopísme, neviem či to preložím správne.“

"$ $Nevadí, použi kúzlo na prepájanie neurónov, proste si to zapamätaj a prelož to čo vieš.“

"$ $Ako keby som to nevedela,"$ $ namrzene odsekla a zahľadela sa do knihy.

"$ $Démon má veľa podôb, jeho schopnosti nie sú, ale špecifikované len na ľudí. Dokáže sa premeniť aj na akékoľvek zviera, či tvora. Tieto legendy sú... Hm, tu je to rozmazané... Tulienka Deľa, je tu niečo, na ďalšej strane, ale v piktopísme. Preložíš to prosím ťa?“

"$ $Čo? Jasné, daj mi to. Vyzerá to ako napísané rukou. Toto by malo byť Jeho Tajomntvo, teda to je s, teda tajomstvo, jeho mhoc, moc teda, to je jedno písmeno, je v Truhliciach troch vôní, čo? Jáj to je farieb, cenného kovu."$ $ Mrmlala Tulienka Deľa a prekladala. "$ $Jeho tajomstvo, jeho moc je v truhliciach troch farieb cenného kovu. Ďalej neprečítam Tarny. Je to príšerne rozmazané."$ $ Tarny sa opatrne toho chytil a sklamane vyhlásil.

"$ $Nie je tam nič. Akoby to tam niekto dopísal. Do toho obrázku. Čo myslíš môžem skúsiť pár kúziel.“

"$ $Mágia zanecháva stopu Tarny, nemali sme tú knihu vôbec vziať. Síce je toto ovzdušie ňou presiaknuté, ale každá na viac...“

"$ $Viem.“

"$ $Ukáž to, skúsim to ešte rozlúštiť, mal by si sa učiť prekladať piktopísmo.“

"$ $Tulienka, na to si odborník ty. Ja mám radšej zmyslovú mágiu.“

"$ $Ale zanecháva magickú silu. Na rozdiel od..“

"$ $Techniky a vedomostí, viem to.“

"$ $Nájdi niečo s tými truhlicami Tarny, aj ja sa to pokúsim zistiť.“

"$ $Dobrý nápad. Ja ako vidím je tu register, samozrejme ručný, keby sa autorom chcelo použiť Hlagenovu teóriu...“

"$ $Aby ste to zneužili!"$ $ Počuli za sebou hlas, ktorý nepochybne patril Deoque. Strhli sa. "$ $Zmyslové kúzla na mňa neplatia Tarny Lietavý."$ $ Vyhlásila. Pauline na nich hľadela polovične nechápavo a polovične nahnevane (práve sa vrátila z lekcie kúziel od Deoque kde sa jej podarilo vytvoriť poriadny Laser a ochrannú stenu). Tulienka sa oborila na Tarnyho.

"$ $Ja som to hovorila!“

"$ $Mňa to nezaujíma!"$ $ Zvreskla Deoque a zas vyčarila Laser. "$ $Vysvetlite mi to!"$ $ Zvolala nenávistne.

"$ $Prepáčte,"$ $ Snažil sa Tarny hovoriť pokojne, hoci to väčšinou nie je možné, keď na vás miery niekto s Laserom pripravený použiť ho.

"$ $Tie knihy čo ste nám dali...“

"$ $Áno?"$ $ Nahnevane ho prerušila Deoque.

"$ $Ak ste tie knihy nepísali vy, tak ich musel písať niekto iný, nepochybne blízky vám, a názory v nich boli tak trochu subjektívne."$ $ Tulienka Deľa ho mykla za ruku.

"$ $Ukradli ste mi knihy!“

"$ $Požičali. Kľudne vám ich vrátime.“

"$ $Mňa to nezaujíma zlodeji!"$ $ Tulienka Deľa sa rozhodla Deoque prerušiť.

"$ $Môžem niečo povedať?“

"$ $Hej.“

"$ $Tarny, ty si idiot!"$ $ Tarny sa, ale nenechal vyviesť s miery.

"$ $Pani Deoque, ide tu o porážku D. Keďže plánujeme unikať, a nikoho do toho nezasiahnuť, potrebujeme niečo vedieť.“

"$ $Máte knižnicu.“

"$ $Nemám platný preukaz na Anglicko, len na strednú Európu, a Tulienka Deľa nemá preukaz vôbec. A ani Pauline. Zavreli by nás."$ $ Na ich všeobecné prekvapenie Deoque prestala z ruky vysielať Laser.

"$ $Len preto, že ste proti D, nie som v spoločenstve a ani nemám nič proti jeho služobníkom, ale mám proti D osobné dôvody. Čo ste našli?"$ $ Spýtala sa chladne.

"$ $Niečo s truhlicami troch farieb, bolo to tam dopísané rukou piktopísmom.“

"$ $Vy viete prekladať piktopísmo?“

"$ $Ja nie len Tulienka Deľa.“

"$ $Ja som nič do kníh nepísala. Mám ich od jednej obchodníčky, mimoriadne výhodný obchod. Vraj pochádzajú zo starej Fentenzie. A teraz odíďte. Nevymažem vám pamäť na polohu tohto domu, lebo sa sťahujem, nechápem ako ste ma našli."$ $ Stáli tam. "$ $Tak na čo čakáte?! Vypadnite!"$ $ Zrúkla.

"$ $Už ideme."$ $ Zašomrala Tulienka Deľa a vyšli z domu.

"$ $Zneviditeľnime sa.“

"$ $A ako?“

"$ $Je to mimoriadne náročné kúzlo, ale je strašne jednoduché, začarujem ťa ja. Budeme ťa vidieť a ty budeš vidieť nás. ostatní nie. Je to náročné najmä v tom, že to musíš uvrhnúť úplne na všetkých aj keď s nimi nie si v kontakte.“

"$ $Ty vlastne seba nezneviditeľníš len všetkým vsugeruješ predstavu, že tam nie si.“

"$ $Problém je taký, že sa takto nedá zakliať techniku, pretože tá nemá zmyslové vnímanie.“

"$ $Tak trochu rozumiem."$ $ Prikývla Pauline. "$ $A kde ideme teraz?“

"$ $Keby sme sa teraz premiestnili do strednej Európy, som schopný kúpiť nejaké knihy, keďže mám preukaz.“

"$ $Nebola som v strednej Európe.“

"$ $Bola,"$ $ odvetil.

"$ $Tak si nepamätám.“

"$ $Skús to.“

"$ $Nespomínam si na nič!“

"$ $Skús napríklad niečo čo ste sa v škole učili a predstav si, že si tam i keď si to naživo nevidela.“

"$ $Dobre, čo napríklad?“

"$ $Česko? Slovensko? Maďarsko?“

"$ $Po maďarsky neviem, Tarny, budeš nakupovať ty, aké vieš slovanské jazyky, lebo predpokladám, že maďarčinu neovládaš, rovnako ako ja.“

"$ $To je pravda. Skús Pauline napríklad Slovensko alebo Česko. Tie jazyky poznám bezchybne.“

"$ $Vybavujem si, napríklad hlavné mesto Českej republiky, Prahu, ako som ju v jednej knihe videla... chyťte sa ma.“

"$ $Okej, len sa premiestni.“

"$ $A nie do vzduchu, nemáme aktivované krídla."$ $ Usmiala sa Tulienka Deľa. Pauline zatvorila oči a videla pred sebou mesto. Premýšľala ako si ho má predstaviť, aby ho neuvideli z vtáčej perspektívy. Tarny a Tulienka Deľa ich zatiaľ zneviditeľnili, ale oni sa stále videli. Pevne sa držali Pauline a zrazu sa z Anglicka ocitli pred zastávkou metra v Prahe.

\begin{center}
*
\end{center}

Rosa a Chen Chengová sa stretli.

"$ $Som pripravená prisahať.“

"$ $Na všetko? A porušiť zákon?“

"$ $Ak prísahu neporuším tak sa nemám čoho obávať."$ $ Rosa prikývla.

"$ $Prisahaj teda,"$ $ Z rúk jej vyšla fialová gulička a približovala sa k Chen, zanechávala pritom tmavomodrú stopu čo pohlcovala Rosu. Chen pochopila o ktorú prísahu ide, o prísahu Ló. Z jej ruky vyšla rovnaká a spojila sa s tou Rosinou.

"$ $Prisaháš,"$ $ Vyslovila to slovo Rosa, ktoré začínalo otázku a žiara fialovej sa rozpálila. "$ $Že neublížiš mne, Rose, ani nič neurobíš čo by k tomu mohlo viesť, bez môjho súhlasu?“

"$ $Prisahám,"$ $ Ticho povedala Chen. V tom zlomku sekundy sa guľôčky rozsvietili a blikli.

"$ $Prisaháš, že pri tom keď si ku mne prišla s tou ponukou nemyslela si na svoje osobné záujmy?“

"$ $Prisahám."$ $ Zas sa to s fialovou opakovalo. Rosa roztiahla ruku a fialová sa jej vpila do ruky. "$ $Prísaha potvrdená."$ $ Povedala zvláštnym hlasom. "$ $Toto znamená, že si súhlasila."$ $ Skonštatovala Rosa. "$ $Kde bývaš?“

"$ $Všade. Nemám pevný domov, preto nebudeš zrejme môcť chodiť do jednej školy a môžem ťa učiť sama. Mám na to patričné vzdelanie.“

"$ $Rozumiem, to si spomínala. A kde budeme teraz?“

"$ $Z Írska odchádzam zajtra. Pôjdeme do Nórska. Ale tam sa zrejme zdržím menej ako mesiac. Predpokladám, že žiadne vzdelanie z mágie si nedostala...“

"$ $Hovorím, že som žila s ľuďmi, aké magické vzdelanie som podľa teba mala dostať?“

"$ $Prepáč, predpokladala som, že nie, ale ako si sa naučila robiť s mágiou?“

"$ $Jednoducho, veď to je iba sila mysle,“

"$ $Si výnimočná, a to nielen medzi ľuďmi.“

"$ $Ďakujem za kompliment. Mám pár otázok. Po prvé, čo je to ten preukaz? Všade ho odo mňa chcú, a ďalej, čo sa budem učiť?“

"$ $Po prvé, preukaz je spôsob ako zistiť kto si. Žijeme v nebezpečnej dobe a preukaz je znakom, že nepatríš k Wymyslensku alebo k Dé. Vybavím ti ho. A po druhé, učiť sa budeš ako normálne matematiku, vedu, angličtinu, Wymyslenčinu, mágiu, vlastnosti a použitie organických a anorganických telies a látok, dejiny, geografiu, staropisy, boj a literatúru.“

"$ $To až toľko?“

"$ $Samozrejme, to je všeobecný prehľad a to sú len základné predmety. Učebnice sú nezávislé na štáte, ale odporúčam si vybrať Algoritmy a Logaritmy, Chémia živých organizmov, English X., Wymyslenčina pre stredne pokročilých, Magické zmysly, Použitie látok do elixírov a ich vlastnosti II., História a dejiny Wymyslenska, Teresovo v kocke, Staré spisy starovekého Egypta, Teória boja III. a Písmo z ostrova Fentenzia. Niektoré mám doma, síce neviem, že či by si mala prejsť v teórii na taký vysoký level hneď, keďže matematika u nás je podstatne zložitejšia, vedu neviem čo vy v ľudských školách preberáte, angličtinu ako vidím, vieš veľmi dobre, ale predpokladám, že si sa z Wymyslenčinou ešte nestretla...“

"$ $O tom jazyku teraz počujem prvý raz.“

"$ $Takže to budeš musieť dobrať, i pres nepriateľstvo ku Wymyslensku je druhý úradný jazyk. Takže aj Wymyslenčiny pre úplných začiatočníkov, začiatočníkov a pokročilých začiatočníkov. Histórie sú aj dejín Spoločenstva M, to sa preberá tretí až šiesty ročník, Wymyslensko je od sedmičky takže teraz sa začína preberať, Z geografie bola už zem, tú predpokladám, brali aj na zemi, mágiu ako vidím, z nej potrebuješ hlavne teóriu, viem, že je nudná, ale patrí to k záväzným učivám. Z prísad do elixírov je treba všetko dobrať, vzhľadom na to, že ľudia tieto vlastnosti nepoznajú a niektoré rastú len na planétach v galaxii, kde je Wymyslensko, mám obe učebnice a na prax je moje obydlie vybavené... Staropisy sa u ľudí nepreberajú predpokladám, takže tam budeš potrebovať všetko učivo. Boj..."$ $ Rosa Chen prerušila.

"$ $Boj? Vy sa vážne učíte bojovať?"$ $ Chen prikývla.

"$ $Súčasná doba to vyžaduje, vieš bojovať?"$ $ Rosa neodpovedala.

"$ $Od ktorého ročníku sa to učíte?“

"$ $Od prvého, ale to je skôr oboznamovanie zo zbraňami. V treťom sa začína skutočný boj, v ňom sú povolené palice, tupé meče, obojručné dýky a normálne dýky.“

"$ $Ale vy vážne bojujete?“

"$ $Hovorím, že je to potrebné.“

"$ $To si neublížite? A prečo používate také zastarané zbrane. Ľudia používajú pištole!"$ $ Zvolala Rosa.

"$ $Je verejným tajomstvom, že firma KKKv tvoria dominantný monopol na trhu zo zbraňami. Marone Tlogenová sa nevzdá svojho zisku ,ani za cenu porážky.‘"$ $ Odfrkla Chen. Rosa blúdila očami.

"$ $Vy... chcem povedať ty máš pištoľ. Netvrdila si...“

"$ $Dá sa to kúpiť aj inde ako v KKKv. Ale sú vekové obmedzenia.“

"$ $Aké?“

"$ $Do dvanástich rokov sú dovolené len tupé meče, palice a dýky. Od dvanástich do osemnástich rokov sú dovolené prvá séria aj kosák, bič a obojručná dýka. Od osemnástich rokov sú dovolené pištole, ale len pre skupinu istých.. ehm ľudí. A tiež poriadne meče a solasprístroje. Ale neplatí toto všade. V Portugalsku od volebného víťazstva anarchistov, ktorí odmietli parlamentný plat a zrušili všetky zákony, tak je tam dovolené kupovať všetko, ale napriek tomu...“

"$ $V Portugalsku je tiež vaše Spoločenstvo?“

"$ $Samozrejme, nie je krajina v ktorej by nebola. Sme federácia zeme. Každý zo štátov v spoločenstve M má iný politický systém, len sú štyri zbierky zákonov platné pre každý štát a jeden federálny parlament s dvoma poslancami za každý štát. Okrem Severnej Kórei a vojnových krajín. Odtiaľ sme ušli.“

"$ $Vy ste po celej zemi? Po celej zemi žijú čarodejníci?“

"$ $Samozrejme, už od praveku. Lenže v stredoveku sa prenasledovanie začalo vyostrovať a boli sme nútení ísť do Wymyslenska, ale to si prečítaj v Histórii. Literatúra nie je len pozemská a takže by som odporučila prečítať niekoľko zväzkov o Literatúre vo Wymyslensku a spoločenstvu M. Ale nezabudni na tú Fentenzíjsku, tá je naozaj prekrásna.“

"$ $Rozumiem."$ $ Rosa si trochu vzdychla. Vedela, že ju čaká veľa učenia. Chcela si to uľahčiť, a tak sa spýtala. "$ $Mohla by som si teraz zobrať nejakú tú knihu?“

"$ $Mám v čítačke niekoľko kníh, požičiam ti je len mi ju nepoškoď!“

"$ $Neboj sa,"$ $ Vzala podávanú čítačku. "$ $Na nič iné sa prosím ťa nepozeraj, dôverujem ti."$ $ Rosa zas prikývla a už bola zahĺbená do čítania.

\begin{center}
*
\end{center}

Morja sa porozhliadla. Šuchot lístia nepočúvala, bola zahĺbená do niečoho iného. Na kameňoch a terasovitých záhradách v starom inkskom meste neboli pre ňu kamene také jednotvárne. Pobyt vo víre ju zmenil.


\chapter{Povedané a nevypočuté}

"$ $Tarny ideš tam ty, ja nemám preukaz a Pauline tiež,"$ $ rázne povedala Tulienka Deľa a strčila Tarnymu do rúk dlhý zoznam s položkami čo mal z kníhkupectva doniesť. "$ $Vieš po česky, anglicky aj Wymyslensky, nemáš dôvod tam neisť. Zato my áno."$ $ Tarny pochopil, že Tulienka Deľa je neoblomná a zrejme má aj pravdu a vstal z lavičky.

"$ $To vážne chceš aj Preklad piktopísma známeho?"$ $ Pobavene sa spýtal Tulienky Deli.

"$ $Všetko sa nám môže hodiť!"$ $ Takmer hystericky skríkla.

"$ $A odkiaľ zoberieš peniaze? Ja mám maximálne na pár z nich a to nezabúdaj, že ďalšie v najbližšej dobe nedostaneme.“

"$ $Kľud, mám v banke účet, v Spoločenstve M, už dlhšie som čakala, že ho budem potrebovať. A tu pri sebe päťsto mačičiek."$ $ Tarny bol stále tak trochu pobavený aj keď si uvedomoval vážnosť situácie. Pauline si prezerala dlhý zoznam kníh, ktorý sa im podľa Tulienky Deli budú hodiť.

"$ $Naozaj potrebujem všetky? Aj Život a smrť Arabely Tlogenovej a Predtým ako sa stanete duchom?“

"$ $Už dlhšie som si ich chcela prečítať. Lenže vo Wymyslensku sa moc zohnať nedajú."$ $ Priznala s trochou červene na sivých lícach.

"$ $Takže tie môžeme vyškrtnúť."$ $ Skonštatovala Pauline.

"$ $Dobre, súhlasím,"$ $ tak trochu neochotne súhlasila Tulienka Deľa.

"$ $A načo sú nám Poklady pozemských dejín, Obranné zručnosti v teórii a praxi a Herbár protijedov a použitie?“

"$ $Po prvé,"$ $ vysvetľovala hystericky Tulienka Deľa. "$ $Tie truhlice, síce nemáme potuchy, o čo tam vôbec ide, ale niečo zistiť predsa môžeme. Po druhé, obrana sa nám zíde, v nej si síce najlepší ty, ale Pauline predpokladám veľa vecí neovláda, ak ju k tomu Deoque nedonútila a... chápeš? A po tretie, nevieš čo sa stane. Protijedy sú potrebné vždy. V herbári sú aj malé množstvá a tak...“

"$ $A ideme zachraňovať svet,"$ $ skonštatoval Tarny.

"$ $Škoda, že tu nie je Sylvia."$ $ Uškrnula sa Tulienka Deľa

 "$ $Ani náhodou, tá najskôr zvrhne Wymyslenskú vládu.“

"$ $O kom tu hovoríte? Kto je Sylvia?"$ $ Prenikla do rozhovoru Pauline.

"$ $Kamarátka, démonka, ako ty a..“

"$ $Mýliš sa,"$ $ prerušil ju Tarny. "$ $Pauline je polodémonka. Sylvia je démonka ako D.“

"$ $Nehovor to nahlas, keby sa niekto dozvedel, kto je, zabila by ma... teda skôr seba, ale nič z toho nechceme.“

"$ $Nikto nás nepočuje. Sme chránení zmyslovým kúzlom.“

"$ $Tie možno zničiť.“

"$ $Viem, ale pokračujme. S Tulienkou Deľou tvoria disidentskú skupinu.“

"$ $Disidentskú?“

"$ $Vo Wymyslensku je disidentom , kto je ochotný povedať, že je načase zvoliť niekoho iného."$ $ Vysvetlil Tarny.

"$ $Sylvia ma prehovorila. Na internete uverejňujeme nejaké články, ale samozrejme z inej IP adresy, keby nás odhalili neviem čo by sa stalo. Nedávno sa jedna skupina odhodlala nahlas povedať, že proti D sa treba spojiť so spoločenstvom a následne boli obvinení z úplatkov a nejasného financovania a odsúdení na päťdesiat rokov.“

"$ $Päťdesiat rokov!? Až!"$ $ Nemohla veriť Pauline.

"$ $Tresty sú vo Wymyslensku aj vo Spoločenstve M väčšie ako na zemi, žijeme predsa len, priemerne dlhšie, a v spoločenstve dokonca večne. Tak dvadsať rokov je relatívne nič.

"$ $Nuž, a Sylvia chce prevrat. Doslova. Revolúciu. Asi tri roky nepretržite plánuje. Mohlo by to byť aj kratšie, ale Sylvia odmieta využívať svoje démonské schopnosti. Hovorí, že by ju to pripravilo o dobrodružstvo... a skrátka nechce. Ona je mierne...“

"$ $Povedzme si, psychicky nevyrovnaná s tým, kým je, a skrátka Sylvia by sa do ničoho nehrnula keby to nebolo nebezpečné a neisté,"$ $ zhrnula Tulienka Deľa.

"$ $Ste fascinujúci."$ $ Povedala so smiechom Pauline.

"$ $A teraz Tarny, môžeš ísť kúpiť knihy a Pauline zatiaľ za nejaké eurá môže kúpiť zmrzlinu.“

"$ $Tu sú české koruny,"$ $ Upozornil ju Tarny.

"$ $Pôjdem zmeniť.“

"$ $A prečo ja? Prečo pre zmenu nemôžeš ty?“

"$ $Lebo ja vyzerám, veď vieš ako! Ľudia na to nie sú zvyknutý!“

"$ $Ale tie vaše zmyslové kúzla!"$ $ Protestovala Pauline.

"$ $Mohla by som to urobiť, ale senzory by ma prezradili, to by som musela mať nejaké špeciálne oblečenie. Ako kór. Dajú sa oklamať nejakým kúzlom, ale keby tu bol niekto zo spoločenstva M, mohol by pre seba zrušiť zmyslové kúzlo a keby to bol niekto z agentov (agent je v spoločenstve aj Wymyslensku označenie pre políciu, vojsko aj význam aký majú agenti u nás), chcel by odo mňa preukaz. A vieš, že nemám.“

"$ $Prepáč, tak idem ja.“

"$ $Odneviditeľním ťa, keď pôjdeme zmeniť peniaze,"$ $ dodala.

"$ $Budú mi rozumieť?“

"$ $Samozrejme! Myslím, že sú tu zvyknutí na turistov. Predsa len je to Praha.“

"$ $Je to menšie ako Chicago. A predsa hlavné mesto."$ $ Pobavene povedala Pauline, keď si prezerala mapu.

"$ $A... to je vážne hlavné mesto?"$ $ Pýtala sa.

"$ $Ty nemáš znalosti z geografie?“

"$ $Niečo som započula, ale predsa, je to také malé mestečko."$ $ Zasmiala sa. "$ $Česká republika nie je až taká dôležitá, aby som ju vedela. Aspoň to som si vtedy myslela.“

"$ $Budeš sa musieť ešte veľa učiť."$ $ Usmiala sa Tulienka Deľa a dodala. "$ $Keď chceme utekať pred spoločenstvom, D a Wymyslenskom.“

"$ $To určite."$ $ Vyprskla. "$ $Tak kde je tá zmenáreň? Máš doláre?"$ $ Spýtala sa keď uvidela, že zmenáreň mení aj doláre na české koruny.

"$ $Tri alebo štyri. Jegrigsen, chcem povedať otec, mi toho veľa nedával. Teraz. Keď už toho ve... teda viac viem, rozmýšľam, prečo ma nezabil. Nebolo by to preňho jednoduchšie?“

"$ $To je zaujímavé. Ak skutočne Goonová proroctiev, tak D v tebe vidí nebezpečenstvo.“

"$ $D je...“

"$ $Nepriateľ. Už sme to hovorili.“

"$ $Jasné.“

Tarny prišiel až o hodnú chvíľu. Tulienka Deľa mu ponúkla čokoládovú zmrzlinu a listovala v novinách.

"$ $Prečo si kúpil Community M News a nie New News?“

"$ $New News mali len po anglicky a Community M News bol anglicky, česky a Wymyslensky a bol elektronický. Ak teraz môžeme mať správ viac."$ $ Tulienka ho už nepočúvala a listovala v čítačke.

"$ $D zas zaútočil v Stonehenge, to vybuchlo atómovým výbuchom, asi v tom má prsty Jean Parvasîe, Jean Parvasîe má asi klon, informácie overujeme.“

"$ $Jean Parvasîe?"$ $ Prerušila ju Pauline.

"$ $Tu máš čítaj."$ $ Povedala Tulienka Deľa a strčila jej do ruky čítačku. 

Jean Parvasîe, služobník D, vyhľadávaný agentúrou, jeden z najmocnejších mágov 20. A 21. Storočia má podľa našich informácií klon. Veľa sa o ňom nevie, ale je podozrenie, že jeho klon, je páchateľom zmiznutia klonovacej a reprodukčnej kliniky v severnom francúzsku a mesta Lowwer. Podľa toho čo na tlačovej konferencii povedal médiám riaditeľ agentúry Rollius Tlogen, Jeanov klon zrejme nespolupracuje s D.

Predbežne aktualizujeme

Pozrite aj:

Kto je Jean Parvasîe

Mesto pri Londýne je preč, našla sa veľká dávka mágie

D zas zaútočil v Stonehenge, to vybuchlo atómovým výbuchom, asi v tom má prsty Jean Parvasîe

"$ $To je strašné.“

"$ $Súhlasím. A toto je a k tomu aj diplomatická a studená vojna s Wymyslenskom.“

"$ $Wymyslensko jej jediný štát na tej planéte?“

"$ $Nie, ale je najväčší a najdôležitejší. Ďalšie sú Teresovo, to riadi Solem Krutý, je vlastne jedna z predĺžených rúk D a tiež Tramtária, je to anarchistická samospráva s jedným diplomatickým reprezentantom, ale uzavretými hranicami, takže sa tam nedá dostať, len odtiaľ.“

"$ $O tých sa akosi nehovorí...“

"$ $Maximálne o Teresove. Tramtária je uzavretá a ako hovorím, nedá sa tam len tak dostať, pretože sa obávajú, že sa naruší ich pokoj.“

"$ $Aj im celkom rozumiem."$ $ Vyhlásil Tarny mieriac očami na titulky.

"$ $Kde pôjdeme Tarny?“

"$ $Niečo najskôr zistíme. Ale nie tu,"$ $ ukázal na ulicu. "$ $Neviete kto nás sleduje, predsa len, Wymyslenčanku, mňa a polodémonku s kopou kníh nevidíte často.“

"$ $Hlavne preto, že svet je veľký a ty si len jeden."$ $ Ironicky Pauline poznamenala a Tulienka Deľa sa zasmiala.

"$ $Hej, zase máš pravdu Tarny, mali by sme ísť do mačičkoesu, ale aj tak je to nelegálne.“

"$ $Viem. A tiež o jednom mieste kde by sme mali byť bez preukazu nestíhaní.“

"$ $Kde?"$ $ Vytreštila naňho oči Tulienka Deľa, keďže vedela, že panika okolo D je veľká, a tak na každom kúte zeme boli agenti, ktorí boli ochotní zatknúť každého za to, že bol tam kde nemal a nemal v ruke kúsok plastu na ktorom boli v čipoch údaje o ňom.

"$ $V Portugalsku."$ $ Povedal úplne vážne. "$ $Tam je anarchia.“

"$ $Áno, a preto je tam sloboda. Nekontrolujú tam preukazy, mačičkoesi...“

"$ $A čo D?“

"$ $To je jediná daň za to. Ale ľudia si vedia poradiť sami. Pár ich policajtov dohliada na bezpečnosť, ale to je len tak, že ak niekto videl D, tak má zavolať. A oni prídu.“

"$ $To vieš odkiaľ?“

"$ $Písal som si z jedným dievčaťom odtiaľ pred rokom, ako medzištátna pošta.“

"$ $Ty to chceš risknúť?“

"$ $Samozrejme, inde nenájdeme pokoj pre vieš prečo.“

"$ $Prehovoril si ma."$ $ Zašomrala Tulienka Deľa.

"$ $Vieš po portugalsky?"$ $ Spýtal sa Tarny.

"$ $Veď si hovoril...“

"$ $Písali sme po anglicky.“

"$ $Vidíš, po portugalsky viem, ale po anglicky sa dohovoríme. Pauline, bola si niekedy v Portugalsku?“

"$ $Nie.“

"$ $A videla si už Portugalsko?“

"$ $Pár obrázkov.“

"$ $Z akej perspektívy?“

"$ $Čo?“

"$ $Vtáčej, alebo ako to vidíš. Aby sme sa nepresunuli do vzduchu.“

"$ $Jeden bol normálny. Lisabon.“

"$ $Dobre. Takže my sa s Tulienkou Deľou zmeníme pomocou zmyslového kúzla na ľudí a potom sa tam premiestnime.“

"$ $Čakám.“

\begin{center}

*

\end{center}

Sylvia študovala plány čo mala otvorené v počítači. Neuložila ich. Aby jej tam niekto neprenikol.

"$ $Už je noc, mali by sme ísť spať, zajtra je škola.“

"$ $Škola? Ja tu nejdem...“

"$ $Ak chceš, aby sa všetko vydarilo tak choď. Už len to, že chýba Tulienka Deľa, bude čudné. Musíš sa tváriť, že o ničom nevieš. Inak sa všetko skazí.“

"$ $V poriadku..."$ $ Zašomrala Loviisa.

"$ $Mám rozvrh niekde tu..."$ $ Sylvia sa prehrabávala v svojich veciach

"$ $Zajtra máme čarovanie, vlastnosti a použitie látok, Wymyslenčinu, matiku, astronómiu, etiketu, ja ju neznášam. A nie, zajtra nám odpadávajú posledné štyri hodiny a máme prednášku od Cecílie. Vyučovanie sa začína o pol ôsmej.“

"$ $Vy máte desať hodín?“

"$ $Hlagenova teória telepatie to umožňuje. Takže nám často odpadnú aj dve hodiny... á..."$ $ Zívla.

"$ $A učebnice?“

"$ $To ti dajú. Len prosím ťa. V škole sa pri mne moc nezdržuj. Cecília mi moc nedôveruje, a myslím, že ma tak trochu podozrieva. Nech nezistí veď vieš čo.“

"$ $Medzi sebou to môžeme volať... napríklad..."$ $ Premýšľala a zívala Loviisa.

"$ $Sklo?“

"$ $Nie, to je moc nápadné, Loviisa. Lepšie buď niečo vôbec nesúvisiace ako domáca úloha.“

"$ $Dobrý nápad. Tak dobrú noc."$ $ Povedala a pobrala sa na svoju izbu ktorú ani neotvorila.

\begin{center}

*

\end{center}

"$ $Vy ste zas vrátili do agentúry Niela?"$ $ Spýtala sa jej tmavovlasá zovero (obyvatelia Wymyslenska, okrem ľudí a dragonov).

"$ $Nie, len teraz robím len pre Izabetu. Potrebujem sa dostať ku záznamom.“

"$ $Máš povolenie?"$ $ Pýtala sa už formálnejšie.

"$ $Áno,"$ $ podávala jej ten istý, ktorý ukazovala v inštitúte zbraní.

"$ $Preukaz."$ $ Niela vytiahla identifikačnú kartu a agentka jej ju preskenovala a rýchlo prečítala to čo sa jej objavilo na počítači. "$ $V poriadku. Ale čo sa týka knihy...“

"$ $Mám povolenie.“

"$ $Ste si istá? Kniha je... a okrem toho tá dohoda s Wymyslenskom...“

"$ $Ja viem. Preto som tu prišla tak unáhlene. Nemôžem si dovoliť pol roka čakať.“

"$ $Samozrejme. Ale predpisy sú nezmenené odvtedy ako...“

"$ $...som tu pracovala."$ $ Dokončila Niela. Nebolo ton žiadne veľké prekvapenie. V spoločenstve M sa nemenilo takmer nič a agentúra bola na tom rovnako.

"$ $Žiadne fotografie, skeny, ani nič také.“

"$ $Čítala som pravidlá, Hynia Zergová.“

"$ $Prepáčte, že vám nedôverujem."$ $ Odvetila chladne a vrátila jej preukaz spolu s kartou na ktorej boli len dva slová. POVOLENÝ VSTUP.

Čierne steny v trezoroch agentúry už Niela dlho nevidela. Ale pamätala si ich stále dobre. Síce bola agentkou v teréne, ale veľa prác robila aj tu.

"$ $Mohli by ste prosím odísť?"$ $ Spýtala sa zdvorilo dvoch agentov čo ju sprevádzali, keď došli ku knihe.

"$ $Sú tu kamery."$ $ Podotkla Niela pri pohľade na ich zamračené tváre.

"$ $Tie môžeš oklamať.“

"$ $Tak isto sa dajú oklamať aj zmysly.“

"$ $Tie sa dajú odčarovať.“

"$ $Ak máš silné kúzlo, tak je to komplikovanejšie ako Teleport, a to viem o čom hovorím. Kamera sa nedá oklamať zmyslovými kúzlami. Vidí to čo robím.“

"$ $Ale dá sa to.“

"$ $Čo myslíte, dala by mi Izabeta povolenie keby mi nedôverovala.“

"$ $My sme v agentúre. Mám právo vedieť čo robíš.“

"$ $Toto je vec medzi Izabetou a mnou."$ $ A vzápätí dodala. "$ $A Morjou Tlogenovou."$ $ Niekoľkí znechutene vyprskli.

"$ $Tou zradkyňou?! Zradila svoje meno, rodinu a česť a aj Spoločenstvo M.“

"$ $Morja nikdy nezradila spoločenstvo!"$ $ Nahnevane vykríkla Niela.

"$ $A čo Jegrigsen?!“

"$ $Morja sama priznala, že to bola chyba, že si nevšimla, že Jegrigsena ovládal D.“

"$ $Kedy? Je stratená, Niela, odišla ako Messia.“

"$ $Morja sa dnes so mnou stretla. Ale zmizla. Na Stonehenge.“

"$ $Načo máme vynakladať naše informácie na nájdenie tej zradkyne!? Nie si ani v agentúre!“

"$ $Práve ste povedali dôvod prečo som túto prácu zanechala. Viac sa budete starať o česť, ako o bezpečnosť.“

"$ $My zachovávame naše staré hodnoty.“

"$ $Ako nenávisť k tkz. zradcom. Ako nedodržiavanie toho, na čo vznikla ušľachtilá myšlienka agentúry.“

"$ $Ako sa opovažuješ...“

"$ $Nečítate správy Bellona? Alebo čítate len tie v ktorých ste chválení, Pani Sarlová."$ $ Tá sa na ňu zamračila. "$ $Asi nepoznáte príslovie poznaj svojho nepriateľa. S obľubou čítam Wymyslenské noviny, o Cecíliiných plánoch je tam viac, než dosť. A raz za čas sa tam zjaví aj pravda.“

"$ $Prečo by sa to malo...“

"$ $Prečo agentúra nepohla ani prstom, keď Jean Parvasîe vyvraždil sám celú Wymyslenskú rodinu, čo tu bola na návšteve u príbuzných, keď tam boli na uliciach celé jednotky agentúry?“

"$ $Sú to Wymyslenčania, ľudia by nás odsudzovali.“

"$ $Niekedy je potrebné nerobiť populistické činy.“

"$ $Sú to nepriatelia!!!"$ $ Až zrevala Bellona Sarlová.

"$ $Viem, ale to je politika a každý život má cenu. Bez ohľadu na to, čo hovoria vodcovia a verejnosť. To je jeden z dôvodov, prečo som z agentúry odišla.“

"$ $Mňa to nezaujíma! Hovoríš nebezpečné slová Niela a nezabúdaj, že ja ťa môžem zatknúť!"$ $ Kričala Bellona a ostatní sa len mračili a Niela tiež, lenže neprikyvovala.

"$ $Teraz ste ako Wymyslensko, aj vy chcete za kritické slovo zatknúť? Znížili ste sa na úroveň Wymyslenskej tajnej polície.“

"$ $Si verejným nebezpečenstvom, v mene výnimočnej situácie ťa zatýkam a beriem ti oprávnenie, v zmysle zákona číslo dvadsaťtri o právomoci agentov vo výnimočných situáciách, zväzok dva."$ $ Niela ostávala pokojná, čo sa o Belone nedalo povedať.

"$ $Ak sa nemýlim, nie je vyhlásená mimoriadna situácia, takže právomoci, ktoré si práve vymenovala nemáš.“

"$ $Nikoho nebude zaujímať, že sa nenájde Morja Tlogenová, ktorá si svoje meno nezaslúži nosiť a jej dieťa! Zradkyňa jedna! A tiež ty, každý kto sa stýka so zradcom by mal byť označený za zradcu!“

"$ $Konáš za chrbtom Izabety, Bellona Sarlová?“

"$ $Morja urazila česť a dobré meno a konala proti tradícii! Je to obyčajná... A ty tiež! Nie si už dôležitá Niela Orbielová! O svoje právomoci si prišla keď si odišla z agentúry! Nie si nič! Akýkoľvek kus techniky tu, má väčšiu cenu ako ty!"$ $ Penila.

"$ $Neodpovedala ste, Bellona! Vy čo tak trváte na zákonoch. Prichádzam od Izabety Tlogenovej a mám poverenie.“

"$ $Určite si sa stretla s démonom, čo na seba vzal jej podobu! A určite to bola tá prašivá zradkyňa Morja, ktorá sa spojila s D a stala sa polodémonkou. Zatýkam ťa.“

"$ $V tom prípade, ak mieniš sama porušovať tvoje posvätné zákony, tak sa hodlám brániť!"$ $ Niela bleskurýchle vytiahla pištoľ s uspávacím dlhodobým efektom a namierila. Bola nabitá.

"$ $Ak sa pohneš ku mne, strelím. Nechcem vás zabiť, odíď Bellona a neposielaj žiadne posily ani nič také, inak budem donútená vytiahnuť Laserovú pištoľ, a tá má smrteľné následky."$ $ Bellonina tvár sa skrivila do šialeného úškrnu.

"$ $O svoje právomoci ste prišli, keď ste odišli z agentúry, vtedy ste prišli o všetky výhody, čo by ste mohli mať, teraz ste len obyčajná radová občianka, a ja sa postarám, aby ste prišli aj o svoje posledné práva. Prečo by ste mali mať právo vedieť veci, čo prislúchajú nám, naše informácie!? Prišli ste o veľa keď ste odišli. V mene..."$ $ nestihla dopovedať. Niela predtým ako vystrelila povedala len jednu vetu.

"$ $To vy ste prišli o veľa."$ $ Nanočastice v guľke nestihla Bellona mágiou zastaviť, zasiahli ju a padla na zem omráčená. Ostatní agenti, akoby si až teraz začali všímať to čo sa deje, vytiahli, každý po dvoch energetických laserových pištoliach, a akoby rozhodnutí Nielu zabiť naraz vystrelili. Niela nečakala a šmahom ruky urobila okolo seba neviditeľnú stenu do ktorej narazila energia a posilnila stenu.

"$ $To ako..."$ $ Vyšlo z jednej.

"$ $Ako som povedala, od toho ako som odišla z agentúry, mala som menej byrokratických obmedzení a zákonných lehôt. Prepáčte, že to musím urobiť, ale sľubujem, že si nič nebudete pamätať."$ $ Po týchto slovách stena vrazila do nich a oni padli na zem, rovnako ako Bellona. "$ $Moja japonská učiteľka mágie mala pravdu, silné to teda je."$ $ Poznamenala a otvorila veľkú knihu proroctiev.

\begin{center}

*

\end{center}

"$ $Je tu celkom pekný hotel."$ $ Ukázal na bielu budovu.

"$ $Zbláznil si sa Tarny?“

"$ $Tu to nik nebude kontrolovať. Inak to skúsim zo zmyslovou mágiou.“

"$ $Hádam sa ti to podarí."$ $ Skôr než Tulienka Deľa stihla odvetiť, urobila to za ňu Pauline.

"$ $Máš peniaze? Keď sme toľko minuli na knihy..."$ $ Pohľadom zastavil na kopu v batohu Tulienky Deli.

"$ $Mám na kreditke minimálne dvesto zlatniakov. To by malo stačiť.“

Hotel bol biela, ale matná budova ešte z dvadsiateho storočia, bol neviditeľný pre ľudí, ako každý správny hotel, či len budova v spoločenstve M. Samozrejme, vždy tu hrozilo riziko, že doň ľudia narazia, ale to sa dalo riešiť tabuľami nevstupovať. Za múrmi neboli strážnici, tých portugalská nevláda nemala povinne, a teda si majitelia nenechali strácať na nich zlatniaky. Tarny, Tulienka Deľa a Pauline prešli bez kontroly ku recepcii kde síce od nich preukazy žiadali, ale preukazy čo im začaroval zmyslovou mágiu Tarny, dokonale oklamali recepčného.

"$ $Izba 245, jedlo nie je v cene."$ $ Odvetil im na platbu kreditkou, čo vybavila Tulienka Deľa. Keď vošli do izby Tarny objavil, že je začarovaná cez Hlagenovu teóriu telepatie.

"$ $No to je super... hlagenovka a za takú cenu..."$ $ Usmial sa.

"$ $Ale... veď zober si verejnú dopravu vo Wymyslensku... Aj tam je Hlagenova teória telepatie...“

"$ $Hlagenova teória telepatie?"$ $ Začudovala sa Pauline. Vedela síce, čo je telepatia, ale čakala, že to bude iba nejaká základná, ako jej telepatizoval Tarny.

"$ $Vynašiel ju istý Hlagenov inštitút vo Felanzii, vo Wymyslensku a je to mimoriadne praktická telepatia, tu vysielaš silné myšlienky a telepatický prijímač myšlienky rozoznáva, vyhodnocuje a vzápätí to odosiela do priestoru kde sa mení priestor v dosahu prijímača, na tvoj myslený priestor."$ $ Odrecitovala Tulienka Deľa práve keď si zložila knihy na stôl a jednu z nich, Preklad piktopísma známeho, si otvorila.

"$ $Ale veď nie je náhodou normálna telepatia...“

"$ $Tá existuje a je veľmi praktická, lenže sa ťažko prenáša cez veľké vzdialenosti, a strašne z toho bolí hlava, raz sme to s Tulienkou Deľou skúšali a nemohol som sa celý deň sústrediť. Preto sa skôr využívajú telepatióny."$ $ Tarny si sadol, tiež vzal jednu knihu, Možnosti zmyslu a začítal sa. Pauline sa k nim tiež pripojila a vzala si obrovskú knihu Mágia obranná a útočná. Vtedy sa zdalo, že je všetko v úplnom poriadku, keď zrazu Tulienka Deľa spozornela a hľadela na okno.

"$ $Stalo sa niečo Tulienka Deľa?"$ $ Spýtal sa Tarny.

"$ $Zdá sa mi niečo... zrejme zmyslové kúzlo. Prosím ťa, ak tam je, zruš ho."$ $ Tarny podišiel k oknu. Zahľadel sa na ulicu a otvoril ho. Videl to. Slabý mihot, ako fatamorgána.

"$ $Veľmi silné zmyslové kúzlo, mali by sme zmiznúť.“

"$ $Myslíš D?“

"$ $Neviem. Silné zmyslové kúzlo, neviem ho prelomiť... Zbaľ knihy, Pauline, musíme zmiznúť, zneviditeľním nás a ty sa Pauline, niekde premiestniš.“

"$ $Kde?“

"$ $Prvé miesto čo ťa napadne, ale aby to nebolo vo vzduchu a napríklad vo vesmíre.“

"$ $Myslíš, že by ma napadlo premiestniť sa do vesmíru?"$ $ Neveriacky pokrútila hlavou.

"$ $Rýchlo,"$ $ sykla Tulienka Deľa a batoh si dala kúzlom na chrbát.

"$ $Premiestime sa, chyťte sa ma."$ $ Naliehala Pauline a myslela. Počuli výkriky a pády na zem.

"$ $Pohni sa!"$ $ Naliehal Tarny a chytil Pauline za ruku. V chvíli keď Jean Parvasîe vyrazil dvere a chcel vyhodiť budovu do vzduchu, zmizli a miestnosť bola prázdna.

\begin{center}

*

\end{center}

Runy. Nie, piktopísmo. Zmiatlo ju to. Inkovia predsa mali uzlíkové písmo a trestali čarodejníctvo. Ako tam mohlo byť piktopísmo? Vedomosti z víru mala v svojej hlave pevne zakorenené. Staroveké piktopísmo. Spoznala ho a čítala.

Ak poznáš tajomstvo prachu, spoznáš tajomstvo vašej ilúzie minula.

Jedna veta bola vyrytá na všetkých stenách a stále dokola. "$ $Ak poznáš tajomstvo prachu... čo to je?"$ $ Premýšľala Morja. "$ $Bude to niečo zo starou mágiou. A s minulosťou."$ $ Morja rozmýšľala. Zrazu ju napadla málo pravdepodobná myšlienka. Keď mala kedysi, kvôli jednej práci v agentúre, prístup do archívu a tam prečítala jeden list... ako sa to len volal...? List Ló Ölverovej Zacaríasovi. Korešpondencia dvoch veľkých filozofov. "$ $Áno, to bude to."$ $ Morja si snažila spomenúť na obsah listu. Pátrala v pamäti. "$ $Podľa mojich myslení dnešných,"$ $ spomínala na začiatok listu. "$ $Ak je čas taký ako vo víre, ak sa niečo stalo, všetko má následky zanechané, tak nejako."$ $ Niečo sa tam hovorilo o prachu, ako v nápisoch. "$ $A zem je tá, ktorá tam bola ak,"$ $ Morja už chápala, je to kúzlo čo dokáže vyvolať niečo čo bolo z niečoho čo je, ale len ako obraz. "$ $Ale do obrazu sa už zasiahnuť nedá,"$ $ ako vraví Ló. Otázka bola do akej doby, či situácie chcela Morja vidieť, a keby, ako by zase zistila ktorá zem je z toho času a musela by jej získať dostatočne veľa, pretože ak by získala zem iba z jednej časti, tak by videla iba minulosť tej časti a to Morja vedela. "$ $Štvorec je dokonalou kockou,"$ $ Spomenula si zrazu na jednu z posledných častí listu. Kocka ktorá bude urobená z bodov zo zeme, kocka má... "$ $Kocka má osem vrcholov a musím ju vytvoriť kúzlom, ale to sa..."$ $ Vtom ju to napadlo, mimoriadne komplikované kúzlo, osobne Morja poznala piatich ľudí, ktorí to dokázali, Izabetu, Arabelu, Rolliusa, Mariona a istú agentku, na ktorej meno si nespomínala. Ona sa o to nikdy nepokúšala. Vždy jej vystačovali zbrane a troch pokročilejšia mágia. Rozhodla sa, že to zvládne, nie že sa o to pokúsi. Lebo ona vedela, že Morju Tlogenovú, ktorá sa raz rozhodla, a bola pevne presvedčená, sa v skutočnosti zastaviť nedá.

Stál tam D, a tiež tam bola Deoque. A tiež bytosť, ktorá nebola človekom, ani zvieraťom, ani zoverom. Morja to len pozorovala, pretože nemohla zasiahnuť. Bol to iba obraz minula. Neprekvapovalo ju, že vidí Dé, ale to, že je tam Deoque, ju naozaj prekvapilo, Morja vedela, že Deoque D nemala moc v láske, presnejšie – nenávidela ho. Ale možno to bolo od... Pozerala ďalej.

Deoque prehovorila.

"$ $Podal si sľub D, a ty to vieš, vieš to."$ $ Bytosť spievala akoby nevnímala okolitý svet nejakú pieseň a vo vzduchu sa tvoril pergamen. Dé sa iba zasmial a Deoque priam zrevala.

"$ $Ak tvoje sľuby nemajú platnosť, tak vedz, že moje majú Démon!"$ $ V ruke zvierala paličku, ktorá bola pôvodne hnedá, ale teraz bola bledá, a vnútri bolo jasne vidieť červenú energiou čo v nej pulzovala. Bytosť spievala a pergamen bol celý, Deoque vrieskala.

"$ $Preklínam ťa Démon!"$ $ Palička sa vzniesla k nemu a zasiahla ho do čela. Nerozpadla sa, hoci na to vyzerala. D sa natiahol po pergamene a Deoque pokračovala. "$ $Preklínam ťa Démon, do troch rokov od vytvorenia tretieho polodémona, zomrieš, Démon! A preklínam ťa, tvojou smrťou zaniknú tvoje Sugero!"$ $ Palička explodovala, ale Démonovi neublížila – nemohla. Ale niečo predsa. Znak, ktorý bol na paličke, znak dvoch polovíc lichobežníka sa mu vyryli na kožu. Bol prekliaty. Skôr než sa premiestnil, mu Deoque zvlášť silnou mágiou, ktorá ublížila i démonovi, rozrezala ruku. 

Koniec videného obrazu. Odtiaľ mal teda D jazvu, bol prekliaty, a koľko polodémonov vytvoril. Deoque hovorila o Sugero. Skrsla v nej nádej, ktorá sa netýkala jej, ani jej dcéry ani D. Týkala sa Jegrigsena Goona. Niekoho koho nenávidela. A napriek tomu... teraz by sa najradšej zbila, pretože ho stále, podvedome mala rada, i cez všetko... prečo? Nadávala sama sebe. Je hlúpa. Stále ho... Nie, nebude na to myslieť... ale... Bojovala sama zo sebou... Človeka, o ktorom bol presvedčená, že ju odporne zradil, a využil. A teraz... možno bol obeťou rovnako ako ona... sakra! Zas si vynadala. Nemôže si ho idealizovať. Teraz hlavne by mala povedať o tom Sugere spoločenstvu M. Rozmýšľala koho má osloviť. Všetci odsudzovali. Ale možno nie. Možno jeden, jedna. Ak nerátala Nielu.


\chapter{Staronová rodina}

Objavili sa na nejakom neznámom mieste, ktoré nepoznal ani jeden z nich.

"$ $Kde sme to?"$ $ Pauline sa zmätene obzerala okolo seba.

"$ $Pauline, ty si nás premiestnila na inú planétu. A dokonca sa tu dá dýchať! Ako... ja som ti hovoril, že nie do vesmíru!

"$ $Toto som nikde nevidela! Naozaj!“

"$ $Skutočne?"$ $ Tarny skúmal čudné prostredie v ktorom sa ocitli. "$ $Takéto rastliny som nikde nevidel, vyzerajú ako kryštáliky.“

"$ $Nedotýkaj sa ich!"$ $ Vykríkol niekto, koho hlas, na svoje veľké prekvapenie poznal. Neveriacky sa obzrel a zbadal svoju matku, ktorá bola niekoľko rokov nezvestná, po nešťastnom boji s D.

"$ $Ako to..."$ $ Bol zmätený. Za Bellou Lietavou išli kryštalické bytosti, absolútne nepodobné tým zo zeme.

"$ $Tarny? Ako ste sa sem..."$ $ Aj ona bola viditeľne prekvapená. Bytosti sa podľa niečo rozsvecovali a zhasínali. Zdalo sa, že Bella Lietavá tomu rozumela lebo im niečo hovorila. Objavila sa za nimi ľudská žena.

"$ $Niečo sa stalo Bella?“

"$ $Ľudia, Leana z Ölverína! Ďalší."$ $ Tá žena, ktorá bola oslovená ako Leana z Ölverína sa usmiala.

"$ $Stalo sa niečo?"$ $ Znepokojene sa spýtala Bella, akoby usmievať sa bola zlá vec. Leana sa stále usmievala.

"$ $Zdá sa, že planéta na druhom konci vesmíru je vítanou, doživotnou destináciou ľudí z Wymyslenska a zo zeme.“

"$ $Ale my tu nechceme ostať doživotne.“

"$ $Ešte sa nevymyslela raketa čo by prešla za jeden život tri miliardy svetelných rokov!“

"$ $Ale my sa premiestnime."$ $ Keď to Paline vyslovila, Leana z Ölverína aj Bella Lietavá na ňu vyvalili oči.

"$ $Si démon?"$ $ Zdesene sa naraz spýtali. Tarny sa ich mienku o Pauline pokúsil zachrániť.

"$ $Matka, polodémoni nie sú všetci s Dé. A rovnako ani démoni. Poznáš predsa proroctvo.“

"$ $Ale kto si potom?“

"$ $A kto si ty?"$ $ Spýtala sa na oplátku Pauline Leany. Bytosti začali priam zúrivo preblikávať.

"$ $Dfefaraa, kľud. Nerozumejú vám. Dajte im prekladače."$ $ Bytosti jej porozumeli, ale predtým blikli. Bella na nich ukázala.

"$ $Tunajší obyvatelia. Dajú vám prekladací prístroj, aby ste im rozumeli. To blikanie ich reč. Potom sa dorozumieme.“

"$ $Áno ale...“

"$ $Stačí Tarny, povedali niečo."$ $ Upozornila Tulienka Deľa.

\begin{center}
*
\end{center}

Možnosti spojenia všetkých krajín Wymyslenska, sa najskôr zdali ako nemožné. Veď čo by sa mohlo aspoň zdať spoločné na štátikoch, ktoré nielen, že mali rozdielne formy vlády, ale poniektoré aj jazyk a písmo. Mega Wemenská tento problém prijímala a snažila sa s ním vysporiadať. Jej sestra Maira, ktorej jazyky učarovali viac ak jej sestre a ktorá tvrdila, že násilím vzniknutý štát, má krátke trvanie. Preto sa rozhodla, že vytvorí vo Wymyslensku, jednotný jazyk. Keďže Wymyslenčania spadajú do rovnakej jazykovej skupiny (Vtedy nepoznali Fentenziu), takže bolo vytváranie jazyku podstatne jednoduchšie. Jeho tvorba bola ale poznačená víťazstvami a prístupnosťou miestnych obyvateľov k cudzincom (pozri Kultúra v predwymyslenských dobách), teda napríklad Verenčina a Tretenčina majú v súčasnej Wymyslenčine minimálny vplyv.

Keď Mega odmietla nástupníctvo v Awerstve (pôvodne dedičná funkcia vládcu dedinu), Maira to ako plnoletá (10 rokov) urobila rovnako, a keď jej sestra Mega, o niekoľko dní neskôr odišla na cestu do clova a jej sestra tiež.

Rosa skončila prvú kapitolu svojej novej učebnice. Začiatky štátu bola vcelku krátka kapitola pojednávajúca o potrebách založenia štátu a dvoch jeho zakladateľkách, Mege a Maire.

"$ $Rosa, už musím ísť domov, ideš?“

"$ $Prisahala si a teda vznikol aj môj záväzok,"$ $ odpovedala Rosa a podala jej čítačku.

"$ $Čo si čítala?“

"$ $Históriu. Celú prvú kapitolu."$ $ Odpovedala jej rovnako formálne ako znela otázka.

"$ $Dobre teda, môžeš napísať stranovú prácu o podmienkach čo prinútili Megu založiť Wymyslensko.“

"$ $Ako domácu úlohu? Teraz?"$ $ Na jej prekvapenie Chen nezareagovala.

"$ $A čo si čakala? Je november a musíš veľa dobrať. Očakávam tú prácu do zajtra. Alebo si si to rozmyslela?“

"$ $Nie, vôbec nie,"$ $ bránila sa Rosa.

"$ $Lenže neviem ako vy takú prácu robíte a vôbec...“

"$ $Naše školstvo je komplikované. To, že nechodíš do školy, bude pre teba tak trochu výhoda, lebo často býva v škole aj desať hodín. A potom sa dáva často z každého predmetu zo jedna domáca úloha alebo práca. Ak to chceš stíhať, radím ti venovať sa celý čas štúdiu.“

"$ $Desať hodín? To ideš do školy a vrátiš sa večer a máš ešte desať domácich úloh?“

"$ $Hodiny nebývajú celé. Využívajú sa podľa toho koľko učiteľ potrebuje. Napríklad ak mu snaží prebrať jednu látku, tak hodina trvá pätnásť minút, potom je prestávka a hneď ďalšia hodina.“

"$ $Neminú sa učitelia?“

"$ $Učiteľov je viac na predmet, a tak sa môže stať, že učí jeden predmet aj traja učitelia. Mňa učili históriu dokonca piati.“

"$ $Skĺbili to?“

"$ $Áno, museli. To patrí k ich pracovnej náplni.“

"$ $Zaujímavé.“

"$ $Keď prídeme do môjho dočasného obydlia, môžem stiahnuť iné, voľne dostupné učebnice.“

"$ $Čo robíš, keď sa budeš sťahovať a tvrdíš, že sa nebudeme zdržovať príliš dlho na jednom mieste?"$ $ Zdalo sa, že táto otázka trochu vyviedla Chen z miery.

"$ $Ja som obchodníčka.“

"$ $Neverím,"$ $ Povedala s istotou Rosa.

"$ $Prečo si premýšľala?"$ $ Na toto už Chen odpovedala s istotou a bez prekvapenia.

"$ $Lebo môj druh obchodu je iný, ako sa bežne berie za obchod."$ $ Rosa premýšľala čo sa spýtať, ale nakoniec nepovedala nič. 

Auto bolo najčudnejšie aké kedy videla. Nielenže ho Chen vytiahla z vrecka a zväčšila, ale aj vyzeralo ako guľa s kolesami a krídlami.

"$ $Mačičkoes. Zatiaľ najúčinnejšia forma prepravy vo Wymyslensku a v spoločenstve M. Toto je AA1123, prototyp dvadsiaty tretí, z kategórie AA, štýl jedenásty. Tento mačičkoes je zároveň niečo typu pozemské auto, dá sa zmeniť na niečo medzi helikoptérou a lietadlom, na čln, ponorku a raketu. Mimoriadne praktické.“

"$ $Zaujímavé. V tom sú čary?"$ $ Chen pokrútila hlavou.

"$ $Nie, nie je to udržateľné. Všetko to je len čistá technika. Zväčša nanotechnológie.“

"$ $Viete predsa čarovať. Načo sa..?“

"$ $To je zložité. Proste trvalo udržateľné čary ešte nikto nevytvoril, preto je praktickejšia technika.“

"$ $Čudné..."$ $ premýšľala Rosa.

"$ $Nie je to čudné. Čary sú ovládanie energie, a ak ich chceš udržať musíš sa sústrediť, na niektoré viac, na niektoré menej. A je komplikované, udržiavať techniku stále.“

"$ $A prísahy?“

"$ $Špeciálny typ mágie. Tie sú zvláštnou výnimkou, rovnako ako telepatia.“

"$ $To sa študuje v škole?“

"$ $Nie, to sa iba skúma.“

"$ $Zaujímavé."$ $ Skonštatovala Rosa. Chen otvorila dvere a Rose sa naskytol pohľad na niekoľko gaučov a jedinú sedačku ako v aute, ktorá bola pred kopou tlačidiel a obrazovkou.

"$ $Môj dočasný byt je v Greater Dublin Area."$ $ Miesto napísala na klávesnici a niečo stlačila. Na obrazovke sa objavila trasa a Chen stlačila tlačidlo autopilot.

"$ $Toto pozemské autá nemajú."$ $ Poznamenala Rosa.

"$ $Máš pravdu. Ale my sme mačičkoesi vymysleli dávno predtým, ako ľudia objavili elektrinu. Predtým boli rotacary, ale tie míňali priveľa paliva, nemali autopilota, nemali niekoľko vymožeností a pohybovali sa iba po zemi – ako autá. A predtým boli drakobusy, niečo ako koče, ale ťahali ich dragony...“

"$ $Sú aj draky?“

"$ $Dragony sú univerzálny názov pre všetky organizmy okrem ľudí, víl a zovero, teda obyvateľov Wymyslenska."$ $ Rosu, ale zaujalo niečo iné.

"$ $Víly? Vo vašom svete sú víly?"$ $ Prekvapene sa spýtala. Noviniek na ňu stále nebolo veľa, ale hlavne preto, že už čosi tušila.

"$ $Hej, ale také ako ich vykresľujú ľudia nie sú. Radšej sa nikomu neukazujú a ak chceš ostať nažive, tak ich nenaštvi. Ale v posledných pár storočiach niekde zmizli."$ $ Mačičkoes náhle zastal a Chen otvorila dvere.

"$ $Tak rýchlo?“

"$ $Rýchlosť je len jedna výhoda mačičkoesov. Pre bližšie informácie odporúčam prečítať si knihu Technika a Mágia."$ $ Takto sa s Rosou ešte nikto nerozprával. Väčšinou ak chcela niečo vedieť odvetili jej nudným a nič nehovoriacim názorom, ktorý jej chceli vnútiť. Chen bola iná. Hovorila maximálne o základnej pravde a chcela, aspoň to tak Rosa vnímala, aby si názor každý spravil sám.

"$ $A ty ju máš?“

"$ $Hej, v mojej knižnici je veľa titulov, mám ju pojazdnú, na čítačke a zmenšenú."$ $ Dodala, keď videla Rosin nechápavý výraz. Chen dočasne bývala v dom, ktorý sa krčil na mieste, ktoré bolo neprístupné.

"$ $Neviditeľný pre...“

"$ $Ľudí,"$ $ dokončila Rosa.

"$ $Vojdi,"$ $ Chen otvorila dvere. "$ $Vitaj v mojom, teda už našom dome.“

"$ $Máš to tu pekné. Lepšie ako moje dva pôvodné bydliská.“

"$ $Rada počujem.“

"$ $Kde je knižnica?"$ $ Vypľula zo seba naraz Rosa.

"$ $Tretie dvere vpravo, chceš niečo na pitie? Napríklad buble banker.“

"$ $A to je?“

"$ $Nápoj, náš. Vyrába sa z bubleníka,“

"$ $Skúsim."$ $ Odvetila a okamžite vystrelila do knižnice.

"$ $Úprimne mi to dievča pripomína mňa."$ $ Zašomrala.

"$ $Idem si uložiť nejaké dôležité veci. Našťastie som si to zamkla."$ $ Povedala sama pre seba Chen Chengová.

\begin{center}
*
\end{center}

"$ $Marone, prečo nás zavolala Arabela?"$ $ Spýtala sa Izabeta, ktorá sa práve vrátila zo stretnutia s Nielou a spýtavo sa zahľadela na dcéru svojej dcéry.

"$ $Netuším. Tvrdila, že niekto príde. Niekto, koho sme už dlho nevideli.“

"$ $Arabela mešká. A aj ten návštevník."$ $ Zamračene povedal Lorion Tlogen a pozrel sa na hodinky.

"$ $Arabela nech nabudúce nezvoláva rodinnú radu Tlogena a Goona. Nevie si ustrážiť čas."$ $ Nahnevane vyhlásila Morna. "$ $A vôbec, kto to je?“

"$ $Arabela príde za chvíľu. Mačičkoes pristáva."$ $ Prerušil ju duch Mariona Tlogena, manžela Arabely Tlogenovej.

"$ $Nemohla potom tú schôdzu mať neskôr...?“

"$ $Je to naliehavé, a okrem toho, Arabela nemôže šoférovať, keďže je duch, takže sa to znamená, že sa to čiastočne skomplikovalo.“

"$ $A to sme nemohli urobiť my?!"$ $ Spýtala sa Morna podráždene.

"$ $Pri tvojej povahe ani nie, odmietla by si to."$ $ Povedal absolútne pokojne, ale to sa nedalo povedať o Morne, ktorá bola maximálne uštipačná.

"$ $Ak to je niekto, kto porušil naše zákony a naše tradície, alebo pošpinil meno rodiny...“

"$ $Ja som vedel, že to tak zoberieš.“

"$ $Vy ste sa opovážili...?"$ $ Nikto ju nezastavil. Očividne bola Morna veľmi prchká povaha a to všetci vedeli, a tak Mornu neprerušili.

"$ $Je to Messia? Zradila rodinu! Zničila jej česť! Ušla s človekom, nič nepovedala, bola proste fuč!"$ $ Nahnevane vyhlásila. Marion ostával stále pokojný.

"$ $Myslím si, že ušla a nič nepovedala lebo by poznala vašu reakciu.“

"$ $Čuš, ty obyčajný Wymyslenčan!“

"$ $Neurážaj, Morna! Marion má občianstvo zo spoločenstva M a patrí do rodiny. A okrem toho, ten návštevník nie je Messia.“

Cez múr prešla Arabela Tlogenová a otvorili sa dvere. Takmer všetky oči na rodinnej rade Tlogena a Goona sa upierali na postavu. Prvá niečo povedala až zas Morna.

"$ $Čo tu do všetkých démonov robí ta zradkyňa Morja?!“

"$ $Morja patrí do rodiny, tak ako ty či ja."$ $ Odvetila Arabela s rovnakým pokojom ako Marion.

"$ $Zradila nás, pošpinila našu rodinu,"$ $ Hovorila teraz Morna o svojej dcére akoby tam nebola, plná nenávisti.

"$ $Odišla som, lebo som vedela čo by ste mi spravili, keby som ostala. Ja som nemala na výber.“

"$ $Nemala si si začínať s..."$ $ Marone nemala silu vysloviť meno brata jej matky, Arabely Tlogenovej.

"$ $Tradície obmedzujú Marone, a my sme proti nim boli, Jegrigsen nebol s Dé od začiatku.“

"$ $A to odkiaľ môžeš vedieť!!!"$ $ Prskala Morna.

"$ $Lebo som bola vo víre. A videla som. D...“

"$ $Mlč! Nič nevieš! Klameš!“

"$ $Nemôžem ti to už dokázať. Ale ty Rollius máš v archíve agentúry list Ló Ölverovej Zacaríasovi. Tam je niečo o vyvolaní minulosti. Alebo si preskúmajte Machu Pichu, starým piktopísmom je tam napísaný nápis."$ $ Teraz sa znova niekto ozval. A Izabeta nehovorila vôbec podráždene ani nahnevane.

"$ $Ako môžeš poznať staré piktopísmo?“

"$ $Hovorím, že som bola vo víre.“

"$ $Vír je mýtus! I to, že niektorý v ňom zmizli! Obyčajný mýtus, Morja!"$ $ Jej meno vyslovila Morna so značným opovrhnutím.

"$ $Ak sme pre toto zvolávali radu, kľudne sa môžeme vrátiť tam, kde by sme mali byť.“

"$ $Vír nie je mýtus. Vír ozaj je. Ako vytrhnutý z času. Je to iné kontinuum.“

"$ $Tieto svety sú iba obyčajnou legendou, a ty to vieš Morja!“

"$ $Nemám dôkaz, ktorý vy budete rešpektovať. Ale hlavne pre moju osobu. Pre nič viac."$ $ Toto Mornu vyprovokovalo.

"$ $Si obyčajná..."$ $ prskala.

"$ $Prestaňte sa všetci urážať!"$ $ Prvý raz bola skutočne rozohnená aj Arabela Tlogenová. "$ $Sme na rodinnej rade, a nie v cirkuse! Ak máme diskutovať o vážnych veciach, tak by sme sa mali prestať urážať.“

"$ $To ona je na vine!"$ $ Vykríkli takmer súčasne Morna a Morja.

"$ $Som tvoja matka!"$ $ Skutočne naštvane vyhlásila Morna. "$ $Tak teraz ma prijímaš späť do rodiny. Keď sa ti to hodí?“

"$ $Mala by si mať...“

"$ $Podľa vašich tradícií, ktoré som zavrhla, keď vy ste zavrhli mňa. Som slobodná bytosť a pre mňa vaše tradície neplatia."$ $ Vyhlásila hrdo. Rada zmĺkla. Morja povedala niečo, o čom sa nerozprávalo. O sile tradícií v spoločenstve M. Toto bola jedna z viacerých vecí čo dokázalo vyburcovať k nesúhlasu takmer celú rodinu.

"$ $Tak odíď, ako si odišla!"$ $ Vyhlásil Someo Tlogen.

"$ $Ja som tu neprišla, aby som sa vrátila do rodiny..."$ $ Začala po chvíli odmlky Morja, keď si uvedomila, že všetky oči na ňu hľadia, ale Marone ju prerušila.

"$ $Tak potom načo?“

"$ $To je tá vaša logika, pokladáte za najdôležitejšiu česť rodiny a človeka. Ja som niečo zistila. Ak to nechcete počuť, je to vaša vec. Ale jedna vec je istá. Vraciam sa do spoločenstva M, i keby som mala vybudovať svoj život a kariéru odznova, nebudú ma brzdiť zvyklosti.“

"$ $Čo si zistila?"$ $ Nadvihol, i keď so značným odporom, obočie Rollius Tlogen.

"$ $SUGERO!“

"$ $Nekrič Morja!“

"$ $Prečo si to nepovedala Morne, Izabeta?"$ $ Izabeta chcela niečo povedať, ale ostala len s otvorenými ústami.

"$ $Neverím ti, zradkyňa."$ $ Vyslovila jasne Morna.

"$ $Tak dovidenia, ale platí, čo som povedala, rodina."$ $ Morja sa zvrtla a odchádzala. Akonáhle za sebou zatvorila dvere začala na seba rodinná rada na seba kričať. Morja, ale vedela čo chce, teraz už áno. Chce nájsť svoju dcéru. A vrátiť sa do agentúry. A ešte Nielu. Premýšľala čo bude ťažšie. Asi všetko.

"$ $Prečo si ju tu pozvala?!"$ $ Vážne nahnevane sa pýtala Morna Arabely.

"$ $Áno, prečo Matka, si ju tu pozvala?"$ $ Zopakovala Marone.

"$ $Lebo Morja ma skontaktovala, a myslím, že skutočne má pravdu.“

"$ $Vari jej neveríš Arabela?“

"$ $Samozrejme verím, veď Morja nie je hlúpa, a mohla čakať, že ju privítate takto. A preto je celkom rozumné predpokladať, že niečo skutočne vedela.“

"$ $Orbielová tvrdila, že Morja sa stratila, tak ako to, že tu je?"$ $ Prehovorila Izabeta.

"$ $Moment, ty si hovorila s Orbielovou?"$ $ Spýtal sa manželky Rollius.

"$ $Áno,"$ $ prikývla. "$ $V Londýne. Tvrdila, že Morja sa stratila a..."$ $ na chvíľu prestala a zahľadela sa im do očí.

"$ $A?“

"$ $A Morja má dcéru.“

"$ $Jegrigsen?"$ $ Toto zaskočila aj Mornu.

"$ $Dva a dva sú štyri a dcéra je len jedna."$ $ Odpovedala.

"$ $Takže...“

"$ $Morja má dcéru?!!"$ $ Šokovane vykríkla Morna.

"$ $Ak Niela neklamala, áno."$ $ Povedala zachmúrene Izabeta.

"$ $Orbielovej nemôžeš veriť Izabeta!"$ $ Vyhlásila zachmúrene Morna.

"$ $Ja si zase myslím, že Orbielovej sa veriť dá, dvanásť rokov pracovala v agentúre."$ $ Zamiešal sa do rozhovoru Rollius.

"$ $Ale odišla!"$ $ Upozornil Someo.

"$ $To viem. A myslím si, že na to dôvod mala."$ $ Marion odvetil za Rolliusa.

"$ $Ale otec..."$ $ Protestovala Marone, ale to už hovorila Arabela.

"$ $Niela Orbielová by nikdy neočiernila Morju, ale nikdy o nej neklamala.“

"$ $Niela bola agentka Rollius. A D po nej išiel. Ako vieš, že neklamala?"$ $ Pochyboval Lorion.

"$ $Klamala by podľa mňa vtedy, keby o Morji nehovorila veci, o ktorých vedela, že sú odsúdeniahodné.“

"$ $Možno svoj postoj už zmenila!“

"$ $Nie, pochybujem. Ale, ešte niečo. Niela spomínala ešte niečo...“

"$ $Áno?“

"$ $Je to logický záver z jej tvrdenia o ktorom sme debatovali."$ $ A doplnila ešte jedno slovo. "$ $Proroctvá.“

"$ $To... neverím!"$ $ Vyšlo z Morny.

"$ $Morja nie je dcérou Tlogena! Už nepatrí do rodiny!“

"$ $Na tom nezáleží, či teraz, ale vtedy. Ona je dieťa dcéry Tlogena a syna Goona, presne ako hovorí proroctvo."$ $ Odvetila Arabela.

"$ $Aj ty si dieťa dcéry Tlogena a syna Goona!“

"$ $Ja viem, ale tu sa ukázalo, že proroctvo nebolo mierené na mňa. A okrem toho proroctvá sa neriadia, podľa rozhodnutia rodinných rád, a to ty vieš, Morna."$ $ Rozhovor sa zvrtol už len na hádku Arabely s Mornou.

"$ $A čo keď sa proroctvo netýka jej?!“

"$ $To nemôžme vedieť, Morna. A ja nedopustím, aby sa niekto, o kom vieme, že sa ho môžu týkať proroctvá, dostal do rúk D!“

"$ $Jegrigsen je s D, Arabela. Myslíš, že by D nechával nažive niekoho, kto ho môže potencionálne ohroziť?“

"$ $Jegrigsen prisahal Morji, že jej, ani potencionálnej Goonovej proroctiev neublíži a nedopustí, aby sa to stalo. Keby D jej chcel ublížiť, musel by zničiť Jegrigsena, a podľa mňa, síce D prehlasuje, že proroctvám neverí, myslí si, že ich môže zmeniť a preto, by chcel, aby sa Goonová pridala k nemu, ale to by podľa mňa, by to považoval za priveľké nebezpečenstvo. A teda chcel, aby sa stala človekom.“

"$ $Myslíš, že by D neobetoval Jegrigsena Goona svojím plánom."$ $ Povedala Morna pochybovačne.

"$ $Keby sa Jegrigsena zbavil, stálo by to veľa špekulácií v jeho radoch, kto by prišiel potom? A jeho stúpenci by ho začali opúšťať.“

"$ $Ako môžeš vedieť, že nepoužije Sugero?“

"$ $Morja to vie, len vy ste jej to nedovolili povedať. Tvrdila, že je istá vec, ktorej sa D bojí, hádam ešte viac ako proroctiev.“

"$ $Veríš Morji?!“

"$ $Áno, a ak vy nie, tak o niečo prichádzate, idem nájsť Morju, pretože ona vie niečo čo my nie.“

"$ $Idem s tebou,"$ $ Povedal Marion.

"$ $Vy zrádzate rodinu.“

"$ $Nie,"$ $ odvetil duch Arabely Tlogenovej. "$ $Ja ju zaceľujem.“

\begin{center}
*
\end{center}

"$ $To blikanie, je ich reč. Prekladače, ktoré vám dáme na oči a do uší vám budú meniť vizuálne vnímanie na zvuk ich jazyka."$ $ Vysvetľovala Bella Lietavá. Tulienka Deľa lamentovala a Tarny ju prerušil.

"$ $Čo je na tom strašné, že nevieš piaty jazyk? Veď toto je úplne neznáma planéta, ako si ho mohla vedieť?“

"$ $To, že o tom viem, je strašné. To, že niečo nevieš a vieš to, je horšie ako to, že niečo nevieš a nevieš to.“

"$ $Ale veď blikajú. To nie je jazyk.

 "$ $Jazyk je komunikačný prostriedok. Vo Wymyslenskej definícii je napísané, že nemusí byť v slovách. Ale keď zistíš, že je jazyk, ktorý nevieš...“

"$ $Máš prekladač.“

"$ $Prekladače sú absolútne nemotivačné k učeniu cudzieho jazyka. A vôbec, nie každý ich má."$ $ Tarny mal toho viditeľne dosť. "$ $Musím sa učiť ďalší jazyk a okrem toho...“

"$ $Mohla by si láskavo sklapnúť, Tulie?“

"$ $Nie som Tulie, Tarny. Tuliena Plavčíková je moja matka.“

"$ $Ja za to nemôžem, že máš také dlhé meno.“

"$ $To som si nevyberala. Tak si moje dlhé meno, láskavo vydiskutuj s mojimi rodičmi, čo mi ho dali.“

"$ $Problém je taký, že sú momentálne vo Wymyslensku...“

"$ $Podľa teórie relativity...“

"$ $...sa neodohrávajú žiadne dve udalosti v rovnakom čase!"$ $ Dokončil. Pauline len s pobavením sledovala diskusiu svojich priateľov a vážne premýšľala prečo sa tí dvaja priatelia stále priatelia.

"$ $Tarny a Tulienka Deľa, nehádajte sa. Ako to, že ste stále priateľmi?“

"$ $Matka my sa nehádame, my normálne debatujeme.“

"$ $Napríklad o tom, že Tarny má vážne medzery v chápaní teórie relativity."$ $ Doplnila Tulienka Deľa.

"$ $Nemám!“

"$ $Tak prečo si potom...“

"$ $Ja som len zjednodušoval...“

"$ $Tak to nerob, pretože hovoríš..."$ $ Pauline došla k tomu, že pri nich sa človek zasmeje, už na ich samotnej existencii. A na ich hádkach, samozrejme.

Keď mali všetci prekladače a Tulienka Deľa konečne, síce po dlhšej debate, uznala, že blikať sa bez prípadného prístroja naučiť nevie, prišli domorodí obyvatelia planéty v sprievode Leany z Ölverína. Tá čo sa predstavila ako Dfefaraa (aspoň to tak rozumeli) hovorila (blikala) a prekladač to prekladal do Wymyslenčiny a Tulienka Deľa tlmočila Pauline do angličtiny.

"$ $Toto je planéta Quert. Sme, ako už bolo spomenuté, tak ďaleko od zeme, že nás vcelku prekvapilo, keď sa tu zjavila Leana z Ölverína. To, že sme si nerozumeli, to sa vyriešilo. Vy tvrdíte, že jedna z vás je démonka.“

"$ $Polodémonka."$ $ Opravila ju Pauline a Tulienka Deľa pretlmočila do Wymyslenčiny, na ktorej prekladač fungoval.

"$ $Prepáčte. Takže polodémon sa je schopný premiestňovať, meniť predmety a meniť sa na iných, podľa našich výskumov a informácií.“

"$ $Nikdy som tu nebola. Ako som sa tu teda mohla dostať?"$ $ Prerušila ju Pauline.

"$ $To skúmame."$ $ Odvetila Dfefaraa.

"$ $Predpokladáme, že ide o nejaké silné silové pole čo pôsobí i na démona. Podľa našich výskumov sa Démon, respektíve polodémon nepremiestni hneď, ale prejde určitý čas, a za ten vás stiahlo sem.“

"$ $A ako sa tu dostala matka a Leana z Ölverína?“

"$ $Bellu tu strhol prúd, keď sa premiestňovala s Démonom, ktorý ju do svojho premiestňovania strhol zo sebou. A ja som sa tu dostala cez vír, a keďže moja miestnosť bola v minulom kuželi ale...“

"$ $Teóriu relativity tu neprednášaj Leana."$ $ Povedala Bella a jedna z kryštalických bytostí dodala.

"$ $Nemôžu sa dostať späť, to sme už hovorili.“

"$ $Je tu pre vás dýchateľná atmosféra, takže tu môžete ostať. Necháme vás zo zeme, osamote. A ešte niečo, neubližujte rastlinám.“

Keď osameli, Bella sa obrátila na Tarnyho.

"$ $Kto je Pauline?“

"$ $Kamarátka.“

"$ $Ako to, že je polodémonka?“

"$ $Ubezpečujem ťa, že s Dé nemá nič spoločné.“

"$ $A ako to...“

"$ $Polodémoni sa môžu narodiť. Jej otec je polodémon.“

"$ $Kto je to?“

"$ $Nepochopila by si to matka, príliš lipnete na rode a pôvode.“

"$ $Ubezpečujem ťa, že i dcéru samotného Jegrigsena Goona by som nezavrhla, ak by nespolupracovala s D."$ $ Tarny sa usmial.

"$ $Trafila si sa mama."$ $ Toto Bella Lietavá nečakala.

"$ $Dcéra Jegrigsena Goona?“

"$ $A Morje Tlogenovej mama, ale tvrdila si, že..."$ $ Bella sa otriasla.

"$ $Áno, ale to som nečakala. Je to... ako sa vám vôbec podarilo skontaktovať?!"$ $ Vyhŕkla zo seba.

"$ $Toto sa ti nebude páčiť mama, ale ak ma neudáš, tak ti to poviem."$ $ Teraz sa Bella zamračila.

"$ $Udať? Z čoho?“

"$ $Nezákonného výskytu a šoférovania bez vodičského oprávnenia."$ $ Bella sa naňho prísne zahľadela.

"$ $Ty si zas zobral otcovi mačičkoes a vybral sa s ním do sveta?“

"$ $É... A je to tu. Sorry, chcel som mu ho vrátiť, ale poznanie, že je Pauline polodémonka a vedomie, čo by to spôsobilo...“

"$ $A teraz pravdu, nebolo to jediný raz.“

"$ $Minimálne veľa krát. Bol som v Amerike mimo premávky na zmyslovú a inú mágiu, a pár razí na výletoch s Tulienkou Deľou a Sylviou... nepoznáš. Ale nič sa nám nestalo..."$ $ Poslednú vetu dodal po Bellinom veľmi prísnom pohľade.

"$ $Chceš povedať, že si riadil mačičkoes mágiou?“

"$ $No, áno. Lenže keď sme oslobodzovali Pauline...“

"$ $A tým myslíš...?“

"$ $Teda, myslíme si spolu s Tulienkou Deľou, že Dé nechcel, aby sa Pauline dozvedela o mágii, ale veľmi mu to nevyšlo. Priplietli sme sa do toho my, a akonáhle sme zistili, že sa urobil dôkaz o nás, nemohli sme použiť mágiu, aby ju nenamerali a ušli sme, chceli sme ísť do spoločenstva, ale ohrozili by sme ich a okrem toho Pauline bola aj je polodémonka. A tak sme sa rozhodli...“

"$ $Vy by ste neohrozili spoločenstvo M, a nechali ste v nebezpečenstve seba?"$ $ Bella nemohla veriť vlastným ušiam.

"$ $Spoločenstvo M má agentov prepánajána! A nikto z vás nemá ani dokončenú školu! Ako chcete vedieť bojovať s D, keď to nedokážu ani kvalifikovaní agenti!“

"$ $Nepoznám jediného agenta ktorý by dokázal rozprávať takmer všetkými pozemskými i fanasskými jazykmi."$ $ Zrakom skončil na Tulienke Deli, ktorá sa mierne zapýrila.

"$ $A tiež,"$ $ Pokračovala zaňho Tulienka Deľa, aby mu vrátila teraz požičané.

"$ $Poznám pár agentov, ktorý dokážu robiť so zmyslami to čo Tarny.“

"$ $Blbosť, Niela Orbielová alebo Rollius dokázali veci o ktorých sa ti ani nesnívalo.“

"$ $Niela Orbielová dala pred pár rokmi výpoveď. A len k tomu, poznáš agenta, ktorý by bol polodémonom?“

"$ $Nie,“

"$ $Takže znásobením...“

"$ $Neblázni! Agenti majú výcvik a...“

"$ $My vieme niečo, čo predpokladám, agentúra nevie.“

"$ $Agentúra vie viac vecí ako si vy myslíte...“

"$ $Neviem o tom, žeby agentúra spolupracovala s Deoque."$ $ Bella sa naňho pozrela ešte prísnejšie

 "$ $Tarny, ak si ty ohrozil ostatných s tým, že ste navštívili Deoque..."$ $ Bella bola zdesená.

"$ $Zas si uhádla.“

"$ $Tarny! Ako si mohol? Deoque je jedna z najnebezpečnejších ľudí na planéte!“

"$ $Náhodou bola vcelku milá, dala nám čítať...“

"$ $A následne nás kvôli Tarnymu takmer zavraždila."$ $ Dodala Tulienka Deľa.

"$ $To si myslela vážne?“

"$ $Samozrejme pani Lietavá. Tarny si čítal knihy Deoque, ktoré mu nedala.“

"$ $Ty máš porušovanie pravidiel nejako v krvi."$ $ Zasmiala sa Pauline, ale Belle to vôbec nepripadalo smiešne.

"$ $Prečo si sa úmyselne priviedol do nebezpečenstva?!“

"$ $Netušil som, že na Deoque neplatia zmyslové kúzla. Ale napriek tomu, sme sa niečo dozvedeli kvôli Tulienkinmu Delinmu nadaniu na jazyky.“

"$ $Tam niečo bolo?“

"$ $Jeho tajomstvo, jeho moc je v truhliciach, troch farieb cenného kovu."$ $ Odrecitovala Tulienka Deľa.

"$ $Čo?“

"$ $Jeho tajomstvo, jeho moc je v truhliciach, troch farieb cenného kovu."$ $ Zopakovala. "$ $To som preložila z piktopísma, bolo to dopísané rukou, takže je veľmi pravdepodobné, že v ostatných exemplároch tej knihy to nebude.“

"$ $Ale ak sa vrátime na zem, tak to agentúra vedieť bude.“

"$ $Agentúra je príliš byrokratická inštitúcia na to, aby to mohla rýchlo reagovať na túto informáciu, a okrem toho, musela by si uviesť zdroj, a predsa len, uverili by zdroju odo mňa, wymyslenčanky a od Deoque, ktorá je ochotná urobiť z vypočítavosti hocičo?"$ $ Bella po dlhom mlčaní prikývla.

"$ $Máte pravdu, nemala by som ťa Tarny obmedzovať, pretože to je vaše slobodné rozhodnutie, ale pár vecí...“

"$ $Áno matka.“

"$ $Odovzdáš mi otcov mačičkoes.“

"$ $Áno matka.“

"$ $A dostanete odo mňa niekoľko vecí, aby sa vám nič nestalo.“

"$ $Ďakujem, tým nám len pomôžeš. Najlepší by bol obrázkový atlas sveta.“

"$ $A načo?“

"$ $Aby sme sa mohli premiestňovať.“

"$ $Aha.“

"$ $A nie s obrázkami z vtáčej perspektívy, to by sme sa premiestnili do vzduchu.“

"$ $Rozumiem. Kedy sa chcete premiestniť?“

"$ $Zrejme dnes, ale neviem ako na tejto planéte plynnú dni."$ $ Prikývla.

"$ $A ešte niečo, zistite od Leany z Ölverína, či chce ísť tiež, a niekde ju premiestnite. Kde by chcela ona.“

"$ $Jasné,"$ $ Prikývol Tarny.

"$ $Idem."$ $ Vybrala sa Pauline.

\begin{center}
*
\end{center}

Nevermore letel nad Londýnom. Videl zmes ľudí a cítil stále nenávisť a hnev, túžbu po pomste. Jean Parvasîe musí zomrieť, to bola akoby jeho život poháňajúca myšlienka. Ničiť, búrať, zabiť. Nenapĺňalo ho to smútkom, výčitkami ani ničím iným. Teraz spoznal len to, že i Jean je rovnako silný ako on. Rozhodol sa pristáť, lebo sa potreboval vyventilovať, vypustiť svoju moc. Zničiť niečo. Zas.

\begin{center}
*
\end{center}

Deoque niečo cítila. Vedela to ako veštica. Hrozbu. Nie D. Hrozbu. Síce Deoque nikdy nebola typom človeka, ktorý bol schopný urobiť niečo nezištne, teraz sa rozhodla hrozbu zastaviť. Samozrejme z čistej vypočítavosti. Na ruke mala ešte stále zosilňovač a pri sebe pištoľ a paličku, ktorá vyzerala na prvý pohľad celkom neškodne. A ešte vymyslieť nejaké klamstvo, čo by jej cestu ospravedlnilo, pretože zmyslová mágia nemusí platiť na každého, a zomrieť zatiaľ nechcela. Zamkla, aj keď vedela, že je to celkom zbytočné, mocný mág by dokázal dvere otvoriť aj bez kľúča, ale nebol by pripravený na to čo by ho čakalo potom. Ľudia bývajúci na Coniston Gardens, si už dávno zvykli, že jeden dom sa javí ako úplne neobývaný, ale napriek tomu tam bol, akoby žijúc svoj vlastný život. Teraz sa pár okoloidúcim zdalo, že sa tam niečo pohlo, ale to považovali za príznak jedného zo slnečných a neupršaných dní v Londýne. Deoque to ani trochu netrápilo a išla cez Londýn, až tam, kde neboli žiadny ľudia zo Spoločenstva M. Urobila okolo seba pole. Vytiahla z vrecka malý, úplne vysušený kúsok dreva a ten sa naraz rozpadol a Deoque letela. Na zemi po nej ostali iba dva kúsočky z toho dreva a veľké množstvo mágie.

\begin{center}
*
\end{center}

"$ $Leana z Ölverína?"$ $ Oslovila ju Pauline a Leana sa strhla.

"$ $Polodémonka...?“

"$ $Nevolajte ma tak prosím.“

"$ $Prepáčte.“

"$ $Ja by som sa len chcela spýtať... ideme späť na zem, a ak by sme sa chceli vrátiť...“

"$ $Áno, prosím, a kde plánujete pristáť?“

"$ $Povedzte kde a zložím vás.“

"$ $V Kralove vo Wymyslensku.“

"$ $Nehovorili mi náhodou priatelia, že vo Wymyslensku...“

"$ $Myslím, že Bella mi o tom rozprávala dosť, a, ale chcem sa vrátiť na Ostrolab. A tam sa nedá premiestniť. Potrebujem ísť čo najbližšie.“

"$ $Nechcem byť dotieravá, ale čo je to Ostrolab?“

"$ $Len pre vyvolených, a vy tam nepatríte."$ $ S náhle stvrdnutým hlasom povedala. Pauline sa ospravedlnila.

"$ $Prepáčte, ale vo Wymyslensku som nikdy nebola a nevedela by som sa tam premiestniť.“

"$ $Tak stačí spoločenstvo.“

"$ $Máme obraz Kralova, môžeš sa premiestniť podľa neho."$ $ Vbehla do konverzácie Tulienka Deľa.

"$ $V poriedku... kedy...?“

"$ $Keď si odpočiniem, som unavená."$ $ Odvetila Pauline, vediac, že tu by ju D nemal hľadať.

\chapter{Deoque a Nevermore}

"$ $Vstávaj Loviisa,"$ $ Budila ju Sylvia.

"$ $Áno...?“

"$ $Škola..."$ $ Loviisa sa snažila zistiť, kto k nej hovorí, a došla k záveru, že ten sen, snom nebol. Pamätala si naposledy zo spoločenstva to ako išla do metra... a potom nič. Bolo tam dievča s menom Sylvia, ktoré chcelo zvrhnúť vládu. Vládu, ktorej členka ju uniesla a vymazala jej pamäť. Aspoň podľa Sylvie. A tej sa, aspoň podľa jej mienky dalo veriť.

"$ $To nebol sen.“

"$ $Nanešťastie.“

"$ $Škola?"$ $ Nešťastne zdvihla obočie.

"$ $Poď, aby si nezaspala, ideme!"$ $ Škola, aspoň to jej mohli odpustiť.

Pred izbou ju čakala Sylvia a súrila ju.

"$ $Zavolala som výťah, škola je v Kralove. Tu je len základná, a ako všetci predpokladajú, pôjdeš do všetko školy...“

"$ $A to je?"$ $ O školskom systéme vo Wymyslensku toho moc nevedela.

"$ $Druhý stupeň vzdelania. Prvý je základná.“

"$ $Ako stredná?“

"$ $Nie, skôr ako osemročné gymnáziá.“

"$ $My v spoločenstve...“

"$ $Ticho. O spoločenstve sa nehovorí... A nezdržuj sa pri mne. Podozrievali by ťa."$ $ Akonáhle sa výťah otvoril, Sylvia zmĺkla a Loviisa sa pokúšala tváriť sa, že ju nepozná. Keď nastúpili, Sylvia sa už zas cítila bezpečne.

"$ $Ahoj Tulienka,"$ $ Pozdravila ženu čo s nimi išla výťahom. Zdalo sa, že sa poznajú.

"$ $Sylvia je kde je Tulienka Deľa?"$ $ Spýtala sa jej.

"$ $Netuším, tvrdila, že to je niečo s Tarnym..."$ $ Žena ju prerušila a ukázala pohľadom na Loviisu.

"$ $To je v poriadku, patrí k nám."$ $ Žena vydýchla a pokračovala.

"$ $Len aby zas nešla na zem."$ $ Povedala Tulienka, ako ju Sylvia oslovila, a pokračovala. "$ $Ako ja proti tomu nič nemám, ale tu ide o bezpečnosť...“

"$ $Ja absolútne rozumiem, Tulie."$ $ Odvetila Sylvia. Vtedy už sa výťah otvoril a nastúpili ďalší, teda hovoriť nebolo bezpečné.

"$ $Kúp si raňajky, v škole nie sú."$ $ Sylvia zas Loviisu prekvapila zvláštnym Wymyslenským systémom.

"$ $Kde?“

"$ $Automat,"$ $ Odvetila a zas sa od nej vzdialila.

"$ $Vpravo."$ $ V rohu boli rady wymyslenčanov, zrejme si kupujúcich v automatoch raňajky.

"$ $Postav sa do radu a nezavadzaj!"$ $ Takmer do nej vrazil Wymyslenčan s bagetou na tanieri.

"$ $Prepáčte..."$ $ Ospravedlnila sa a on zrazu zhíkol.

"$ $Si zo spoločenstva!"$ $ Loviisa sa naňho vyjavene pozerala a nechápala čo spravila zlé.

"$ $Prečo si to povedala po anglicky?"$ $ Ticho sa k nej pretlačila Sylvia a šepkala.

"$ $Automaticky.“

"$ $Teraz si si urobila zlé meno. Nenávidia ľudí zo Spoločenstva M.“

"$ $Prečo?“

"$ $Potom... Alebo... vieš používať telepatiu?“

"$ $Áno...“

"$ $Tak telepaticky, ale radšej nie, pretože telepatické hovory sú odpočúvané. Potom."$ $ Odtiahla sa. Všetci sa na ňu pohoršene alebo zhrozene dívali. Nový život bude krutý.

\begin{center}
*
\end{center}

Deoque si pozorne pozerala ulice najskôr okrajov, až potom centra Londýna. Zatiaľ nevidela nič, čo by bolo podobné jej vízii. Nevšimla si miesto, a keby aj, Deoque toľkokrát Londýn nevidela, aby vedela spoznať o akú časť sa jedná, len pohľadom. Bolo už nadránom a jej už dohárala energia. Bola na okraji Londýna, a síce aj tam po uliciach chodili ľudia, táto ulica bola takmer vyprázdnená, len vzadu bolo pár ľudí a v po chodníku išiel muž. Taký akého ho videla. Vedela čo nesmie a čo smie robiť. Z druhej strany ulice išlo policajné auto. Zastavil ho vo vysokej rýchlosti bez zaváhania a policajta sa dôrazným hlasom spýtal kde je Jean. Deoque, stále neviditeľná, zosadla a v tej chvíli, sa už dôraznejšie a s hnevom v hlase pýtal to isté. Deoque už nečakala. Vystrela ruku k Nevermoreovi a kým by stihol niečo urobiť, zneviditeľnila ho a uväznila ho tak, aby nemohol použiť svoju obrovskú moc. Nevermore sa na ňu, s veľmi nahnevaným výrazom, zahľadel.

"$ $Kto si?"$ $ Deoque sa spýtala, kontrolujúc systém. Mlčal a snažil sa prelomiť putá. Ale ako náhle vzniklo kúzlo, vtiahlo sa do pút. Bolo vysaté.

"$ $Čo to..."$ $ Nevermoreovi sa zaleskli oči a šepkal ako šialenec.

"$ $Jean, kde je Jean?"$ $ Deoque tušila, že to k ničomu nevedie, a preto sa rozhodla, že mu dá poslednú šancu.

"$ $Kde je Jean?!"$ $ Zas, ako zmyslov zbavený, začal opakovať jednu vetu. "$ $Zabijem ho, zabijem ho."$ $ Tvár mu naplnila myšlienka pre ktorú žil a pre ktorú podľa neho samého jedine žil. On predsa vedel, že Jean zomrie jeho rukou. Deoque sa nateraz snažila, aby ju neprevládla paranoja, pretože teraz o to ani prinajmenšom nestála, teraz potrebovala zistiť čo najviac o Nevermoreovi.

"$ $Kto si?"$ $ Nevermore ju nepočul. Bol v záchvate šialenstva, keď jedinou jeho myšlienkou bola pomsta. Netušil za čo.

"$ $Si s Dé?"$ $ Len zavrčal. Deoque pochopila, že teraz nie je vhodná šanca na zisťovanie. Bola unavená ona sama a hlavne z toho, že bolo niečo čomu nechápala, bola predsa veštica! Určite ju chce D zdiskreditovať! Určite! Určite tu poslal jeho, aby prenikol do jej systému, aby sa infiltroval so systému, ktorý tak dlho vytvárala. Musí sa odsťahovať a jemu vymazať pamäť, ale nie, to D dokáže zvrátiť! Chybu urobila už v tom, že im prepáčila ukradnutie jej kníh! Určite pomáhali D a ona si to ani neuvedomila. Určite chceli vedieť, aké má informácie o ňom, a ako je ďaleko k jeho zničeniu a teraz to vyšlo nazmar, čo bude robiť? Žiadnu paniku, žiadnu paniku... keby prepadla teraz panike. On by mohol utiecť a potom...! Nezniesla pomyslenie. Keby mala predpoveď na to čo chcela a len na to, ale to nie to potom... Racionálne jadro Deoque prikazovalo, aby sa vzchopila a prestala veriť niečomu nepodloženému, alebo ňou nepredpovedanému. Jej dedukcia zlyhávala, rovnako ako analýza a Deoque vedela, že ak si chce zachovať zdravý rozum, nesmie veriť ničomu nepotvrdenému, alebo ňou nepredpovedanému. Vedela, že pomaly začína prestávať rozlišovať svoje predpovede od jej predstáv, a tak i jej toto pravidlo v počiatočnom štádiu paranoje prestávalo stačiť. Zabiť nechcela, pretože potrebovala informácie. Pomaly už, ale nevedela komu má veriť. Informácie zistené ňou musela vidieť, nie vydedukovať, na to si snažila dávať si pozor, ale... Verila knihám. Ale čo ak klamú? Deoque začala mať pochybnosti. Ale teraz nie! Nie! Nemohla teraz prejaviť slabosť. Potrebovala ho omráčeného. Ale v stave, v akom sa teraz nachádzala to nedokázala. Nie, to si nemala povedať.

To čo Deoque prežívala nebolo zúfalstvo, ani nerozhodnosť. Vedela o svojich chybách, ale nedokázala si ich priznať. Vtedy by poprela dokonalosť, ktorú na sebe videla a ostatní by našli jej najväčšie chyby a slabiny a mohli by to využiť. Je veštica, veštica! To keď priala toto poslanie, vtedy ešte bola v poriadku, teraz sa zmenila. Ona si to uvedomovala, nie tušila to. Pre jej hrdosť to bolo neprípustné. Dlho svoje konflikty dávala do úzadia, až vyšli na povrch, a to až príliš bolestne jasne.

Áno, Deoque sa bála. Bála sa sama seba, budúcnosti, predpovedí a pritom po nich túžila. V jej mozgu prebiehal samostatný súboj dvoch Deoque. Jej vešteckou osobnosťou, ktorá sa bála budúcnosti, hoci by si to nikdy nepripustila, ktorá bola presvedčená o sebe. A jej racionálnou osobnosťou, pragmatickou a vypočítavou, až cynickou. Doteraz sa tieto osobnosti nejako prekrývali, ale teraz sa každá chcela, až násilne dostať na povrch. Deoque ich prestala mať pod kontrolou. Už sa to nedalo zakryť. Deoque sa sama sebe vymkla z rúk. Paranoja, kombinovaná s rozdvojenou osobnosťou, a tak trochu i s neurózou jej prerástla cez hlavu. Vybuchovali z nej kúzla. Rozbila ulicu Laserom. Prestala ho vysielať a nahradila ho Solanom. Zasiahla ním Nevermorea. Hlavnou príčinou bol strach pred tým, aby sa stal svedkom toho, čomu sa Deoque tak úporne bránila. Nedokázala viac byť (nie žiť) s predstavou jej v tomto momente. Namierila Solan na seba.

\begin{center}
*
\end{center}

Loviisa mala ráno predtuchu, že sa tento deň nevydarí. Po afére pri raňajkách, keď sa takmer všetci stali jej nepriateľmi, sedela v mačičkobuse, na ceste do školy. Sylvia jej pri vstupe pripomenula, že sa má tváriť ako keby je nepoznala. 

Klebety sa na zemi šíria rýchlo. Wymyslensko nie je ako zem, ale napriek tomu je rýchlosť šírenia informácií takisto vysoká.

Keď dostala Loviisa lístok, v mačičkobuse si tí, už usadení, zväčša zo Žblnkotaničkova, začali šepkať a vrhať na ňu rôzne druhy pohľadov. Ľutujúce, nenávistné, posmešné, zhovievavé i opovržlivé. Keby nebolo dohody so Sylviou, že sa bude snažiť neupútavať na seba pozornosť, by im Loviisa najradšej niečo odsekla. Ale teraz nemohla. Nechcela.

Prežila celú cestu až do Kralova. Loviisa by sa chcela najradšej zavrieť v izbe a vymýšľať plány na získanie jej pamäte. Ale nebol na to čas. Sylvia jej povedala, že ešte pred vyučovaním má sa ísť prihlásiť ku riaditeľke po dodanie učebníc. Chcela nenávidieť, ale nechcela, zakazovala si to, pretože si bola istá, že potrebuje mať čistú myseľ. Wymyslensko mala prečo nenávidieť. Alebo aspoň časť jeho obyvateľov a Cecíliu Žblnkotaničkovú. Ona bola všetkému na vine. Svoje myšlienky si, ale nechávala pre seba.

Ani si nevšimla kedy, stála pred kanceláriou Lusen Stikruovej, riaditeľky Kralovskej všetko školy.

Školy sa vo Wymyslensku delili na základné, stredné a vysoké. Súkromné školstvo takpovediac neexistovalo, a tak bol jeden, jediný typ základných škôl. Pestrejšie paleta bola už u stredných. Systém stredného školstva fungoval asi tak, že z každej školy, každých päť rokov vyberali desiatich najlepších zo všetkých ročníkov od tretieho (aby po dokončení školy mal študent minimálne desať, čo sa vo Wymyslensku považuje za dospelosť (zovero majú zrýchlený duševný i fyzický vývin), priali desiatich najlepších bez ohľadu na vek, ale s ohľadom na vedomosti) do ôsmeho. Z deviateho ročníka sa išlo na strednú školu ako v spoločenstve M. Stredné školy sa delili na Odborné školy (všetky boli s Wymyslenskou obdobou maturity), Mystické školy (Zo zameraním hlavne na vedu, ale spolu s druhom mágie – mystikou), Čarodejnícke školy (Zamerané hlavne na mágiu), Kráľovské školy (niekedy volané aj Vládne školy – zamerané na manažment, diplomaciu, a iné veci, čo človek potrebuje na politiku a bytie byrokratom) a Všetko školy (zamerané na všetko, čo sa učí na školách ostatných). Zo všetkých stredných škôl, sa dalo ísť na vysokú školu a niektoré odbory sa vyučovali na univerzite. Vysoké a univerzitné školstvo bolo veľmi benevolentné k uchádzačom, ktorých prijímali. Priali všetkých, čo prešli cez úvodné sito (zložené zo zisťovania základných vedomostí o téme – teda žiadne vzdelanie nebolo potrebné). Približne po tretine semestra (dĺžka záležala od typu štúdia) museli prejsť ďalšími sitami v podobe esejí, skúšiek a testov. Kto prešiel všetkými, šiestimi až štrnástimi kolami sít a prešiel záverečnými skúškami obdržal študovaný titul a nikoho nezaujímalo, či už má urobenú základnú školu. Vo Wymyslensku sa zdalo, že platí vo vzdelaní pravidlo, že na veku nezáleží, a dalo by sa pripojiť, že ani na povolaní nezáleží, lebo keď ste chceli, aby vám vykachličkovali kúpeľňu, len málo z pracovníkov nemali pred menom napísaný titul IngKach (inžinier – kachličkár).

Lusen Stikruová mala v ten deň vcelku všedný deň. Jedna nová osoba, pár telefonátov, byrokratické nezmysly a káva. Práve jej niekto klopal na kanceláriu. Správne predpokladala, že to bude nová.

"$ $Ďalej,"$ $ povedala, a ani sa neobťažovala pozrieť kto to vlastne je. Čakala, že sa návštevník ohlási sám, pretože ďalej, bolo kúzelné slovo, ktoré vás zbavilo všetkej starosti plynúcej zo zoznamovania. Počula len pozdrav a nič. Kúzlo nefungovalo. "$ $Ďalej,"$ $ zopakovala.

"$ $Huber Ayd!"$ $ (čítaj. Habr ejd – po wymyslensky dobrý deň). Stále sa nikto nepozdravil. Lusen Stikruová musela preraziť škrupinu zoznamovacieho stereotypu a preto prehovorila.

"$ $Tu riaditeľňa Všetko školy v Kralove, Lusen Stikruová, kto ste vy?“

"$ $Loviisa Räkkänová. Vraj si tu mám ísť po učebnice a rozvrh.“

"$ $Správne."$ $ Lusen Stikruová siahla po balíku, zabalenom v bielom baliacom papieri, a podala ho Loviise.

"$ $Ďakujem.“

"$ $Máte tam učebnice, rozvrh, kartu od automatu na jedlo, čip do knižnice."$ $ Vymenovávala. Keď Loviisa odchádzala, riaditeľka dodala.

"$ $A samozrejme, trieda štvorka B.“

Čo si Loviisa spomínala, prvá hodina má byť čarovanie, ale rozhodla sa, že sa radšej pozrie do balíka, skôr ako sa začne vyučovanie. Samotná trieda triedy Štvorky B ako trieda slúžila len na Wymyslenčinu, inak používali rôzne učebne a prednáškové haly. Každý mal v triede samostatnú lavicu, s veľkým priestorom na prácu a na uloženie pomôcok. Loviisa si pamätala, že oni, v spoločenstve M mali jednu triedu na oveľa viac predmetov, ale to bolo hlavne spôsobené tým, že spoločenstvo M malo svoje zamaskované priestory vo veľkomestách a tie nemohli byť na veľmi veľkých plochách, preto, aby nevzbudili u ľudí podozrenie. Prirodzene, mohli využívať výškové budovy, ale tie sa využívali maximálne na absolútne dôležité miesta, ktoré obrovské byť museli. Ako napríklad agentúra, parlamenty a nákupné centrá. Veľké budovy boli nepraktické i pre komplikovanú evakuáciu, ktoré sa pravidelne nacvičovala aspoň raz mesačne. Z vyšších poschodí mohli odlietať zneviditeľňovaný, ale keďže sa zneviditeľňovanie učilo až na druhom stupni školy, mali vysoké základné školy nevýhodu v evakuácii.

Lavice vyzerali akoby ich práve priniesli z výroby. V triede bolo jedenásť miest a jedno voľné. Cecília musela mať všetko naplánované. Všetko! Najradšej by nadávala na ňu, na Wymyslensko a studenú vojnu medzi ním a spoločenstvom. Ale nemohla. Loviisa sedela v strednom rade. Pár ľudí ju obkolesilo a pýtali sa jej na všeličo. Všimla si, že Sylvia sedí úplne vzadu. Sama. Ju to netrápilo. V škole svoje myšlienky nemohla dať do počítača (písali na počítače, ale text sa kontroloval a prístup na internet nebol a okrem toho by to bol priveľký risk, robiť protirežimné plány v škole), ale napriek tomu niečo na ňom písala. Loviisa by najradšej prišla k nej, ale vedela, že nemôže. Nech si radšej každý myslí, že s ňou nič nemá, že nenávidí spoločenstvo M z ktorého prišla hľadať vo Wymyslensku azyl, že... V každom prípade ich o tom musí presvedčiť. O tom, že je oddaná Wymyslenčanka, milujúca svoju novú vlasť. Nech sa to rovná čomukoľvek.

V štvorke B sa do rozhovoru s ňou zapojili takmer všetci okrem Sylvie, ktorá ticho niečo dopisovala, okrem jedného chlapca, ktorý si niečo opakoval a tichého dievčaťa, sediaceho v lavici a čítajúc knihu.

"$ $To je Brumia, s ňou sa nebav, má polovicu rodiny v spoločenstve M, vo Fínsku. Utiekli tam za Solemy. Zradcovia. Mali pomáhať oslobodiť svoju vlasť."$ $ Čiernovlasý Wymyslenčan, ktorého meno je Jemed. Loviise je jasné, že tento bude kvázikamarát, ktorých by mala zohnať čo najviac, aby sa vyhla podozreniam. A mali by byť čo najviac prowymyslenský. A čo najviac proti spoločenstvu M.

"$ $Aha."$ $ Prikývla a tvárila sa, že ju to zaujíma, ale rozmýšľala o tom dievčati, Brumii. Rozmýšľala, či pozná jej rodinu, či sa usídlila v Helsinkách, a ak áno tak či v tom istom odseku spoločenstva M.

"$ $A kto je tamto?"$ $ Ukázala na chlapca. Zámerne sa vyhla Sylvii.

"$ $To je Brownear. Ten je fajn, ale niekedy hovorí hlúposti o tom, že by sme mali vymeniť vládu. Ale len pred voľbami. Inak ho to prejde. Ale aj tak je to mimoriadne čudné."$ $ Viac ako "$ $aha"$ $ Loviisa nekomentovala.

Keď išli do učebne čarovania Loviisa vedela čosi o všetkých. Síce všetci boli posudzovaní Jemedom, hlavne podľa príslušnosti k Wymyslensku a sympatiám k spoločenstvu M, ale to, čo robili bolo rovnaké, aspoň to predpokladala.

Sylvia bola opísaná ako zradkyňa ktorá chce zvrhnúť Wymyslensko a zanechať všade spoločenstvo. Brownear bol popísaný ako blázon, ktorý nechápe úžasnosť Cecílie. Brumia ako chudák, ktorú zradila jej vlastná rodina. Jemed bol nimi vykreslený ako absolútne najlepší človek (možno okrem Cecílie), ktorý chápe, že Wymyslensko je najlepšia krajina a spoločenstvo chce ovládnuť svet a priniesť nadvládu D. Na päty mu dýchala Pipia s Pásikom, Cecíliiným synom. To, že je Cecília úžasná a mala by vládnuť a spoločenstvo pomáha D verili Rukyn a Kissa. To, že Spoločenstvo je zlé a Wymyslensko dobré, teda v čiernobiely svet verili Dino a Tamara. Loviisa premýšľala, kde je v rebríčku ona. Podľa nej si ešte nezaradili. Túžila po tom, aby ju nechali na pokoji a nechali ju so Sylviou, aby sa mohla s ňou skontaktovať. Nemohla. Musela sa vyhnúť podozreniam, stále si pripomínala. Aby ju nespájali so Sylviou, tak potrebovala byť nenápadná a zapadnúť do davu podporovateľov Cecílie.

Čarovanie vyučoval Modmes Žiarivý. Bol to mladý Wymyslenčan, ktorý hneď ako zbadal, že v triede je niekto nový, pridelil mu lavicu a strávil pätnásť minút vypytovaním sa na rôzne veci, od pôvodu cez rodinu na to ako ďaleko je z čarovaním. Loviisa zistila, že je približne na rovnakej úrovni ako oni, a tak sa po zmarení polovice hodiny konečne začali učiť. Celá hodina aj tak vyzerala, že zdvíhali predmety a krúžili nimi nad hlavami. Ako domáci úlohu dostali napísať praktické využitie krúžiacich predmetov. Sylvia vynikala. Ale to si nikto nevšimol. Maximálne si šepkali, že to zistila pri agentoch zo spoločenstva a Démonovi.

Po hodine čarovania, vlastnostiach a použití látok, Wymyslenčine, matematike, astronómii a etikete sa Loviise zdalo, že všetci sú k nej milý z nejakého rozkazu, ako keby to mali prikázané. Keď včera verila Cecílii, tak sa ju ľahko dalo presvedčiť na opačný názor. Loviisa premýšľala, či Cecília nedala priamy rozkaz učiteľom, aby ju presvedčili. Premýšľala, či ich tu mala dosadených, či boli len všetci taký, že si nechali hovoriť, čo majú robiť. Po astronómii mala podozrenie, ale po katastrofe na etikete, keď pokazila čo sa dalo a ešte bola ospravedlňovaná, sa v tomto názore začala utvrdzovať. Ešte mali prednášku a konečne odtiaľ mohla odísť.

Cecília dlho, dve dlhočizné hodiny rozprávala o naoko rôznych témach, ako zdroje energie, nebezpečenstvo od Démona, zlo minulého režimu a prirodzene, o hrozbe vojny so Spoločenstvom M. Ale všetky boli preplnené propagandou, nenávisťou a vlastenectvom. Wymyslenské zásoby ropy míňajú polodémoni, ktorých si vytvorilo spoločenstvo, aby destabilizovalo Wymyslensko. Tí tiež skupujú Wymyslenské produkty a potom ich ničia, aby destabilizovali trh. Fentenzíjčania by mali mať obmedzené práva, pretože za minulého režimu kolaborovali so spoločenstvom... Až napokon začala hovoriť, už ani trochu rozumne, Loviisa v jej prejave nenašla žiadny fakt, o ktorom by stopercentne vedela, že je pravdivý. Jej spolužiaci, s výnimkou Sylvie, Brumie a Browneara hltali Cecíliine slová s veľkou fascináciou.

"$ $A vy viete,"$ $ Ukázala na nich. "$ $Že nás neporazia. Že svoju armádu polodémonov, ktorí nás majú zničiť, budovali nadarmo. Vyničíme tú démonskú pliagu ktorá nás chce ovládnuť."$ $ Rozohnila sa. Toto bolo na Sylviu priveľa. Loviisa videla, ako vstala a rozkričala sa na Cecíliu, ktorá ju ešte včera presviedčala, o úžasnosti vzťahov spoločenstva M a Wymyslenska.

"$ $Klamete! Vytvárate si konšpirácie na svoju propagandu, ktorou vyhrávate voľby, vymývate mozgy wymyslenčanom zo svojej zaslepenosti!"$ $ Hala zhíkla. Sylvia sa odvážila kritizovať Cecíliu, najvyššie postavenú osobu vo Wymyslensku a dokonca spochybniť jej slová o spoločenstve. Cecília sa na ňu zlovestne usmiala a povedala.

"$ $Vy si teda myslíte, slečne Mänchenová, že spoločenstvo nezbrojí, že nemá armádu proti nám, nechce nás zničiť a nepomáha D?“

"$ $Áno, to si myslím, len o tej armáde, spoločenstvo má armádu, ale priznajte si, nemá ju Wymyslensko tiež?"$ $ Cecília sa na ňu zlovestne pozrela a pokračovala.

"$ $Ste diplomatka slečna Mänchenová? Alebo poznáte politickú scénu a údaje z tajnej služby?“

"$ $Nie, ale tie si môžete vymyslieť aj vy."$ $ Pokrčila plecami Sylvia.

"$ $Prečo by som asi mala?“

"$ $Zoberte si to, je to dobré vašim preferenciám, poviete niečo, ľudia vám uveria a keďže oni sami nemajú údaje, musia sa spoliehať na vás. A prečo by ste hovorili niečo čo vám ublíži? Prečo neprekrúcať pravdu...?“

"$ $Ale prečo ju prekrúcať?“

"$ $Už som to povedala. Pravda je často nevýhodná. A klamstvá krajšie. Aspoň pre niekoho.“

Loviisa na Sylviu hľadela z obdivom. Hádala sa osobne s najmocnejšou osobou Wymyslenska, a keďže Wymyslensko bolo najmocnejšie na Fanase, tak zrejme s najmocnejšou osobou tejto planéty. Ona by to nedokázala. Videla, ako sa počas prednášky Sylvia kontrolovala, aby nič nepovedala, ale po Cecíliiných slovách o démonoch sa neudržala. Pretože ona sama bola démonka.

"$ $Koniec prednášky. Slečna Mänchenová je zaslepená, a možno by sme ju aj mali ľutovať. Ona nechápe..."$ $ Sylvia jej skočila do reči.

"$ $Ani vy."$ $ Usmiala sa a vstala. Vyšla z haly a ešte zakričala Cecílii. "$ $A nič neostane ako je. Ani kameň na kameni!"$ $ A odišla. Doslova vypochodovala zo sály. Nepozorovane strčila Loviise do ruky papierik.

Cestou z Kralova sedela v mačičkobuse. Naschvál si vybrala taký kde nebude Sylvia, aby si nikto nemohol ju s ňou spájať. Ani nemal kto. V mačičkobuse bola len jedna osoba z jej triedy. Brumia Spachtošová.

Skôr ako by si to mohla rozmyslieť, prihovorila sa jej tak, aby to nikto nepočul. "$ $Tvoji príbuzný sú vo Fínsku? Pretože moji tiež."$ $ Brumia sa na ňu neveriacky pozrela, pretože Loviisu ten deň videla v skupine okolo Jemeda, ktorú nenávidela.

"$ $Chceš ma mi posmievať?“

"$ $Nie, to čo bolo v triede je pretvárka, aby si ma nevšimli. Aby ma nepodozrievali.“

"$ $Ako ti mám veriť?“

"$ $Nijako. Nemám dôkaz. Cecília ma uniesla.“

"$ $V rámci akcie O+?“

"$ $To je čo?“

"$ $Cecília podľa niektorých wymyslenčanov unáša deti a mladistvých, aby zvýšili populáciu sebe a znížili spoločenstvu M.“

"$ $Asi to.“

"$ $Ja som si myslela, že Cecília vymazáva pamäť.“

"$ $To áno. Nepamätám si.“

"$ $To je strašné. Keby sme mohli..."$ $ Vtedy Loviisu napadol šialený nápad.

"$ $Chceš sa zoznámiť so Sylviou.“

"$ $Ty ju poznáš. Šikovná pretvárka.“

"$ $Áno a... uvidíš.“

\begin{center}
*
\end{center}

Deoque nebola v stave, v akom by dokázala napraviť ulicu ktorú zničila laserom. Bola pri vedomí, ale nechcela byť. Nedokázala žiť s vedomím, že nezvládla situáciu. Že zlyhala. Deoque ležala na studenej podlahe a triasla sa. Nechcela byť raz takto. Nechcela byť ničím. Chcela vrátiť čas, nestať sa vešticou, odhodiť, zničiť, spáliť alebo odovzdať veštecký prsteň. To sa bez smrti nedalo. A to ešte nechcela. Potrebovala...

Ešte z kúsku sily čo jej ostala vyvolala zapnutie jej samoopravovacieho systému a zas samu seba omráčila.

\begin{center}
*
\end{center}

Na planéte kde pristáli, pre nedostatok pozemskej stravy, ich stravná jednotka boli nanotablety s požadovaným množstvom proteínov, vitamínov, vlákniny, cukrov a bielkovín pre osobu. Vyvinula ich Leana z Ölverína ešte pred dávnymi rokmi. Pauline bola nadstavená na pozemský deň a pozemskú noc, a tak zaspala aj napriek tomu, že podľa času, ktorý používali na Querte bolo práve poludnie. Quertský kalendár sa delil na 7 časových úsekov – Obdobie svietivosti slnka, Obdobie padajúcich hviezd, Obdobie noci, Obdobie kryštalika, Obdobie veľkých mesiacov, Obdobie elektriny a Obdobie mágie. Práve keď tam boli, tak bolo obdobie svietivosti slnka, cez ktoré (spolu s Obdobím kryštalika) mali kryštalické bytosti na planéte najviac energie. Quertský rok bol sedemkrát dlhší ako zemský, či Wymyslenský, takže dĺžka jedného obdobia sa rovnala jednému roku. Deň bol dvakrát dlhší ako na zemi, a teda rok mal celkom 1278 dní a každý tretí rok bol priestupný. Na planéte Quert počítali v trojkovej sústave a písali zvláštnymi farebnými štvorcami. Rozoznať ich bolo pre mimoquerťana takmer nemožné, pretože Querania používali také odtiene farieb, že pre nevytrénovaného pozorovateľa bolo ich rozlíšenie takmer nemožné. Komunikácia, pre absenciu sluchu, pozostávala z farebných znakov, ktoré vysvecovali. Doslova písali čo videli, ako komunikačná jednotka. Ako ľudia so synestézou, ktorí počuli farby a dokázali ich napísať.

Leana tu žila tak dlho, že dokázala vymyslieť prijateľný prekladač, ktorý farby menil na zvuky a zvuky na farby. Pauline tušila, že pre ňu bude ťažké odtiaľto odísť. Bolo zvláštne lúčiť sa s miestom, ktoré bolo tak neuveriteľne nádherné, s tým, že ho už nikdy neuvidíš. Pauline ku Quertu nemala žiadny zvláštny vzťah po jednej noci (pozemskej) čo tam strávila, ale bolo jej ľúto odtiaľ odísť. A ani trochu tomu nepomáhal fakt, že sa ich na Zemi pokúša niekto zabiť. A keby to bol len tak niekto, ale toto bol samotný D, nepriateľ spoločenstva a Wymyslenska dokopy. Viac nevedela. A možno to bolo aj pre ňu dobré.

Bella mala pripravené veci na odchod a zhovárala sa s Tarnym. Ten sa hral s kryštalickou trávou, odtrhával ju mágiou a menil jej naoko farby. Pauline si až teraz uvedomila čo bude mať na následok pozemský včerajšok. Po prvé, budú sami, po druhé, nemôžu sa spoliehať na pomoc žiadnej inštitúcie. Tieto veci vedela už včera, lenže naplno si ich uvedomila až teraz. Po tretie, prenasleduje ich démon, ktorý ich chce zabiť, spolu so svojou armádou, v ktorej je zhodou náhod aj jej otec. Po štvrté, niečo majú nájsť, len ešte netušia čo. A po piate, ich presuny záležia od nej, a takže je ich poslednou záchranou. Pauline si naplno uvedomovala to, že je najhoršia z celej trojice. Tarny bol geniálny so zmyslami a Tulienka Deľa zase čo a jazykov týka. A ona? Jediné čim bola potrebná, bola to, že bola polodémonkou. Žiadne schopnosti, ktoré sa naučila, žiadny talent, ale len genetický kód, ktorý zdedila. Preto si Pauline dala úlohu niečo nájsť v čom by im mohla pomôcť. Okrem premiestňovania.

\begin{center}
*
\end{center}

"$ $Čo uvidím?"$ $ Spýtala sa stále nedôverčivá Brumia.

"$ $Nemôžem ju zradiť.“

"$ $Aj tak je to len trik. Už to na mne skúšali.“

"$ $Ver mi!“

"$ $Nemôžem. Ty si sa spojila s nimi."$ $ Brumia Loviisou pohŕdala, lebo pokladala jej pretvárku za pravdivú. Loviisa si pomyslela, že má aspoň Sylviu.

\begin{center}
*
\end{center}

Rozdiel medzi chemickými nemagickými a magickými látkami bolo to čo Rosu momentálne najviac trápilo. Verila Chen. Síce vedela, že by nemala byť úplne dôverčivá, len magická moc je ako dôkaz dostačujúca. Pila buble banker a stihla si stiahnuť do tabletu niekoľko učebníc. Podľa jej mienky boli najzaujímavejšie predmety mágia a vlastnosti a použitie organických a anorganických telies a látok. Boj v knihách vyzeral podľa Rosinej mienky vcelku staromódne, pretože sa tam vyučoval boj s mečmi, palicami, dýkami, bičmi a kosákmi. Rosa predpokladala, že aspoň šerm by mohol byť celkom zaujímavý. Keď prišla do Írska chcela veľmi šermovať, ale jej matka ju odbila s tým, že je to drahé a dievčatá nešermujú. Teraz sa nad tým len usmievala a chcela, aby jej "$ $matka"$ $ videla tento zaujímavý systém. Keďže bolo poobedie a večer odlietali do Nórska, musela si vystačiť s teóriou. Do tabletu si zaznačila posledný údaj z kapitoly o fanasských rastlinách a rozhodla, že si ide poprezerať stránky z koncovkou .ym, teda z fínčiny yhtesiö M, čo je spoločenstvo M. Našla niekoľko stránok s mailmi, nejaké médiá, niekoľko blogov, oficiálne stránky inštitúcií, či len osobné stránky. Jej prvou dlhšou zastávkou bol denník Community M news, new news a vedecký magazín Dragons \& Space. Keď zapadlo slnko už sedeli spolu s Chen v mačičkoese, ktorý letel do Nórka.

\chapter{Škrt cez rozpočet}

Loviisa vystúpila z mačičkobusu pred megadomom vo Žblnkotaničkove a kúpila si v automate na jedlo obed. Výťahom sa vyviezla na jej poschodie a zložila si v izbe knihy. Zamkla, a keďže na chodbe nikto nebol, zaklopala na Sylviinu izbu. Bolo počuť šramot a za chvíľu otvorila Sylvia a Loviisa počula ako si vydýchla. "$ $Rýchlo, poď dovnútra. Kým nás nikto nesleduje!"$ $  Vtiahla Loviisu dnu a zamkla dvere. Začala zo skríň a spod postele vyťahovať kufríky a knihy. Keď otvorila jeden, bol plný wymyslenských bankoviek, mačičiek, v hodnote sto. Kniha sto zaručených spôsobov ako sa stať lupičom bola zmenšená a zabalená v ruksaku.

"$ $Čo sa deje Sylvia?"$ $ 

"$ $Pokazila som to! Chápeš? Dnes som sa mala ovládať, ale jej slovám o démonoch sa neoponovať nedalo! Neovládla som to a je to moja vina, čo teraz bude! Cecília ma už považuje za nebezpečenstvo! Musím ujsť! Je len otázkou času kedy ma prídu zatknúť. A vtedy im už neujdem."$ $ 

"$ $Si démonka predsa! Môžeš sa premiestniť."$ $  Sylvia sa zamračila, ako vždy, keď sa niekto zmienil o démonoch.

"$ $A to vyhlásia, že som D. O existencii druhého démona vie pár ľudí. A ver mi, že Cecília medzi nimi nie je."$ $  Loviisa pochopila, že Sylvia odchádza. Jediná osoba ktorej teraz dôverovala...

"$ $Idem s tebou. Si moja jediná šanca!"$ $  Sylvia sa na ňu pozrela, ako na človeka, ktorý ničomu nechápe.

"$ $Nemôžeš tak riskovať!"$ $ 

"$ $Ale môžem!"$ $  Oponovala Loviisa.

"$ $Nechápeš, že ak ťa nájdu, tak ťa zabijú?! Mňa nemôžu lebo..."$ $  Nevyslovila to.

"$ $Ja viem, a som si toho vedomá."$ $ 

"$ $Ale..."$ $  Sylvia chcela protestovať. Nechcela mať na svedomí obete.

"$ $Sylvia. Prosím. Ujdeme spolu."$ $ 

"$ $Ale buď si vedomá, že nezískam tvoju pamäť."$ $  Pokúšala sa ju zastaviť.

"$ $Potom. Teraz nie je čas."$ $ 

"$ $Ty to uznávaš?"$ $ 

"$ $Logicky. Toto je čas na útek, nie na revolúciu."$ $  Sylvia neochotne súhlasila, pretože, zas, Wymyslensko by mohlo vypočúvať Loviisu a to by nemohla dopustiť.

"$ $Zbaľ si veci, čo najrýchlejšie."$ $ 

"$ $Berieš ma?"$ $ 

"$ $Ako vidno, musím."$ $  Uznala Sylvia.

"$ $Idem. Odomkneš mi?"$ $ 

Po niekoľkominútovom kontrolovaní, či na chodbe nikto nie je, Sylvia odomkla dvere a stihla pripomenúť, aby klopala. Loviisa mala veci v balíku, takže sa moc baliť nemusela. O pätnásť minút zaklopala a ohlásila sa u Sylvie. Tá jej odomkla, a keď bola vnútri, zamkla.

"$ $Nedotýkaj sa guľôčok na dlážke!"$ $  Zakričala Sylvia a Loviisa sa strhla, pretože jednu práve chcela chytiť do ruky. "$ $Sú naprogramované na explóziu, takže sa ich nechytaj!"$ $  Ešte dodala. Guľôčky boli zošúverené a trochu žiarili. Sylvia vysvetľovala.

"$ $Vydržia ešte okolo niekoľkých mesiacov, potom explodujú sami od seba. Tak pozor."$ $ 

"$ $Ako ich plánuješ prenášať?"$ $ 

"$ $Nedotknem sa ich, budú v kufríku."$ $ 

"$ $Tam neexplodujú?"$ $ 

"$ $Nie, fungujú na princípe teploty a dotyku. Kufrík je izolovaný."$ $ 

"$ $Ako si to urobila?"$ $ 

"$ $Veľmi zložitá mágia."$ $ 

"$ $Aká?"$ $ 

"$ $Nie je čas. Vieš si zmenšiť veci?"$ $ 

"$ $Viem."$ $ 

"$ $V poriadku, zmenši si tie najpotrebnejšie veci, ja napíšem Mrane, že nás má nasledujúcich dvadsaťštyri hodín čakať."$ $ 

"$ $Ideme teda k nej... Kde býva? A vie, že si...?"$ $  Vrhla na ňu škaredý pohľad. Sylvia nenávidela to kým je, a spomínanie toho rovnako.

"$ $Od prvého zoznámenia. Prekukla ma. A býva vo Felanzii v jednom z menších miest. Má svoj vlastný dom a pracuje ako majiteľka softvérovej firmy."$ $ 

"$ $Si si istá?"$ $ 

"$ $Verím jej. Ak príde k najhoršiemu, budem bojovať. Občianskym menom sa volá Keria Oetová."$ $  Sylvia si vypla notebook a zmenšila ho. Následne zopakovala proces so svojimi knihami a kufríkmi. Jej zbrane zo školy mala v osobitnej taške. "$ $Máš zbrane Loviisa?"$ $ 

"$ $Tu nie."$ $ 

"$ $Problém, nejaké ti požičiam..."$ $ 

"$ $Zbrane si privolávame, ale nie som si istá, či by zvládli takú veľkú vzdialenosť."$ $ 

"$ $Nemôžeme to skúsiť, aby to nezbadali."$ $ 

"$ $Máš pravdu."$ $  Sylvia mala všetky svoje kufríky zmenšené a poskladané na sebe. Loviisa svoj práve úspešne zmenšila a hodila si ho do vrecka.

"$ $Vieš robiť štít?"$ $  Loviisa prikývla.

"$ $Solan?"$ $  Prikývla. "$ $Laser?"$ $ 

"$ $Nie. U nás je to zakázané, pretože je to nebezpečné. Ak to nie je v ohrození života."$ $ 

"$ $Tak bojuj Solanom a vždy si udržuj štít."$ $  Zadávala bojové inštrukcie Sylvia.

"$ $Budem."$ $ 

"$ $Vyjdeme odtiaľto na krídlach. Mám tu ukryté dva nanoplášte, prispôsobujú svoj vzhľad okoliu. Nasaď si krídla."$ $  Krídla boli zmenšené a položené pod dvoma kuframi. Sylvia svoje zväčšila, vykukla podvedome z okna a vydýchla.

"$ $Máme ďalší problém. Je tu už polícia. Našťastie, ako vidno nie sú tu najlepší zmysloví mágovia, tak sa mi ich podarilo bloknúť. Asi budem musieť..."$ $  Tvár sa jej skrivila odporom...

"$ $...premiestiť sa."$ $  Dohovorila za ňu Loviisa.

"$ $Asi áno."$ $  Nešťastne prisvedčila. "$ $Krídla si, ale nasaď. Nemám dostatočnú prax a možno sa premiestime do vzduchu, nemám prax, a ani ju mať nechcem."$ $  Chcela ešte niečo povedať, ale vtom... "$ $Počujem kroky!"$ $  Schmatla zmenšené kufríky a dala si ich do tašky ktorú mala v ruke. Tam kde boli predtým kufríky sa leskol zlatý panáčik, vyzeral ako nejaká hračka. Loviise sa podobal na pozemských vojačikov, ak nerátame to, že bol väčší a bol z kovu a diamantov.

"$ $Čo to je?"$ $ 

"$ $Neviem! Zober to a chyť sa ma."$ $  Naliehala, a ani sa nepozrela čo Loviisa myslí. Niekto vchádzal na chodbu. Loviisa v jednej ruke zvierala panáčika a druhou sa chytila Sylvie vo chvíli, keď zas použila svoje schopnosti.

\begin{center}

*

\end{center}

Fequel bol členom jednotky už desať rokov. Keď už nebol podľa majora schopný na misie v teréne D a E, poslali ho späť do mesta. Jeho misia na dnes bola jednoduchá, chytiť a umlčať Mänchenovú. Nikdy ju nikto nepovažoval za nebezpečnú pre vnútornú bezpečnosť, až kým dnes neprerušila Cecíliin prejav v Kralovskej strednej Všetko škole. Bola pravda, že vždy nebola prowymyslenská, ale takto nie. Každý, kto propagoval spoločenstvo propagoval démona. A propagácia D bola trestná. Veď Démon ich chcel zabiť.

Dvere boli zamknuté, ale on ich rýchlo otvoril a zneviditeľnený, s použitím zmyslovej mágie na dvere, aby sa zdali ako zatvorené, a aby nikto nepočul zvuky vošiel do miestnosti. Bola prázdna. Snažil sa zistiť, či niekto nepoužíva zmyslové kúzla, ale nič nevyrušil. Miestnosť bola prázdna. Celú ju presvietil Solanom a nič. Žiadny dôkaz. Ak by tam niekto bolo, Solan by ho zasiahol, pretože štít by vyrušil. Ledaže by to bol nejaký extra silný zmyslový mág čo bolo vylúčené. Ak niekto prejavoval veľké známky talentu, bol sledovaný. A Mänchenová nikdy... Mohla to síce prirodzene zakrývať, ale... Vytiahol kameru a premietol si obraz. Nikto. A technika sa oklamať nedala a on necítil žiadnu čerstvú mágiu. Rozhodol sa, že ju odmeria neskôr. Prehliadol si všetky izby tak, aby ho nezbadali. Bola tam len už končiaca stredoškoláčka Milka Capková, dcéra Registra Wymyslenskej federálne republiky (register je niečo ako minister zdravotníctva). Nič podozrivé. Ďalej niekoľko prázdnych izieb a na konci chodby izba Mokrej-Plavčíkovej. Tá bola podľa databázy sledovaná, ale pred dvoma dňami zmizla. Veľmi záhadne. Možno to malo niečo s Mänchenovou... Niektorí si ju spájali s Spoločenstvom M. Ak to bola pravda, bola nepriateľom štátu. Fequel sa rozhodol, že keď chytí Mänchenovú, získa záznamy Mokrej-Plavčíkovej. A presadí jej sledovanie. Predsa len zmiznúť, bolo podozrivé. A to platilo dvojnásobne, ak to bol potenciálny nepriateľ. V izbe nikto nebol. Žiadne stopy po Mänchenovej. Posledná izba patrila novej, získanej v operácii O+ v Fínsku. Fequel premýšľal, prečo zachránenú cez akciu O+ nechali bývať na tomto nebezpečnom poschodí. Pri Mänchenovej a Mokrej-Plavčíkovej! Preskúmal izbu. Nikto tam nebol. Skener DNA zachytil stopy zachránenej. Končili ako stopy, ktoré boli najnovšie v izbe Mänchenovej. Tam na jednom mieste nebolo nič. Zatelepatizoval na veliteľstvo.

"$ $Zmizla, Mänchenová je preč."$ $  Oznámil.

"$ $Netrepte somariny, kde je?"$ $ 

"$ $Zmizla, Nikde nie je. Mám podozrenie, že zmizla spolu s novou z akcie O+."$ $ 

"$ $Räkkänová?"$ $  Fequel sa pozrel na dvere a prikývol.

"$ $Áno."$ $ 

"$ $Nemohla len tak zmiznúť! Ste si istý, že neodletela alebo nepoužila zakrívače DNA?"$ $ 

"$ $Senzory nič neodhalili."$ $ 

"$ $A ako podľa vás zmizla?! Mänchenová je predsa hrozba pre národnú bezpečnosť, nemôžete ju nechať zmiznúť!"$ $ 

"$ $Teraz ma niečo napadlo, čo ak Mänchenová bola démonka, teda D, alebo polodémonka, ako sa v jej spise píše, že sa dostala do Wymyslenska?"$ $ 

"$ $Prišla tu, proste prišla. A, síce je vaše vysvetlenie zmysluplné, prečo by tu bol D a zmizol, to nedáva logiku Fequel! Okamžite príďte na veliteľstvo, ste degradovaný na radového vojaka, pošlem tu jednotku, čakajte na ňu."$ $ 

"$ $Dovoľte mi ju nájsť."$ $  Veliteľstvo sa chvíľu odmlčalo a po chvíli sa zas ozvalo.

"$ $V poriadku, ale ak ju nájde naša jednotka skôr ako vy, ste degradovaný, ak ju nájdete vy, ste povýšený, ak zomriete, je to vaša vina. Prijímate podmienky?"$ $  Fequel nerozmýšľal.

"$ $Áno."$ $ 

"$ $ Pošlem jednotku. Nájdite ju Fequel Caposlav."$ $ 

\begin{center}

*

\end{center}

Ako prvé sa rozhodla nájsť Nielu. Tá jej dôverovala. Morja nevedela kde má Nielu hľadať, pretože sa rozdelili a terajšia poloha Niely jej nebola známa. Systematicky hľadať neprichádzalo v úvahu pre vysoký počet miest kde sa mohla Niela nachádzať. Použitie telepatiónu neprichádzalo v úvahu hlavne preto, že Morja telepatión momentálne nemala a Niela ho nepoužívala. Keby chcela si byť stopercentne istá možno by jej pomohla jedna z veľkých kníh, hlavne preto, že poznala jazyk v akom boli napísané. Premýšľala o agentúre, ale to nepripadalo v úvahu. Napokon Morja dospela k tomu, že nemá nič a kedykoľvek ju môžu napadnúť. Teda zamierila do najbližšieho obchodu, ktorý patril k spoločenstvu. KKKv.

\begin{center}

*

\end{center}

Proroctvá sa nedali vyčítať bez znalosti jazyka, ale kniha ich automaticky negenerovala. Boli už predtým vyslovené, každé jedno, kniha ich len odhaľovala. Podľa nejakého záhadného kľúča, na ktorý ešte nik neprišiel. Niela by najradšej bola, keby tam bola Morja, ktorý by jazyk už mohla vedieť.

Vyvolávať proroctvá bolo nemožné, a tak jediné čo Niele zostávalo, bolo spoliehať sa na knihu. Bola veľmi hrubá. V koženej väzbe, ktorá nevyzerala, že by sa na nej nejako prejavil čas. Na jej obálke bol vytlačený ornament, nejaké písmeno z piktopísma, v tvare zvlneného hada pretnutého napoly spojeného tenkou čiarou. Položila ruku na knihu. A tá sa rozžiarila. Celá bola zlatá a vychádzal z nej hlas.

"$ $Príde, príde proroctiev koniec,

Spália sa na uhoľ,

Zostarnú,

A kým tri kovy budú jasnúť,

Dovtedy nie je bezbranný,

Tajomstvá sa odhalia,

A možno bude stáť nad,

A kto bude tým predmetom,

Dozvedia sa snáď."$ $ 

Žiara vyprchala. Niela si proroctvo zaznamenala, ale nepochopila mu. Kniha predpovedá svoj koniec, alebo niečo také? Bolo to jasné: "$ $Príde, príde proroctiev koniec, Spália sa na uhoľ, Zostarnú,"$ $ . Toto proroctvo hovorilo o konci proroctiev, hoci tie vedeli byť záludné, a to si Niela uvedomovala. Bolo veľmi pravdepodobné, že sa jednalo o metaforu alebo len niečo, čo sa stane po splnení niečoho. Časti o troch kovoch neporozumela. Keď bola v agentúre ešte ako teoretička, mala prístup do databázy spísaných proroctiev, pretože agentúra všetky nahrané proroctvá zaznamenávala. To isté aj Wymyslensko, ale oba archívy neboli spojené, a tak vo wymyslenskom mohlo byť niečo o kovoch, pretože Niela si nič také nepamätala, ale ani ona samozrejme nepoznala všetky proroctvá. Keby mala teraz prístup do ich databázy... analyzovala ďalej. Časť o tajomstvách (5-6 verš) odhalila len to, že bude niečo odhalené nad. Chýbal jej v tej vete predmet. Nad čím budú stáť, a hlavne, kto? Či zas sa jedná o metaforu? Kniha veľa metafor nepoužívala, väčšinou len tajomné tvrdenia, ktoré sa napokon rozriešia. Posledné dva verše hovorili niečo o predmete. A ten predmet bol niekto. Kto je predmet, to bolo záhadou, podľa proroctva by sa to mali dozvedieť. Ale kto a kedy?

Niela zas položila ruku na knihu, v nádeji, že jej kniha vypľuvne ďalšie proroctvo. Ale bola to hlúposť. Kniha nefungoval na povel. Bez znalosti piktopísma sa bolo možné spoliehať len na nádej. A to bolo trochu, či viac zlé.

Nič. Kniha ostala chladná a nezmenená. Niela pochopila, že má dve možnosti, ktoré jej pomôžu. Buď sa dostane archívu, alebo nájde Morju. Pretože agentúra jej veľmi naklonená nebola, rozhodla sa pre druhú možnosť.

\begin{center}

*

\end{center}

Premiestnili sa do Felanzíjskeho lesa. Loviisa zvierala v jednej ruke panáčika a druhou sa stále držala Sylvie.

"$ $Pusť ma prosím."$ $  Požiadala ju.

"$ $Bolo to zvláštne."$ $  Skonštatovala Loviisa.

"$ $Už to nikdy nespravím. To bolo naposledy."$ $  Vyhlásila Sylvia pevne. Tak ako to už raz urobila, spomenula si.

"$ $Nič nehovorím."$ $  Povedala Loviisa, pretože sa jej nechcela nijako dotknúť. Sylvii vtedy padol zrak na panáčika.

"$ $Čo to je?"$ $ 

"$ $To čo som zobrala."$ $ 

"$ $Nie je to nejaká bomba alebo pasca?"$ $  Nedôverčivo si to Sylvia prehliadala.

"$ $Pusť to."$ $ 

"$ $Prečo?"$ $ 

"$ $Idem zistiť toho nezávadnosť."$ $  Panáčik spočinul na zemi a Sylvia použila zmyslovú mágiu. Nič. Zobrala si spoza opasku nejakú paličku a niečo urobila. Paličkou sa dotkla panáčika a nič sa nestalo.

"$ $Načo to robíš?"$ $ 

"$ $Keď sa dotkne naprogramovaného objektu, zapípa."$ $ 

"$ $Aha."$ $  Prikývla.

"$ $Tento nemá v sebe zabudované kúzlo. Preverím ho po technickej stránke."$ $  Paličku si zas dala za opasok a zobrala ďalšiu sa znova sa ňou dotkla panáčika. Zas nič. "$ $Vyzerá, že je v poriadku."$ $  Chytila ho a zistila, že má odnímateľné topánky. Nič sa zas nestalo. "$ $Detská hračka."$ $  Podala ju Loviise, ktorá sa za ním z nejakého zvláštneho dôvodu natiahla.

"$ $Tadiaľto."$ $  Ukázala na malý chodníček. "$ $Je možné, že nás hľadajú, použi zmyslovú mágiu."$ $  Sylviu sa premenila a Loviisa tiež. Tá sa hrala cestou s panáčikom. Zistila, že má odnímateľnú hlavu. Len tak, ju odňala. A vtom z krku vystrekol Laser, ktorý, ale nepresekol Loviisu, ale zastal pri jej ruke. Tá bola mierne zmätená. Sylvia zastala a vyjavene sa pozerala na panáčika.

"$ $To je lasermeč, jediný na svete, už viem prečo nefungoval pri mne. Je tvoj. Zatvor ho prosím."$ $  Fascinovane vyhlásila Sylvia.

"$ $Prečo nefungoval pri tebe?"$ $ 

"$ $Zatvor ho! Okamžite, prosím! Prezradí nás."$ $  Na Loviisinej ruke laser zastal a bezpečne ho zatvorila. Sylvia začala vysvetľovať.

"$ $Nie som majiteľkou meča, ten si vybral teba. Teda, ho nemôžem používať. Vždy keď majiteľ zomrie, meč sa premiestni k osobe, ktorá ho podľa jeho uváženia najviac potrebuje. Netuším ako meč myslí, ale podľa mňa je to nejaké zložité programovacie kúzlo, a keď myslím zložité, tak naozaj zložité. Podľa mňa tam sú použité nejaké iné programovacie jazyky ako MPL."$ $ 

"$ $Čo je MPL?"$ $ 

"$ $Mrana, teda Keria ti to vysvetlí. Volaj ju radšej Keria, Mrana je internetová prezývka."$ $ 

"$ $Aha."$ $  Loviisa si dala lasermeč do vrecka a išla za Sylviou. Po pätnástich minútach chôdze došli k opevnenému domu na okraji mesta. Nebol svojím strážením až tak veľkou výnimkou, pretože z piatich domov, na ktoré videli, mali opevnenie tri.

Wymyslensko nemalo za svoju existenciu nikdy úplne ružovú bezpečnostnú situáciu a možno z výnimkou pár rokov sa na štát spoľahnúť nedalo. V každom štáte mali obyvatelia iné bezpečnostné opatrenia. Od veľmi benevolentnej Fentenzíjskej kultúry až po tvrdé bezpečnostné opatrenia na Felanzii. Tá mala dosť pohnutú históriu, a tak sa jej obyvateľom moc nečudovali. Na Felanzii prebiehali posledné vojny, posledné zvyšky Tretenskej ríše, a keď sa k tomu dajú aj nehostinné podmienky, tak tam obyvatelia získali nejakú vrodenú paranoju.

Múry obopínajúce dom Kerie mali jednu malú bráničku, zabudovanú v múre, ktorá mala v sebe zabudovaný skener očí, DNA a otlačkov prstov. Zvonček mala tiež, pretože všetky údaje boli nakonfigurované pre Keriine údaje. Sylvia zazvonila a zasvietila malá obrazovka na ktorej sa zjavili pokyny.

"$ $Identifikujte sa."$ $  Sylvia priložila ruku k čítačke otlačkov prstov a skener jej presvietil oko a na obrazovke sa objavil nápis Čakajte. Bránička sa otvorila a vošli do úzkej chodbičky, zvukovo izolovanej od ostatného sveta.

"$ $Zvuková identifikácia, varovanie, je tu filter mágie."$ $ 

"$ $Ohlás sa po mne aj ty."$ $  Šepla Loviise Sylvia.

"$ $Sylvia Mänchenová, hlásený návštevník a ešte jeden nehlásený."$ $ 

"$ $Loviisa Räkkänová."$ $ 

"$ $Moment..."$ $  Ozval sa robot. Presunuli sa, ani nevedeli ako, do prázdnej miestnosti.

"$ $Vieš čo to je?"$ $  Spýtala sa Loviisa.

"$ $Neviem. Keria hovorila o opatreniach, ale u nej som ešte nebola. Keby niečo, vytiahnem môj kufor."$ $  A po chvíli zamyslenia Sylvia dodala.

"$ $Myslím, že mi to nehovorila kvôli Hlagenovke. Keby sme sa dostali do hlagenovho pásma."$ $ 

"$ $To je to pásmo kde..."$ $ 

"$ $Kde sa zaznamenávajú tvoje myšlienky a požiadavky a na základe nich sa mení priestor. To je presná učebnicová definícia. Sú dobré, ak sú na veci, ako menenie typu sedadla v mačičkobuse. Ale povráva sa, že ich Cecília používa na špehovanie disidentov. Preto Keria nechcela, aby som vedela všetko. Myslím, že Keria vytvára vo firme programy napojené na Hlagenovku."$ $ 

"$ $Programovanie sa dá robiť v kombinácii s mágiou?"$ $  Loviisa síce žila vo svete spoločenstva a teda aj mágie, ale bola skôr pasívnou osobou. Programovanie bolo pred pár dňami na jej zozname priorít niekde úplne vzadu.

"$ $Áno. Na tomto príklade fungujú telepatióny. Ono to je vlastne pozostatok programovacích kúziel, ale tie sú zložité na sústredenie. Preto existuje TPL. Je o niečo ťažší ako bežné programovanie pre dotýkanie sa mágie, ale nemusíš sa naň sústrediť, ako keď programuješ v MPL."$ $  Keď Sylvia dohovorila vetu, otvorili sa dvere a v nich stála Mrana. Loviisu to trochu zaskočila, pretože čakala viac bezpečnostných opatrení. V časoch keď začínalo spoločenstvo, ešte bez armády, bola v ňom bezpečnostná panika.

"$ $Ahoj Sylvia. To je Loviisa, ktorá sa dostala do Wymyslenska v rámci O+?"$ $ 

"$ $Áno."$ $ 

"$ $Verím ti... ale netvrdila si, že prídeš do dvadsaťštyri hodín?"$ $ 

"$ $Malé komplikácie."$ $ 

"$ $Jasné. Musela som vás preveriť Hlagenovkou, aby nedošlo k ničomu. Na teba sa... no nemôže D ani nikto iný zmeniť, ale..."$ $  Spýtavo hľadela na Sylviu, či môže pokračovať, lebo hovorila o veci, na ktorú bola Sylvia veľmi citlivá.

"$ $Loviisa o tom vie. To, že som... Démon."$ $  To posledné slovo vyhlásila veľmi ticho, tak, že ho takmer nebolo počuť.

"$ $Myslela som si, že ..."$ $ 

"$ $Ja viem čo. Ale musela som urobiť, veď vieš čo, aby sme unikli."$ $ 

"$ $Tvrdila si predsa..."$ $ 

"$ $Ja viem. Bolo to naposledy."$ $  Mierne zvýšila hlas.

"$ $V poriadku. Poď za mnou."$ $  Mrana akceptovala to, že Sylvia o tom, že je Démonka, nechcela ani počuť.

Keria išla do obývačky, kde nejaký robot písal na počítači program a ďalší práve dorábal obed.

"$ $Píše to za teba, vás..."$ $  Loviisa nevedela ako má Keriu osloviť.

"$ $Tykaj mi."$ $ 

"$ $V poriadku."$ $  Keria pokračovala.

"$ $Je to spravené cez Hlagenovo pole. Nemusím písať, ale stačí mi mať v hlave to, čo chcem, aby to napísalo a ono sa to napíše. Včera som to spravila, a tak to skúšam."$ $  Hovorila a vtom sa spolovice prekvapene a spolovice fascinovane zahľadela na lasermeč. "$ $Ty máš..."$ $ 

"$ $Lasermeč. Dnes som ho získala."$ $ 

"$ $Fascinujúce. Ukážeš mi ho prosím."$ $ 

"$ $Nebude ti fungovať. Funguje len pri majiteľovi, a tomu neublíži. Teda pri mne."$ $  Povedala Loviisa. Mrana si skúmavo lasermeč prezerala.

"$ $Vyzerá ako hračka."$ $ 

"$ $Je to maskovanie. Nehovor, že si nikdy nečítala sto zaručených spôsobov ako sa stať lupičom, kapitola 26. Predmety magického využitia, odsek osem. Alebo Magické predmety, ktoré ste nikdy neprogramovali, kapitola trinásť."$ $ 

"$ $Ty si pamätáš celé tie knihy?"$ $  Nedokázala to pochopiť Loviisa.

"$ $Len približne. Viem kde mám tie veci hľadať. Keby som ich naraz potrebovala."$ $ 

"$ $Ja používam indexy, ktoré som naprogramovala v hlagenovom poli."$ $  Pripojila sa Mrana.

"$ $Tak to hej, ale keď nerobím ešte s poliami. V čítačke som sa pokúšala niečo spraviť, tam je batéria, ale normálne... nechcem si po energickom kolapse zničiť knihy."$ $ 

"$ $Ty tu máš knihy?"$ $ 

"$ $Logicky. Zmenšené samozrejme. Niektoré sú vzácne a nemohla som ich nechať Wymyslensku, pretože nasvedčujú mojej nelegálnej činnosti."$ $ 

"$ $Hlagenovým poľom knihy nezničíš."$ $ 

"$ $Viem, ale s ním som ešte nepracovala. Pracujem s MPL."$ $ 

"$ $To je ťažšie. Ale je v tom viac možností."$ $ 

"$ $Samozrejme, pretože TPL funguje s osobami a ich mysľou. Takto nefunguje MPL. TPL je ako zmyslová mágia."$ $ 

"$ $Veď Telepatia teoreticky do zmyslovej mágie patrí. Inak, čo tu máš za knihy? A s tou nelegálnou činnosťou, to si myslela Sto zaručených spôsobov ako sa stať lupičom?"$ $ 

"$ $Nie len to. Mám tu nejaké texty od Ló, Zacaríasa, od Oka I. až XVII. A nejakých pozemských filozofov a čínskeho teoretika vojny. Ďalej tu mám všetky knihy od Nestela Sargata, Kritiku moci od Ajsie Ašanovej. A nejakú poéziu. A asi poslednú papierovú knihu tu mám text od istej autorky, ktorý je vo Wymyslensku neoficiálne zakázaný a to Prekliatie tejto krajiny. V čítačke mám... chceš vedieť?"$ $  Zasmiala sa, rovnako ako Mrana.

"$ $Ďakujem. Teraz nie. Potom, keď sa tu nejako ubytujete a môžeme sa najesť. Inak Sylvia, požičiaš mi toho teoretika vojny a Kritiku moci? Ajsiine knihy sú dobré, ale nikde na internete nie sú zdigitalizované. A to som hľadala cez hlagenovo pole, ale nikde a to nikde to nie je."$ $ 

"$ $Ty robíš v svojej nelegálnej sfére aktivít s hlagenovkou?"$ $ 

"$ $Internetovou hlagenovkou. Je to špeciálny typ vírusu čo vytvorí okolo daného počítača v sieti hlagenovo pole."$ $  Spresnila.

"$ $Hlagenovky sa dajú robiť aj cez počítač? Myslela som si, že počítač ich len udržuje."$ $ 

"$ $No, prišla som na spôsob, ako počítačom vytvoriť hlagenovku, ale ono to je vcelku zložité, pretože to sú hlavne dáta, a už na čistej úrovni elektróny a trochu vyššie NORy, XORy, NANDy a iné."$ $ 

"$ $To viem. Ale mágia je predsa na čisto bunkovej a energetickej úrovni."$ $ 

"$ $A to funguje na alelách, molekulách a fotónoch. Nedokážeme ju vytvoriť bez tej správnej kombinácie alel, ale dokážeme ju naprogramovať. Ono to umelo vytvorené hlagenovo pole nie je ani mágia, skôr je to niečo na nanoúrovni a mágia je vo víruse. Tak som získala všetky tie dokumenty."$ $ 

"$ $Jasné. Dobrý nápad."$ $ 

"$ $Inak čo si mi včera písala o tej krádeži, to sa asi..."$ $ 

"$ $To záleží na nej."$ $  Ukázala na Loviisu ktorá sa zasekla v ich rozhovore niekde na druhej vete. "$ $Sú to Loviisine spomienky. Ale teraz sme v nebezpečenstve a ja veď vieš čo využívať nebudem."$ $  Chvíľu bolo ticho a potom ho prerušila Loviisa.

"$ $Nepovedala si Keria, že... no, že by sme sa mali ubytovať?"$ $ 

"$ $Samozrejme, prepáč Loviisa, ale hlagenovky sú veľmi zaujímavé, inak Sylvia, celý princíp ti ukážem."$ $ 

Keria vstala a keď boli všetci na nohách zaviedla ich na druhý koniec chodby, kde boli dve postele a to je všetko.

"$ $Je tam hlagenovo pole, teda hlagenovka."$ $  Dodala. Sylvia si tam dala všetky kufríky a veci čo mala stále v rukách. Loviise sa cez hlagenovku vytvoril písací stôl kde sin zložila pár vecí čo mala. Lasermeč si nechala vo vrecku a bundu si odložila do novovzniknutej skrine.

"$ $Hlagenovka je fascinujúca najmä v tom,"$ $  povedala Sylvia pomedzi vyberania a zväčšovania si svojich kníh. "$ $Že ti objekty ostanú až po zmenu myšlienky, totiž je v nej minimalizovaná entropia."$ $ 

"$ $Pochopila som dve slová."$ $  Zasmiala sa Loviisa.

"$ $To si v spoločenstve M nič nerobila?"$ $ 

"$ $No..."$ $  Zdráhavo začala Loviisa. Nechcela moc, aby Sylvia vedela čas svojho doterajšieho života bola s kamarátmi von a dokopy skutočne nič nerobila. Aspoň podľa Sylviinho mienenia, keď niečo robiť znamenalo programovať, učiť sa nejakú fyziku, či dačo aké, plánovať revolúciu a písať články. "$ $Áno."$ $  Priznala.

"$ $Takže čas čo tu strávime, by bol najlepšie využitý tvojím vzdelávaním. Hlavne keď máš teraz lasermeč. Máme zraň čo oni nie."$ $ 

Prečo? Napadlo Loviisa. Prečo chcela Sylvia stále robiť len toto...?

"$ $Ty chceš vojnu."$ $ 

"$ $Hovorila som revolúciu."$ $ 

"$ $Možno prídu obe."$ $ 

"$ $Možné to je. Ale je isté, že Cecília tu vládne až pridlho."$ $ 

"$ $O čom hovoríš? Myslela som si, že ti urobila škrt cez rozpočet. Nám."$ $ 

"$ $To je pravda. Ale nič neostane len tak. Vieš čo je efekt motýlích krídel?"$ $ 

"$ $Ani nie. To je s tým hurikánom a motýľom, že?"$ $ 

"$ $Presnejšie je to niečo, že niečo tak malé, ako je trepotanie motýlích krídiel, môže v konečnom dôsledku vyvolať tornádo, napríklad aj niekde na druhej strane Zeme."$ $ 

"$ $Aha."$ $  Pokúšala si to Loviisa usporiadať si v hlave. "$ $Tak my mávneme krídlami, a..."$ $  Nadýchla sa a vydýchla. Medzitým začala zas rozprávať Sylvia.

"$ $My spôsobíme revolúciu, pretože moc sa nemôže udržať. Pretože moc je ilúzia väčšej slobody a jej obmedzenia. Je konečný počet slobody a my nechceme spôsobiť jej skladovanie. Nestel Sargat, Tlejúci oheň, strana stoprvá."$ $  Loviisa hľadela na Sylviu s otvorenými ústami. Knihu Tlejúci oheň mali ako povinné čítanie pred rokom, ale pamätala si maximálne jej zápletku, ktorej aj tak nepridávala veľký význam.

"$ $Ty si to všetko pamätáš? Ako to...?"$ $  Loviisa bola veľmi prekvapená.

"$ $Normálne. Prečítam si knihu, vidím ju, pamätám si ju. Ty nie? Tulienka Deľa toto vedela s jazykmi."$ $ 

"$ $Všetci nie sme géniovia."$ $ 

"$ $Nehovor. Alebo po fínsky, Älä sano."$ $ 

"$ $Vieš po fínsky?"$ $ 

"$ $Trochu sa na mňa nalepilo z toho čo sa učila Tulienka Deľa."$ $ 

"$ $Ona vie po fínsky?"$ $ 

"$ $Hej, ale lepšie sa je pýtať čo nevie, lebo takto sa budeš pýtať veľmi veľa."$ $ 

"$ $Teda čo nevie?"$ $ 

"$ $Hovoriť štyrmi jazykmi. Strašne ju to štve a odmieta použiť prekladače."$ $ 

"$ $Čo za jazyky?"$ $ 

"$ $Klikavčina, Maďarčina, Lapončina a jeden ázijský jazyk, ale neviem aký."$ $ 

"$ $Ty aj niečo nevieš?"$ $ 

"$ $Tak áno. Napríklad to prečo mi nevychádzajú moje rovnice o vzniku vesmíru."$ $ 

"$ $What?"$ $ 

"$ $Nepasujú mi do nich viaceré veci. Mágia, démoni, anomália medzi hmotou a antihmotou..."$ $ 

"$ $Ty pracuješ na rovniciach o vzniku vesmíru?"$ $ 

"$ $Áno, už dlhšie. To nebol sarkazmus."$ $  Povedala úplne vážne.

"$ $Aha, zdalo sa to."$ $ 

"$ $Ach tak."$ $  Loviisa chcela dostať debatu do, pre ňu, zrozumiteľnejšej roviny. "$ $Čo ideme robiť..."$ $  Toho sa chytila Sylvia.

"$ $Ideme za Mranou? Chcem vidieť tú hlagenovku."$ $ 

"$ $Jasné."$ $  Prikývla Loviisa, tak trochu nezúčastnene.

\begin{center}

*

\end{center}

Fequel premýšľal čo má robiť. Ak je Mänchenová démonka, teda D, stopy po sebe zmyť dokáže všade. On slúžil Cecílii a bezpečnosti štátu. Nájsť Mänchenovú je jeho povinnosť. Zlyhanie si nepripúšťal. Notebook, ani iný údaj o nej, tam Mänchenová nenechala. Logicky. Mänchenová pochádzala odnikadiaľ, jej pôvod bol neznámy, alebo skôr neuvedený. Fequel sedel v mačičkoese a prezeral si jej spis. Ešte nebol aktualizovaný. Otvoril vyhľadávač a zadal tam meno Mänchenovej. Preskočil výsledky v spise už zahrnuté a prešiel na súkromnú korešpondenciu. Vedel, že sa k nej nedalo dostať. Mala ju blokovanú. Podobný príklad ako D. Ďalšia podobnosť. Napadlo ho. Nezobrazoval sa ani text správy, ale ani príjemca, čo bežné nebolo, pretože ich systém bol pomerne prepracovaný od informatikov z agentúry na ktorých sa spoľahnúť dalo. Cez tento systém našli niekoľkých nepriateľov Wymyslenska, ktorí ho mali za úlohu zničiť a rozložiť zvnútra. Len Mänchenová bola až príliš tvrdým orieškom. Ak je démonkou, čo je nanajvýš pravdepodobné, je to D, alebo jeden z jeho polodémonov, či už patriacich aj ku spoločenstvu, alebo nie. Ale Spoločenstvo s D aj tak spolupracovalo, takže to bolo jedno. Tri správy boli nové, z tohto dňa. Veľkosť správ sa pohybovala na úrovni niekoľko desiatok až stoviek telebajtov. Mänchenová zrejme čosi tušila o ich systéme na odšifrovanie správ či skladanie správ z tkz. telepatickej stopy. Jeho senzory práve zachytili hlagenovo pole. Vychádzalo z počítača. Niečo nezvyklé. Pole... Ale to bolo predsa nemožné! Telepatické polia sa nedali predsa urobiť strojom, museli byť vytvorené niekým. Alebo bola vytvorená tak zložitá technológia, ktorá dovoľovala vytváranie poľa na diaľku, alebo bez prítomnosti mágie? Fequel vzorku údajov o poli odoslal centrále. Do druhého, paralelného vyhľadávača osôb z registra, teda osôb o ktorých mali spis, teda drvivá väčšina obyvateľov Wymyslenska a niekoľkých zo spoločenstva, zadal kľúčové slová Hlagenovo pole, Programovanie, TPL, Informatika, Magické technológie a iné slová, ktoré boli otagované v systéme a viazali sa na hlagenovo pole. Ako prvé nechal filter zobraziť osoby pod dohľadom. Otvoril údaje a zadal ich do programu, ktorý vyhodnocoval pravdepodobnosť.

\begin{center}

*

\end{center}

Jej telepatický okruh Morju Tlogenovú nenašiel ani v okruhu sto kilometrov. Ak si nechcela nechať vybuchnúť hlavu, čo hrozilo pri používaní telepatie bez telepatiónu na neúmerne veľké vzdialenosti, musela s tým prestať. Otázne bolo, že či je Morja vôbec v jej dimenzii. Premiestniť sa mohla s doskou úplne kdekoľvek. Ak niečo, ale vychádzalo pravdepodobné, tak to bolo to, že sa Morja premiestnila do jedného z bájnych miest, v ktorých plynie čas inak ako na zemi. Napríklad vír bolo mýtické miesto, kde vraj zmizla jedna z kňažiek miesta osudu, ako nazývali útočisko knihy osudu. Ozaj kniha osudu, Niela ešte keď bola členkou agentúry, žiadala väčšiu ochranu knihy. Lenže väčšina ľudí neverila ani na jej existenciu. To, že sa D zameriaval zatiaľ len na proroctvá, to neznamenalo, že svoj okruh rozšíri aj ďalej. Prešla cez pár kontrol z agentúry, za občasných pohľadov a zamierila von. Prešla poslednou bezpečnostnou kontrolou a práve vo chvíli keď chcela odletieť, uvidela Arabelu Tlogenovú.

\begin{center}

*

\end{center}

"$ $Sme v Nórsku."$ $ 

"$ $V poriadku."$ $  Odtrhla zrak od knihy Rosa. Bola skorá noc a jej sa ešte nechcelo spať. Bola síce pravda, že už vyše dvadsať hodín nespala, ale bola na spánok až priveľmi zaujatá knihami.

"$ $Tu nasledujúcich pár dní bývame,"$ $  povedala Chen, keď vošla do trojizbového bytu. Rozložila interaktívnu knižnicu kde si Rosa hneď sadla. "$ $Potom choď aj spať."$ $  Rosa sa usmiala.

"$ $Samozrejme, len musím toho veľa dohnať."$ $ 

"$ $Neboj sa, ak si za čas čo tie knihy máš, stihla už prečítať Algoritmy a Logaritmy, Použitie látok do elixírov a ich vlastnosti II. a Teóriu Boja, tak to zvládaš, a ak si si to navyše aj zapamätala, tak budeš o chvíľu lepšia ako tvoji rovesníci, čo to študujú od začiatku."$ $ 

"$ $To je málo. Je to všetko len teória."$ $ 

"$ $To je mi jasné. Ak chceš prax, bude ti poskytnutá. V boji ti môžem pomôcť. Jednu dobu som predávala a vyrábala maloobchodnú sieť na výrobu zbraní. Aj skúšky v polovici štúdia, ktoré sa u nás berú ako niečo ako skúšky na strednú školu, som mala z boja."$ $ 

"$ $To je v poriadku. Ďakujem."$ $  Rosa sa usadila sa kresla a pokračovala v čítaní. Chen situáciu ohodnotila ako prijateľnú a vrátila sa ku práci.

\chapter{Kovové zjavenie}

Na zemi pršalo. Pauline sa premiestnila ku domu patriacemu Lietavým. Tarny odovzdal mačičkoes a Bella im odovzdala to čo sľúbila. Zhodli sa na tom, že do Kralova je premiestňovať sa vcelku nebezpečné, a tak od nich Leana odišla hneď ako sa premiestnili. Tarry, Tarnyho otec nebol doma, ale dom bol stále nadstavený na Tarnyho a Bellin genetický kód, a tak mohli vojsť. Tulienka Deľa si do svojej čítačky z Interaktívnej knižnice rodu Lietavých nahodila niekoľko skenov starších kníh v piktopísme. Tarny, napriek tom, že bola vcelku neskorá noc, ešte vďaka vplyvom z Quertu, necítil únavu a tak spolu s Tulienkou Deľou listovali v knihách a hľadali niečo nezvyčajné. Tulienka Deľa si pri čítaní strany v skene prastarých Poznaniach Osudu všimla niekým dopísané štyri nejasné piktopísmové znaky. Nevedela čo to je za druh piktopísma.

"$ $Pani Lietavá...? Bella? Máte nejaké slovníky piktopísma? Zišli by sa!"$ $ 

"$ $Len jeden vzácny rukopis z ôsmeho storočia pred naším letopočtom, a nie je v ňom automatický vyhľadávač!"$ $ 

"$ $To nevadí, kde je?"$ $ 

"$ $Kategória Vzácnosti, sekcia slovníky, názov Piktopísmo Jasnovidky srdca a tiež jej predkov."$ $ 

"$ $Ďakujem."$ $  V objemnom slovníku, našťastie digitalizovanom, si prešla Tulienka Deľa Jasnovidkine piktopísmo a keď v žiadnom zo stopäťdesiatich znakov neidentifikovala ani jeden zo znakov v knihe, tak začala s vylučovacou metódou. Po polhodine oznámila Tarnymu.

"$ $Je tu napísané buď Meden12 alebo Meď33 44. Dvanástka a štyridsať štvorka sú podobné znaky."$ $ 

"$ $A čo to podľa teba znamená?"$ $ 

"$ $Je možné, že to je spojenie na to sme čítali u Deoque alebo proste si niekto zapisoval čo má nakúpiť. Alebo niečo úplne iné."$ $ 

"$ $Počítaj zatiaľ s prvou alternatívou. Čo to podľa teba ukazuje."$ $ 

"$ $Na ktorej strane sme našli ten text u Deoque?"$ $ 

"$ $Nie som Sylvia, Tulienka Deľa."$ $  Usmial sa.

"$ $Prepáč, stále si myslím, že aj ty si tak geniálny."$ $  Uškrnula sa a pokračovala. "$ $Tridsaťtri možno evokuje stranu na ktorej máme to hľadať. Alebo niečo iné. Máš nejakú asociáciu ktorá sa týka týchto dvoch čísiel?"$ $ 

"$ $Ak si spomeniem.. Asi takto, asi pred dvoma týždňami v škole sme mali na literatúre nejakú čudnú Fentenzíjsku báseň, ktorá, povedzme si, je čudná, ale potom niekto prišiel na to, že každé štyridsiate štvrté slovo treba čítať a vyjde ti niečo ako opis krajiny. Je to vraj vychádzajúce z jedného proroctva, ktoré bolo vypočuté Okom III."$ $ 

"$ $Nenapadá ťa nič na to proroctvo?"$ $ 

"$ $Mne pripadalo ako reklama na kovy, Celé proroctvo bolo rozdelené na tri časti, z ktorých každá má jedenásť riadkov."$ $ 

"$ $Tridsaťtri! Je tam to číslo. A kovy! To dáva zmysel! Je to..."$ $ 

"$ $Je to blbosť."$ $  Povedal.

"$ $Ale nie je, to musí mať nejaký význam!"$ $ 

"$ $Máš nejakú psychickú poruchu? Väčšina vecí nedáva význam."$ $ 

"$ $Ale toto áno. Mne to pripadá ako priveľmi nepravdepodobná náhoda."$ $ 

"$ $Rozmýšľaj, prečo by sme sa to učili v škole? Je to báseň ako každá iná."$ $ 

"$ $Nie je, pamätáš si niečo s tej skrátenej verzie?"$ $ 

"$ $Na triednom dorozumievači to máme, kuknem."$ $  Po chvíli začal Tarny recitovať. "$ $Krajina prebitá riekou, vytesaná do skaly, stojí už mnoho rokov. Skrýva kov. Pod krajinou tou. Za morom oddelený, Vesmírom spojený."$ $ 

"$ $Dáva to zmysel. Dáva to zmysel!"$ $  Takmer od radosti vykríkla.

"$ $Tulienka Deľa, ja si nemyslím, že..."$ $ 

"$ $Nechápeš!? Odkazuje to na nejaký kaňon v krajine za morom."$ $ 

"$ $Tak potom Wymyslensko."$ $ 

"$ $Posledná časť básne! Vesmírom spojený. Je to myslené ako Zem. Sme vesmírom, teda červou dierou spojený. Krajina za morom je Amerika."$ $ 

"$ $Si si istá, že Oko III. vedelo čo je červia diera a to, že to bolo písané z Európy?"$ $ 

"$ $Nie, ale je fakt, že ľudia sa na Wymyslensko dostali už predtým cez časopriestorové tunely."$ $ 

"$ $To je pravda ale..."$ $ 

"$ $Mne na tom pripadá väčšina vecí logických."$ $ 

"$ $Mne nie."$ $ 

"$ $Musíme to skúsiť."$ $ 

"$ $Príliš riskantné. Ani nevieš kde to vôbec máš hľadať, je to príliš riskantné."$ $  Poslednú vetu zdôraznil.

"$ $Ale čo... Deoque... to snáď riskantné nebolo?"$ $ 

"$ $Ale to muselo..."$ $  Bránil sa.

"$ $Netrep."$ $ 

"$ $Ja hovorím. Toto nemusíme..."$ $ 

"$ $Musíme... kto iný? My sme to zistili..."$ $ 

"$ $Dobre uznávam,"$ $  prikývol Tarny a pokračoval. "$ $Ale kde to je?"$ $ 

"$ $Grand Canyon. Tá truhlica je tam."$ $  Automaticky odvetila.

"$ $Ako to môžeš vedieť?"$ $ 

"$ $Myslím si to. A musíme to skúsiť. Máme poslanie Tarny. Zajtra vyrážame. Upovedomím Pauline."$ $  Tarny už nestihol nič povedať.

\begin{center}

*

\end{center}

Rosa zaspala s knihou v ruke. Keď sa prebudila, chvíľu premýšľala kde vlastne je, pretože udalosti posledných dní sa jej zdali až príliš neuveriteľné. Je to teda pravda, odišla a to čo dokáže je mágia. A nie je jediná. Usmiala sa a pozrela sa na digitálne hodiny nad knižnicou. Dvanásť hodín. Prespala pol dňa! Vstala a išla hľadať Chen. Našla v kuchyni len lístok.

Prepáč, nechcela som ťa budiť, ale dnes sa do piatej večer nevrátim. Nemusíš ma nikde hľadať, radšej ostaň tu, aj keby som tu do noci nebola. Ak chceš dennú tlač, máš v čítačke pravidelne aktualizované informácie. Jedlo si priprav sama.

Chen

Rosa dočítala list. Následne si ešte raz prečítala druhú vetu. Nemusíš ma nikde hľadať, radšej ostaň tu, aj keby som tu do noci nebola. Formulácia naznačovala, že Chen nevylučovala možnosť, že by sa nevrátila. Ale prečo? Nórsko bolo relatívne bezpečné, ale nevedela ako to je v Spoločenstve. Alebo robila niečo riskantné? Chen hovorila, že je obchodníčka, ale vtedy ju tá otázka tak trochu zarazila. A potom tvrdila, že je nejaká zvláštna obchodníčka. S čím? Keď to bolo také nebezpečné...? S dušami? Napadla jej jedna poviedka, čo v noci čítala. Tá predstava bola mierne od veci, ale... Nechala sa unášať svojou fantáziou.

Obchodník s dušami príde po poslednom zvone a bude ťa chcieť zničiť. Ale ty mu to nedovoľ, nesmieš. Bude tvojou ilúziou a napokon ťa zabije. Na to si pamätajte, Ness.

Napadol jej úryvok. Ale to bola hlúposť. Ale aj tak. Usmiala sa a pokračovala vo svojej úvahe.

Obchodník s dušami ti ich bude chcieť predať, aby si mu za ne dala svoju. Ale ty sa nedaj, lebo budeš poslednou čo prežije. Chápeš to Ness? Nezastavuj sa a ži ďalej. Nedaj sa oklamať obchodníkovi s dušami, bude tvojím duchom pokroku, a napokon ťa zničí. Neklam samú seba. Ness...

Nebola v tom síce žiadna paralela, ale aj tak sa tou predstavou zaoberala. Rovnako ako pravdou o Chen. Rosa mala trochu pocit, že jej Chen klamala. Síce sa s ňou zviazala prísahou, ale čo ak... Mala pochybnosti, ale tie nechcela vysloviť pred Chen. Zatiaľ nie. Možno je zbytočne podozrievavá. Nechala si ich pre seba a rozhodla sa, že sa tento deň ide venovať mágii a histórii.

\begin{center}

*

\end{center}

"$ $Arabela?"$ $ 

"$ $Vítam ťa Niela,"$ $ 

"$ $Nevrátila som sa."$ $  Pevne vyhlásila, aj keď Arabela sa jej to nepýtala, a ani pýtať neplánovala.

"$ $To som si vedomá."$ $ 

"$ $Čakala si ma?"$ $ 

"$ $Tak trochu."$ $ 

"$ $Prečo Arabela? Ja mám prácu."$ $ 

"$ $Morja sa vrátila, a ty si to vedela."$ $ 

"$ $Je to kvôli mne,"$ $ 

"$ $Aj to viem."$ $ 

"$ $Kde je, Arabela?"$ $ 

"$ $To neviem. Viem len kde bola, a podľa mňa je veľkú veľká pravdepodobnosť, že je niekde tam."$ $ 

"$ $Kde, Arabela?"$ $ 

"$ $Poď, Niela. Ako sa Morja vrátila, veľa vecí je nových. Ako to, že žije dcéra Morje a Jegrigsena Goona a ak proroctvá neberú v úvahu rodinné rady, ale krv, je možné, že Goonová proroctva sa už narodila."$ $ 

"$ $To viem a som si toho vedomá,"$ $  odvetila.

"$ $A to ma znepokojuje Niela. D si je nebezpečenstva preňho vedomý, a tak prečo by to dieťa nezabil?"$ $ 

"$ $Ako vieš, že žije?"$ $ 

"$ $Je to moja domnienka, a rovnako Morjina. Jeg prisahal. A prísahy sa viažu naveky."$ $ 

"$ $Morja ti niečo hovorila?"$ $ 

"$ $Áno, veľa vecí."$ $ 

"$ $Kedy...?"$ $ 

"$ $Pár hodín dozadu."$ $ 

"$ $Kým som sa sem dostala chvíľu to trvalo."$ $ 

"$ $Kde chceš ísť?"$ $ 

"$ $Nie je to zrejmé? Za Morjou, lebo mám pocit, že nám zamlčala viac ako povedala."$ $ 

"$ $Si si istá? To je aj mojím cieľom."$ $ 

"$ $Tak máme spoločnú cestu. A povedz, bola si niekedy na našej rodinnej rade? Keď sa tam začali s Mornou hádať.."$ $ 

"$ $Viem si predstaviť."$ $ 

"$ $Morna je, aká je, to je fakt. Jej povaha nám k mnohému pomohla, ale teraz skôr prekáža. Musíme nájsť zas Morju."$ $  Niela prikývla.

"$ $Mali by sme ísť."$ $ 

"$ $Súhlas, tak nasadni a povedz, čo prezradila kniha?"$ $ 

\begin{center}

*

\end{center}

 Videla zničenú miestnosť. Väzňa a strop. Deoque sa prebrala a snažila sa spomenúť na to, čo sa stalo. Išla zachrániť Londýn, a teraz má v dome väzňa. Jej dom je rozbitý, určite to spravil on. Určite ju prišiel zabiť. Ale ona zistí kde je D... Postavila sa a prezrela si miestnosť. Vyvolala minulosť pomocou mágie a videla. Kto je ten, ktorý je vyslancom D. Desila sa seba, ale nedala to najavo, teraz nie. Potrebovala vedieť pravdu o Nevermoreovi.

\begin{center}

*

\end{center}

"$ $Pauline, ráno, ideme!"$ $  Zobudila Tulienka Deľa Pauline. Tarry nebol doma, a tak o ich prítomnosti a odchode vedela len Bella, ktorá ich rozhodnutie s miernou nevôľou, ale predsa akceptovala.

"$ $Mám sa premiestniť do Grand Canyonu?"$ $  Spýtala sa Pauline.

"$ $Ešte kniha."$ $  Tarny jej podal knihu a Tulienka Deľa prikývla. Bol čas. Vo chvíli keď odbili hodiny sedem zmizli.

\begin{center}

*

\end{center}

Atmosféra u Mrany bola poznačená vedomím toho, že každú chvíľu môže prísť polícia a ony budú musieť zmiznúť. Výstrahy, ktoré Mrana nadstavila, by ich mali včas upozorniť, ale čo ak? Sylvia využila svoje schopnosti s veľkou nevôľou a vedela, že je šanca, že polícia si zráta dva a dva a pochopí, že ona je démonka. Alebo polodémonka. Ale na tom až tak nezáleží. Bola až priveľmi nervózna, aby mohla ísť spať, by vôbec zatvorila oči. Bolo päť hodín ráno, aj s dvojhodinovým časovým posunom späť a ona neustále premýšľala. Vedela, že obe, Mranu aj Loviisu dotiahla do nebezpečenstva. Mrana by ju až tak netrápila, tá s tým súhlasila dobrovoľne, ale Loviisu do toho zatiahla veľkou náhodou. Kvôli tomu, že sa potrebovala vyrozprávať... a ani nevedela prečo ju oslovila. Urobila priveľkú chybu, že sa jej snažila všetko vysvetliť, mala byť radšej ticho a venovať sa svojej revolúcii. Bála sa tiež o Tulienku Deľu a dúfala, že nepríde späť. Spojiť sa s ňou nedalo pre jej priveľkú vzdialenosť a telepatióny s internetovou komunikáciou boli sledované, a zvlášť teraz, keď bola na úteku pred políciou. Tulienky Deli nebolo už vyše dvoch dní a ona o nej nemala informácie. Len tak odišla za Tarnym, bez toho, aby ju vzala. Kvôli tomu, že ona, Sylvia potrebovala jeden dokument zo suterénu. A tak Tulienka Deľa odišla bez nej, a nevrátila sa. Netušila kde je, ani čo zažíva. A či vôbec stále žije. To ju mierne zabolelo. Všetko, čo sa mohlo pokaziť, by malo až priveľmi zlé následky. Už nedokázala byť sama zo svojimi myšlienkami. A k tomu v tichu, ktoré pomáhalo jej zlým myšlienkam. Potrebovala ich nejako vyrušiť. Pustila si hudbu a snažila sa zaspať. Z jej chmúrnych myšlienok ju vyrušila Mrana.

"$ $Sylvia, si blázon?!"$ $ 

"$ $Aj áno, prečo sa pýtaš?"$ $ 

"$ $Skutočne chceš zaspať pri death metale?"$ $ 

"$ $Čo máš proti death metalu?"$ $ 

"$ $Nič, pár skupín aj počúvam, ale spať pri tom?!"$ $ 

"$ $Nesnažím sa spať."$ $ 

"$ $A čo teda?"$ $ 

"$ $Zničiť sa."$ $ 

\begin{center}

*

\end{center}

"$ $Grand Canyon je až príliš veľký, Tulienka Deľa, to ho chceš celý prejsť meter po metri a rozkopať ho? Nechceš ľudom predsa zničiť ich národný park?"$ $ 

"$ $Nechcem,"$ $  odsekla.

"$ $A ako to chceš hľadať? A vôbec, v takej tme? Je okolo polnoci, do tvojho plánu si nezarátala časový posun."$ $ 

"$ $To využijeme na utáborenie sa a svietiť vieme. Pokiaľ viem, nejaké baterky sme mali."$ $ 

"$ $Hm.."$ $ 

"$ $Čo?"$ $ 

"$ $V mačičkoese."$ $ 

"$ $Fuck,"$ $  Zakliala. Pauline sa trochu neisto spýtala.

"$ $Chceš povedať, že nemáme svetlo?"$ $  

"$ $Máme, ale len s mágiou, a tú som moc využívať nechcela."$ $ 

"$ $Tak čo spravíme?"$ $ 

"$ $Si polodémonka, sakra! Prepáč, že nadávam, ale... premiestnime sa niekde kde je deň, kúpime za doláre nejaké baterky a premiestnime sa späť!"$ $ 

"$ $Tulienka Deľa, odstrašuješ Pauline. Za to, že si absolútne nepripravená na toto, neviň ju, ale seba."$ $ 

"$ $Sklapni Tarny. Pauline, premiestni sa, prosím do nejakého mesta a kúpime baterky. Respektíve kúpiš ich ty a my ťa budeme chrániť."$ $ 

"$ $V pohode."$ $ 

"$ $Tarny ostaneš tu a..."$ $ 

"$ $Tulienka Deľa, pri tvojej nálade radšej ostaň pri veciach ty, sama sa obrániš, ale Pauline..."$ $  Paulina mala pocit akoby tam nebola.

"$ $Ja som tu."$ $ 

"$ $Viem, zlož si veci a niekde sa s Tarnym premiestni. Bude neviditeľný, ty ho vidieť budeš."$ $ 

"$ $Fajn."$ $  Povedala Pauline a o chvíľu sa premiestnila.

Objavili sa v Londýne. Obchod, ktorý hľadali, si Pauline všimla na fotke a tak sa premiestnili rovno predeň. Tarny Pauline mierne zmenil výzor, aby zapadla do davu, pretože potrebovali byť nenápadní. Všetko prebiehalo zatiaľ dobre. Keď vychádzali z obchodu, Tarny sa zrazu zahľadel na jedného muža v dave a zatelepatizoval Pauline.

"$ $Pauline, nepremiestňuj sa a nemysli na nič. Jean by dokázal zistiť kde sa premiestnime! Netelepatizuj!"$ $ 

Nemohol použiť mágiu, pretože Jean, a vôbec väčšina lepších mágov ju dokázala odhaliť. To, že je v dave človek s zoverom so spoločenstva M by nemalo byť nič nezvyčajné, ale Jean Parvasîe ich už raz zabiť mal, a to mu unikli na Quert. Premýšľal čo urobiť. Jean zrejme vie ako vyzerajú a mal až príliš silný magický potenciál. Bol silnejší ako oni. Tarnymu bolo jasné, že útok by neprežili. Vytvoril obranné telepatické pole, chytil sa Pauline a zatelepatizoval jej.

"$ $Premiestni sa rýchlo na randomné miesto."$ $  Tarny urobil čo najsilnejšie telepatické pole, ktoré bolo blokované pred zásahmi z vonku a premiestnili sa. Jean Parvasîe si ich všimol. Jeho Laser ich už, ale nezasiahol.

"$ $Pauline?"$ $  Zatelepatizoval Tarny Pauline.

"$ $Tarny!"$ $  Otvoril oči. Krajina okolo neho bola neznáma. "$ $Kde sme Tarny?"$ $  Pomaly mu to dochádzalo.

"$ $Tam kde si sa premiestnila."$ $ 

"$ $Randomne."$ $  Bránila sa.

"$ $To myslené ako, premiestni sa, kde chceš, ale snaž sa rozumne."$ $ 

"$ $Ja som prepla na random. Ako vtedy v Portugalsku."$ $ 

"$ $Teda sme na inej planéte."$ $  Sucho utrúsil.

"$ $Je to tvoja vina!"$ $  Obvinila ho Pauline.

"$ $Prečo si nepochopila správne moju inštrukciu?!"$ $  Vyčítal jej.

"$ $V ohrození života proste človek nerozmýšľa! Robí intuitívne!"$ $ 

"$ $Mali sme čas!"$ $ 

"$ $Nedal si mi ho! Povedal si RÝCHLO!"$ $ 

"$ $To áno..."$ $ 

"$ $Ale čo? Nevyhováraj sa!"$ $ 

"$ $Mohli by sme sa prestať hádať. Baterky máme, a premiestniť sa vieme."$ $ 

"$ $Neodbočuj!"$ $ 

"$ $Chcem sa len prestať hádať!"$ $ 

"$ $To aj ja, uznaj si chybu!"$ $ 

"$ $Chybu máme obaja!"$ $ 

"$ $Ja ti nevidím do hlavy!"$ $ 

"$ $Fajn, ideš ma tu nechať a premiestniš sa sama? Zachránil som ťa pred Jeanom. On nehľadal mňa, ale teba. Si v neustálom nebezpečenstve!"$ $ 

"$ $Toho som si vedomá."$ $ 

"$ $Mier?"$ $ 

"$ $Mier."$ $ 

"$ $Tak sa premiestni."$ $ 

"$ $Chyť sa ma."$ $  Pauline zatvorila oči a myslela na Grand Canyon.

Otvorila oči. Nič sa nestalo.

"$ $Čo sa stalo?"$ $  Tarny sa mračil.

"$ $Práve to, že neviem."$ $ 

"$ $Fuck!"$ $ 

"$ $Čo sa mohlo stať...?"$ $  Hovoril si Tarny nevšímajúc si Pauline.

"$ $Myslíš si, že už nie som polodémonka?"$ $ 

"$ $Práve to, že nie. Premeň sa na niekoho."$ $  Pauline sa bez veľkej námahy premenila na Tarnyho a zas späť.

"$ $Stále si polodémonkou."$ $ 

"$ $Ale prečo sa odtiaľto nedá premiestniť?"$ $ 

"$ $Práve to. Neviem."$ $ 

"$ $Myslíš, že inopole?"$ $ 

"$ $Čo?"$ $ 

"$ $Inopole. Nepoznáš?"$ $ 

"$ $Nikdy som o ňom nepočul. Odkiaľ to máš?"$ $ 

"$ $Neviem, proste si myslím, že je to inopole."$ $ 

"$ $A čo to je?"$ $ 

"$ $Iný svet. Proste ako paralelný svet."$ $ 

"$ $Odkiaľ to máš? Myslel som si, že paralelné svety sú doménou nás s Tulienkou Deľou."$ $ 

"$ $Viem to, proste, nejako to viem."$ $ 

"$ $Čo ti spravila Deoque?"$ $ 

"$ $Podceňuješ ma."$ $ 

"$ $Pauline..."$ $ 

"$ $To nebola výčitka, ale konštatovanie. Je mi jasné, že vy ste obdarení akosi, väčšou inteligenciou ako ja."$ $ 

"$ $Alebo ti náš svet nikto neukázal. Naučila si sa sama nejaké veci."$ $ 

"$ $Nejaké, to každý."$ $ 

"$ $Ani nie... Pochopila si pár vecí za pár dní, ktoré my sme sa učili roky."$ $ 

"$ $Blbosť."$ $ 

"$ $Ani nie."$ $ 

"$ $Nevrav."$ $ 

"$ $Podceňuješ sa sama."$ $ 

"$ $Je to pravda."$ $ 

"$ $Viem, že je toho veľa, Pauline. Na taký krátky čas. Ľudom s naším génom sa už neodhaľujeme, bol to pre nich až priveľký šok. Pre teba nie. Ty si to možno tušila."$ $ 

"$ $Chápala som veciam, ktoré ľudia nie. A tiež som niečo dokázala, čo ľudia nie. A tiež videla."$ $ 

"$ $Teraz je systém, že keď nájdu niekoho s naším génom, kto nepatrí do spoločenstva a nenašiel nás, monitoruje ho agentúra. Pre bezpečnosť štátu, totiž D by bol schopný ich presvedčiť o tom, že on je tá správna strana a oni sú vyvolení. Na to ľudia skočia. D dokázal využiť časť týchto ľudí. Často sa stávalo, že odmietli nás rešpektovať ako fakt a ich nerozhodnosť využil D, a to až príliš skryto, aby neodhalil ich pravú podstatu."$ $ 

"$ $Pozná pravdu, odhalí lži, možnosť na nikdy nedostane späť, navždy budeme mŕtvi."$ $ 

"$ $Čo?"$ $ 

"$ $Neviem. Nejako... nejako mi to napadlo. Proste vedela som čo mám povedať. Ako inopole."$ $ 

"$ $Vážne ti nič nespravila Deoque?"$ $ 

"$ $Nie,"$ $  odsekla.

"$ $Si si istá?"$ $ 

"$ $Absolútne."$ $ 

"$ $Mne sa to nezdá."$ $ 

"$ $Nebude v tom niečo, čo je len tu?"$ $ 

"$ $Myslíš nejaká cudzia mágia?"$ $ 

"$ $Tak dáko."$ $ 

"$ $Môžem ju skúsiť detekovať, ale nemám žiadny prístroj."$ $ 

"$ $Veď do sa dá..."$ $ 

"$ $To je pravda, ale spotrebuje sa na to zjavná dávka mágie, ešte väčšia ako na laser, a to je čo povedať."$ $ 

"$ $Aha."$ $ 

"$ $Ale bolo by v pohode sa o to pokúsiť. Ak by sa niečo dialo, upozorni ma prosím. Keby hocikto prišiel. Nevieme kto tu je, a kde sme. Rozumieš?"$ $ 

"$ $Samozrejme."$ $ 

"$ $Prípadne použi Solan alebo Laser a urob ochrannú stenu."$ $ 

"$ $Nie som si istá, či to dokážem."$ $ 

"$ $Dokážeš to. A tu máš meč, keby niekto zaútočil. Vieš s tým narábať?"$ $ 

"$ $Meč? Nie sme v stredoveku, že?"$ $ 

"$ $Pozri sa na to takto – pre Marone Tlogenovú je v jej obchodnom záujme brzdiť trh. V každom prípade, tento je trošičku prerobený. Má v sebe zabudovanú laser pištoľ, takže ak stlačíš tlačidlo, vystrelíš."$ $ 

"$ $Rozumiem."$ $ 

"$ $Ale v každom prípade..."$ $ 

"$ $Ťa upozorním."$ $ 

"$ $V pohode."$ $  Tarny zatvoril oči a hľadal mágiu...

\begin{center}

*

\end{center}

Tulienka Deľa sedela na kraji ich provizórneho obydlia a čakala na nich. Už boli preč vyše pol hodiny! Keby neboli tak ďaleko, tak by im zatelepatizovala. Len z nudy si zobrala do ruky noviny a zapojila internet. Chcela sa upokojiť, ale hneď najnovšia správa ju znepokojila.

V Londýne zaútočil Jean Parvasîe, obete nehlásia

V Londýne dnes, o v neskorých večerných hodinách zaútočil jeden z najsilnejších a najobávanejších prívržencov D, Jean Parvasîe. Podľa svedkov zaútočil Laserom na dvojicu, ktorá sa mu, ale vyhla a niekde zmizla. Škody na majetku sú približne na úrovni päťtisíc zlatniakov, ale ešte nie sú všetky hlásené. Parvasîe poškodil stenu jednej z budov spoločenstva, rovnako i časť ľudského obchodu a za obeť mu padla aj časť cesty. Jean po nezasiahnutí svojho cieľa, vyslal ešte niekoľko Laserových lúčov a zasiahol jedného človeka a spôsobil mu ľahké zranenia, ktoré ošetrila pozemská zdravotná služba, keďže išlo o človeka. Následne Parvasîe odišiel, podľa očitých svedkov išiel po niekom konkrétnom, kto mu, ale zrejme unikol. Záznam z kamier polícia analyzuje.

Dodávame ešte ...

Tulienka Deľa vcelku tušila, kto boli tí, čo unikli. Na koho by chcel D útočiť... Už ich prenasledoval dosť dlho... Tarny s Pauline sa iste premiestnili ale... Prečo nie tu? Áno, je pravda, že Jean mohol zistiť, kde sa premiestňujú, predsa len on bol až príliš mocný... Ale prečo sa proste niekde nepremiestnili, a odtiaľ k nej? Veď Jean sa nemohol premiestňovať, nebol polodémonom. Tarny nebol hlúpy, ale kde teda je? Nemohla ich ísť hľadať, to bolo až príliš nebezpečné, čo ak by sa vrátili?, ale aj ostávať na jednom mieste... Odkaz im nechať nemohla, to bolo až priveľmi riskantné. Otázkou, ale bolo, že či stále žijú...

Premýšľať nad tým bolo až príliš smutné a znepokojujúce. Tulienka Deľa si nepripúšťala, že by boli mŕtvi, veď zmizli, ale kde sú teraz? Kde? Celé to bol jej nápad, jej! Ona to spôsobila!, ale vedela, že čakať len tak na mieste je nanič. Keď už raz sú tu, treba začať. Detekcia mágie bola jednou z možností. Prístroj mala ona, Tarny si ho nebral, takže mohla začať. Všetky veci mala pre istotu, pre potrebu rýchleho úteku zabezpečené a zmenšené. Nič nenasvedčovalo nebezpečenstvu, ale D mohol byť hocikde. Pustila zariadenie a čakala na výsledok.

\begin{center}

*

\end{center}

Otvoril oči.

"$ $Nedá sa mi použiť mágia. Akoby nebola..."$ $ 

"$ $Ale veď..."$ $  Pauline vytvorila na zemi mágiou čiaru.

"$ $Si polodémonka. Na teba to obmedzenie zrejme neúčinkuje."$ $ 

"$ $Inopole je zrejme viazané na gén."$ $ 

"$ $Myslíš, že je to podmienené tým čo som povedal?"$ $ 

"$ $Asi tak. Ale nedá sa odtiaľto premiestňovať."$ $ 

"$ $To je problém. Sú v inopoliach niečo ako brány do iných polí a inopolí?"$ $ 

"$ $Inopolia musia mať nejaký spoj z minimálne jedným inopoľom, či poľom na možný prenos. Inak je to už paralelný svet. Medzi paralelnými svetmi sa nedá premiesťovať."$ $ 

"$ $A nemohlo dôjsť k anomálii?"$ $ 

"$ $Pozri, paralelné svety sú myslím ako nejaké dve nekonečná."$ $ 

"$ $Myslíš sú paralelné a nikdy sa nespoja? Ako keď máš množinu nekonečno bodov a dve podmnožiny po nekonečne bodov, pričom žiadny bod sa nenachádza v dvoch množinách súčasne?"$ $ 

"$ $Skôr ide o dve množiny po nekonečno bodov ktoré, ale nie sú podmnožinami. Nemajú nič spoločné. Sú nezávislé, ale môžu byť rovnaké, ale nikto z množiny A nevie nič o množine B. To ako: v inopoliach a poliach jednej paralely sa všetko dá popísať jedným fyzikálnym zákonom. V paralelách nie."$ $ 

"$ $Ale veď tu nefunguje mágia."$ $ 

"$ $Tvoja, moja mágia funguje. To, že sú niekde odchýlky alebo výnimky je v poriadku. Tu ide o skôr o ten všeobecnejší model."$ $ 

"$ $Ako energia – fotóny a veci ako gravitácia a zloženie z hmoty ktorá je z kvarkov a následne atómov a iných častíc?"$ $ 

"$ $Porozumela som gravitáciu, ale ak tú vetu chápem správne, tak asi tak."$ $ 

"$ $Vzdelávaj sa Pauline."$ $ 

"$ $Teraz? WTF?!"$ $ 

"$ $No, najskôr zistíme, kde je prechod do nejakého iného sveta. Vieš niečo o tom?"$ $ 

"$ $Brány bývajú väčšinou na jednej kope v inopoliach a v poliach sú roztrúsené. V inopoliach je menej brán..."$ $ 

"$ $Inopole a pole je rozdiel vlastne aký?"$ $ 

"$ $Veľkosť a tiež vlastne to, že inopole je akoby jeden povrch, teba napríklad všetko je jedna veľká chodba."$ $ 

"$ $A toto považuješ za?"$ $ 

"$ $Inopole, zatiaľ. Keď to preskúmame, možno by sa to dalo posúdiť lepšie."$ $ 

"$ $Ideme? Načo tu len tak stáť?"$ $ 

"$ $Bude to bezpečné? Ty nemôžeš používať mágiu..."$ $  Tarny si z vrecka vytiahol niečo vo veľkosti zápalkovej škatuľky a zrazu to vystrelilo lúč, ktorý na zemi vytvoril malé spálené miesto.

"$ $Funguje to aj tu."$ $  Spokojne konštatoval.

"$ $Čo to je? Bella ti to predpokladám nedala."$ $ 

"$ $No... Požičal som si to..."$ $ 

"$ $Spôsobom ako mačičkoes?"$ $ 

"$ $Asi tak..."$ $ 

"$ $A čo to je?"$ $ 

"$ $Zbraň. Je síce stará pár rokov, ale slúži a mala by nám pomôcť. Ty niečo vyčaruj. Ako laser alebo Solan, ak by niečo. Alebo vyčar ochranu."$ $ 

"$ $Pokúsim sa. Ešte to úplne neviem. Ochrana chráni i pred nemagickým útokom?"$ $ 

"$ $Hej."$ $ 

"$ $To je dobre, keby náhodou nevedeli čarovať."$ $ 

"$ $To je pravda."$ $  Tarny išiel náhodným smerom, keďže sa nemali podľa čoho orientovať. Po pár minútach narazili na skaly...

\begin{center}

*

\end{center}

Mágia sa vyskytovala len v malých chuchvalcoch, ktoré zodpovedali prirodzenému rozptýleniu. Zdalo sa, že na mieste, kde bola nič zvláštne nebolo. Celá mapa rozptylu mágie hovorila len to, že tu už dávno nikto mágiu nepoužil. Mágia zanechávala stopy veľa tisícok rokov po svojom použití. Prirodzene, všetko záležalo od množstva mágie. Tulienka Deľa predpokladala, že vec čo hľadajú, v sebe bude mať obrovské skladisko mágie. Grand Canyon bol až príliš veľký na systematické hľadanie. A hlavne, keď sa naliehavo potrebovala spojiť s Tarnym a Pauline. Ale keď dospela k záveru, že truhlica je tam, kde bola, musí tam byť predsa nejaká jasnejšia indícia! Zneviditeľnila sa a vytiahla čítačku.

\begin{center}

*

\end{center}

"$ $Vyzerá to, akoby tu už dávno nikto nebol."$ $  Skonštatoval.

"$ $Alebo vôbec."$ $ 

"$ $Je to stále inopole, ale už pole?"$ $ 

"$ $Stále to vyzerá ako inopole, ale nevideli sme všetko."$ $ 

"$ $Keď si hovorila o poliach a inopoliach, tak si to myslela ako s tou veľkosťou?"$ $ 

"$ $Pozri, vesmír je pole. Zem je časť poľa. Keby bola len zem a prechody, tak by bola skôr inopole."$ $ 

"$ $Takže sa máme pripraviť na niečo veľkosti zeme?"$ $ 

"$ $Môže byť aj väčšie. Veľkosť, ale nie je určujúca."$ $ 

"$ $Určujúce sú tie druhy prostredia a tiež rozptýlenie a počet brán do ďalších polí a inopolí."$ $ 

"$ $Myslíš si, že keď sme sa niekde premiestnili, tak tam, ak sa jedná o inopole je brána a teda sú tam aj ostatné?"$ $ 

"$ $Ani nie, my sme sa zrejme premiestnili do blízkosti brány, resp. sme sa tu dostali cez bránu a premiestňovacia zotrvačnosť nás dostala trochu ďalej. A okrem toho, skôr by som tipovala brány na tie skaly. Bývajú uzavreté..."$ $ 

"$ $Ideme to preskúmať?"$ $ 

"$ $Skúsme. Len aby sme sa tam dostali..."$ $ 

"$ $Ja tu mám krídla, len... sú zmenšené. Vedela by si ich zväčšiť?"$ $ 

"$ $Nie."$ $ 

"$ $Musíš sa skúsiť ak sa nechceme dolámať."$ $ 

"$ $Ale ja to neviem!"$ $ 

"$ $Musíš to vedieť. Pozri..."$ $  Tarny vzal do ruky kamienok a ukázal ho Pauline.

"$ $Zmenši ho."$ $ 

"$ $Nechceli sme to zväčšovať?"$ $ 

"$ $Nateraz to zmenši."$ $ 

"$ $Ako?"$ $ 

"$ $Na to príď sama. Dôležité je, aby každý detail odtiaľ zachovaný."$ $ 

"$ $Aj ten najmenší?"$ $ 

"$ $Aj ten najmenší. To je to umenie."$ $ 

"$ $Ale ako?"$ $ 

"$ $Sústreď sa. Proste to musí ísť cez teba. Cez tvoje vnútro. Musíš cítiť tú mágiu. Ty to vieš. To sa nedá naučiť. Musíš to v sebe objaviť."$ $ 

"$ $Tak načo sa učí mágia?! Keď sa to nedá naučiť?!"$ $ 

"$ $Dá sa pomôcť s objavením spôsobu, ktorým čaruješ. Ale narábať s ním musíš sama."$ $ 

"$ $Ja to neviem."$ $ 

"$ $Skús to."$ $ 

"$ $Musím?"$ $ 

"$ $Chceš sa zlámať na skalách? Toto nie je hra, Pauline, toto je otázka života a smrti. Toto je naozaj."$ $ 

"$ $Ale ja to neviem!"$ $ 

"$ $Proste to skús, namiesto rečí o tom, že to nevieš."$ $ 

"$ $Ale..."$ $ 

"$ $Pauline, ide o život! Keď ho stratíš nemáš ďalší.."$ $ 

"$ $Tak to skúsim..."$ $ 

"$ $Skús..."$ $  Pauline zobrala do ruky kamienok a zahľadela sa naň. Snažila sa vidieť všetky jeho detaily, i tie voľnému oku neviditeľné. Keď jej Tarny povedal, aby ho zmenšila, myslela si, že zredukuje počet atómov, ale to by nikdy nedal zväčšiť do pôvodnej podoby s detailmi aké mal. Nie, ona musela zmenšiť atómy, elektróny, protóny, neutróny, kvarky... Ona ho mala zmenšiť, ale nie ako to pôvodne myslela. Ona nemala zničiť hmotu, ona ju mala len dočasne zredukovať, ale ani nie hmotu, ale len jej rozmer. V tom spočíval spôsob. Zmeniť rozmer, nie veľkosť. Prúdila cez ňu magická energia...

Kameň sa začal zmenšovať, až sa zastavil na veľkosti nechtu. Pauline otvorila oči a pozrela sa na Tarnyho.

"$ $Môže byť?"$ $ 

"$ $Šikovná, a že to nevieš."$ $ 

"$ $Donútil si ma."$ $ 

"$ $Možno. Ale urobila si to ty. Máš v sebe viac mágie ako si myslíš. Len sa nepodceňuj."$ $ 

"$ $A teraz? Mám zväčšiť tie krídla?"$ $ 

"$ $Pomaly! Zmenšila si to, ale teraz ho zväčši. Na presne takú veľkosť ako bol predtým."$ $ 

"$ $To sa dá?"$ $ 

"$ $Samozrejme. Na to musíš prísť. A nevzdaj sa..."$ $ 

"$ $Tebe sa to ľahko povie, všetko je na mne!"$ $ 

"$ $Nemôžem za to, že si sa premiestnila na random!"$ $ 

"$ $Povedal si náhodne!"$ $ 

"$ $Nechaj to tak. Čaruj..."$ $  Pauline zatvorila oči a pokúšala sa dosiahnuť pokoj. Ľutovala už to, že s nimi odišla, že prišla do magického sveta, ktorý bol tak nebezpečný a zvláštny... Ale bola to jej voľba, a napriek tomu to mierne ľutoval... kedy tušila čo ju čaká... Mierne to ľutovala, aj keď vedela, že jej tento svet, ktorý jej patril vzali. A to, že teraz ho dostala späť...

Už si nepamätala veľkosť, ani všetky detaily, ale ona ich predsa nemenila, ona menila rozmer, a to bol rozdiel... Videla rozmer a pomaly ho zväčšovala. Videla ich všetky naraz, videla ich okolo seba a všetko v nich. Menili sa, kameň rástol v ich rozmere, až kým sa nedostal na zhodu. To videla, aj keď si nepamätala jeho veľkosť. Zastala.

"$ $Akým spôsobom zväčšuješ?"$ $  Tarny hľadel na kameň a premýšľal.

"$ $Ako to myslíš?"$ $  Nechápala.

"$ $No, ja zväčšujem a zmenšujem predmety spôsobom záznamu. V niečom, ako jednej častici sa zakóduje poloha atómov, zmení sa ich veľkosť a následne sa zmení veľkosť tak, že sa požije teória relativity, ale okolitému systému pritom slabne energia..."$ $ 

"$ $Teda to robím zle?"$ $ 

"$ $Nemyslím... Neklesla tu vôbec energia, a to je čudné..."$ $ 

"$ $Ja... ty si povedal, že musí byť zachovaný každý rozmer, a tak som premýšľala ako zmeniť pre nás veľkosť, aby som nijako nezasiahla okolie a zároveň zachovala každý detail. A tak som zmenila rozmer, nie veľkosť..."$ $ 

"$ $Rozmer som menil len potom, ako už detaily zasiahnu superstruny."$ $ 

"$ $Neviem čo to je, ale je mením rozmer celkovo."$ $ 

"$ $To sa dá...?"$ $ 

"$ $Asi áno."$ $ 

"$ $Ako to robíš?"$ $ 

"$ $Vidím rozmery..."$ $ 

"$ $Popravde... zrejme tak sú urobené aj zmenšovače... ja som debil! Ja som debil Pauline!"$ $ 

"$ $Čo je Tarny?"$ $  Začudovane, a mierne s pobavením sa ho spýtala.

"$ $Som debil, chápeš? Nemali sme tu byť, nemali sme ísť nikde?! Mysleli sme si o sebe veľa, ale v skutočnosti nevieme nič!"$ $ 

"$ $Hm...?"$ $ 

"$ $Robíš to, čo ja neviem! Sakra!"$ $ 

"$ $Robím niečo viac? Alebo hádam nie niečo zlé...?"$ $ 

"$ $Nie! Nechaj tak! To ja som debil! Ty si len prišla na niečo čo neviem, a to je zlé... ja som debil! Spôsob, ktorým mením veľkosť, je zlý! Nedá sa ním urobiť nič dokonale! Tvojím... ty si skombinovala rozmery! Svety! Vieš viac ako ja!"$ $ 

"$ $É... Prepáč."$ $  Ospravedlňovala sa.

"$ $Nemáš sa mi za čo ospravedlňovať. Jediný, kto je tu debil som ja!"$ $ 

"$ $Nie si... odhalil si Jeana. A vieš viac ako ja."$ $ 

"$ $Ale nie natoľko, aby sme stačili utiecť na normálnejšie miesto!"$ $ 

"$ $Predtým v hoteli..."$ $ 

"$ $Ale i tak... ušli sme na poslednú chvíľu! Matka mala pravdu! Nemali sme odchádzať!"$ $ 

"$ $Ale odišli sme, nevyčítaj si to. Teraz zväčším krídla a ideme hľadať."$ $ 

"$ $Počkaj! Si si istá, že ich zväčšíš v poriadku?"$ $ 

"$ $Neviem..."$ $ 

"$ $Sú naše jediné..."$ $ 

"$ $Tak zmenším skaly."$ $ 

"$ $Nie! Odlákaš na nás pozornosť. Verím ti. A predtým nás zneviditeľni... čo ak tam niekto bude? Stále sme ako na zemi."$ $ 

"$ $Ako sa zneviditeľňuje?"$ $ 

"$ $Prídi na to... a nezačínaj s nami!"$ $ 

"$ $Jasné."$ $  Rozumela tomu, prečo je treba začať s niečím jednoduchším, ale nechápala, prečo jej to Tarny musel tak zdesene pripomínať. Pozrela sa na kameň, čo mala v ruke. Ako mala docieliť, aby zmizol? Ale to predsa nemala spraviť! Mala docieliť, aby nebol viditeľný! Ale ako? Ako mala doňho zakódovať neviditeľnosť, ale tak, aby ho ona videla?

"$ $Čo robíš?"$ $  Zrazu na ňu skríkol.

"$ $Neviem! To je nemožné urobiť!"$ $ 

"$ $Nie je!"$ $ 

"$ $Je! Toto je... no, neracionálne!"$ $ 

"$ $Je to jednoduché!"$ $ 

"$ $Ako to robíš?"$ $ 

"$ $Normálne... proste zabrániš vidieť."$ $ 

"$ $Ale ako!? Sakra!"$ $ 

"$ $Je to čiastočne programovacia mágia, to súhlasím, ale napriek tomu je to jednoduché!"$ $ 

"$ $A čo to tá programovacia mágia vlastne je?"$ $ 

"$ $Druh mágie. Vieš nejaký programovací jazyk?"$ $ 

"$ $Nie..."$ $ 

"$ $Predstav si... ako by som ti to vysvetlil..."$ $  Tarny sa na chvíľu odmlčal a pokračoval. "$ $Mysli na vec čo chceš, aby sa stala. Vtlač ju do toho predmetu..."$ $ 

"$ $Ale ja ho nechcem poškodiť..."$ $ 

"$ $Nepoškoď ho."$ $ 

"$ $Ale ako mám získať energiu?!"$ $ 

"$ $Z okolia... Trochu ho ochlaď a máš ju..."$ $ 

"$ $Ale ako to tam sa má vtlačiť bez poškodenia..."$ $ 

"$ $Pozri sa na to, ako na energetické pole. Nie hmotnosť, iba energia..."$ $ 

"$ $To sa dá?"$ $ 

"$ $Samozrejme."$ $  Pauline chcela niečo namietnuť, ale radšej si povedala, že to skúsi. Snažila sa vidieť energiu, ako jej to povedal Tarny. Ale nič sa jej nedarilo. Všetko vyzeralo ako predtým...

\begin{center}

*

\end{center}

Poznaje sú dverami brán. Ich miesta sú nes.... na mieste kde ri... tvorila ...... Ich energia je sústredením moci Sustevelíny v jej po..... vraku. A kľúčom je ten, ktorý tvorí oko.

Tulienka Deľa hľadela na piktopísmo a snažila sa zistiť význam viet, ktoré boli popretkávané neúplnými a možno zle preloženými slovami. Originál, ktorého sken čítala bol poriadne starý, možno viac ako tisíc rokov, i keď si nebola istá, či Bella nemá iba kópiu, aj keď v starších wymyslenských rodinách a rovnako i v spoločenstve M bolo zvykom vlastniť originály starých písomností. Skôr kópie bol výnimkou. Skôr v štátnych archívoch sa uskladňovali kópie. Vlastniť spisy bola pre rod otázka prestíže. Často rodiny ani len nevedeli, čo sa v spisoch nachádza, alebo len základné informácie, pretože najhlavnejšie bolo predsa samotné vlastníctvo spisu. Ale, keďže ich mali digitalizované, mohla rátať s nejakou vedomosťou o ich obsahu. Táto konkrétna časť textu v piktopísme bola vpísaná do obrazca, zrejme niekým, kto spis vlastnil neskôr. Tulienka Deľa nečakala, že by Bella alebo Tarry, či niekto z ich rodiny vedel tento druh piktopísma. I ona sama ho objavila náhodou. Na jednom z ich nelegálnych výletov so Sylviou a Tarnym, pár rokov dozadu, na jednej mayskej pyramíde. Sama mala iba svoj vlastný obraz pyramídy s textom a svoj prepis, ktorý, ale nedokončila celý, pre agentúru, ktorá musela prísť akurát vtedy, keď jej ostávalo ešte pár slabík. Zobrazila si fotografiu a zahľadela sa na časť, ktorú neodpísala. Práve tá, bola nie úplne zaostrená, čo ju vcelku štvalo. Porovnala si fotografiu s textom a všimla si pár vecí. Poznaje, slovo, ktoré jej predtým nedávalo zmysel sa poznanie, brán na hrán a do neúplného slova "$ $ri...."$ $  zistila, že posledné písmeno je "$ $a"$ $ . Prvá veta, "$ $Poznanie sú dverami hrán."$ $  je pripomenulo frázu vo Fentenzíjčine "$ $dvere hrany"$ $ , ktorá označovala obrazec, ktorý mal magický význam. "$ $Ich miesta sú nes... na mieste kde ri....a tvorila .... ."$ $  Celú vetu si čítala kým nevytvorila nejaké jej možné významy. Nes opravila na pres a najmožnejší možný význam jej znel ako ich miesta sú presne na mieste kde rieka tvorila krajinu. To by sedelo na Grand Canyon. "$ $Ich energia je sústredením moci Sustuvelíny v jej po... vraku."$ $ . Po pár prečítaniach jej na tú vetu došiel význam "$ $Ich energia je sústredením moci Sustuvelíny v jej posvätnom mraku. A kľúčom je ten, ktorý tvorí oko.

Došlo jej to. Sustuvelína bola v dobách pred Ja'Sno\v{}vid na Fentenzii jedna z hlavných bohýň fentenzíjskeho panteónu. Zodpovedala za slnko, svetlo a jej oltáre boli medené, pretože podľa povesti, jej démon chcel vziať medený, tepaný kov a použiť ho proti bohom, a preto ho skryla, ale keďže jej čary si vyžadovali meď, tá jej bola obetovaná. Jej znak bol znak na erbe Fentenzie. Sedelo to. Sustuvelína a meď – medená truhlica, démon, to bude zrejme D a preto bola truhlica skrytá. Uvedomovala si, že je možné, že to je legenda, ale všetko predsa sedelo. Znak a jeho oko. Tam to malo byť. Tulienka Deľa bola šťastná, že si tú Fentenzíjsku filozofiu študovala spolu s jazykom naozaj pozorne, pretože hľadať na pračudesnom abstraktnom obrazci oko, bolo ako hľadať ihlu v kope sena.

Okom obrazu sa nazýval vo Fentenzíjskej kultúrnej náuke hlavný párnopočetný uzol obrazu s obkolesením presne jedného bodu. V dobe pred Ja'Sno\v{}vid i v rannej dobe, keď už tradícia Oka bola, ho musel mať každý správny symbol. Symbol Sustuvelíny nebol výnimkou. Jeho oko sa vynímalo v strede symbolu.

Otvorila si satelitné zábery Grand Canyonu z najpresnejšej družice od spoločenstva M. Ak bol jej predpoklad správny, mal by byť v Canyone niečo čo je akoby ten symbol a na mieste kde je oko, tam je truhlica. Zábery si najskôr pozrela ako celok, ale nenašla nič, čo by aspoň vzdialene pripomínalo symbol Sustuvelíny. Nevzdala to. Každý jeden záber si pozorne prezrela do detailov. Nič. Stále to nevzdala. Vedela, že existujú aj iné typy máp... a bolo to tak, ako to písala osoba, čo písala piktopísmo... Ale ak mala víziu... To bolo tiež možné ale... Vtedy si na niečo spomenula... na to, že tam zrejme už veľmi dlhú dobu nikto nečaroval... A možno až také... údaje na merači ešte boli... priemerný rozptyl bol pod priemernou dávkou mágie... akoby jej niečo zabraňovalo... Podľa dávky, sa dalo približne prísť na posledné použitie mágie... zadala údaje a odpoveď jej nahrávala do teórie. Okolo päťtisíc rokov, a keď ešte sa pripočíta to, že dávka mágie pri tom ako sa tam dostala truhlica mohla byť obrovská, tak ten čas približne odpovedal. A autor, ktorý písal do knihy piktopísmom na to zrejme prišiel...

\begin{center}

*

\end{center}

Sakra, mysli na to čo tam chceš dať! Pokúšala sa ovládať Pauline. Asi jej piaty pokus sa skončil neúspešne a ona prichádzala o nervy. Vytvorila si myšlienku a pokúsila sa myslieť iba na ňu. Cez energiu ju vlievala do kameňa a naraz počula tresk...

"$ $Čo preboha robíš?!"$ $ 

"$ $Ja... čo sa stalo?"$ $ 

"$ $Explodoval kameň. Si šikovná. Prestaň... ja zistím, ako sa inak zneviditeľniť..."$ $ 

"$ $Čo si načerpal mágiu?"$ $ 

"$ $Nie... Ale možno niečo mám... ak som to niekde nedal..."$ $  Pauline sa trochu cítila vinne.

"$ $Ja sa ospravedlňujem... vážne prepáč... ja... neviem to... to je ťažké!"$ $ 

"$ $Nechaj to tak... Nerozbi niečo ďalšie... naučíš sa to potom, keď budeme mať čas..."$ $  Tarny si začal vyhadzovať veci z vreciek. Keď si už myslel, že asi skončili mal v ruke nejaké podlhovasté predmety a zmenšeninu oblečenia. Vydýchol si.

"$ $Je to tu. Zväčši to... to hádam dokážeš.."$ $ 

"$ $Samozrejme."$ $  O chvíľu boli na zemi dva obleky, vyzerajúce ako priehľadná kombinéza a tri spreje.

"$ $Toť nanospreje. Spôsobia, že nejaký predmet je ako neviditeľný, to oblečenie tak isto."$ $  Odpovedal na nepýtanú otázku.

"$ $Načo ty, veľký zmyslový mág, nosíš toto?"$ $ 

"$ $Pretože zmyslová mágia platí iba na živé predmety, nie na kamery atď."$ $ 

"$ $Aha."$ $ 

"$ $Zväčši krídla a obleč si to."$ $ 

"$ $Ideme teraz?"$ $ 

"$ $Na čo čakať?"$ $ 

"$ $To je pravda."$ $  Prisvedčila.

\begin{center}

*

\end{center}

Mapa mágie Grand Canyonu splnila jej predpoklad. Mágia mala tvar symbolu. A oko bolo na mieste, resp. pri mieste kde bola. "$ $To je šťastie..."$ $  Miesto oka bolo na pár metrov od nej... Ale prečo to teda neukazovalo? Možno tam bolo niečo čo tomu bránilo... predsa len... bolo určite treba skryť truhlicu čo najlepšie... Možno niečo nájde a možno sa už nevráti. V každom prípade, bolo by im dobré zanechať odkaz, aby ju nehľadali. Do tohto ide sama. Potrebovala niečo, z čoho by nikto nič nevyčítal. Aby neporučila pravekú mágiu a jej tvar, rozhodla sa, že to spraví hlagenovkou. Ešteže ju to Mrana naučila.

\begin{center}

*

\end{center}

Skaly mali podľa Tarnyho odhadu okolo tisíc až viac metrov. Týčili sa nad nimi ako niečo, čo sa tam naraz náhodou desublimovalo. Boli asi v polovici. Keďže a nanoobleky im zabraňovali vidieť sa, Pauline menila farbu vzduchu a vytvárala tak cestu.

"$ $Pozri tam je priechod!"$ $  Povedal Tarny Pauline.

"$ $Myslíš medzi tými dvoma skalami?"$ $ 

"$ $Áno. Ideme tadiaľ?"$ $ 

"$ $V pohode."$ $  Prechod prešli hladko a naskytol sa im pohľad na niečo modré uprostred čoho bol ostrov. Tarny zrazu zastal a začal niečo hovoriť, ale veľmi ticho a Pauline mu nerozumela.

"$ $Čo hovoríš?"$ $  Tarny si ju ani nevšimol.

"$ $V jazere z modrosivej rieky vo svete inom. Vo svite zo slnka boha, v strede je krvavá misa mede, napi sa z nej a zomri, alebo získaj to o čom sa sní."$ $ 

"$ $Čo?"$ $ 

"$ $To bolo v jednej knihe, čo sme raz objavili a Tulienka Deľa to preložila. A teraz sme tu. Všetko sedí! Milujem ťa Pauline!"$ $ 

"$ $Čo?!"$ $ 

"$ $No, raz sme boli na takom, no výlete pri mayských pyramídach a toto sme našli. Konkrétne niekoľko mayských kníh, ktoré prezieraví mayskí čarodejníci ukryli s tým, že ich nemôže nájsť nikto, kto nemá magický gén. A keďže Tulie je Tulie, ona to musela zobrať a prečítať. A našla pri tom túto pasáž. A toto jazero všetko spĺňa... Aj to, že je to v svete inom, i to, že je to jazero, i to, že je to v obklopení skál, a dokonca aj presný tvar! Našli sme to!"$ $  Pauline jeho nadšenie nezdieľala.

"$ $A čo to má byť?"$ $ 

"$ $Meď! Truhlica! Máme to!"$ $ 

"$ $Hm?"$ $ 

"$ $No... ten text hovorí o tom, že meď je v chráme zasvätenom bohovi slnka na ostrove."$ $ 

"$ $Myslíš si to naisto? Nemôže sa nám nič stať?"$ $ 

"$ $To áno... zomri alebo získaj. My ideme získať. Ideme!"$ $ 

"$ $Si si istý?"$ $  Miernila ho.

"$ $Samozrejme!"$ $ 

"$ $Ak zomriem, zabijem ťa!"$ $ 

"$ $V pohode... Ale ak zomrieš ty, zomriem aj ja, teda nebudeš mať koho zabiť!"$ $  Neprestával žartovať.

"$ $O tom sa nežartuje!"$ $ 

"$ $Ale áno! Poznáš termín čierny humor?"$ $ 

"$ $Hej!"$ $  Ignoroval ju.

"$ $Ideme..."$ $ 

"$ $Ale..."$ $ 

"$ $Ideme odtiaľto prečo!"$ $  Vychrlil zo seba. Pauline nevedela , čo má povedať, a tak súhlasila.

"$ $To hej..."$ $ 

"$ $Na ostrov."$ $  Tarny spomalil krídla a začal klesať. Pauline ho teda nasledovala, lebo nemala lepší plán.

\begin{center}

*

\end{center}

Miesto oka nebolo na pohľad výnimočné. Tulienka Deľa naň vyšla, po zmenšení všetkých svojich vecí, a cítila energiu. Magická energia bolo akosi blokovaná, ale bola tam. Ponorila sa do energetického poľa a všimla si to, čo energiu blokovalo. Mala približné tušenie, čo zničenie toho môže spôsobiť, ale chcela to risknúť. Všetku energiu, čo dokázala použiť, použila na zničenie energetickej blokácie...

Odhodilo ju to dozadu a na chvíľu omráčilo. Bolela ju hlava a cítila brnenie. Magická energia ju obklopovala a ona bola až zarazená z jej prudkosti a veľkosti. Otvorila oči a zdesene sa pozrela okolo seba. Cítila prudkú bolesť. Jej ruky boli popálené mágiou. Svoje zvyšky mágie použila na vyliečenie seba. Vstala. Výbuch mágie spôsobil toho viac. Okolo nej bola zem síce na nerozoznanie, že by sa tam niečo stalo, ale tam, kde bolo oko obrazu bola diera. Bez dna. Vypĺňal ju jeden veľký vír točiaci sa stále dookola a nevychádzajúci.

"$ $Fuck! Čo na toto niekto povie?"$ $  Bola možnosť, že by to bolo opatrené zmyslovou mágiou, ale o tom pochybovala. Rozmýšľala čo spraviť. Mohla vojsť do víru s nepredvídateľnými následkami, mohla čakať alebo mohla odísť. Snažila sa prísť na to, čo je najvhodnejšie. Myslela si, že to čo hľadá bude tam kde bola, ale ukazovalo sa to zložité. Ale teraz jej napadlo, že to by bola blbosť. Veď kto by schovával len tak, niečo tak dôležité! Zrejme to bude v tom víre. Ten vír je cesta k tomu. Alebo to je pasca pre tých, čo sa chcú zmocniť predmetu. Napadlo Tulienke Deli v okamihu keď vstúpila doňho...

\begin{center}

*

\end{center}

Jazero menilo farbu viac na červenú, čím bolo bližšie k ostrovu. V jeho strede bol postavený chrám zo zlata.

"$ $Zosadneme."$ $ 

"$ $Si si istý?"$ $ 

"$ $Samozrejme."$ $ 

"$ $Si si úplne istý?"$ $ 

"$ $Samozrejme."$ $  Zas odpovedal rovnako.

"$ $A čo ak zomrieme?!"$ $ 

"$ $Samozre... čo hovoríš?"$ $ 

"$ $Čo ak tam zomrieme?! Tarny, počúvaj ma!"$ $ 

"$ $Ja ťa počúvam..."$ $  Bránil sa.

"$ $Nemyslím. Odpovedz mi na otázku."$ $ 

"$ $Tak zomrieme."$ $ 

"$ $Pred chvíľou si mi hovoril, či chcem zomrieť?! A teraz...!"$ $ 

"$ $Toto mení situáciu. Máme možnosť sa odtiaľto dostať, aj získať truhlicu!"$ $ 

"$ $A sakra načo! Načo sme ich išli získať?!"$ $ 

"$ $Aby sme porazili D."$ $ 

"$ $Ty ho chceš poraziť?!"$ $ 

"$ $Nie, ty."$ $ 

"$ $Ja?"$ $ 

"$ $Si Goonová proroctva. Ty si vyvolená. Takže nezomrieš, pretože proroctvá sa plnia. A až príliš presne."$ $ 

"$ $Si si istý?"$ $ 

"$ $Samozrejme. A napokon, ako sa inak odtiaľto chceš dostať. Máš pár dní a ak nenájdeš vodu a jedlo zomrieš, tak či tak."$ $ 

"$ $Tak prečo nehľadáme niečo iné?!"$ $ 

"$ $Lebo máme možnosť. A ja ju nehodlám premárniť. Ideš?"$ $ 

"$ $Ale ak zomriem, je to tvoja vina!"$ $ 

"$ $My nemôžeme zomrieť Pauline! My zo spoločenstva nie..."$ $ 

"$ $Ale ja nie som..."$ $ 

"$ $Si... máš krv so spoločenstva M, i keď si neprisahala na dýku."$ $ 

"$ $Na čo?"$ $ 

"$ $Na Izabetinu dýku. To musí každý, kto vstúpi do spoločenstva."$ $  

"$ $Ja som..."$ $ 

"$ $Ale tvoji rodičia, resp. Izabeta áno."$ $ 

"$ $A čo s tým?"$ $ 

"$ $Prísaha zabezpečuje nesmrteľnosť. A to je už dedičné."$ $ 

"$ $Si si istý?"$ $ 

"$ $Stopercentne."$ $ 

"$ $Tak..."$ $ 

"$ $Ideme."$ $  Tarny a Pauline zleteli priamo nad ostrov.

"$ $Stále máme obleky?"$ $ 

"$ $Kde ich chceš vyzliecť? Stále signalizuj ako teraz."$ $ 

"$ $V pohode. Ale vyčerpáva ma to."$ $ 

"$ $Neskolabuj."$ $ 

"$ $Veď máme baterku Tarny! A môžeš navigovať ty!"$ $ 

"$ $Ja som debil. Kde je?"$ $ 

"$ $Tu."$ $  Podala mu ju.

"$ $Že sme si na to nespomenuli skôr."$ $ 

"$ $Však?"$ $ 

"$ $Hm..."$ $  Nasledovalo tiché klesanie na ostrov. Keď boli tesne nad ním Tarny prehovoril.

"$ $Neklesajme úplne na zem, nemôžeme vedieť, či nie je otrávená. Držme sa pár centimetrov nad zemou."$ $ 

"$ $Pravda."$ $  Na ostrove nerástlo žiadne rastlinstvo, aspoň ho nevideli. Chrám bol jasným monumentom.

"$ $Čo myslíš, ako dlho tu nikto nebol?"$ $ 

"$ $Tak... veľa rokov..."$ $  Zamyslel sa Tarny. Okolie chrámu bolo rovnako tepané zlatom ako chrám samotný. Na jeho stenách bolo vyryté piktopísmo, obrazce slnka a neznámeho symbolu podľa fentenzíjskych zvyklostí.

"$ $Keby tu bola Tulienka Deľa... Tá by nám to všetko preložila..."$ $ 

"$ $A na čo to je potrebné?"$ $  Tarny sa na ňu nechápavo zahľadel, ako ju mohlo také dačo len napadnúť.

"$ $Pauline, začínam pochybovať o tvojej inteligencii..."$ $ 

"$ $Tarny!"$ $ 

"$ $Hm?"$ $ 

"$ $Prestaň ma ponižovať láskavo!"$ $ 

"$ $Ja konštatujem."$ $ 

"$ $Prestaň!"$ $ 

"$ $Prečo?"$ $ 

"$ $Lebo... ma podceňuješ!"$ $ 

"$ $Možno, ale pravdivo..."$ $ 

"$ $Čo som ti spravila?"$ $ 

"$ $Potrebnosť piktopísma!"$ $ 

"$ $A?"$ $ 

"$ $Nechápeš! Veľa vecí, ktoré takto nezistíme, alebo zistíme ťažkou cestou sme mohli vedieť!"$ $ 

"$ $A si si istý, že to nie je nejaký nápis, v ktorom je napríklad zoznam potravín, čo nakúpiť?!"$ $ 

"$ $Piktopísmo sa nikdy nepoužívalo len tak. Na to je príliš zložité."$ $ 

"$ $Ako vieš, že tu sa ním normálne nepísalo?"$ $ 

"$ $Na chrám?!"$ $ 

"$ $A?"$ $ 

"$ $Toto je podľa mňa tak pusté miesto, že by tu niekto bol nakupovať."$ $ 

"$ $Ale nevieš, či tu niekde nie je civilizácia?!"$ $ 

"$ $Pozri, tu by sa podľa mňa nedostali, ako..."$ $ 

"$ $Možno mali krídla tiež!"$ $ 

"$ $Tu už nikto dávno nebol! Nie sú tu žiadne stopy!"$ $ 

"$ $Mohli lietať. Alebo sa vznášať! Nevieš aké stopy zanechávajú."$ $ 

"$ $Toto je ako tvrdenie: gravitáciu spôsobujú ružový trpaslíci ktorých nemožno detekovať."$ $  Pauline sa naňho naštvane pozrela a on vtom povedal.

"$ $Prečo sme sa vlastne začali hádať?"$ $ 

"$ $Pravda."$ $ 

"$ $Čo?"$ $ 

"$ $To nebola rečnícka otázka?"$ $ 

"$ $Nie."$ $ 

"$ $Ale..."$ $ 

"$ $O čom sme sa to začali hádať?"$ $  Zopakoval.

"$ $Urazil si ma."$ $ 

"$ $Nie, zapochyboval som o tvojej inteligencii, po tvojej... no, hlúpej otázke."$ $ 

"$ $Ja som normálne spýtala!"$ $ 

"$ $Nehádaj sa, spýtala si sa možno normálnym tónom, ale otázka to bola neuveriteľne hlúpa. Piktopísmo je neuveriteľne zložitá vec, a kto by do toho dával niečo nedôležité?"$ $ 

"$ $A čínština hádam nie je zložitá?! A hovorí ňou vyše miliarda ľudí!"$ $ 

"$ $Ale rozprávajú ňou od mala."$ $ 

"$ $Ale aj piktopísmom môžu písať od mala!"$ $ 

"$ $No..."$ $ 

"$ $Zamotal si sa."$ $ 

"$ $Nie, piktopísmo je z Fentenzie, nie odtiaľto."$ $ 

"$ $A čo keď je? Nevieš to!"$ $ 

"$ $Je..."$ $ 

"$ $Počúvaj ma!"$ $ 

"$ $Ja ťa počúvam. Len ty hovoríš niektoré veci neuveriteľne.. no..."$ $ 

"$ $Podľa teba má všetko nejaký význam!"$ $ 

"$ $Ale všetko musí predsa..."$ $ 

"$ $Počúvaj ma konečne! Vo všetkom vidíš nejaký význam, aj keď tam nie je! Proste všetko nemusí mať nejaký hlboký význam! Akosi priveľa snívaš Tarny!"$ $ 

"$ $Nie, len..."$ $ 

"$ $Čo len?! Sakra! Ty v úplne všetkom, vrátane miliónteho dvetisíc päťdesiateho siedmeho zrnka piesku vidíš niečo hlboké, niečo čo ti pomôže zachrániť svet! Prečo máš neustálu potrebu robiť so seba hrdinu?! Prečo sa stále chystáš zachraňovať svet? Ja mám toho po krk! Stále hovoríš svoje reči o svojej úžasnosti, ale skutočne si nič! Si jeden z mnohých! Si presvedčený, že si niečo absolútne úžasné, ale čo si naozaj?! Čo si o sebe myslíš? Snažíš sa stále hľadať v úplne všetkom niečo nadprirodzené a byť presvedčený, že to ty si výnimočný! Prečo?! Prečo si myslíš, že všetko zapadá do tvojho vyvolenia, ktoré si si vytvoril sám?! Si schopný nájsť niečo úžasné v každom jedinom písmenku každej knihy, podľa mňa i zbierky básní pre deti! A Tulienka Deľa je taká istá! Ste presvedčení o vlastnej výnimočnosti, čo máte nejaký komplex alebo čo?! Sakra, prečo ma do toho ťaháte? Prečo?!"$ $  Pauline zmĺkla a nadýchla sa. Tarny na chvíľu ostal ticho a následne povedal.

"$ $Ja nehľadám vo všetko niečo výnimočné. Ja len pozorujem. A piktopísmo je taký nevídaný zjav, že..."$ $ 

"$ $Nevídaný? Veď ho nachádzame všade, sakra! V knihách, existujú slovníky atď..."$ $ 

"$ $Sakra, Pauline, ja chcem povedať, že.."$ $ 

"$ $Čo?! Čo chceš povedať...?!"$ $  Tarny sa nadýchol a premýšľal ako sformulovať, čo chcel povedať. Počas rozmyslu sa pozrel na oblohu a niečo uvidel. Chcel zatelepatizovať Pauline, ale uvedomil si, že nemôže. Tak ju schmatol a vtiahol do chrámu.

"$ $Čo...?"$ $  Takmer zvreskla, keby jej Tarny nezapchal ústa.

"$ $Tmhnhrr!"$ $ 

"$ $Ticho."$ $  Šepol a takmer nedýchal.

"$ $Čo..."$ $  Nahnevane sa na ňu pozrel.

"$ $Ticho... Niečo je vonku, aby nás to nezbadalo..."$ $  Pauline si uvedomila, že je naozaj vydesený. Takého ho ešte nevidela.

"$ $Čo...?"$ $ 

"$ $Pst!"$ $  Sykol. Z jedného vrecka vytiahol pero a napísal si na ruku zreteľné slová pre Pauline.

"$ $Čítaj..."$ $  Ticho pošepol.

Vonku je niečo. Vyzeralo to ako neľudská bytosť, ale tiež aj agresívne. Buď prosím ťa ticho a snaž sa nedýchať nahlas. Nečaruj, nevieš, či to nemôže zachytiť mágiu.

Pauline dočítala a pozrela sa na Tarnyho. Stále bola s ním v stave hádky, ale teraz to musela odložiť nabok. Zadržiavala v sebe hnev, ale uvedomovala si, že je v ohrození života. Druhý raz na pár hodín! Tú chvíľu čo tam nehybne sa vznášali, pár centimetrov nad zemou, rozhodla sa využiť na porozhliadanie sa. Sme stále neviditeľní, napadlo jej keď Tarnyho nikde nevidela. Alebo on zmizol... Ale prečo potom videla jeho ruku...? Vtedy sa v jej hlave ozval jeho telepatický hlas.

"$ $Je tu koncentrovaná mágia, aká človeka aj zabije! Ostaň tam kde si a ja to tu preskúmam. Nechcem, aby sa ti niečo stalo."$ $ 

"$ $A prečo si ma potom..."$ $ 

"$ $Teraz sa nehádaj, aspoň na chvíľu..."$ $ 

"$ $S hádkou si začal ty..."$ $ 

"$ $Neobviňuj, perkele!"$ $ 

"$ $A čo..."$ $ 

"$ $Teraz nie, potom sa dohádame, neboj sa nezabudnem..."$ $  Zatelepatizoval s náznakom sarkazmu Tarny. 

"$ $Kde si?"$ $ 

"$ $Neuvidela by si ma. Ani ja nevidím teba."$ $  Pauline okolo seba videla oltáre a sochy, ale niečo sa jej nezdalo. To, že to bolo dávno opustené, nebolo pochýb. Napriek tomu na hlavnom oltári svietilo svetlo, vyzerajúce akoby sa ho už dávno nikto nedotkol. Tarny sa už dlho neozýval. Na pravej strane od nej bolo niekoľko obrazov zo zlata. Z vzdialenosti akej od nich bola, usúdila, že je to obraz oblohy nad ktorou sú dva páry očí, pozorujúce svet a píšuce do knihy. Na knihe bola kopa symbolov a piktopísma. Teraz musela súhlasiť s Tarnym, že Tulienka Deľa nebola až tak na zahodenie, a písanie písma do obrazu muselo mať nejaký význam, i keď, tak momentálnu náladu umelca. Na druhom obraze bol výr ukladajúci sa do truhlice z medi. I keď bol obraz zo zlata, na truhlici mal zvláštnu, medenú farbu. Po dlhšom prezeraní zistila, že farba slabo preblikuje medzi zlatou a medenou. Rovnako aj vír bol miestami červený, inde a inokedy žltý a niekedy oranžový. Niečo sa jej nezdalo. Vtedy dostala správu od Tarnyho.

"$ $Okamžite príď k oltáru! Okamžite! Som tam!"$ $  Vtedy uvidela Tarnyho pri oltári, tiež zo zlata, snažiaceho sa na niečo prísť.

"$ $Čo je?"$ $  Zabudla na to, že má byť ticho. A rovnako i Tarny.

"$ $Chrám o chvíľu bude zničený. Tie netvory privábil pulz vybuchujúcej mágie. Sú to deti smrti. Prosím ťa. Okamžite!"$ $  Rýchlo dýchal a bol zjavne znepokojený a vystrašený. "$ $Pauline!"$ $ 

"$ $Čo je?!"$ $ 

"$ $Tu si! Máme päť minút, a teraz už menej, na dostanie sa odtiaľto. Inak to tu exploduje a mágia zničí štít!"$ $ 

"$ $Čo?!"$ $ 

"$ $Ak neprídeme na to ako odtiaľto vypadnúť, tak zomrieme!"$ $ 

"$ $Ale veď.."$ $ 

"$ $Výbuch mágie ničí všetku mágiu! Niekto nepredpokladal, že do vyčerpania mágie nepríde ku koncu sveta. Čo to len bolo v tom proroctve...?"$ $  Nepýtal sa Pauline. Bol pod stresom a pokúšal na niečo prísť skôr ako im vyprší čas. Pauline sa pokúšala na niečo racionálne prísť tiež, predsa len, zomrieť nechcela. Spomenula si na to čo Tarny nepríčetne opakoval.

"$ $Krvavá misa mede, napi sa a zomri alebo získaj čo chceš. Napiť sa! Kde je tá meď?!"$ $ 

"$ $Si si istý?"$ $ 

"$ $Zomrieme aj tak."$ $  Tarny začal skúmať chrám a čas pomaly odbíjal. Pauline tiež zmätená pobehovala a nevedela čo má robiť.

"$ $Mám truhlicu! Je tu. Je v nej vír. Jediné čo si pamätám z piktopísma."$ $  Na truhlici bolo piktopísmom napísaný nápis "$ $Prechod Mede"$ $ .

"$ $Lenže ako to roztaviť? Ten zámok, fuck! Sakra! Perkele!"$ $ 

"$ $Mágiou?"$ $ 

"$ $Nedá sa! Je príliš pevný iba ak..."$ $ 

"$ $No?"$ $ 

"$ $Nemôžem to urobiť sám. Ide o to rozbiť celý chrám, uvoľniť pritom mágiu a zároveň pritom uvoľniť výr z truhlice."$ $ 

"$ $Hm?"$ $ 

"$ $Minúta. Ideme do toho alebo čakať na istú smrť?"$ $ 

"$ $Skúsime to samozrejme!"$ $ 

"$ $Urob štít!"$ $ 

"$ $Nehovoril si..."$ $ 

"$ $Štít! Okolo nás oboch!"$ $  Čas pomaly, ale iste odbíjal. Pauline zatvorila oči a urobila štít. Počula akoby sekundy do smrti.

\begin{center}

*

\end{center}

Zdalo sa je to ako večnosť, že zastal čas. Cítila, že má otvorené oči, ale nevidela nič. Bola ticho, ale nič nepočula, ani len svoj dych. Necítila žiadny vnem, ale vedela, že žije. Cítila akoby plávala v tmavom tuneli. Premýšľala kde sa dostala. Čo keď to bola naozaj pasca a už sa odtiaľ nedostane? To čo urobila bolo nepremyslené. Mala čakať na Tarnyho ale... Oni zmizli. Tma okolo nej bola absolútna. Nevidela ani seba, len vedela svojím vedomím, že zrejme je, keď to nebolo isté. Kľudne mohla zomrieť a toto bol život po smrti. Kľudne to mohli byť jej halucinácie. Alebo niečo úplne iné. Plávala čudesnou hmotou bez pohybu. Bola to pasca! Nadávala sama sebe. Vtom ju oslepilo svetlo a počula hlas Tarnyho a Pauline.

\begin{center}

*

\end{center}

"$ $... na istú smrť?"$ $  Tarnyho hlas. Bolo to zdanie alebo realita?

"$ $Skúsime to samozrejme!"$ $  Pauline hovorila o nejakej možnosti. Mala Tulienka Deľa halucinácie?

"$ $Urob štít!"$ $  Vyzýval Tarny Pauline. Ak to je pravda, musí ísť o niečo, kde to Tarny nemôže urobiť. Teda niečo vážne.

"$ $Nehovoril si..."$ $  Pauline sa niečo pýtal. Začínalo Tulienke Deli šibať? Pýtala sa sama seba a uvedomila si, že o sebe hovorí v tretej osobe.

"$ $Štít!"$ $  Tarny mal vystresovaný hlas. Niečo veľmi vážne.

"$ $Okolo nás oboch!"$ $  Buď nemohol čarovať on, ale to by bolo čudné, alebo robil niečo dôležité, čo si vyžadovalo celý jeho magický potenciál. Vtedy, keď o tom premýšľala a chcela vedieť, či je zo pravda, a čo sa vlastne s nimi deje, svetlo začalo vrhať jasný obraz...

\chapter{Fequelova stopa}

Videla ich siluety a všetko. Bolo to ako skutočné. Ak nemala halucinácie, alebo sa nezbláznila, videla ich. Išlo im o život. Mohla sa ich pokúsiť dostať k nej, ale zdalo sa jej to nemožné. Vychádzalo v tej rýchlosti ako najvýhodnejšia možnosť. Volala na nich. V zúfalosti, zdalo sa jej to všetko strašné... a nemožné...

\begin{center}

*

\end{center}

Zámku tavila mágia rýchlejšie ako sa Tarny nazdával. Chytil Pauline za ruku, nevedel či to stihne, už sa pripravoval na zničenie mágiou. Keď vtom počul hlas Tulienky Deli.

"$ $Tarny! Tarny!"$ $  Nesústredil sa. Chcel sa pokúsiť dostať do víru v priebehu pár zlomkov sekúnd, ale uvedomoval si, že to nestíha...

\begin{center}

*

\end{center}

"$ $Tulienka Deľa...!"$ $  Počula jeho hlas a desila sa toho, že ich stráca. Navždy. Musela si zachovať chladnú hlavu. Chcela ich vziať späť. Naťahovala za nimi ruku, i keď vedela, že je to nemožné... Už nebudú...

\begin{center}

*

\end{center}

"$ $Mrana je dnes doma, ale nevyrušuj ju, pracuje."$ $  Oznámila ráno Sylvia Loviise.

"$ $Si v pohode?"$ $  Spýtala sa jej Loviisa. Sylvia vyzerala strhane. Ako niekto kto sa práve vrátil zo zničujúceho boja, ktorý trval až pridlho. Pomyslela si.

"$ $Hm..."$ $  Odvetila neprítomne a malátne Sylvia.

"$ $Spala si vôbec?"$ $  Spýtala sa, keď videla veľké kruhy pod jej očami.

"$ $Pár minút."$ $ 

"$ $To sa ti ako podarilo?"$ $ 

"$ $Normálne. Nemohla som zaspať a potom prišla Mrana a mali sme rozhovor."$ $ 

"$ $O čom?"$ $ 

"$ $Programovanie, Wymyslensko, D, spoločenstvo M, fyzika, význam života...vyber si."$ $  Odvetila strhane.

"$ $Nechceš si ísť ľahnúť?"$ $ 

"$ $Nie. Idem sa učiť."$ $ 

"$ $Po prebdenej noci?"$ $ 

"$ $Človek ľahko zabudne na všetko..."$ $ 

"$ $Ale nie si unavená?"$ $ 

"$ $Pozri,"$ $  odvetila mierne zmätene. "$ $Raz som sa vrátila z výletu s Tulienkou Deľou a Tarnym. Nespala som pár dní, pretože sme sa zaplietli s D a bol z toho zásah agentúry, a po tomto tu som chodila celý týždeň do školy. Nie, nie som unavená."$ $ 

"$ $Vyzeráš malátne... nemáš horúčku..? Alebo, prosím ťa, nepila si niečo?"$ $  Sylvia sa na ňu zahľadela a potom odvetila.

"$ $Tri fľaše modrého buble bankeru. Človek veľmi dobre zabudne..."$ $ 

"$ $Človeče! Nie si plnoletá! Nemôžeš..."$ $ 

"$ $Mrana hovorí, že ak človek stále dokáže napísať softvér, ktorý ti ustelie posteľ podľa teba, nie je to zlé. Ja som sa neopila, len povedzme... prehnala to... trochu..."$ $ 

"$ $Ale je to nelegálne..."$ $ 

"$ $No a? Koľko nelegálnych činností robím ja? Tri pollitre modrého... idem si čítať.."$ $  Odvetila zničene a padla do najbližšieho kresla. Loviisa ju sledovala a rozmýšľala, s kým to vôbec je. Ona sama modrý buble banker, teda nízkoalkoholický nápoj, pôvodom z Wymyslenska, nikdy neochutnala. Nemala chuť porušovať zákony. Nápoj to bol síce s nízkym obsahom alkoholu, ale keď to niekto prehnal ako napríklad Sylvia, dopadlo to tak, ako dopadlo. Ale napokon. Pri všetkom čo ona robila, toto bolo asi najmiernejšie. Aspoň čo sa potenciálneho trestu týka. Nanešťastie.

\begin{center}

*

\end{center}

Fequel sa pozrel na výsledky. Prístroj vyhodnotil najmožnejšiu polohu osoby ako Fentenzia, Kralovo, Felanzia, lovo alebo Tramtária. Posledné zatiaľ vylúčil, resp. Tramtária bola preňho nedostupná. Tá patrila iným, a keby sa tam Mänchenová, alebo možnoD (ako ju sám nazýval) premiestnila skazila by mu plány. A to do bodky. Nepredpokladal, že bude v Kralove, odtiaľ práve predsa odišla a mohli ju spoznať. Skôr niečo odľahlejšiu. Fentenzia bola možno označovaná ich nepriateľmi ako "$ $slobodná"$ $ , ale v skutočnosti ničila slobodu, ktorú Cecília získala porážkou Solemy. Ako to chcela aj Mänchenová. A ak by tam išli, mohli čakať s tým, že ich tam budú hľadať. Veď vo Fentenzii za posledný mesiac strojnásobili ich jednotky, pre rozmáhajúcu sa protištátnu činnosť. Akýsi "$ $Priatelia Spoločenstva"$ $ ! Títo ľudia si zjavne neuvedomovali, že spoločenstvo je Démon v najčistejšej podobe, pripravujúcej pre nich smrť. Ale títo ľudia si akosi nedali povedať! Pre nich bolo spoločenstvo a Wymyslensko nejaké dva spriaznené štáty! Fentenzíjčania boli naozaj čudní! Z anatomickej stránke sa viac podobali na ľudí ako zovero! Rovnako ako tí anarchisti, tramtaríjčania. Akoby mali tú nezdravú rebéliu v krvi! Oba národy mali v sebe to nezdravé až choré odmietanie systému. Temer všetci s ktorými mali z Fentenzie starosti, mali reči plné tej čudesnej magickej slobody, ktorú nedokázali nájsť v tej, čo priniesla Cecília porážkou Solemy. Fequel síce nemal sympatie k Soleminmu režimu, ale nápad niekoľkých funkcionárov v jej poradnom orgáne zmiasť fenenzíjčanov zo sveta, sa mu vcelku pozdával. Alebo ich aspoň obmedziť. Keby ho povýšili (v čo dúfal), mal by poradné právo v rade. Určite by navrhol nenápadne zrušiť funkciu Oka, ktorá tieto choré bludy z Fentenzie a Fentenzíjskej komunity vo Wymyslensku podporovala. A tiež zakázať Fentenzísky anarchistický myšlienkový smer. Za corlovne pred Solemou, Mileny Verighe'tovej, prvej corlovne pôvodom z Fentenzie, sa tento smer výrazne propagoval. Ani jedna z nich, Solema, či Milena, nebola nič dobré pre Wymyslensko. Solema aspoň bránila jeho česť. Ale Milena... Tá prinášala nové parazitické kultúry, ktoré ničili ich systém. Tá verila v spojenectvo so spoločenstvom a na iné podobné bludy. Fequel ju považoval za zmätenú. Veru, tí Fentenzíjčania by vážne nemali byť, alebo aspoň nie ako rovnoprávny občania. Tak, ako podradný štát, to by vyhovovalo. Rovnako ako Wymypata, ostrov nakazený Lenistickou filozofiou tkz. slobody. To, že tam strážili kolaborantov so spoločenstvom zástupy polície akosi nestačilo. Tí nenápadne kazili občanov Wymyslenska a prevracali ich proti ich vlasti. Veru, s touto pliagou sa nemožno zahrávať, treba proti nim použiť hrubú silu. Rovnako ako proti Mänchenovej. Posledný výsledok sa javil Felanzia. Felanzia bola odjakživa vcelku poslušný kraj. Občania sa starali o svoje záujmy a chápali dobro, ktoré im Cecília priniesla.

Zas sa preniesol k Mänchenovej. Keďže v miestnosti, kde bývala takmer nič neostalo, mal dobrý dôvod predpokladať, že ich čakala, alebo bola pripravená na príchod polície. Ak to bol naozaj D, ich chcela zničiť nie kvôli svojej naivnosti, a tomu, že ju oklamalo spoločenstvo, alebo rovno bola jeho agentkou, resp. bol, ale pretože chce zničiť ľudstvo ako také. D pomáhal spoločenstvu, či skôr spoločenstvo jemu. Išlo mu o zničenie Wymyslenska. A presne na to bol D ako Mänchenová. Fentenzia bola dostatočne skazená a Kralovo nebezpečné. Pripadala mu ako jediná rozumná možnosť Felanzia. Tam mieril.

\begin{center}

*

\end{center}

Mierili ku nej dve tmavé postavy. Jedna mala kapucňu ktorá jej zakrývala tvár. Obe mali na sebe plášte. Ani jednej nevidela do tváre a obe ju volali. Ona sama ležala na zemi a nevedela prečo. Zomrela si predsa, nie? Spýtala sa sama seba. Alebo niekto z tvojich priateľov. D ťa našiel. Hovorila sama sebe Sylvia. Takto zrejme vyzerá život po smrti, ak to nie je nejaká predsmrtná ilúzia. Matne si spomínala na minulé udalosti, len na to, že musela odísť. Všetkých som ich uvrhla do nebezpečenstva, všetko som pokazila, všetko je zle. Napadlo jej. Teraz prišiel čas zomrieť. Alebo niečo horšie. Ona ani zomrieť takmer nemohla. Len v rovnakom momente ako D. A to bola šanca veľmi malá. Vymanila sa z nenávisti ku svojej vlastnej podstate a začala sa pozerať na postavy. Boli čoraz bližšie. Pokúsila sa vstať, ale niečo ju zastavilo. Cítila studené prsty na koži. Niekde v diaľke počula známy krik. Nedokázala sa im postaviť...

"$ $Si v poriadku? Sylvia?!"$ $  Triasla ňou Loviisa. Sylvia vystrašene otvorila oči a zahryzla sa jej do ruky.

"$ $Au! V pohode Sylvia?"$ $ 

"$ $To bol len sen..."$ $  Pre zmenu zahryzla do svojej ruky.

"$ $Sylvia...?"$ $ 

"$ $Čo je?"$ $ 

"$ $Nie si v pohode."$ $ 

"$ $To viem! Nevyzerám tak?"$ $  Zamumlala stále si hryzúc ruku a vtom sa začala biť.

"$ $Nerob to!"$ $ 

"$ $Prečo?! Hanbím sa!"$ $  Odvetila.

"$ $To nie je zdravé, biť a hrýzť samú seba. A k tomu opitá. Nechceš proti tomu niečo?"$ $ 

"$ $Ja som..."$ $  Chcela namietnuť, ale Loviisa ju prerušila.

"$ $Nie si v pohode. Vidím ťa."$ $ 

"$ $Hm..."$ $ 

"$ $Chceš niečo proti účinkom buble bankeru?"$ $ 

"$ $Ďalší prosím."$ $  Odvetila mierne neprítomne.

"$ $Nie."$ $ 

"$ $Prečo?"$ $ 

"$ $Pozri sa na seba, došľaka!"$ $ 

"$ $Čo?"$ $  Loviisa spravila pohyb, a Sylvia bola až príliš neprítomná, aby si uvedomila čo robí.

"$ $Au! Sakra, Loviisa, prečo ma páliš?"$ $  Loviisa zatvorila Solan a Naštvane sa na ňu pozrela.

"$ $Práve si to prehnala, Sylvia Mänchenová, nezdá sa ti?"$ $ 

"$ $Ani nie."$ $ 

"$ $Hm?"$ $ 

"$ $Aj tak všetci zomrieme!"$ $  S radostným výrazom na tvári zas zaspala.

"$ $Zabiť ťa."$ $  Odvetila jej Loviisa. Sylvia nebola na tom akosi psychicky dobre a Loviisa to videla. Niečo ju ničilo a bolo asi len otázkou času, kedy z nej nezostane nič. Ako Deoque. Napadlo Loviise.

\begin{center}

*

\end{center}

Precitla. Žila. Mierne jej povypadávala pamäť, ale pamätala si, ako sa im míňa čas a počuje Tulienku Deľu. Zomrela?

"$ $Tarny?"$ $ 

"$ $Žiješ?"$ $ 

"$ $Ako to?"$ $ 

"$ $Asi si pri tej energii našla zas pole na premiestňovanie a odišla si aj so mnou. Ale kde sme?"$ $ 

"$ $Ja som nezapla random..."$ $ 

"$ $Ale urobila si to nepredvídane..."$ $ 

"$ $Ja som myslela na Tulienku Deľu a... kde sme Tarny?"$ $ 

"$ $Otvor oči."$ $  Pauline sa postavila. Okolo nej bol tunel ožiarený mágiou. Bola to číra magická energia čo pulzovala a vlievala do tunelu svetlo.

"$ $Tulienka Deľa?"$ $  Pauline poslala na tunel telepatickú správu a čakala. Vtedy uvidela Tarnyho. Ten stál pri zdroji mágie spolu s Tulienkou Deľou.

"$ $Pauline!"$ $  Zavolali na ňu. Tá sa na nich ešte raz pozrela. Nevyzerali skutočne. Skôr ako tiene alebo niečo také. Ako tiene z magického svetla.

"$ $Čo čakáš? Teraz máme šancu!"$ $  Zavolala na ňu Tulienka Deľa – tieň. Pauline sa to celé nepozdávalo. Poobzerala sa okolo seba. Rozmýšľala. Čo ak toto celé bola iba ilúzia uvoľnenej mágie? Čo ak sa zachránila len ona a Tarnyho nezachránila? Toto bola najbolestnejšia alternatíva. Veď si nepamätala, či ho vôbec držala. Čo ak...? S Tarnym sa síce hádala, ale nepriala mu smrť. Nech bola akokoľvek rýchla. Pred ňou na ňu Tulienka Deľa a Tarny – tiene stále volali. Ako si im dochádzala trpezlivosť. Pauline, ale stále premýšľala o tom, kde je. Za ňou bol tmavý tunel a všetko pred ňou osvetľovala tá žiara.

"$ $Pauline!"$ $  Začula slabý hlas spoza seba. Nevenovala mu, ale pozornosť. Tarny – tieň sa na ňu obrátil a prehovoril zmeneným hlasom. Pauline začínala tušiť, že niečo nie je v poriadku.

"$ $Prečo si neprišla keď sme ťa volali?!"$ $  Zacítila ostrú bolesť a niečo ju ťahalo ku žiare.

"$ $Pauline!"$ $  Obrátila hlavu, ale nevidela nikoho. Namiesto toho ju stále a stále ťahalo ku žiare. Tulienka Deľa – tieň sa začala meniť a už to bol len tieň.

"$ $Neprišla si!"$ $  Začal tieň kričať a tieň čo bol pred chvíľou Tarnym ju chytil. Takú bolesť už Pauline dávno nezažila. Horela. Ruky jej vzbĺkli a vedľa nej videla miešanie sa tieňov do rôznych tvarov a podôb. Okolo nej sa samovoľne otvárali a zatvárali bráni do polí a inopolí. A pri nej bola medená platňa, ktorá len tak stála uprostred celej mágie a niečo hovorila. Pauline jej nerozumela. Mágia ju pálila a napadlo ju, že je to predsa len stále v chráme a zomiera...

Zacítila bolesť a naraz jej otupenie. Žila, alebo aspoň mala vedomie. Teda žila. Pauline na nijaký život po smrti neverila, a tak došla k záveru, že ju niekto musel zachrániť, alebo to všetko bolo sen. Keby tak celý jej život bola len nočná mora, z ktorej sa o chvíľu prebudí... Keď chcela otvoriť oči, začula známy hlas. Hlas, ktorý ju pred chvíľou volal...

"$ $Neotváraj oči, prosím ťa. Ešte tam máš stále nános mágie."$ $ 

"$ $Ty..."$ $ 

"$ $A radšej ani nehovor. Telepatia."$ $ 

"$ $Prečo nemám hovoriť?"$ $ 

"$ $Pozri, si celá popálená od mágie, a ak nechceš, aby sa ti niečo stalo, radšej sa ani nehýb. Tarny na niečo príde, ako tú mágiu zničiť..."$ $ 

"$ $Ty chceš zničiť mágiu?"$ $ 

"$ $Celú nie, ale aspoň kúsok."$ $ 

"$ $Ako to myslíš?"$ $ 

"$ $No, keď použiješ mágiu, nejakú vytvoríš. Táto mágia sa pomaly rozpadá a stále vzniká nová. A je jej stále viac. Kedysi sa s nadbytkami mágie vysporiadávali ukladaním jej do kovov, pretože tie jej vedeli pri vhodnej štruktúre zabrániť prejsť na povrch. No a tieto skladiská mágie sa teraz zrejme pokúsi využiť D, pretože ich je teoreticky využiť jednoduché, ak sa pritom nespáliš ako ty. Lenže zničiť to, je tak náramne komplikované, že neviem či to vôbec ide..."$ $ 

"$ $Čo? O čom hovoríš?"$ $ 

"$ $Ty si sa ma pýtala, či chcem zničiť mágiu. Tu máš odpoveď: Tarny sa tu snaží prísť na spôsob ako odstrániť mágiu čo sa dostala do teba.

"$ $Prečo ju treba zničiť?"$ $ 

"$ $Pretože mágia je ako rádioaktivita v malom, ale pri obrovských množstvách má podobné účinky."$ $ 

"$ $Je teda bezpečné používať mágiu?"$ $ 

"$ $To áno, lebo telo si pri jej používaní robí štít. Ale ide o to, že ty si štít nemala. A pri takom obrovskom množstve by asi aj tak nepomohol..."$ $ 

"$ $Ale ja som nikdy nerobila štít, ani keď čaroval niekto iný..."$ $ 

"$ $Vieš, sú množstvá mágie, pre ľudské telo úplne prirodzené, ale tisícročia ukladaná mágia zo Začiatku... to je už niečo iné. To je mágia ktorá sa vymyká spod kontroly každému a žije svojím vlastným životom. Otvára brány do iných svetov.."$ $ 

"$ $Polí a inopolí."$ $  Opravila ju Pauline.

"$ $Do čoho?"$ $ 

"$ $Vesmír je pole, tam kde sme boli s Tarnym, to je inopole, inopolia sú minisvety prepojené s poliami a inopoliami bránami, tak, že sa z každého inopoľa alebo poľa dá prejsť do iného nejakým súborom brán. Takýto súbor navzájom prejditeľných polí a inopolí je jedna paralela. Paralely medzi sebou nie sú prepojené."$ $ 

"$ $Ok, rozumiem, aspoň trochu. No a to magické pole čo vytvára veľa nazhromaždenej mágie, vytvára náhodné brány do polí a inopolí, vytvára ilúzie na základe podvedomí, a vybuchuje. A samozrejme ničí. Náhly náraz mágie dokáže človeka odhodiť až rozpustiť úplne na u kvarky, d kvarky a elektróny."$ $ 

"$ $Na čo?"$ $ 

"$ $Proste je to sila, náhle uvoľnená mágia, čo dokáže rozbiť protóny a neutróny na kvarky. Rozumieš?"$ $ 

"$ $Tak trochu."$ $ 

"$ $A do toho si sa dostala ty."$ $ 

"$ $Myslíš do výbuchu mágie?"$ $ 

"$ $Hej. Tie tiene čo si videla, vychádzali z tvojho podvedomia. V tomto vie byť mágia pomerne krutá."$ $ 

"$ $A ako to, že som sa nezmenila na kvarky?"$ $ 

"$ $Kvarky máš v sebe stále."$ $ 

"$ $Ale ako protóny a neutróny."$ $ 

"$ $Ja viem... Ale k tomu, prečo si stále človek, asi to, zasiahli sme v čas. Tarny na teba volal, ale nereagovala si, a jemu došlo, že máš pred sebou niečo zo zmyslovej mágie, a keď si tam išla a zasiahla ťa mágia, oddelil ju od teba."$ $ 

"$ $On to dokáže?"$ $ 

"$ $No... áno, aspoň na chvíľu. Kým som ti neodstránila najväčšie zranenia a potom ju uzavrel."$ $ 

"$ $Ale prečo sa uvoľnila?"$ $ 

"$ $No, my sme ju uvoľnili..."$ $ 

"$ $Ale ako to, že sme vôbec s Tarnym prežili? A čo sa stalo?"$ $ 

"$ $Tak obrovská energia otvorila bránu, alebo rovno tam presiaklo tak veľa zo sveta iného, teda poľa, či inopoľa, že sa dalo premiestňovať a obaja ste sa premiestnili tu. Teba náraz mágie odhodil preč a Tarnyho ku mne. On bol pri vedomí a hľadali sme ťa. Zachytili sme tvoju telepatickú správu práve vo chvíli, keď si ty už bola mierne vyvedená z miery tieňmi. A následne Tarny na teba volal, ale ty si už nereagovala."$ $ 

"$ $Ja som počula mierne volanie, ale myslela som si, že mám halucinácie..."$ $ 

"$ $Nabudúce si to už radšej nemysli. Teraz máš magické popáleniny tretieho stupňa!"$ $ 

"$ $Ale prežijem, nie?"$ $ 

"$ $To by už malo byť dobré, len Tarny príde na to čo s mágiou. Potrebujeme nejakého silného mága, ktorý by tú mágiu zničil."$ $ 

"$ $Ale keď ju zničíš pomocou mágiu, vytvoríš ďalšiu mágiu!"$ $ 

"$ $Ale..."$ $ 

"$ $Je to tak. Mágiu nemožno zničiť, len použiť."$ $ 

"$ $Sakra Pauline, kedy ťa to napadlo? Veď mágia sa rozpadá!"$ $ 

"$ $To je pravda, ale mení sa na inú energiu a veľmi pomaly. Nemôžeš mágiu zničiť!"$ $ 

"$ $Ale veď sa ničí!"$ $ 

"$ $Pomaly! Môžeš ju uzavrieť, ale nie ju zničiť. Mágiu dokážeš detekovať len s využitím ďalšej mágie, resp. použitím génov. Ale zasahovať do nej ako uzavierať, to sa bez mágie, len s génmi dá, ale nie ničiť ju. Nemôžeš tak veľmi urýchliť ten proces."$ $ 

"$ $Prečo?"$ $ 

"$ $Lebo na to musíš použiť energiu! A magickú. Ty mágiu vlastne len nahradíš!"$ $ 

"$ $Ale nie úplne..."$ $ 

"$ $Platí zákon zachovania energie."$ $ 

"$ $A pri rozpade?"$ $ 

"$ $Vtedy sa energia rozkladá na nemagické fotóny."$ $ 

"$ $A naraz to urobiť nemôžeš?"$ $ 

"$ $Myslím, že nie, pretože pri tom vytvoríš inú energiu, a pri takom množstve energie sa ti môže nekontrolovateľne premieňať na mágiu."$ $ 

"$ $A to máš odkiaľ?"$ $ 

"$ $Proste to viem. Niečo mi to hovorí. To je ako s poliami a inopoliami."$ $ 

"$ $Hm?"$ $ 

"$ $Viem o nich, a nikto mi v živote o nich nič nehovoril."$ $ 

"$ $A ako vieš, že je to pravda?"$ $ 

"$ $Myslím si to."$ $ 

"$ $A čo ak zle?"$ $ 

"$ $Pri poliach to fungovalo."$ $ 

"$ $Ale prečo to potom neskúsiť, keď nevieš, či sa to stopercentne nedá?"$ $ 

"$ $Lebo môžeš urobiť niečo... povedzme oveľa horšie."$ $ 

"$ $A to vieš ako?"$ $ 

"$ $No, tá mágia je tu určite obrovská, a keď vytvoríš tak veľa energie, zdvihneš teplotu ovzdušia a krajiny tak... že to bude horšie ako s mágiou."$ $ 

"$ $To dáva logiku."$ $ 

"$ $Viem."$ $ 

"$ $Počkaj na chvíľu. Predebatujem to s Tarnym. Pokús sa nehýbať, lebo ťa tie popáleniny budú bolieť."$ $  Tulienka začala hovoriť to, čo jej povedala Pauline Tarnymu. Ten chvíľu premýšľal a napokon povedal.

"$ $Tak ju znova vrátime do kovu. A odnesieme odtiaľto. D to nesmie nájsť, a bolo to ukryté veľmi jednoducho. Zoberieme to zo sebou."$ $ 

"$ $Ale veď potom sa dostaneme do ešte väčšieho nebezpečenstva. D bude viac záležať na tom, aby nás zabil."$ $ 

"$ $Ale musíme to odtiaľto dostať. Rozdelíme to na viac kúskov a zoberieme zo sebou a niekde skryjeme."$ $ 

"$ $To je naivné. Vieš dobre, že nie sme nejaký super mágovia. Ako to chceš urobiť?"$ $ 

"$ $Ale čo s tým? Nemôžeme to nechať tu, ale... keby sme to premiestnili do nejakého iného poľa alebo inopoľa."$ $ 

"$ $D to nájde. Môže sa premiestňovať, rovnako ako Pauline."$ $ 

"$ $A čo chceš teda urobiť."$ $ 

"$ $Skryť to."$ $ 

"$ $A kde?"$ $ 

"$ $No... keby sme to vymazali z času..."$ $ 

"$ $A to chceš ako urobiť?"$ $ 

"$ $Počul si niekedy slovo vír?"$ $ 

"$ $Je to prírodný úkaz. Veterný vír, magický vír, vodný vír... Ale čo to má spoločné so skrývaním mágie?"$ $ 

"$ $Nie taký vír, Tarny."$ $ 

"$ $A aký?"$ $ 

"$ $Vír je jeden zo svetov, či inopolí, či ako to Pauline nazýva."$ $ 

"$ $A čo s tým?"$ $ 

"$ $Vo víre je podľa fentenzíjskych legiend miestnosť so všetkými momentmi a miestami sveta. Funguje na princípe hlagenovky."$ $ 

"$ $Chceš tam ísť?"$ $ 

"$ $A do miestnosti budúceho momentu, keď bude tá mágia potrebná dáme ju."$ $ 

"$ $To myslíš vážne? Veď to je legenda."$ $ 

"$ $Zmizla v nej jedna Oko."$ $ 

"$ $Zas len legenda."$ $ 

"$ $Ale fentenzíjska!"$ $ 

"$ $A čo s tým?"$ $ 

"$ $Fentenzíjske legendy majú veľký základ v pravde, dokonca vo väčšine príbehov ide o takmer neporušenú začiatočnú reálnu skúsenosť. Táto skúsenosť sa opiera o najmä bohaté Fentenzíjske dejiny, a tiež o to, že ich viera sa zakladá hlavne o reálnu skúsenosť a ich mytológia je založená na histórii, nie na panteóne bohov. Citácia z knihy, Úvod Do Fentenzíjskej Filozofie a Kultúry."$ $ 

"$ $Veríš nejakej knihe? A aj tak, bolo tam – väčšinu a takmer. A vír je..."$ $ 

"$ $Inopolia existujú naozaj. A vír je jedným z nich."$ $ 

"$ $Ale čo ak nie je? Čo ak zoberieme mágiu, a vír nebude?"$ $ 

"$ $Vír je."$ $ 

"$ $Nič nie je na sto percent. Ja jeho existenciu nevylučujem, len o nej dosť pohybujem."$ $ 

"$ $Jasné, že nič nie je isté na sto percent. Ani to, že keď to skryjeme inde, že to D nenájde. Ale keď to dáš do iného času, vymažeš to z tohto času, a ono to vznikne až potom."$ $ 

"$ $Zmeníš budúcnosť."$ $ 

"$ $Budúcnosť pre nás ešte nie je. Tvoria ju všetci, vrátane nás."$ $ 

"$ $Ale tu nejde o nás. Tu ide o všetkých. Tá mágia sa tam objaví len tak a čo potom?"$ $ 

"$ $Objaví sa vtedy, keď bude potrebná."$ $ 

"$ $Ale ako to budeš vedieť?"$ $ 

"$ $Hlagenovka."$ $ 

"$ $Ale to byť potrebný je vysoko abstraktný pojem, a taká obyčajná hlagenovka..."$ $ 

"$ $Ale to nie je obyčajné. Je to nasiaknuté mágiou a pole je vytvorené nie cez obyčajný program ako normálna hlagenovka."$ $ 

"$ $Ale telepatické polia sú založené na programe..."$ $ 

"$ $Nie, nutne byť nemusia, to je obyčajné zjednodušenie. Aj my môžeme telepatizovať bez umelo vytvoreného poľa."$ $ 

"$ $Vytvorili sme ho my."$ $ 

"$ $Mágia, ktorú ovládame. A keďže mágia má tendencia správať na nepredvídateľne."$ $ 

"$ $Je to energia."$ $ 

"$ $Je tu entropia. No a keďže mágia sa správa ako sa správa, vytvorila aj vír a mimoriadne silné hlagenovo pole."$ $ 

"$ $Ale ako vieš, že to naozaj je? Čo keď to je rozprávka alebo výmysel? Zoberieš mágiu zo sebou a potom čo? A ako sa tam chceš dostať? A čo keď spravíme ešte väčšiu blbosť ako teraz? Mágia musí byť v bezpečí, a to nie len pred D. Je veľa ľudí, ktorí by to využili. A to nie len ľudí."$ $ 

"$ $Cecília?"$ $ 

"$ $Nielen. Ide o to, že sme sa dostali do väčšieho problému ako sme boli, a teraz musíme konať rýchlo. Veľmi rýchlo. Vrátim tú mágiu späť do kovu. Ale rozdelím kovy. Nech má každý pri sebe jeden. Pauline sa dokáže premiestniť, a my sme zas lepší v mágii. Dolieč Pauline čo najrýchlejšie tie zranenia a medzitým ideme domov. Cestou sa dohodneme a potom môžeme hľadať tvoj vír."$ $ 

"$ $A čo s tou mágiou? Nepochyboval si o ňom?"$ $ 

"$ $Možno, ale ak existuje, tak je riešením. Ale nepremiestnime to do toho hlagenovho poľa, ale necháme to vo víre. A necháme to zmiešať so zvyškom mágie."$ $ 

"$ $Ale nespraví to niečo?"$ $ 

"$ $Inopolia a polia sú z takého množstva energie, že im toto podľa mňa príde ako kvapka v mori."$ $ 

Ozvala sa telepaticky Pauline, ktorá počúvala ich rozhovor.

"$ $Takže to by mohlo byť. Tarny..."$ $ 

"$ $Viem."$ $ 

Energia nazhromaždená za tisíce rokov žila svojím vlastným životom. Tarny ju pozoroval. Zmietala sa, ale neprešla cez blokáciu, ktorú jej vytvoril. Inak by sa asi rozliala po okolí. Kov, v ktorom bola držaná ležal na kraji. Tarny nemohol použiť mágiu, aby ju nerozmnožil, a tak ju privolal. Mágia kovu neublížila. Bol to zvláštny kov. Tarny sa aj mierne desil toho, aby nepoškodil jeho vlastnosti rozdelením na tri časti, pretože mágiu museli do niečoho vložiť. Dostal sa do magického poľa. Čiastky mágie netvorili chumáče ako bol zvyknutý, ale oblaky, cez ktoré nebolo vidieť. Síce sa mágia bežne považuje mimo magického poľa neviditeľné, pri vysokej koncentrácii ju možné vidieť. A to i cez blokáciu. Samotná mágia sa riadila zákonmi, o ktorých už boli nespočetné množstvá vedcov presvedčené, že konečne vyriešili jej posledné tajomstvo, ale potom došli na to, že jej premenlivosť je až príliš veľká na jednu rovnicu. Moderní mágofyzici sa už ani nesnažili vytvoriť jednotnú teóriu mágie, či nedajbože ju spojiť s kvantovou, či makrofyzikou, ale pokúšali sa prísť na to, presne čo sa ešte musí preskúmať. Problém mágofyziky dneška bol hlavne v tom, že mnohé udalosti, ktoré by im pomohli vo výskume, sa stali až príliš dávno, aby boli stopercentne dokázané, a nechceli čerpať z legiend, čo celej mágofyzike viac škodilo ako pomáhalo.

Tarny čítal niekoľko kníh o vlastnostiach mágie, či už od starších autorov, či z modernej mágofyziky. Ale nikde nič nebolo o uzavieraní mágie. Čakal, že to bude niečo ako programovacia mágia, ale to by tam musel niečo, nejaké kúzlo, čo je len zlomok mágie do toho zakliať a zvyšok? Keby mohol nasýtiť ten kov mágiou, tak, aby nebolo riziko výbuchu, ale to nevedel. Najprv sa to zdalo veľmi ľahké a naraz... Zas sa ponoril do magického poľa a zachytil mágiu, čo presakovala cez blokáciu a následne našiel kov. Ten mal v sebe nejaké neznáme častice, ktoré síce vyzerali ako mágia, ale neboli nimi. Alebo boli, ale ostávali pevne v kove, nepoletovali ako mágia. Mágiu čo ovládal priblížil ku kovu a vtedy sa udialo niečo zvláštne. Mágia sa vsiakla doňho a on videl ako sa vytvorilo niekoľko ďalších neznámych častíc. Antimágia. Napadlo mu. Ale nevytvára fotóny, alebo aspoň nie také ako poznal. Išli pomalšie, veľmi pomaly, rovnako ako mágia, ktorá mala veľmi nízku rýchlosť, napriek obrovskej energii. Tento jav objavili pomerne nedávno, presne pred dvadsiatimi siedmimi rokmi, spomenul si Tarny. Ale napadlo mu aj niečo iné. Teraz, keď videl ako sa kov správa pri kontakte s mágiou, a teda ho nemohol rozdeliť. Musel ostať celistvý. Aspoň sa nemusel starať o to, čo by sa mohlo stať pri jeho rozdelení. Teraz aspoň fyzika stála pri ňom.

\begin{center}

*

\end{center}

Bolo predpoludnie, a takmer už obed a Sylvia stále spala a striedavo sa zobúdzala a kričala nezmysly. Loviisa sa na ňu už nemohla pozerať na to, ako Sylvia má sebadeštruktívne sklony, tak zobrala jednu zo Sylviiných kníh, tú čo mala Sylvia otvorenú na posteli a začala čítať. Názov knihy znel Zabudnutia, a Loviise sa nechcelo hrabať Sylvii vo veciach, a tak si vzala knihu túto. Nechala záložku Sylvii tam kde mala knihu otvorenú a otvorila na prvú stranu. Autorkou diela bola nejaká Fentenzíjčanka, aspoň podľa mena. Kniha bola striedavo vo fenzenzíjčine, wymyslenčine a angličtine. Fentenzísky Loviisa nevedela, a tak preskočila prvé tri strany a pustila sa do čítania. Básne. Sylvia číta básne? Odkedy? To ju prekvapilo. Ale zas, pomyslela si potom, nepoznám ju nejako dlho. Po prvej básni ju mierne prekvapenie upustilo.

"$ $Krik V Tvojej Nočnej More

Kto ťa spoznal a kto ťa nenávidel,

Kto ťa zničil a kto to chcel,

Kto ťa preklínal a kto ťa zabil?

Posledný z posledných zastali naveky,

Zavri sa, zostaň sám,

Väzniteľ samého seba,

Tmavý tieň ľútosti,

Ale kto ťa už bude ľutovať?

A kto ťa nenávidí?

A ktorých viac?

Už ťa nikto nebude ospevovať,

Už ťa nikto nebude milovať,

Znič svoju ilúziu hermeli,

Zakrič a už len navždy mlč,

Zatvor oči alebo sa znič!

Nikto ťa nezničí,

To spravíš ty!

Ktorý tieň ti zahmlil myseľ,

Ktorý cit ťa zblúznil?

Poznáš sa, či sa len ľutuješ?

Smrť si za tebou prídeš,

Či chceš alebo sa skryješ.

Roztrhal som ti oči,

Tak prečo stále vidíš?

Zničil som ti myseľ,

Tak prečo stále myslíš?

Zabil som ťa,

Tak prečo stále žiješ?

Ostávaš živý so svojimi snami,

Nezomrel si keď si sa rozložil,

Svojho konca sa dožil,

Krič vo svojej nočnej more, je to zle, či dobre?"$ $ 

Loviisa dočítala báseň a došla k tomu, že sa to k Sylvii vcelku hodí. Začal s ďalšou a ďalšou a to ju čím ďalej utvrdzovalo v tom ,že sa Sylvii tá knižka určite nedostala náhodou. Autorka písala ešte smutnejšie, kritickejšie, nenávistnejšie, sebadeštruktívne básne, že sa Loviisa vôbec nechápala, ako sa jej to môže páčiť. Keď dočítavala pasáž z obzvlášť dlhej básne, niekto jej poklepal po ramene. Sylvia.

"$ $Ja som sa ťa zľakla... Prepáč že..."$ $  Vydýchla.

"$ $To je v poriadku... Nemáš nejaké lieky na bolesť hlavy? Je asi pravda, že nie som v stave, aby som to vyliečila mágiou..."$ $ 

"$ $To je pravda. Spýtam sa Mrany?"$ $ 

"$ $Tá pracuje..."$ $ 

"$ $Ja nič nemám..."$ $ 

"$ $Nevieš tú nejako zmierniť?"$ $ 

"$ $Ani nie..."$ $ 

"$ $Sakra..."$ $ 

"$ $Neľahneš si?"$ $ 

"$ $Ale ja nie som unavená... práveže spať nechcem, lebo sa mi stále vracajú tie mory."$ $ 

"$ $To je problém..."$ $  Sylvia pokračovala. "$ $A nemôžem sa, ale ani na nič sústrediť."$ $ 

"$ $Nemáš ty nejaké lieky?"$ $ 

"$ $Tak... nie. Ja nebývam chorá."$ $ 

"$ $A teraz si...?"$ $ 

"$ $Len ma bolí hlava."$ $ 

"$ $A? To snáď nie je choroba?"$ $ 

"$ $Len symptóm..."$ $ 

"$ $A z toho je choroba..."$ $ 

"$ $To je zo stresu..."$ $ 

"$ $A? Aj z toho bývajú choroby."$ $ 

"$ $Ok.. nehádam sa. Vážne nemáš nič?"$ $ 

"$ $Spýtaj sa Mrany, teda Kerie..."$ $ 

"$ $Ale..."$ $ 

"$ $Poď..."$ $ 

"$ $Môžem si za to sama..."$ $ 

"$ $Tak ti nemám pomôcť?"$ $ 

"$ $Ale áno... ja..."$ $  Chcela sa udrieť, ale potom sa pozrela na Loviisu, ktorá pokrútil hlavou a zastavila sa.

"$ $Fajn... teda idem za Mranou."$ $ 

"$ $Idem s tebou."$ $ 

"$ $Díki."$ $ 

Mrana sedela pred počítačom. Po doprogramovaní programu na skúmanie myslí ľudí v okolí senzoru, a s tým spojené automatické zbalenie, na ktorom pracovala už dlhšie sa chcela vrhnúť do niečoho pre firmu, i keď si bola vedomá, že programov, ktoré pomaly vypúšťajú ako nové a nové aplikácie, majú dosť. Nechcela, aby jej firma bola nejaká výnimočná. Pre ňu bolo dôležité, aby si ju polícia nevšímala. Zbežne prebehla senzory. Vtedy jej zatelepatizovala Sylvia.

"$ $Mrana, nemáš niečo proti bolesti hlavy?"$ $ 

"$ $Tretia polica doprava, miestnosť 5G. Ja som ti hovorila, že nemáš piť."$ $ 

"$ $To nie je z toho..."$ $ 

"$ $Je, ale nechaj tak. Ja za chvíľu prídem."$ $ 

"$ $Maj sa."$ $  Mrana sa zahľadela na senzory. Nahodila do nich jej a pozrela si výsledky. Zatiaľ nič. Našťastie. Keď chcela ísť už za nimi, všimla si, že niečo jej umelé hlagenovo pole, rozšírené v sieťach zaznamenalo. Bolo to so Sylviou. Vedeli, že sú vo Felanzii.

Mrana zapla alarm a následne nechala sledovať zariadenie z ktorého vyšla správa. Zatiaľ to mali. Alarm na princípe hlagenovho poľa bol zapnutý. Nemohla strácať čas. Išla za nimi.

Do miestnosti 5G ich navigovala hlagenovka. Sylvia už bola triezva, ale hlava ju stále bolela.

"$ $Čo to bola za kniha?"$ $  Spýtala sa Loviisa Sylvie.

"$ $Zabudnutia od Gli'isi\v{}je Ch\v{}g¸hiefovej. Rok vydania – tisícdeväťstodeväťdesiatosem. Počet strán..."$ $ 

"$ $Ale čo to je?"$ $ 

"$ $Kniha, prirodzene."$ $ 

"$ $To viem."$ $ 

"$ $Kniha básní."$ $ 

"$ $Aj to mi došlo."$ $ 

"$ $Tak čo chceš vedieť?"$ $ 

"$ $Ako si sa k nej dostala, čo si o nej myslíš..."$ $ 

"$ $Aha.. tak prečo si sa tak nespýtala?"$ $ 

"$ $Lebo... lebo to bola obkľučná otázka."$ $ 

"$ $A prečo si ju použila?"$ $ 

"$ $Lebo... neviem... používa sa na začatie rozhovoru."$ $ 

"$ $Ale prečo?"$ $  Loviisa pokrčila plecami a Sylvia pokračovala. "$ $Nemôžeš to proste nepoužívať? Šetrí to čas."$ $ 

"$ $Ok.. nehádam sa... Ale odpovieš mi...?"$ $ 

"$ $Na.. aha. No. Fuj, čo, no.."$ $  Preglgla. "$ $To mám od Tarnyho... nepoznáš. On mi to pred tromi rokmi dal na čítanie. A vcelku v pohode. Osobne mám radšej Fentenzíjsku klasiku, ale toto je tak... pekné a iné. Je to písané za Cecílie, tak... predsa len, ale je to dobré. Obzvlášť milujem básne ako Krvavé poznanie, alebo Nevinnosť Hriechu, či Zapredaj Sa. Ale mala ostať len vo fentenzíjčine."$ $ 

"$ $To by som ničomu nerozumela."$ $ 

"$ $Nevieš po Fentenzísky?"$ $ 

"$ $Ani nie."$ $ 

"$ $Ani slovo?"$ $ 

"$ $Tak veci ako Ja'Sno\v{}vid..."$ $ 

"$ $To vie každý..."$ $ 

"$ $Asi tak... Inak, toto je už 5G, nie?"$ $ 

"$ $Hej... kde je tretia polička?"$ $ 

"$ $Si pred ňou."$ $ 

"$ $Aha... Ten môj výpadok si nevšímaj, je mi blbo..."$ $  Povedala a vtedy jej skočila do ruky, vďaka hlagenovmu poľu, rovno celá škatuľka s liekmi proti bolesti hlavy.

"$ $Už je lepšie?"$ $ 

"$ $Uhm... hej..."$ $ 

"$ $Potom mi preložíš niečo z tej knihy?"$ $ 

"$ $Skúsim... Ale nebude to moc poetické... na to je skôr Tulienka Deľa..."$ $ 

"$ $Hm... Ale preložíš?"$ $ 

"$ $Potom... Chcem ešte niečo s Mranou."$ $ 

"$ $Jasné..."$ $ 

"$ $Uvidím kedy sa..."$ $  Vtedy vošla Mrana.

"$ $Tu ste! Niekto zistil, že sme vo Felanzii."$ $ 

"$ $Sakra! Je to moja..."$ $ 

"$ $Nie je... Bola to otázka času. Ideme preč, alebo ideme bojovať?"$ $ 

"$ $Preč. Nechcem vás ohroziť."$ $ 

"$ $My sme do toho išli dobrovoľne, ale súhlasím. Tak... Ideme teraz mačičkoesom, alebo potom..."$ $  Pozrela sa na Sylviu, ktorá pochopila čo myslí.

"$ $Nikdy – viac – v živote."$ $  Dostala zo seba.

"$ $V poriadku. Toto prenechám môjmu hlagenovmu poľu, mohlo by ich to zadržať. Musíme sa zbaliť a vypadneme."$ $ 

"$ $Jasné. Ale kde...?"$ $ 

"$ $Fentenzia?"$ $ 

"$ $Príliš nebezpečné."$ $ 

"$ $A čo... Tramtária...?"$ $ 

"$ $Je to... nie..."$ $ 

"$ $A čo... Loriatar?"$ $ 

"$ $Ty tej povesti veríš, Sylvia?"$ $ 

"$ $Tulienka Deľa raz cez rozbor piktopísma zistila kde brána doňho je."$ $ 

"$ $Šibe ti?"$ $ 

"$ $Trochu."$ $ 

"$ $Tak Tramtária."$ $ 

"$ $Tam je Loriatar."$ $ 

"$ $Tak."$ $ 

"$ $Kedy sa kde sa stretneme?"$ $ 

"$ $Tam kde ste prišli na začiatku, o.. desať minúť najneskôr."$ $ 

"$ $Samozrejme."$ $ 

"$ $Rýchlo."$ $ 

\begin{center}

*

\end{center}

Fentenzia mala síce porovnateľnú rozlohu s ostatnými štátmi, ale celé jej obyvateľstvo bolo sústredné do pár miest a dedín okolo nich. Felanzíjčania boli veľmi dobrý kraj, starali sa o seba a neriešili Spoločenstvo. A tak to má byť. Pomyslel si Fequel. Program podľa útržkov hesiel o Mänchenovej došiel ku miestam, kde by mohla Mänchenová byť a nasmeroval tam mačičkoes. Malo to vyjsť.

\begin{center}

*

\end{center}

Sylvia bleskurýchle zmenšila všetky svoje knihy a hodila ich do kufra. Keď tam chcela Loviisa pridať svoj Lasermeč, Sylvia ju poklepala po pleci.

"$ $Nie! To si nechaj. Je to asi najlepšia zbraň čo máme a je prakticky nevyčerpateľná a do tvojej smrti s tým nedokáže robiť nik iný. Tak nech nepríde skoro. Nechaj si to. V ruke."$ $  Loviisa nenamietala, pretože Sylviine dôvody sa jej zdali rozumné.

"$ $Ideme?"$ $  Sylvia sa posledný raz pozrela na izbu a vyhlásila.

"$ $Áno."$ $ 

Mrana tam dorazilo o chvíľu.

"$ $Mačičkoes je kde?"$ $ 

"$ $Tu."$ $  Mala ho v ruke.

"$ $Odkiaľ ideme?"$ $ 

"$ $Z príletiska. Zamkla som a spustila senzory. Firmu riadim z domu, takže nikto si nevšimne, že tam nie som."$ $ 

"$ $A nemáš strach o pracovníkov? Wymyslensko je schopné všetkého."$ $ 

"$ $Viem,"$ $  priznala Mrana.

"$ $Majú všetci už asi týždeň platené voľno."$ $ 

"$ $Nájdu ich."$ $ 

"$ $Proti tomuto už nič nespravím."$ $  Trochu smutne povedala.

"$ $Ako s tým dokážeš žiť... ale... A ďalej?"$ $  Prvú vetu si povedala Sylvia len sama pre seba.

"$ $Cestou."$ $  Mrana zamierila k príletisku a rozprávala. "$ $Nič usvedčujúce tu nezostalo. Všetky dáta z počítača mám na disku. Počítač samotný som zmenšila a zobrala."$ $ 

"$ $Roboty?"$ $ 

"$ $Mód neviditeľný, plus zničená pamäť na vás a na všetko nelegálne. Tie by nemali byť nejako nápadné. Nahodila som na celý dom program, ktorý sa zapne naším odchodom, ktorý bude vytvárať v mysli postavu ktorá bude hovoriť veci podľa analýzy jeho mysle a to budú tie, ktoré program vyhodnotil ako bezpečné."$ $ 

"$ $Je to spoľahlivé?"$ $ 

"$ $Testované. A úspešne."$ $ 

"$ $Si si istá?"$ $ 

"$ $Program nevie o nás, preňho neexistujeme, takže nás nemôže prezradiť."$ $ 

"$ $Tak je to v pohode."$ $ 

"$ $Sme tu."$ $  Mrana zväčšila mačičkoes a nasadli. "$ $Zbohom."$ $  Povedala pre seba keď odlietali Mrana. Už len videla, ako sa zatvoril posledný vchod do jej bývalého domova. Nikto ich nemal vidieť. Smer Loriatar, alebo aspoň Tramtária.

\begin{center}

*

\end{center}

Fequel zastal mačičkoesom pri dome Kerie Oetovej. Tento objekt vyhodnotil program ako najpravdepodobnejší. Bol neviditeľný. Brána k domu odpovedala Fentenzíjkym. Vtedy jeho senzor zameral Hlagenovo pole.

Tá Oetová iste pomáhala Mänchenovej. Napadlo mu. Hlagenovo pole nebolo proti zákonu, ale kto ho používal, bol minimálne podozrivý. Nikto nemohol vyčítať wymyslenčanom, že sa chceli brániť. Veď nikto netušil, kedy Spoločenstvo bude chcieť zaútočiť. Ale používať Hlagenovo pole? Nebol veľký počet programátorov s Hlagenovým poľom, a to valná väčšina bola pod štátom. A kto by si chcel nechávať programy pre seba a nepomáhať štátu. Veď bezpečnejší štát, bezpečnejší jednotlivec. Nechal si poslať údaje o Oetovej. Programátorka. Pomerne neznáma. Firma s primárne telepatickým zameraním. To mohol čakať. Nesledovaná. Minulosť: neznáma. Ľudí s nezdokumentovanou minulosťou predsa sledovali nie? Pomyslel si. Inak by to bolo riziko. Ako teraz. Údaje pokračovali: Možné spojitosti: možná neidentifikovaná internetová nelegálna aktivita. Teraz sa Fequel zarazil. Pri Oetovej bola možná nepriateľka štátu, a nikto ju nesledoval, ani nekontroloval. Pracovala ako povolanie, ktoré je zväčša štátne a mohlo vyústiť do toho, aby štát mal menšie možnosti obrany ako je možné. Ale napriek tomu, nebola sledovaná. Kto to dopustil? Bol presvedčený, že akonáhle toto skončí podá hlásenie o nedisciplíne. Ale teraz čo s tou hlagenovkou? Hlagenovo pole sa dá zrušiť zničením zdroja alebo odrezaním od zdroja energie. Ani jedno nemal k dispozícii. Hlagenovo pole sa nedá oklamať. Alebo dá? Je možné sám od seba vytvoriť hlagenovo pole, ale musí to byť mocný mág, pretože na vytvorenie a udržanie poľa je potrebná obrovská mágia. Toto pole sa dá zlomiť. A oklamať. Ale počítačom udržované... veď preto ho používali oni. Pomyslel si. Teraz sa na nich pokúšala Oetová útočiť tým, čo používali oni. Ale ako tam vniknúť? Bol mimo hlagenovho poľa, ale čo keď sa už zalarmovali tí vnútri. Musel riskovať, aby bol v budove čo najrýchlejšie. Všetko, čo by mohlo jeho pokus ohroziť si vytesnil z hlavy. Keď prišiel k bráne, nejaký hlas ho oslovil.

"$ $Identifikujte sa."$ $  Bol tam skener DNA a otlačkov prstov. Na toto bol pripravený a systém zatiaľ oklamal.

"$ $Hlasová identifikácia."$ $  Diktoval systém. To mi mohlo napadnúť. Pomyslel si Fequel. Povedal meno, niekoho, kto chýbal, tak, že ešte si nechal skresliť hlas, aby to bolo uveriteľné.

"$ $Tuliena Deľa Mokrá – Plavčíková."$ $  Toto meno patrilo ďalšej, ktorá sa stýkala s Mänchenovou a momentálne bola asi na úteku, alebo snovala nejaké plány proti Wymyslensku... alebo to bola rovno agentka.

"$ $Nie ste ohlásený."$ $  Oznámil mu hlas.

"$ $Je to súrne."$ $ 

"$ $Identifikujte sa."$ $ 

"$ $Tuliena Deľa Mokrá – Plavčíková."$ $ 

"$ $Požívate lož."$ $  Oznámil mu hlas. Prekukli ho.

Dopekla! Pomyslel si Fequel. Oetová iste vie, že tu je. Toto mala byť rýchla akcia. Ale ony, Oetová a najmä Mänchenová, mu kazili plány. Polícia mala povolenie vykonať domovú prehliadku, ale oficiálne len s povolením súdu, alebo tkz. rýchlosúdu, ktorý vydával len priebežné opatrenia. A toto malo prebehnúť hladko. Mänchenová má zmiznúť zo sveta a bolo by vykonané. Ale ako vidno, jej sa to očividne nepozdávalo. Musel to skúsiť inak. Naozaj. Napokon, i tak o ňom vedia. Vedel, že za takýto riskantný pokus ho v prípade neúspechu isto čaká ešte väčšia degradácia ako v prípade "$ $len"$ $  neúspechu. Ale bolo to možné. Prišiel k bráne. Hlas neprehovoril. Zrejme keď viac razí povedal nepravdivú informáciu, hlas sa naňho zablokoval. V duchu zanadával. Teraz vedel, že sám ju nedostane. Musel si priznať, že Mänchenová prvé kolo vedie. Ale nevyhrá! Pomyslel si. To nemohol dopustiť, aby vyhralo Spoločenstvo a D. Poslal do veliteľstva správu o súčasnej situácii.

Odpoveď bola jasná. Nesplnil príkazy, ale našiel kde je. Alebo kde bola.

"$ $Čakajte na jednotku. Pozorujte sídlo. Pri akýchkoľvek pokusoch o únik – zatknite. Nesmú vám ujsť. Skontrolujte vzdušný priestor. Jednotka príde o chvíľu."$ $  Fequel oznámil akceptáciu a nechal hlagenovo pole skontrolovať vzdušný priestor. Mal prístup k informáciám leteckého ústavu. Rovnako si vyžiadal informácie o vozidlách patriacich Oetovej. Jeden mačičkoes. Dal si zistiť jeho polohu. Neznáma. Bolo viac možností, čo sa s tým mohlo spraviť. Buď už bol zničený, ale Oetová to nenahlásila, alebo... zničila GPS... to sa mu zdalo, už len kvôli skúsenostiam a zamestnaniu Oetovej vcelku pravdepodobné. A to teda znamenalo že mohli byť kdekoľvek. A rovnako sa mohli premiestniť okamžite, pre Mänchenovú. To by, ale znamenalo, že je nezatknuteľná... Vlastne tomu všetko nasvedčovalo. Ale stále bolo možné ju vydierať. Ale, ak sa jednalo o D, tak by toto bolo nanič platné. Ak sa jednalo o D, bolo možné ho len zabiť. Teda ju. Vtedy ho napadlo, že polohu D vlastne sledujú. Hodil požiadavku s jeho polohou do hlagenovho poľa okolo neho, ktoré pochádzalo z jeho telepatiónu. Démona sledovali agenti Wymyslenska už dlhšiu dobu, ale jeho polohu nebolo väčšinou možné poznať, hlavne vďaka jeho schopnostiam, ktoré mu dovoľovali premiestniť sa. Po chvíli hľadania v systéme prišla z poľa odpoveď.

Poloha Démona z 00:04 vášho času, dnes, je Zem, Austrália, Uluru.

To Fequela neuspokojilo. Kľudne to mohla byť stále Mänchenová. Z predvčera, dňa, kedy Mänchenová začala figurovať na zozname osôb poznámkou "$ $nebezpečný/á pre štát, okamžite zatknúť"$ $  si vyžiadal polohu D pre časy od jej "$ $prezradenia"$ $  a to dovolenia si spochybniť Cecíliu a pre čas, kedy prichádzal ju zatknúť a kedy ušla, teda premiestnila sa.

Poloha Démona z 11:54, 11.11. 2013 vášho času, je Zem, Anglicko, Stonehenge.

Poloha Démona z 13:12, 11.11. 2013 vášho času, je Zem, Anglicko, zóna A43B/23.

Zem! Nie Wymyslensko. Bol to určite D, pretože zmyslovú mágiu sa odhaliť dá, a na démona sa polodémon nemôže zmeniť. Jeho myšlienka že Mänchenová je D, bol zrejme nesprávna... rozmýšľal. Ale stále je určite s ním. Je predsa so spoločenstvom a snaž sa rozvrátiť Wymyslensko. Ale to neznamená, že ju nemožno vydierať. Najskôr ju, ale treba nájsť. Prešiel na mapu vzdušného priestoru a nechal hlagenovo pole prehľadávať dráhy neoznačených letov. Ak sa teda Mänchenová nepremiestnila. Zároveň s tým dal zas prehľadávať jej minulosť.

\chapter{Vypočutie}

Ostala v hoteli. Vedela, že do svojho starého bytu, je sa vrátiť až príliš nebezpečné a doma by ju nepriali. To jej bolo jasné. Rovnako jej bolo jasné, že si bude potrebovať nájsť zamestnanie, pretože zlatniaky a eurá sa jej pomaly, ale iste míňali. Do agentúry by ju nevzali. Mala na sebe až prílišnú stigmu. Otázne bolo, že či sa jej niekedy podarí zbaviť. Vo všetkých, alebo takmer všetkých zamestnaniach, na ktoré mala kvalifikácia bola potrebná previerka minulosti. A síce bolo prikázané nediskriminovať, otázne bolo, čo tým bolo myslené. Nediskriminovať bol až príliš všeobecný pojem. Asi preto tam bol zavedený. A bolo jej jasné, že s pozemskou minulosťou na seba doniesla stigmu minimálne takú veľkú ako ju už mala. Bolo jej jasné, že musí nájsť Nielu. Ale kde tá bola? Morja si uvedomovala, že D ju môže nájsť. A keď ich už raz takmer zajal, teraz to bolo pravdepodobné, že sa o to pokúsi zas. A okrem toho chcela nájsť svoju dcéru. Musela. Aspoň o tom bola presvedčená. Ale netušila kde je. Nebola si ani istá, či žije. Zrejme áno, ale aj ak, bolo až príliš veľa miest kde mohla byť. Bolo jej jasné, že na to, aby ju našla sama, musela by mať veľké šťastie. A to ju akosi obchádzalo. Morja neverila na osud, ale metaforu zo šťastím používala. Bolo jej prinajmenšom jasné, že sama, bez Niely, alebo agentúry, to čo hľadá, či kôr tú čo hľadá, nikdy nenájde. Musela nájsť Nielu. Povedala si zas.

\begin{center}

*

\end{center}

"$ $A posledný verš proroctva bol: dozvedia sa snáď. Potom už kniha odmietala čokoľvek povedať."$ $ 

"$ $To je pre ňu prirodzené."$ $  Odvetila Arabelin duch. "$ $Kniha je tak neskutočne zmagizovaná, že myslí ako živá osobnosť a vie, kedy čo prezradiť."$ $ 

"$ $Toho som si vedomá. Ale nie je to vlastne iba hlagenovo pole?"$ $ 

"$ $Nie, a to je na nej to fascinujúce. Nenamerali pri nej ani zvýšenú hladinu mágie, ani žiadne pole."$ $ 

"$ $Čítala som tú štúdiu. Bolo a stále je viac teórií. Podľa mňa najzvláštnejšia, ale zároveň asi tá, čo najviac odpovedá tomu ako sa kniha správa, je to, že je to živý organizmus. Ale zas..."$ $ 

"$ $Ktorý živý organizmus vydrží, možno okrem stromov, tisíce rokov, ako kniha. A vôbec... neprijíma živiny, ani nič takého charakteru."$ $ 

"$ $Viem. Ani energiu?"$ $ 

"$ $Čítala si, Niela, štúdiu švajčiarskeho ústavu mágofyziky, ktorý skúmal vlastnosti knihy?"$ $ 

"$ $Z ktorého roku?"$ $ 

"$ $Asi dvadsať rokov staré."$ $ 

"$ $Nie."$ $ 

"$ $Nuž, tam sa objavujú zvláštne javy. Za ten rok skúmania, kniha nikdy proroctvo neprezradila, ak tam nikto neprišiel z cieľom získať proroctvo, a ak tento cieľ bol len z čisto výskumného hľadiska, kniha proroctvo neprezradila. Kniha, akoby vedela o motivácii žiadajúceho, čo spĺňa definíciu telepatického poľa Hlagenovho inštitútu. Ale žiadne pole, ani náznak zmeny energie, či pohltenia energie knihou, vymykajúci sa bežnej odchýlke nebolo zaznamenané. Tá kniha... je akoby odniekiaľ... nie odtiaľto."$ $ 

"$ $Veríš v..."$ $ 

"$ $Tu nejde o vieru, ale o to odkiaľ tá kniha je. Je akoby nie z tejto paralely, ale to predsa nie je možné. Vysiela mágiu, ale nie energiu niečo ju pohlcuje, pridávajú sa do nej proroctvá, poukázateľne vyslovené, ale žiadna akoby komunikácia nebola zameraná. Tá kniha..."$ $ 

"$ $Som si vedomá toho, aká je..."$ $ 

"$ $Nehovor, to nie je nikto. A nepredstierajme poznanie, keď ho nemáme. O knihe máme menej informácií, ako o tom, kedy D zaútočí, takže..."$ $ 

"$ $A o tom nejaké informácie vôbec máme?"$ $ 

"$ $Práveže ani nie,"$ $  pokúsila sa o úsmev Arabela, ale toto bola až príliš vážna informácia. Bolo pravdou, že o plánoch D, nemali ani tušenia, nie to o tom, kedy chce prebrať moc. A to ich stavalo do roly obete. Mohli byť vyzbrojení viac ako on, ale moment prekvapenia, to bola asi najväčšia zbraň D. Chvíľu mlčali. Vtedy vlastne Niele napadlo, že Morja žije, a zrejme v tomto čase, a niečo vedela.

"$ $Čo vlastne hovorila Morja, že sa dozvedela?"$ $ 

"$ $Práve to neprezradila. Morna sa nezmenila."$ $ 

"$ $A čo by sa dalo čakať?"$ $ 

\begin{center}

*

\end{center}

"$ $Mám to."$ $ 

"$ $Už si...?"$ $ 

"$ $Hej."$ $  Prikývol Tarny. Tam, kde Pauline predtým prišla ku zraneniu, teraz ležala medená doska, zas plná mágie. Pauline neostávali už takmer či žiadne popáleniny a tak mohli ísť. Tarny ešte skorigoval mierne zničené okolie zmyslovou mágiou a mohli sa premiestniť.

\begin{center}

*

\end{center}

Vedela kde býva Niela, ale nečakala, že by bola doma. Zas prerátavala kde by ju mohla nájsť a kde by mohla vôbec ísť. Agentúra bola pre ňu nedostupná. Rovnako ako rodinné sídlo. Vtedy si vlastne uvedomila, že nemôže ísť všade. Preukaz síce mala, keď odišla, na celú Európu plus Ameriku, ale otázne bolo, či mal takú platnosť i dnes. Niečo jej napadlo. Čo ak sa celý čas pohybovala po svete s neplatným dokladom? Áno, mala kopu falošných, ale teraz, keď sa už "$ $oficiálne"$ $  vrátila...? Vytiahla ho a prezrela si ho. Jej myšlienka bola správna. Platnosť sa skončila asi pred tromi rokmi.

Bolo vôbec dobré, že ho mala pri sebe, pretože tým, že odišla zo Spoločenstva, ho mohla rovno zahodiť. Ale z nejakého neznámeho dôvodu ho mala pri sebe. Prečo, to nevedela. Akási predtucha, či iba podvedomé želanie návratu. Pretože ona sa vlastne aj vrátiť chcela, ale uvedomovala si, že by ju to zničilo, tak ako sa to už dialo. Odohnala mraky minulosti a musela premýšľať čo teraz. Keď preukaz vypršal, bola povinnosť si ho vymeniť ho, kým neprestane platiť. Ale čo s tými, čo nemohli? Napríklad agenti, nezvestní a obete únosov a prípadne nový občania, ktorí si spoločenstvo našli. Morja nespadala ani do jednej kategórie. Nikde síce nebolo napísané, že preukaz môžu dostať iba tieto kategórie, ale v Spoločenstve M platilo až príliš nepísaných zákonov. Napriek tomu ho potrebovala. A bolo to až príliš dôležité, aby ju zastavili tradície a byrokracia.

\begin{center}

*

\end{center}

"$ $Kde ideme?"$ $  Po chvíli mlčania prehovorila Niela, keď videla, že mačičkoes mení smer.

"$ $No, dostala som pred chvíľou správu, že Bella Lietavá sa vrátila."$ $ 

"$ $Bella zmizla spolu s D, pri..."$ $ 

"$ $Vraj sa premiestnila na nejakú odľahlú planétu a teraz sa dostala späť."$ $ 

"$ $Ale ako? A čo to má spoločné..."$ $ 

"$ $Hovorila o istej polodémonke. Dcére Jegrigsena Goona..."$ $ 

"$ $A Morje... Prečo..."$ $  Niela prerušila svoju neopýtanú otázku, pretože prišla na odpoveď. Jegrigsen Goon je predsa polodémonom...

"$ $Čo je?"$ $ 

"$ $Nič, už mi to napadlo, ale aj tak.. to bude..."$ $ 

"$ $Čo?"$ $ 

"$ $Ak je dcére Morje polodémonka,"$ $  ďalej premýšľala nahlas. "$ $Tak to je iba horšie... Vieš predsa, že..."$ $ 

"$ $Ja si dobre uvedomujem, že už len to že je dcéra Morje, bude problémom. Kým bude v rodine Mrana, tak ju do rodiny neprijmú, ale meno jej nezoberú. Ale to, že je polodémonkou, to všetko ešte viac komplikuje..."$ $ 

"$ $A čo som povedala?"$ $ 

"$ $Len rozvíjam."$ $ 

Niela poznala väčšinový vzťah spoločenstva M ku démonom a polodémonom všeobecne. Bol jednoznačne záporný. Z veľkej časti za to mohol D. Keďže väčšina ľudí žiadneho iného démona, ani polodémona, okrem D a niektorých jeho služobníkov nepoznala, boli presvedčený, že všetci Démoni a polodémoni sú s D. Alebo s Wymyslenskom, keďže Wymyslensko bolo verejným nepriateľom číslo jedna, spolu s D. Nemožno sa im ani čudovať, pretože polodémonské, a už vôbec, démonské gény sa v populácii vyskytovali veľmi zriedka, až tak, že nebolo vôbec známe, že polodémonstvo je dedičné, a tak za jediný spôsob, ako sa stať démonom, bol pokladaný služba D.

Tento prístup bol oddávna. Strach z Démona, ale nebol jediný dôvod nenávisti a strachu. Ľudia sa odjakživa zväčša báli nového a hlavne nepoznaného, a démoni a polodémoni tým boli. A ak k tomu započíta aj to, že démoni (a polodémoni samozrejme) majú väčšiu moc ako oni, a toto je nemenné... Povaha ľudí. Pomyslela si.

\begin{center}

*

\end{center}

Nepoznaj ma. Nikdy si ma nevidela, Ness, nikdy. To si pamätaj, nikdy si ma nevidela, ani nepočula, a už vôbec nikdy som s tebou nehovorila. Keď príde obchodník s dušami, tak odolaj. Si jedinou kto ostal. Budú ťa chcieť zničiť...

Loviisa, začítaná do Obchodníka s dušami, úplne zabudla na čas. Zdvihla zrak na hodiny. Osem hodín a päť minút večer. A Chen tam stále nebola. V liste ju upozornila, že je možné, že bude meškať. Ale tri hodiny... Rada by vedela, kde je. Chen jej nič nehovorila, a to sa jej zdalo podozrivé. Síce vedela, že na to sú aj rozumné vysvetlenia, no nezdalo sa jej to. Rozhodla sa, že zatiaľ jej bude dôverovať, ale keď sa toto bude opakovať... Ale Chen, keď sa vráti, žiadne pochyby nedá najavo. Rozhodla sa, že prezatiaľ si bude cvičiť mágiu.

\begin{center}

*

\end{center}

"$ $Tarny?"$ $  Spýtala sa Tulienka Deľa, práve vo chvíli, keď stáli pred ich domom, stále neviditeľní.

"$ $Hm?"$ $ 

"$ $Uvedomuješ si, že ja aj Pauline sme tu nelegálne. Vôbec, v celom spoločenstve M. Ja som formálne Wymyslenčanka a Pauline..."$ $ 

"$ $To mi ani nenapadlo... A nemôžeš si zmeniť občianstvo? A Pauline tak isto..."$ $ 

"$ $V druhom prípade máš pravdu, ale..."$ $ 

"$ $Hm?"$ $ 

"$ $Nie som plnoletá, a to by sa ešte dalo vyriešiť, ale... celú rodinu mám vo Wymyslensku a vieš čo..."$ $ 

"$ $Ja si uvedomujem, čo robí Wymyslensko..."$ $ 

"$ $Tarny! Ich môžu zatknúť i za to že som zmizla... a vôbec Sylviu a všetkých... a keby som sa vrátila, zase mňa!"$ $ 

"$ $A nemôžu prísť sem...?"$ $ 

"$ $Sylvia nikdy nepríde, a ich by si nepresvedčil... sú našou spojkou, potom čo vaši odišli..."$ $ 

"$ $To mi je..."$ $ 

"$ $A chápeš...?"$ $ 

"$ $A čo chceš teba robiť? Na čo narážaš?"$ $ 

"$ $Nemôžem sa pridať ku spoločenstvu, teda si zlegalizovať pobyt. Cecília to zistí... a..."$ $ 

"$ $Nemôžeš sa, ale ani vrátiť, pretože by ťa zatkli. Perkele."$ $  Ozvala sa Pauline.

"$ $A čo budeme teda..."$ $ 

"$ $Nad tým premýšľam."$ $ 

"$ $Môžeme ďalej robiť to, čo sme začali. Hľadať vír, a ďalšie..."$ $ 

"$ $Potrebujeme odpočinok, ale D môže kdekoľvek zaútočiť."$ $ 

"$ $Nemôžeme ostať na tom tak ako sme na tom."$ $  Dokončila Tulienka Deľa.

"$ $A čo...?"$ $ 

"$ $Ostaneme na dnes asi u mňa, hádam to matka pochopí..."$ $ 

"$ $Vystavujeme sa..."$ $  Chcela namietnuť Tulienka Deľa, ale Tarny ju prerušil.

"$ $Menej, ako keď sme vo svete."$ $  Prikývla. "$ $Zatelepatizujem matke."$ $ 

\begin{center}

*

\end{center}

Úrad pre dokumenty a doklady, ako znel oficiálny názov miesta, kde sa Morja nachádzala, bol preplnený. Po pár najnovších útokoch D a jeho služobníkov, sa snažili ľudia získať čo najviac miest, kde moli byť. A samozrejme, kvôli zvýšenej hrozbe D, úradníci každého preverovali viac, a tak sa vydávalo menej dokladov a dlhšie sa čakalo. Úradníkov bolo síce dosť, ale priestory malé, a tak ich počet nebol za moc platný. Pred ňou bolo ešte päť ľudí. Pri tempe vybavovania, Morja čakala, že sa dostane na rad tak zajtra.

\begin{center}

*

\end{center}

Bella Lietavá si už stihla obnoviť preukaz, za Izabetin príhovor, a zavolať do agentúry správu o návrate. Bolo jej jasné, že bude toho musieť ešte veľa vysvetľovať. Tarry sa ešte nevrátil z konferencie, kde podľa Tarnyho bol, a keďže to bolo až v Japonsku, teda až príliš ďaleko na telepatizovanie (telepatióny a emaily mohli byť odpočúvané, tak ich nepoužívala) a Arabela Tlogenová jej volala, že príde, neinformovala ho ešte o svojom návrate. Keď premýšľala, kde sú, zatelepatizoval jej Tarny.

"$ $Čo sa deje, Tarny?"$ $ 

"$ $Nuž, našli sme prezatiaľ, čo sme hľadali, a teraz..."$ $ 

"$ $Tak rýchlo?"$ $  Prerušila ho.

"$ $Bola to náhoda. Veľká náhoda. A čo som chcel, teraz som pred naším domom, len uvedomili sme si problém..."$ $ 

"$ $A o čo ide?"$ $ 

"$ $O to, že Tulienka Deľa, ani Pauline nie sú občianky Spoločenstva a teda nemajú preukaz. A tu ide o to, že Tulienka Deľa sa nemôže vrátiť do Wymyslenska, pretože by tu bola otázka kde bola a mohli by ju zatknúť a zas, keby sa pridala ku spoločenstvu, mohla by ohroziť svoju rodinu..."$ $ 

"$ $A načo mi to hovoríš, keď si povedal, že nie je možná ani jedna alternatíva?"$ $ 

"$ $My sme si povedali, že budeme pokračovať v tom, čo sme teraz začali, len na to momentálne nie sú sily... a či by sme tu nemohli ostať, aspoň na dnes, bez toho, aby niekto vedel, že... vieš čo."$ $ 

"$ $Rozumiem... len..."$ $ 

"$ $O čo ide?"$ $ 

"$ $Príde tu teraz Arabela Tlogenová, a..."$ $  Tarny sa nachvíľu odmlčal.

\begin{center}

*

\end{center}

"$ $Sakra!"$ $ 

"$ $Čo sa stalo?"$ $  Znepokojene sa spýtala Tulienka Deľa.

"$ $Príde tu o chvíľu samotná Arabela Tlogenová, a ty asi vieš, čo to znamená..."$ $ 

"$ $Nemusí vedieť, že sme tu... A Arabela nie je ako ostatný Tlogenovci... Ona..."$ $ 

"$ $Kto to je?"$ $  Spýtala sa Pauline, pretože nevedela o kom sa rozprávajú.

"$ $O istej tvojej príbuznej... myslím praprastarej matke."$ $ 

"$ $To..."$ $ 

"$ $Nikto by o tebe, možno okrem tvojej matky nemal vedieť. A Morja je už dlho preč..."$ $ 

"$ $A čo to znamená...?"$ $ 

"$ $Že radšej, aby o tebe nikto nevedel."$ $ 

"$ $A...?"$ $ 

"$ $Potom. Musím sa ohlásiť matke. Ale..?"$ $ 

"$ $Potrebujeme niekde ísť... Arabela nemusí vedieť, že tam sme."$ $ 

"$ $Nateraz súhlasím."$ $  Ozvala sa vtedy Bella.

"$ $Tak čo?"$ $ 

"$ $Ak nezistí, že sme tam..."$ $ 

"$ $Nemusí."$ $ 

"$ $Tak v poriadku. Stále si oficiálne preč. Aj keď nechcem klamať. Ja som..."$ $ 

"$ $Ja rozumiem matka. Ideme."$ $ 

Ich dom, bol tradičný, nenápadný dom, ako väčšina domov zo spoločenstva M. Keď vošli do chodby, ozvala sa Bella.

"$ $Arabela je tu o pár minút. Choďte niekde do domu, a to, či to niekto zistí, že ste tu, je iba na vás. I keď pochybujem, že..."$ $ 

"$ $Ale bezpečnosť je vždy na mieste."$ $  Povedal Tarny a vybrali sa do jeho časti domu.

"$ $Sme tu,"$ $  oznámila Arabela. Niela Bellu poznala, ale nikdy u nej nebola. Tá ich už čakala.

"$ $Tu ste."$ $ 

"$ $To by som skôr mohla povedať ja."$ $  Usmiala sa Arabela.

"$ $Teší ma Niela, vy ste sa..."$ $  Bella síce už dlhšie žiadne správy o spoločenstve nemala, ale Niela odišla z agentúry ešte predtým, ako zmizla ona.

"$ $Nie, len momentálne pracujem pre Izabetu..."$ $ 

"$ $Myslíte to tak, že..."$ $ 

"$ $Mám od Izabety prácu. Nič viac."$ $ 

"$ $Aha. Tak som si myslela."$ $  Bella sa vtedy zas pozrela na Arabelu.

"$ $Čo potrebuješ, keď si tu?"$ $ 

"$ $Prišla si, a to už nikto nečakal. Veď..."$ $ 

"$ $Mohlo to vyznieť zvláštne, predsa len, bol v tom D, ale v ten záver po tých rokoch už nikto neveril Skôr to vyznievalo na tvoju smrť... Tak prečo, a hlavne, ako si sa vrátila? A kde si vôbec bola?"$ $ 

"$ $Na Querte..."$ $  Arabela i Niela sa zamračili.

"$ $A to je...?"$ $ 

"$ $Planéta."$ $ 

"$ $Kde? Také mená..."$ $ 

"$ $Ešte ju, aspoň myslím, naši astronómovia nenašli. Je vzdialená tri miliardy svetelných rokov, ale neviem presne kde, a ak sme ju už objavili, tak má iné meno."$ $ 

"$ $Ale ak to je tak vzdialené a nie je tam červia diera..."$ $ 

"$ $To nie je."$ $  Prikývla.

"$ $Tak potom..."$ $ 

"$ $Potom, ako si sa dostala tam a ako späť?"$ $ 

"$ $Tam sa mi nejako podarilo dostať sa s D. Nejako, keď sa premiesťoval, tak... som sa premiestnila s ním... a on sa zjavil na Querte a vtedy, keď sa odmiestňoval, vtedy som tam už ostala..."$ $ 

"$ $Prečo si sa premiestnila s D?"$ $  Niela by nečakala, že sa Bella spojila s D, ale napriek tomu, radšej jej stále naplno nedôverovala.

"$ $Nedopatrenie. Bol to boj. A v tom..."$ $  Odmlčala sa a pokračovala. "$ $Podozrievaš ma, že, Niela?"$ $  Pozrela sa jej do očí a Niela po chvíli prikývla.

"$ $Ty dobre vieš, že situácia je, aká je. A stratiť ostražitosť je to posledné, čo človek urobí. Nejako zvlášť si nemyslím, že by si sa spojila s D, ale po Jegrigsenovi... ten bol ten posledný, na ktorého by som dala spojenie s D..."$ $ 

"$ $Morja niečo o ňom hovorila... že sa nestal..."$ $ 

"$ $Morja ho nenávidí. Keď sa pokúsil nás uniesť, v Anglicku včera, Morja mala čo robiť, aby sa držala plánu, ktorým sme ušli. Zabila by ho..."$ $ 

"$ $Morja niečo videla, a to bolo niečo, čo..."$ $ 

"$ $Nepovedala to, lebo Morna je aká je... Ale odišli sme od témy."$ $ 

"$ $Nie som s ním. Nikdy... Prisahala by som na svoj život..."$ $ 

"$ $To je zakázané..."$ $  Arabela sa mierne usmiala.

"$ $Ty mi niečo hovor... kto to kontroluje, a hlavne, na čo to je?"$ $ 

"$ $Nikto, už si nepamätám, za posledných dvesto rokov, že by bol niekto stíhaný za zakázanú prísahu,"$ $  pokračovala vážnejšie. "$ $Tento zákon, bol vytvorený, aby sa zamedzilo tomu, aby boli vytvárané prísahy na život zo žartu."$ $ 

"$ $Kedy sa naposledy niečo také stalo?"$ $ 

"$ $Presný údaj ti nepoviem, ale ešte asi sto rokov dozadu boli bežne traja až piati mŕtvi z toho, že sa zaviazali prísahou na život za nejakú úplnú somarinu, ako to, že sa niečo stane, napríklad nejaké mužstvo vyhrá zápas, ale nič, ako to o čom tu hovorila teraz Bella. Proste, vymklo sa to trochu spod kontroly a..."$ $ 

"$ $Rozumiem... Ale nechápem, ako to niekto mohol urobiť."$ $ 

"$ $Prísahy na život by mohli byť povolené, ale... a okrem toho, oficiálny dôvod pre ten zákon bola mienka, že Démonovi sa prisahá na život, a tak v parlamente vládlo presvedčenie, že zákaz prísah zabráni prechádzaniu k D, ale skôr sa to tak verejnosti iba predkladalo."$ $ 

"$ $To jasné. Aj keby tie fámy o prísahách boli pravdivé, tak tí ľudia sa rozhodli odísť zo Spoločenstva tak či tak. Veď už prejdením k D porušili zákony."$ $ 

"$ $Proste číry populizmus."$ $  Prikývla Bella a pokračovala. "$ $Ale skončili sme na tom, ako som ostala na Querte."$ $ 

"$ $Aký máš dôkaz?"$ $ 

"$ $Svedkov a pár vecí čo mám z Quertu."$ $ 

"$ $Hm?"$ $ 

"$ $Niekoľko vzoriek ich flóry a hornín. A môžu mi to dosvedčiť..."$ $ 

"$ $Kto? A ako si sa vôbec odtiaľ dostala? Keď hovoríš, ako je to ďaleko... či si stala polodémonom?"$ $ 

"$ $Nie. Nikdy. S Démonom som nikdy nespolupracovala, ani polodémon nie som."$ $ 

"$ $Ale ako si sa odtiaľ dostala? Ak tam nie je červia diera..."$ $ 

"$ $Nie aspoň o nej neviem."$ $  Pozrela sa Arabele do očí, kde sa ešte stále črtala jej podozrievavá otázka. "$ $Na Querte som sa zmierila s tým, že sa už nikdy nevrátim, ale potom sa tam nejako dostala, teda dostali oni..."$ $  Odmlčala sa a nadýchla sa.

"$ $Kto oni?"$ $  Bella vedela, že ich bude musieť prezradiť. Ale s tým predsa museli rátať. A Arabela je dostatočne inteligentná na to aby to pochopila.

"$ $Môj syn, istá Wymyslenčanka a dcéra Morje a Jegrigsena Goona..."$ $  Niela ju prerušila.

"$ $Morjina dcéra? Odkiaľ to o nej vieš? Tak predsa žije? Je polodémonkou, však?"$ $  Bella prikyvovala.

"$ $Povedala mi to. A niečo aj môj syn, ktorý sa tam premiestnil s ňou."$ $  Tulienku Deľu nechcela spomínať, pretože tá bola oficiálne vo Wymyslensku, a nechcela jej narobiť problémy, i keď o tie sa už úspešne starala sama s Tarnym.

"$ $Tvoj syn? Čo v tom má...?"$ $ 

"$ $On... skontaktoval sa s ňou. A potom sa dostali na Quert."$ $ 

Lietaví boli príbuznými Tlogenovcov, i keď ich meno nikdy nenosili, a ani ich príbuzenstvo nebolo oficiálne. Matka Tarryho bola Izabetina dcéra, i keď to Izabeta nikdy nepriznala. Ale stále platilo, že Lietaví mali požehnanie Izabety a teda i štátu samotného.

\begin{center}

*

\end{center}

Tarny počúval.

"$ $Prečo to robíš?"$ $  Spýtala sa Pauline z nudy, ale Tarny len sykol.

"$ $Pst!"$ $ 

"$ $Ale..."$ $  Tarny na ňu vrhol nahnevaný pohľad.

"$ $Pauline, toto je o bezpečnosti. Nikomu never. Nikomu sa nezveruj. Buď tiež počúvaj, alebo buď ticho, aby nás nenašli..."$ $ 

"$ $Ale..."$ $  Chcela niečo povedať, ale Tulienka Deľa ju prerušila.

"$ $ Potom."$ $  Pauline na ňu škaredo pozrela zas sa začítala do knihy, zatiaľ čo Tarny stále počúval rozhovor Arabely, Niely a jeho matky.

\begin{center}

*

\end{center}

"$ $Si si istá, že to bola naozaj ona? Neoklamal ho D, ale niekto od neho?"$ $ 

"$ $Nemyslím, že by ma D chcel späť na zemi."$ $ 

"$ $Je nevyspytateľný."$ $ 

"$ $Nesledovali ste ho?"$ $ 

"$ $Čas?"$ $ 

"$ $Včera okolo večera až obeda."$ $ 

"$ $Vtedy bol útok na Stonehenge."$ $  Zachmúrene povedala Niela.

"$ $Tam bol aj D? Nie len ten Nevermore, ktorého agentúra nevie nájsť?"$ $ 

"$ $Bola som tam osobne. Bol tam Jean, D aj ten, čo sa označuje ako Nevermore."$ $ 

"$ $O tom som počula, ale kto to vôbec je?"$ $ 

"$ $Nevermore?"$ $  Prikývla.

"$ $Pred pár dňami vybuchla klinika klonovania vo Francúzsku. Vtedy nikto nevedel, kto to spôsobil a..."$ $  Bella ju prerušila.

"$ $Myslíš súkromnú kliniku klonovania? Tú, ktorá bola v roku môjho zmiznutia obvinená zo spolupráce s D? Nezavreli ju?"$ $ 

"$ $Mali dobrého právnika. Presvedčil súd. A myslím, že v tom boli aj peniaze. Klinika ich mala až dosť. A tak fungovala až kým nebola doslova zrovnaná so zemou. Prežila presne jedna osoba. Bola to skaza, akú za sebou zanecháva Jean, ale keďže vtedy tam mal podľa našich zdrojov klon, bolo by to od D, čudné i naňho. A tak predpoklad je, že to vykonal ten jeho klon. Nie je isté, čo mu robili s mozgom, ale podľa Niely niečo, pretože jeho správanie neodpovedala žiadnej normálnej osobe, i keď tento opis na Jeana moc neplatí, ale pokračujme."$ $  Niela sa ujala slova.

"$ $Následne bolo včera zničené mesto pri Londýne, tak isto po ňom nič neostalo, len obrovská dávka mágie, maximálne neprirodzená. A následne bolo zničené Stonehenge. Vtedy tam bol aj Jean a D, ale tí následne sa premiestnili, a on ho zničil. Bol pri tom úplne, ako to povedať... nie chladný, ale... kričal a bol mimoriadne nahnevaný, ale nič viac, bol vcelku chladný, preto si myslím, že mu poškodili centrum emócií. Akoby D chcel mať neľútostného zabijaka, bez emócií. A potom odišiel, či skôr odletel a zatiaľ nič nezničil."$ $ 

"$ $Teda ďalší nepriateľ."$ $  Skonštatovala Bella zachmúrene.

"$ $A k tomu nepredvídateľný. Agentúra znásobila hliadky a pokúšame sa ho nájsť."$ $ 

"$ $Je nebezpečný... to je jasné... nespolupracuje s D?"$ $ 

"$ $Nie... on si podľa mňa ani len neuvedomuje, že D je. Jemu ide o Jeana..."$ $ 

"$ $Ako to myslíš?"$ $ 

"$ $Chce ho zničiť."$ $  Odvetila Niela. "$ $Keď prišiel na Stonehenge, niekoľkokrát sa pýtal kde je Jean. A neznelo to nejako priateľsky."$ $ 

"$ $Ty si tam bola?"$ $ 

"$ $Aj Morja, ktorá, ale zmizla. Mne sa podarilo zachrániť silným ochranným poľom."$ $ 

"$ $Kde zmizla Morja? Nehovorila Arabela, že s ňou rozprávala?"$ $ 

"$ $Podľa vlastných slov bola vo víre. Odtiaľ sa vrátila a volala mne. A Morna sa nezmenila, takže to čo tam Morja našla, nikto, asi okrem nej nevie."$ $ 

"$ $Prečo ste boli s Morjou v Stonehenge...?"$ $ 

"$ $Najskôr sa Morja vrátila, resp. presvedčila som ju. Následne som sa dozvedela, že má dcéru a následne sme došli na to, že je možné, že je to Goonová proroctva. A aj to, že stále žije, čo sa nám, koniec koncov, takmer isto potvrdilo."$ $  Pozrela sa na Bellu a pokračovala. "$ $Nikto z nás netušil, kde ju nájsť, resp. kde vôbec je, ani nič o nej. A keďže kniha je kniha, a mohla by to vedieť, bolo by dobré sa spýtať jej. Ale keďže je nepredvídateľné, bolo by dobré nájsť nejaký preklad jej piktopísma. Tak sme došli k nejakému staršiemu textu a došli sme k tomu, že preklad skrýva Stonehenge. A tam sa pri Nevermorovom uvoľnení mágie, takpovediac otvorila zem a vyšla odtiaľ zlatá doska. Či skôr tabuľa. A Morja sa jej dotkla... a zmizla. Asi tak."$ $ 

"$ $A Arabela, ty tvrdíš, že..."$ $ 

"$ $Áno, objavila sa, pred asi pol dňom. Potom odišla, nevedno kam. A agentúre už nechcem dávať príkazy súvisiace s Morjou, keď tu je na to Niela, a okrem toho, agentúra by to moc dobre nepriala. Ten príkaz myslím. Teraz sú už nahnevaní, už len pre tú dohodu s Cecíliou ohľadne knihy. Bellone, a rovnako aj pár ľudom z vedenia sa to nepozdáva dávno, ale je to predsa len, asi jediný diplomatický ústupok, ktorý dokázala Cecília, a rovnako i my vykonať. A obe vieme, že to nemá nič súvis s tým, že by sa vzťahy Spoločenstva a Wymyslenska nejako lepšili. Bol to čisto pragmatický ústupok, lebo D si na tú knihu brúsi zuby, a to by nedopadlo dobe..."$ $ 

"$ $Ani pre jednu, ani pre druhú stranu. D by vzájomné oslabenie využil na útok. Koniec koncov, očakávam, že to tak napokon urobí, a ten moment vzájomného oslabenia raz príde, ak Cecília a časť vedenia spoločenstva nezmení svoju rétoriku. To spájanie druhého s D je už smiešne a nezaručuje to bezpečnosť, ale skôr naopak..."$ $ 

"$ $Ale voličské hlasy áno, a o to, koniec koncov ide. I keď Cecília by si dokázala to víťazstvo získať i inak, ale..."$ $ 

"$ $Ľudia by neboli spokojní. Ono, Wymyslenčania sa aj vzbúria, a to viem, predsa len, žila som tam, i za porážky Solemy. Vtedy nespokojnosť vrcholila a bolo len otázkou času, kedy sa nájde niekto, ako Cecília. Wymyslenčania chceli mať hlavne pokoj. A v tom hádam mentalita väčšiny Wymyslenského národa nezmenila od časy Megy. Jej myšlienka zo spájaním kmeňov bola možno trochu dobrá, ale aj tak jej išlo najmä o osobný prospech..."$ $ 

"$ $A komu by nešlo. Wymyslenčan, človek či Fentenzíjčan, všetkým ide hlavne o osobné šťastie a úspech, nech proklamujú hocičo iné, akúkoľvek snahu o kolektívne blaho, ale sme tvory vypočítavé, a to je fakt."$ $  Bella prikývla.

"$ $Máš pravdu, i kolektívnosť funguje na princípe snahy o dosiahnutie osobného prospechu, alebo teda ilúzie osobného úspechu. Veď ktorý z vodcov režimov, ktoré hlásali rovnosť, žil život obyčajných ľudí? Ani u ľudí, ani u nás. To si netreba namýšľať..."$ $  Povedala Arabela.

"$ $A Wymyslensko je umelo vytvorený štát. Keby to bol len Wymyslenský kontinent, ale pripojením Fentenzie..."$ $ 

"$ $Fentenzia a jej vtedajšie Oko mali naivnú predstavu, že ich, Fentenziu nikto ku žiadnemu štátu pripojovať nebude. Ale Mega chcela vytvoriť čo najväčšie impérium. A síce Fentenzia je malá, jej kultúrny a hospodársky význam je nesporný. Oko s Megou len vyjednalo poloautonómiu, ale išlo len o zanedbateľný ústupok od Megy..."$ $ 

"$ $Mega nebola hlúpa... áno, urobila niektoré chyby, ale nebola hlúpa. Keď presviedčala Fentenziu, mohla zvoliť aj vojenský útok, ale neurobila to, aj keď mala dostatok armády na hladké víťazstvo. Ale vojna by spôsobila jej nepopularitu vo Fentenzii, rovnako ako štátu. Toto si na Mege cením, že nevolila vždy vojenské riešenie, a nie ako Cecília..."$ $ 

"$ $Tá by sa skôr vyhrážala, čo vlastne robí..."$ $ 

"$ $Mega bola veľmi ovplyvnená Mairou, i keď si to asi nikdy nepriznala. Podľa listov z ich lovskej korešpondencie, Maira sestru umierňovala v jej plánoch o vojenskom ovládnutí Wymyslenska. Ona sama presadzovala spojenie, s rovnakým, alebo aspoň približne rovnaký dôvodom ako Mega, ale presadzovala prirodzenú cestu spoločným jazykom. A jej plán je úspešný, aspoň z časti, keďže Wymyslensko stále je. To sama napísala v liste Mege z Verenova..."$ $ 

"$ $Maira mala pôvodne v pláne vytvoriť jednu celosvetovú, teda celofanasskú ríšu, ale Teresovo z pochopiteľných dôvodov nechala Mega po sestrinej smrti na pokoji a Tramtária, tá po pričlenení Fentenzie sa radšej formálne zmenila na republiku..."$ $ 

"$ $Tramtária je asi najspoľahlivejšia z celej Fanasy, ale nikoho nového z vonka do nej neprijímajú..."$ $ 

"$ $Asi práve preto..."$ $ 

\begin{center}

*

\end{center}

Sestra, tvoje predstavy o získaní nadvlády nad lovom sú vcelku rovné realite, aspoň v časoch keď som ešte nebola v Alcheme. V tvojom minulom liste si spomínala tvoje plány o získaní si dôvery awqerov a následnom získaní moci s ich čiastočnými kompetenciami. Chcela by som ku tvojmu plánu povedať, že podľa mňa, ak nechceš svoju moc upevňovať potlačenými povstaniami, ktoré by ti na popularite nepridali, či medzi tvojimi spojencami z radou awqerov pridávali, by si si mala získať dôveru medzi ľuďmi a presvedčiť ich o svojej nenahraditeľnosti, či o tom, že si ich novým spasiteľom, ktorí ich ochráni, dá im lepšiu budúcnosť a pomôže im. Ale, nesľubuj priveľmi nesplniteľné sľuby, pretože pamätaj, zatiaľ si len občianka lova s cieľom o vládu nad ním. A nepokúšaj sa byť presvedčená o tom akú máš moc, nemáš ju sestra, zatiaľ nie. K tým sľubom, všimla som si počas mojich ciest, že obyvatelia síce majú radi sľuby, ale sú k nim pomerne kritický, resp. veria im, ale dávajú si pozor na ich nereálnosť a to, či dotyčná vláda plní sľuby. Na to by si si i ty mala dať pozor, sestra. Pred pár dňami som bola svedkom toho, ako občania istej dediny v severnej časti Alchemskej ríše obesili ich Awqera za to, ako občanom tej dediny prisľúbil úrodu i na predaj, a sfalšoval čísla, ktoré udávali hospodárstvo dediny. Občania pracujúci na poliach si to všimli a tohto Awqera zvrhli s požehnaním samotného Alchemského sudcu. Ľudia sa vedia brániť a vzbúriť, a to nielen v malom. Celý tvoj plán môže byť úspešný len keď si získaš podporu, a predovšetkým dôveru občanov. A tú si získaš hlavne propagandou a (alebo) naplnením toho, čo od teba občania očakávajú (alebo ilúziou toho, ale to sme zas pri propagande), a (alebo) predstavou u istej skupiny, že si DOBROM a tu musí platiť to, že skupina je vplyvná na osoby vo svojom okolí a na spoločnosť, vtedy je dôležité označiť tvojich oponentov ako ZLO a tým vytvoriť pocit nepriateľa a TY budeš tá, za ktorú treba bojovať. Občania podporujúci "$ $ZLO"$ $  sú považovaní za čudných, a teda za niekoho, koho nehodno podporovať. Občania ti veria, lebo ty si vtedy DOBRO a oni ťa podporujú. Je síce pravda, že toto je tiež propaganda, ale davová. Ja ti osobne doporučujem kombináciu všetkých troch. Používanie len propagandy a davu je síce účinné, ale keď príde niekto, kto by bol lepší ako ty, a oslovil by občanov lepšie, rýchlo by si stratila všetko. Ak budeš, ale trestať akékoľvek snahy o zmenu, odpor bude rásť, len keby si to robila s takou efektívnosťou, že je nemožný akýkoľvek náznak odporu. Ale na toto potrebuješ mať už absolútnu vládu, alebo mocného spojenca, ktorý ti vojensky a hospodársky pomôže. Predsa len, občania, ktorí neveria tomu čo robia, nepodávajú taký výkon pre štát ako občania s vierou v to, že to pre čo robia má svoje opodstatnenie, význam i správnosť. So štátom spokojní i v štát veriaci občania podávajú pre štát väčšie výkony, ale samozrejme netreba zabudnúť, že každý myslí v prvom rade na svoju spokojnosť. Preto je dôležitá táto spokojnosť pre tvoju stabilnú vládu. Preto si myslím, že popularita a dôvera občanov, by mala mať základ aj nie v propagande, ale v reálnych výsledkoch. Ďalší možný zdroj popularity je aj sklamanie občanov u iných vládcoch a vládach, ale pozor, táto dôvera a popularita je značne nestála. Tiež si myslím že v občanoch by mala byť istá spolupatričnosť, niečo čo spôsobuje, že sa nechcú rozdeliť na viac štátov. Niečo, čo aj napriek historickej, a istej kultúrnej odlišnosti spôsobí, že občania sa budú cítiť v jednej ríši zajedno, nie len s tým o DOBRE a ZLE. Došla som už v love na to, že jeden zo spájacích nitiek je spoločný jazyk a písmo. Keďže jazyk a písmo je primárny komunikačný prostriedok, obyvatelia hovoriaci jedným jazykom dokážu bez bariér komunikovať a tým sa vytvára u kultúry hovoriacej jedným jazykom istá spolupatričnosť. Preto zastávam názor, že by v tvojom štáte, ktorý chceš založiť mal byť jeden štátny jazyk, ktorý by bol ľuďom materinský. Ale, keďže zatiaľ ním zrejme ešte nik nehovorí, bude potrebné, aby sa ho ľudia naučili, a aby nebol až príliš ťažký, aby si ho osvojil každý, aby mal základy z každého jazyka krajín, ktoré majú byť v štáte, a zároveň aby to nebola nezrozumiteľná hatlanina. Nad týmto konceptom práve pracujem. Preto sa sťahujem momentálne do severnej časti kontinentu, aby som zistila niečo o ich tradíciách a kultúre aj osobne, nie iba z listov a spisov odtiaľ. Buď trpezlivá a plán si poriadne premysli. Maira.

\begin{center}

*

\end{center}

Morja písala abecedu piktopísma. Ešte bolo pred ňou pár ľudí, a ona ju chcela maž napísanú. Mala ju síce vcelku pevne v pamäti, ale keby sa jej niečo stalo... Práve dopisovala do tabletu "$ $Jí"$ $  keď vtom niečo začula. Obzrela sa. Nič, asi sa jej to zdalo... Alebo? Prezrela si okolie pozornejšie. Nejaké vlnenie vzduchu. Zbystrila. Nebolo leto, ani nejaké horúčavy, a tak fatamorgánu vylučovala. Zmyslová mágia? Sústredila sa na miesto, kde videla vlnenie. Stále tam bolo a čoraz jasnejšie. Teraz si bola takmer istá, že ide o mágiu. Zamerala sa na to, aby ju zrušila. Matne, ale predsa vedela o koho ide. Jegrigsen Goon zas prišiel.

\begin{center}

*

\end{center}

"$ $Nič nové o knihe?"$ $ 

"$ $Kniha sa stále odovzdáva, ale to som už predsa..."$ $ 

"$ $Nie Knihe proroctiev... Myslím knihu osudu. Alebo už agentúra prehlásila, že tá neexistuje?"$ $ 

"$ $Nie, stále sa po nej pátra, ale agentúra ju mierne odsunula na druhý koľaj. Stále oficiálne existuje, teda nie pre verejnosť, ale mala by byť. Ale ľudia z vedenia o nej nechcú ani počuť. Vraj teraz to nie je hrozba, ale D o tej knihe určite vie a nedá mu to ju nehľadať. Ale Rolliusa sa akosi nedá presvedčiť pre jej hľadanie a bezpečnosť. Už som mu to pripomínala viacej krát. A stále to je tak, ako si zažila ty."$ $ 

"$ $A nejaký posun v hľadaní, Arabela?"$ $ 

"$ $Lokalita je zrejme na zemi, ale nič viac."$ $ 

"$ $Skúmali ste koho spisy? Niekoho iného ako Ló?"$ $ 

"$ $Zacaríasa a spisy Oka I. až po Ja'Sno\v{}vid srdca. Ale všetko to vedie maximálne na zem. Predpokladala som, že by Deoque mohla mať nejaké spisy, ale vedenie o tom moc počuť nechcelo. A tak sa postupne pokúšame odšifrovať nejaké správy a niečo nájsť. Ale veľa toho je napísané v starom piktopísme, a je málo ľudí čo ho vedia, a v agentúre taký, ktorý by dokázal preložiť tie spisy zatiaľ nie je..."$ $ 

\begin{center}

*

\end{center}

"$ $Čo je kniha osudu?"$ $  Spýtala sa Pauline, ktorá chvíľu po tej menšej hádke začala počúvať tiež.

"$ $Dačo charakteru veľká kniha proroctiev. Ale je oveľa menej známa a podľa viacerých odborníkov aj viac presná. Vraj funguje na princípe mimočasu."$ $ 

"$ $Čo to je?"$ $ 

"$ $Čas tam nie je. Všetko je v jednom momente."$ $  Odvetil Tarny. "$ $Dorozprávame sa potom, keď odídu. Tiež je niekoľko vecí čo mi napadlo."$ $  Odpovedal po náznaku telepatie z jej strany. Pauline tušila čo má na mysli.

\begin{center}

*

\end{center}

Osem hodín, dvadsať minút. Chen už poriadne meškala, a Rosa bola presvedčená, že toto je niečo, čo sa jej pokúšala zatajiť. Už sa jej nechcelo donekonečna vytvárať falošné obrazy, zvuky a vône, ktoré sa naučila robiť, ale ani čítať. Bola mierne nervózna. Netušila čo sa mohlo stať, že Chen tak dlho meškala. Napadlo ju, že je zrejme trochu aj domýšľavá, ale aj tak. Rozhodla sa, že musí zistiť, aká je bezpečnostná situácia v spoločenstve.

Okamžite zareagovala. Bolo možné, že tam nebol kvôli nej, a nevedel o nej, ale to sa jej zdalo nepravdepodobné. Zavrela počítač a zmenšila ho. Pozorovala ho. Ak sa nestala paranoidnou, približoval sa smerom k nej. Vytvorila okolo seba ochranné pole a čakala, pripravujúc sa naňho.

\begin{center}

*

\end{center}

"$ $A čo Goonová proroctva? Nemáš ju hľadať Niela?"$ $ 

"$ $A rovnako Morju, tá má ešte čo vysvetľovať."$ $  Pritakala.

"$ $A vieš niečo?"$ $ 

"$ $Nič. Takmer nič. A ty si tvrdila, že si ju stretla, a dokonca ťa dostala z Quertu. Nevieš niečo?"$ $ 

"$ $Potom odišli. Niečo hľadať. Niečo, o čom sa dočítali u Deoque, aspoň podľa nich... Niečo s D..."$ $ 

"$ $Bože, oni si myslia, že..."$ $ 

"$ $Nepoznáš môjho syna? Veď si ho párkrát stretla, Arabela. Oni... povedzme si to tak, sú o sebe vcelku presvedčení..."$ $ 

"$ $Nemohla si..."$ $  Bella pokrútila hlavou.

"$ $Som presvedčená o Fentenzíjskej tradícii. Nie."$ $ 

"$ $To rešpektujem. Ale nemáš ani predstavu kde šli...?"$ $ 

"$ $Odišli v noci. Neviem kde. Tak sme sa dohodli."$ $  Bella nemala v úmysle klamať Arabelu s Nielou, ale ani prezradiť, že tí, o ktorých polohe by Niela chcela vedieť, sú v dome, kde Niela práve bola, a tak nepovedala celú pravdu.

"$ $Nič ti o tom, čo našli nepovedali?"$ $ 

"$ $Hovoril o niečom s porazením D. To je všetko. To kde išli netuším..."$ $  Bella takmer povedala "$ $boli"$ $ , ale opravila sa. Neklamala.

"$ $Bolo to skutočne niečo s Deoque? Oni boli za Deoque?"$ $ 

"$ $Podľa môjho syna áno."$ $ 

"$ $Ale prečo, to čo chceli nájsť, nenechali na niekoho iného a viac kompetentnejšieho?"$ $ 

"$ $To sa musíš spýtať niekedy ich. Tvrdili, že agentúra nebude dôverovať zdroju od Deoque a nedali sa presvedčiť."$ $ 

"$ $Hm... nemáš ani potuchu o čo tam ide?"$ $ 

"$ $Nie."$ $  Pravdivo odvetila.

"$ $Takže, vieš ešte o nich niečo?"$ $ 

"$ $Čo konkrétne?"$ $ 

"$ $Spojili sa s tebou?"$ $  Bella premýšľala čo odvetiť. Nechcela Arabele klamať, ale ani prezradiť ich, pretože Tarnyho prianie rešpektovala.

"$ $Áno."$ $ 

"$ $A? O čo išlo?"$ $ 

"$ $Je to moja súkromná komunikácia."$ $  Upozornila ju Bella.

"$ $Bella, správaš sa, akoby si niečo tajila! Si s D, alebo čo? Alebo s Wymyslenskom, či čo?! Sakra, tu ide o budúcnosť a proroctvá! Nejako si sa zmenila!"$ $ 

"$ $Arabela... Ja som so spoločenstvom a kľudne ti to na život odprisahám. Arabela, to ty si sa zmenila... Uvedomuješ si to? Hovoríš už ako Marone, či Morna."$ $  Arabela sa mierne zamračila, ale Bellinu poslednú vetu ignorovala.

"$ $Tak prisahaj,"$ $  vyzvala ju a pokračovala. "$ $A vysvetli, prečo to nepovieš..."$ $  Bella nereagovala, a až po chvíli prehlásila kamenným hlasom.

"$ $Prisám, prisahám na svoj život, že nespolupracujem s Démonom, zvaným D. Prisahám, prisahám na svoj život, že nespolupracujem s Wymyslenskou vládou."$ $  Dopovedala prísahu, a mihol sa okolo nej prísažný plameň. Arabela mlčky prikývla.

"$ $A teraz ma zatkneš?"$ $  S miernym sarkazmom sa spýtala.

"$ $No... premýšľam nad tým."$ $  S rovnakým sarkastickým tónom odvetila. Bella sa usmiala.

"$ $A teraz vysvetľuj."$ $  Belle zmizol úsmev z tváre, ktorá zas bola kamenná.

"$ $Tu nejde len o Goonovú a môjho syna. Je s nimi ešte niekto, kto by mal problémy, alebo aspoň jeho, teda jej rodina, keby sa o ňom dozvedelo Wymyslensko."$ $ 

"$ $Vyzerám ako agent Wymyslenska?"$ $  Arabela bola stále pokojná. "$ $A vôbec, kto to je?"$ $ 

"$ $To práve. Čím menej ľudí vie o tom, že je tu, tým je to bezpečnejšie."$ $ 

"$ $O koho tu dočerta...?"$ $ 

"$ $Poznáš Tulienu Plavčíkovú – Mokrú."$ $  To nebola otázka, ale Arabela i Niela prikývli. "$ $Jej dcéra..."$ $  Arabela zdvihla obočie.

"$ $Ona? Čo s tým má...?"$ $ 

"$ $Je s nimi. Stačí. Wymyslensko sa nesmie dozvedieť o tom, že je tu, pretože potom... vieš čo to bude znamenať pre jej rodinu a aj pre nás..."$ $ 

"$ $To si dobre uvedomujem... Ale teraz, čo ti tvoj syn hovoril, či odkázal, keď..."$ $  Bella sa nadýchla.

"$ $Tu. Sú tu. V tomto dome..."$ $ 

\begin{center}

*

\end{center}

"$ $Matka nás prezradila."$ $  Zatelepatizoval im Tarny, práve vo chvíli, keď ich to všetkých napadlo.

"$ $Čo chceš robiť?"$ $  Zašepkala Pauline.

"$ $Čo myslíš? Máme vcelku dobrý dôvod im nedôverovať, a teraz vedia kde sme. Musíme zmiznúť..."$ $ 

"$ $Kde? Je to nebezpečné! A čo je ten dôvod...???"$ $ 

"$ $Potom, Pauline. Napadá ma niečo... a teba Tulienka Deľa?"$ $ 

"$ $Quert?"$ $  Ozvala sa Pauline.

"$ $Dobre Pauline, konečne rozumný nápad."$ $  Tá sa naňho zle pozrela.

"$ $Berte veci!"$ $  Rýchlo šepla Tulienka Deľa. Chytili sa Pauline a tá sa premiestnila na Quert.

\begin{center}

*

\end{center}

Nikto iný si ho zatiaľ nevšimol. Buď nepozorovali okolie, alebo bolo zmyslové kúzlo príliš silné. Bola pripravená na obranu a pozorovala ho. Pole okolo seba stále držala. Nechcela zaútočiť, hoci by to niekedy inokedy urobila. Približoval sa. Bol už na úrade, v jeho priestoroch. Agenti na ulici si ho očividne nevšimli. Alebo tam neboli. Prešiel cez miestnosť, že bol na pár metrov od nej. Jej predpoklad bol správny. Uvedomovala si, že nebude brať ohľad na ostatných, ktorí o ňom nevedeli. Nemala zbraň, ale mala mágiu. Rýchlo rozšírila ochranné pole tak, aby zastalo pred Jegrigsenom. Vedela, že pole určite zachytí, ak teda neodíde, ale to sa rovnalo pravdepodobnosti jeho premeny na pečený zemiak, alebo niečo takého charakteru a jeho reakciu nevedela odhadnúť. Hlavné bolo prečo prišiel. Predpokladala, že kvôli nej, ale mohla sa aj mýliť, ale s tým momentálne nerátala. Určite o nej vedel. Došla k tomu, že najlepšie odhadne jeho reakciu, ak ho bude sledovať. Síce tým mierne ohrozí aj seba, ale v jej situácii jej to pripadalo najlepšie. Pozrela sa naňho. Konečne sa jej podarilo, ale iba pre ňu, úplne zrušiť zmyslové kúzlo. Pozrela sa mu na chvíľu do očí, a ich pohľady sa stretli.

\begin{center}

*

\end{center}

"$ $Kde tu? Prečo..."$ $ 

"$ $Arabela, ty to chápeš, a rovnako aj ty, Niela. O tom, že je na zemi dcéra Tulieny sa nesmie dozvedieť Wymyslensko, lebo to ohrozuje nielen ju, ale aj naše záujmy."$ $ 

"$ $Ja si to dobre uvedomujem, ale prečo si to pred nami zatajovala?"$ $ 

"$ $Arabela, ty si v agentúre? To by si mohla chápať."$ $  Arabela sa zachmúrila. Odvetila Niela.

"$ $Čím menej ľudí to vie, tým lepšie. A hlavne ľudí, ktorých je v agentúre kopa. A tým som Arabela, nemyslela teba, ale plno iných byrokratov, ako tých, čo sa rozhodli konať za tradíciu proti Izabetinej vôli. Ako Bellona a krídlo v agentúre okolo Alvara Pysyvyysnena. A z nich by ju vyhostili a Cecília... tá by sa to dozvedela..."$ $ 

"$ $Keď sa budú len tak túlať po svete, to riziko nehrozí?"$ $ 

"$ $Máš pravdu, ale... je to ich voľba..."$ $ 

"$ $Ale tu ide o viac než o nich! O štát a..."$ $ 

"$ $Zas hovoríš ako Morna."$ $ 

"$ $Nie. Morna by hovorila o tradícii."$ $ 

"$ $A čo je štát? Tento štát. Tradícia štátu. Je to ilúzia, Arabela a my to vieme, ale napriek tomu sa o ilúziu staráme, napájame ju a udržujeme. A bránime."$ $ 

"$ $Štát..."$ $ 

"$ $Teraz nie je čas na slová o tomto."$ $  Prerušila ich Niela. "$ $Podľa mňa si Tuliena aj Delyan dobre uvedomujú, čo ich dcéra robí na zemi, a tak sú ostražití. Máme viac agentov vo Wymyslensku..."$ $ 

"$ $A týchto stratíme pre ňu?"$ $ 

"$ $Je neskoro o tom hovoriť. Ak sa tu zdrží dlhšie, Wymyslensko ich bude podozrievať. Ak sa prezradí, rovnako a rovnako i keď sa vráti. A ak sa im podarilo to, čo chcú..."$ $ 

"$ $To sa nestane. Nenamýšľaj si."$ $ 

"$ $Nie, nenamýšľam si. Poznáš ich. Je to polodémonka, dokážu sa dostať takmer všade, a ona... Tulienina dcéra pozná takmer všetky jazyky... A on... Už len to, že sa im podarilo cestovať bez toho aby..."$ $ 

"$ $Oni porušili zákony ohľadne cestovania bez preukazu? Neplnoletí. S Goonovou alebo...?"$ $ 

"$ $Bez."$ $ 

"$ $Ale porušili zákon..."$ $ 

"$ $Cestovaním porušili zákon, viem. Ale kto ho neporušil... Tým, že Morja bola v Anglicku a predpokladám nemala platný preukaz..."$ $ 

"$ $Ale sú to ešte deti! Bella... Nemajú..."$ $ 

"$ $Je to ich rozhodnutie..."$ $ 

"$ $A to, že sa niekto pridá k D nie je..."$ $ 

"$ $To, že sa niekto pridá k D, je podľa mňa hlavne chyba tohto systému a jeho sklamania v ňom... alebo proste len sebadeštruktívnych a deštruktívnych chúťok."$ $ 

"$ $Ty obhajuješ...?"$ $ 

"$ $Nie. Nemyslím, že je to správne. Ale je to na rozhodnutí človeka. Ak ho k tomu D, prirodzene nedonúti. Ale to sa už nerozprávame o slobodnom rozhodnutí."$ $ 

"$ $Tým hovoríš, že ak by ľudia nechceli spoločenstvo, ale D alebo Wymyslensko...?"$ $  Zachmúrená Arabela zdvihla obočie.

"$ $Samozrejme. Štát je pre ľudí, nie ľudia pre štát."$ $ 

"$ $Tým chceš povedať, že spoločenstvo..."$ $ 

"$ $Je tu preto, lebo ho chceli ľudia. Inak by boli inde."$ $ 

"$ $Alebo bolo spoločenstvo menšie zlo."$ $ 

"$ $Mohli byť mimo, ako Deoque."$ $ 

"$ $Nie každý je paranoidná veštica, ako ona. Ľudia hľadajú spoločnosť..."$ $ 

"$ $Myslela som ani na jednej strane. Deoque nie je s nikým."$ $ 

"$ $Dostali sme a od Goonovej zas ku politike,"$ $  upozornila Niela. "$ $Tým nechcem povedať, že toto nie je téma rovnako závažná, ak nie viac, ale sme tu kvôli niečomu inému. Na debaty o ideálnom zriadení štátu a politológii nie je vhodný čas."$ $  Na chvíľu zmĺkli. "$ $Doriešime to potom."$ $  Povedala Arabela a jej duch sa zahľadel na Bellu.

"$ $Môžem sa aspoň trochu pozhovárať s Goonovou? A Niela rovnako..."$ $ 

"$ $Je to jej rozhodnutie. Idem po ňu, alebo..."$ $ 

"$ $Poďme všetci..."$ $ 

"$ $V poriadku..."$ $ 

Izba bola prázdna. Arabela sa pozrela na Bellu, ktorá chvíľu mlčala a napokon vydýchla.

"$ $Museli to počuť. Aspoň časť z rozhovoru... A odišli..."$ $ 

"$ $Čo? Niektoré časti rozhovoru boli až príliš nebezpečné pre naše záujmy, i pre štát..."$ $  Bella mlčala a nastalo zas ticho. Belle prechádzali myšlienky hlavou. Vedela, čo mohli počuť a rovnako poznala Tarnyho.

"$ $Dúfam... že nie..."$ $ 

"$ $O čom...?"$ $  Nadvihla obočie Niela. Bella sa jej pozrela do očí.

"$ $Ak ide o to, čo si myslím..."$ $  Odmlčala sa a hľadala slová. "$ $Ak počuli o knihe..."$ $  Zas sa nadýchla a vydýchla. Tušili čo má na mysli, ale nechali ju dopovedať. "$ $Oni sú toho schopní... Oni... ju išli hľadať..."$ $  Pre Bellu to nebolo až také veľké prekvapenie, pretože ich, alebo aspoň Tarnyho, dobre poznala. Ona sama zastávala Fentenzíjsku filozofiu, ale toto... Knihu nenašli agenti, D... prečo si Tarny myslel, že to dokáže on?
\chapter{Návštevník}

Videl ju a hľadel jej do očí. Chvíľu na seba nenávistne hľadeli, akoby každý by toho druhého najradšej zabil – kedy mohol. Morja pomaly vstala. Jegrigsen sa jej stále díval do očí. Ostatní čakajúci si jej zvláštne správanie už všimli, predsa len tak stála a s prižmúrenými očami hľadela, ale len pre nich, do prázdna. Jegrigsen bol očividne presvedčený, že má prevahu. Ona ostatných ignorovala; sústredila sa iba naňho a na jeho reakciu. Jeho oči ostávali chladné, tak ako ho videla naposledy a rovnako výsmešné, smerom k nej. Dýchala, aby sa upokojila a snažila sa nájsť v jeho očiach náznak útoku, alebo naopak, ľútosti. Ani jedno ani druhé. Neprestávala sa naňho sústrediť a rovnako, na svoju možnú reakciu. Jegrigsen tam len stál, a hľadel na ňu, výsmešným a nenávistným pohľadom. A vtedy, znenazdajky niečo zašepkal. Morja dobre vedela čo. Jej reakcia bola okamžitá.

"$ $Laser!"$ $ Vykríkla. Prestala si uvedomovať to, že okolo nej sú ľudia, ale vnímala len Jegrigsena. On na ňu vyslala Solan, čo bolo mierne prekvapivé, ale predsa len, zrejme ju má len uniesť. Ale to sa mu nepodarí, to sa Morja zaviazala sama sebe. Všetci sa na ňu dívali, a viac aj na Laser, ktorý sa zrážal s odnikiaľ vyčareným Solanom. Morja stála, a upierali sa na ňu mierne vydesené pohľady, iní vytvárali ochranné polia a niektorí boli pripravení zaútočiť. Morja to vôbec nevnímala. Sústredila sa na Jegrigsena, dívala sa mu do očí a pokúšala sa prečítať jeho reakciu. Jegrigsen bol nevidený ostatnými pre silné zmyslové kúzlo. Solan nestihol začarovať, pretože Morja jeho kúzlo zničila už v zárodku. Pomaly jej dochádzali sily. Jegrigsenovi tiež, ale pomalšie, keďže zmyslové kúzlo potrebovalo sústredenie iba pri jeho vytváranie, inak bolo viac-menej automatické. Ona sama musela udržiavať vcelku silné ochranné pole a Laser. Vedela že ju to vyčerpáva. Čisto magický súboj by takmer iste prehrala z dôvodu vysilenia. Zmenšila ochranné pole len na niekoľko metrov od seba a vyhlásila tak, aby ju počuli ľudia, ale i Jegrigsen.

"$ $Kryte sa! Je tu Jegrigsen Goon."$ $ A vzápätí sa zas pozrela nenávistne Jegovi do očí a povedala.

"$ $Skap, ty hajzeľ!"$ $ Jegrigsen Goon ostával pokojný. Usmial sa, ale videla to len ona. Vtedy v sekunde Morja vytiahla zosilňovač Laseru a Solanu, natiahla si ho na druhú ruku a tou začala vysielať Laser a zároveň všetku zvyšnú energiu, ktorú predtým dávala do ochranného poľa, použila sa zničenie Jegrigsenovho zmyslového kúzla. Podarilo sa jej to. Mierne sa zapotácala, ale udržala Laser, ktorý presne mierila do Solanu. Jega už všetci, čo ho vtedy vidieť mohli, videli. Stál tam, pozeral sa na Morju, ktorá sa rýchlo spamätala a už zas pevne stála. Vyslala naňho Solan z druhej ruky. On bleskovo zareagoval. Na chvíľu len na seba nehybne hľadeli, a jediný dôkaz toho, že sa tam niečo deje, bol Laser pohlcovaný Solanom a naopak. Ľudia boli od nej v bezpečnej vzdialenosti, ale žiadny sa nezapojil do súboja Morje a Jegrigsena. Tí na seba hľadeli a mlčali. V diaľke odbíjala polnoc.

\begin{center}

*

\end{center}

Rosa sa prebudila a pozrela na hodinky. O desať minút polnoc. Zívla a vstala z kresla, kde zadriemala, pri čítaní momentálnych správ istého spravodajského portálu zo spoločenstva M. Zas zívla. Oboma nohami stála pevne na zemi, len občas chcela driemať. Naliala si pohár vody, a vtedy si uvedomila, že Chen tam stále nie je. Čosi sa muselo stať. Bezpečnostná situácia v spoločenstve nebola vôbec ideálna. Práve naopak. D, podľa všetkých médií bol najväčší nepriateľ spoločenstva M, spolu s Wymyslenskom, ktoré ale, aspoň podľa dojmu Rosy, nepredstavovalo takú bezpečnostnú hrozbu. Na www.knowthing.ym/en/politics našla základné informácie, ale zdalo sa jej, že sú viac propagandistické, ako pravdivé. Začínala sa obávať, že Chen sa niečo vážne, veľmi vážne stalo. Ale v správach nič nebolo. Vypila druhý pohár vody a zas si zívla. Stále ju trápilo veľa otázok. Kde je Chen? A kto je vôbec Chen?

\begin{center}

*

\end{center}

Sylvia odtrhla oči od knihy.

"$ $Preložíš mi nejaké tie básne, prosím?"$ $ Požiadala ju Loviisa.

"$ $É... Musí to byť teraz?“

"$ $Ak nechceš, tak... tak to pochopím..."$ $ Sylvia to akoby ignorovala, rovnako, ako svoje predchádzajúce slová.

"$ $Zničení, povstaní z popola, nič už nevytvoria odznovu. Prišli ťa zabiť, zničiť a roztrhať. Už nič nespravíš, roznosia ťa, zjedia... mocnosti tvojej viery. Zabudni na to, čo ‘s veril..."$ $ Odrecitovala s kamennou tvárou a zatvorenými očami. Napokon ešte dodala. "$ $Môj vlastný preklad polovice básne Len'i'nge\v{}rg, v preklade Ničiteľ, strana dvesto osemnásť, preklad bol do veršu deväť. Napísané v roku tisíc deväťstoosemdesiatpäť. Publikované v knihe Zabudnutia v roku...“

"$ $Nemusíš to všetko vymenovávať..."$ $ Prerušila ju Loviisa. Sylvia, mierne namrzene, prestala. Mrana ich nevnímala, ale niečo písala a sledovala obrazovku.

"$ $Koľko ti to trvalo?“

"$ $Čo, preložiť a prebásniť to?“

"$ $Hej,"$ $ prikývla. Sylvia pokrčila plecami.

"$ $Tak... rátaj odvtedy, čo si ma s tým oslovila. Toto je moja obľúbená, tak toto bolo... ľahšie..."$ $ Loviisa sa pokúsila o náznak úsmevu, ale Sylvia to akosi ignorovala, zas začala čítať, a miestami sa pozerala z okna, alebo sa prechádzala po mačičkoese.

"$ $Máš v sebe nepokoj..."$ $ Skonštatovala Loviisa pri pohľade na ňu. Toto akosi Sylvia nezvládla.

"$ $Tak čo, mám lietať po mačičkoese, smiať sa a rozdávať úsmevy?"$ $ Naštvane vybuchla. "$ $Sakra, prenasleduje nás Wymyslenská polícia, ja som prekliaty tvor, ktorý nosí skazu a nešťastie, a ty sa mi hovor niečo o vnútornej pohode! Mala som ísť za D, vytiahnuť Laser a preťať nás oboch!"$ $ Loviisa bola mierne zo Sylvie vystrašená.

"$ $To nehovor... si inteligentná, šikovná...“

"$ $Hovno mi z toho! Som démonka! Démonka, vieš čo to znamená! Som prekliata! Prekliata naveky, až kým nezabijem D. Ja ho zabijem, a s ním aj seba. Pretože už sa seba nezbavím! Nenávidím sa! Nenávidím sa, ako veľmi môžem. Všetko čo robím, je to ilúzia na zakrytie nenávisti a zabudnutie!, ale to sa aj tak nestane! Nechcem sa! Nenávidím sa! Som odporné stvorenie tohto univerza! Zabi ma, zabi ma, Loviisa a zachráň ma! Nemôžem žiť, všetci kvôli mne zomierajú a..."$ $ Vtedy sa otočila aj Mrana.

"$ $Nechcem ťa poučovať Sylvia, ale teraz preháňaš. Kto kvôli tebe zomrel? Alebo som niečo neprekukla?“

"$ $Nie... zatiaľ nikto... Ale všetci kvôli mne zomriete! Všetci! Ja som skaza..."$ $ Mrana sa tentoraz zamračila. "$ $Sylvia... toto nie je zdravé... myslím si...“

"$ $Sakra! Sakra! Mne je tak...“

"$ $Sylvia...“

"$ $Prestaň ma poučovať, dobre?! Nenávidíte ma, alebo ma aj ľutujete, ale nikto...“

"$ $Teraz si namýšľaš. To, že som ti pomohla, rozprávam sa s tebou a... vôbec. To nie je ľútosť. Toto nemá nič spoločné s tým, že si démonka. Ty si namýšľaš.“

"$ $Nenamýšľam!"$ $ Bránila sa. "$ $Moja matka ma nenávidí, môj otec ma nenávidí. Každý ma nenávidí... každý kto ma naozaj pozná...“

"$ $To nie je dôvod na nenávidenia seba a vôbec nie zovšeobecňovania. Nechcem ti nič prikazovať, poznáš ma. Ale prosím ťa, nenechaj sa zničiť sebou."$ $ Sylvia sa na ňu pozrela, mierne zamračená; chcela niečo povedať, ale prehltla to. Do očí sa jej mierne vhrnuli slzy, ale utrela si ich, aby ich nikto nevidel. Nechcela plakať. Nevedela čo chcela. Zdalo sa jej, že je za všetko vinná ona. Ale za čo všetko?

\begin{center}

*

\end{center}

Deoque sa prebudila po niekoľkých hodinách zas, ale teraz už prevládala jej racionálna, nie veštecká osobnosť. Otvorila oči, aby prekontrolovala to, čo urobila. Nemohla si dovoliť nijakú medzeru, či niečo, čo by ohrozilo jej bezpečnosť. Osoba, ktorú zajala tam stále bola. Bol v bezvedomí, po jej zásahu Solanom. Jej pohlcovač mágie fungoval, čo posudzovala podľa toho, že dom nebol poškodenejší, ako ho zničila ona, a hlavne – stále stál. Všetko bolo opravené, aspoň tak, ako to v svojom vtedajšom stave dokázala opraviť mágiou. Zakázala si na to myslieť. Bola predsa Deoque, Deoque, veštica, nezávislá, nebude počúvať svoju zmätenú myseľ! Skontrolovala prístroje. Zatiaľ žiadne nebezpečenstvo. Zajatý, ako sa ho rozhodla nazývať, nejavil známky vedomia. Dýchal. Na ruke, hrudi a časti čela mal popáleninu, od Solanu. Ležal spadnutý na časti prevráteného kresla a pár popadaných knihách. Kúzlami ho presunula do jednej z holých miestností v dome. Škody neboli také, že by sa nedali opraviť jednoduchým kúzlom. Pozrela sa na ulicu. Pás, vypálený laserom sa končil na konci celty, nezasiahol iný dom a už bol takmer opravený, ale menšia stopa na ceste ostávala. Tú okamžite opravila, takže okrem zvýšenej hladiny mágie nebolo nič zvláštne. Knihy zničené vôbec neboli. Jej kúzla boli dostatočne účinné. Popadané kreslá na zemi. Prevrátený stôl. Bolo okolo polnoci. Premýšľala, koľko bola v bezvedomí... deň? Nie, viac. Keď konečne opravila škody, ktoré si sama spôsobila, pozrela sa na seba. Na ľavej dlani mala spálenú kožu, a popáleniny mala rovnako i na lýtku a ramene. Takmer ich necítila, ale nie preto, že by boli malé, či zanedbateľné, to nie. Deoque vedela, že sa na nej odrazil vplyv jej obranných kúziel. Nude si musieť na to dávať pozor, napadlo ju, pretože bolesť, hoci je nepríjemná, pred niečím ešte horším varuje. Niečo privolala a v spojení s kúzlom jej popáleniny zmizli. Zas skontrolovala kamery, alarmy a hlásenia. Nič zaujímavé. Stav Zajatého bol stabilizovaný, ale bol v bezvedomí. Nemala poňatie, či sa za čas, čo ona bola v bezvedomí, neprebudil. Tep mal v poriadku aj všetky potrebné životné funkcie. Akurát na hrudi mal škaredú ranu. Od nej. Po druhom – kontrolnom použití jej medicínskych magických kúzel. Bezvedomie zrejme spôsobila popálenina. Alebo bola jeho hlavnou príčinou. Deoque nepotrebovala, aby bol Zajatý v kóme, alebo mŕtvy. Aspoň nateraz. Potrebovala od neho dôvod jeho prítomnosti v Londýne a rovnako i príkazy od D. Ak teda s ním bol. Deoque predpokladala, že áno. Pôvodne mu nechcela odstraňovať rany, ale napokon došla k tomu, že pre jeho príchod k vedomiu bude najlepšie, ak žiadne veľké rany mať nebude. Solan sa nijako zvlášť nerozširoval. Prepálil jedno miesto, kde zanechal ranu, ak bol príliš silný. Ale teraz bola mierne, a to si priznala i sama sebe Deoque, hoci priznanie si slabosti považovala pomoc pre súpera, pretože každé hlagenovo pole, ktoré bolo aspoň trochu dobre naprogramované, ich dokázalo zachytiť vo forme myšlienky. A to bolo to posledné, spolu so smrťou, čo Deoque chcela – aby jej nepriatelia vedeli o jej slabostiach.

\begin{center}

*

\end{center}

V polospánku sedela v kuchyni na stoličke, klipkala očami, zívala, pozerala sa na hodiny, ale zaspať nevedela. Netušila kde je Chen. A či ju vôbec uvidí. Zaspávala. Vždy ju zo spánku vyrušili kroky, alebo niečo čo za kroky považovala, ale Chen to nikdy nebola. I teraz. Polnoc a päť minút. Sledovala sekundovú ručičku, ktorá pomaly, ale iste ohlasovala Ďalšiu minútu a predlžovala Chenino meškanie. Po niekoľký krát zas zatvorila oči, keď ju prebudili kroky. V byte. Naozajstné. Nevedela, či je to Chen, predsa len, ten nepriateľ spoločenstva M mohol vniknúť i do ich bytu. Alebo to bol iba obyčajný zlodej. Rosa sa zneviditeľnila, ako sa to naučila v predchádzajúci deň a vstala zo stoličky. Zvuk, ktorý bol niečo medzi vŕzganím a buchotom pritom zakryla rovnako kúzlami. Vyplazila sa s kuchyne, pričom kroky "$ $zametala"$ $ kúzlami. Uvidela Chen, ktorá jej kúzla očividne prekukla a pritom si jasne vydýchla.

"$ $Rosa! Tu si! Prečo sa... nechaj tak! Rýchlo si vezmi veci, odchádzame...“

"$ $Čo sa...?“

"$ $Istá naliehavá udalosť."$ $ Odvetila rýchlo. Bez zaváhania. Napriek tomu Rosa cítila jej strach, ktorý Chen ovládala, ale jeho náznak v jej reči bol. Rose sa niečo nedalo, ale počúvla ju. Prisahala predsa. Do minúty boli pred ich dočasným obydlím, ktoré Chen zmenila na prepravovateľnú súčiastku a nastúpili do mačičkoesu. Všetko bolo tak rýchle, že to Rosa takmer nevnímala. Zaspávala, ale zvedavosť jej spať nedala.

"$ $Čo sa stalo?“

"$ $Istý obchod. Ale je až v Austrálii, takže... a má to byť čo najrýchlejšie... prepáč, že som ťa zobudila..."$ $ Rosa ju prerušila.

"$ $Kde si bola? Meškáš..."$ $ Pozrela sa na hodiny. "$ $Vyše päť hodín..."$ $ dokončila. Rosa okamžite odvetila.

"$ $Prepáč, mala som jedno obchodné rokovanie a mierne, teda poriadne sa to pretiahlo... a musím ísť okamžite do Austrálie, prepáč že...“

"$ $Bojíš sa niečoho Chen. Cítim to. Znepokojuješ sa, ale nedávaš to najavo, aspoň sa o to pokúšaš."$ $ Odvetila jej Rosa a vážne sa jej pozrela do očí a čakala na vysvetlenie. Chen tušila že Rosu dobre neodhadla. Bola viac citlivá a dokázala lepšie odhadnúť emócie ako to čakala. Na toto nepomýšľala. Mal pár sekúnd na dôvod prečo. Nadýchla sa odvetila. "$ $Peniaze. Všetko len peniaze.“

Rosa sa mierne mračila, pretože Chenina verzia udalostí sa jej nepáčila.

"$ $A čo na tom je tak desivé?“

"$ $Obchod. Korupcia. Vydieranie.“

"$ $A čo z toho sa stalo tebe?“

"$ $Dnes?“

"$ $Samozrejme.“

"$ $Asi všetko. Pozri, to s čím obchodujem je vcelku dôležité a z toho plynie aj nebezpečenstvo. Prepáč, že som do toho zatiahla aj teba.“

"$ $Mne to nejako zvlášť neprekáža.“

"$ $Skutočne?“

"$ $Vážne. Je to oveľa, oveľa lepšie, ako predtým... A je tu isté nebezpečenstvo, ktoré je... povedať to ako... no... mierne zaujímavé... vzrušujúce... neviem presne ako to povedať, ale hádam chápeš ako to myslím?"$ $ Chen prikývla. Dúfala, že sa Rosa nebude vypytovať, ale už mala vytvorený príbeh natoľko, aby ju nepodozrievala. Rosa tiež chvíľu mlčala a premýšľala nad tým, čo sa ešte môže spýtať, a čo je vôbec na tom, čo povedala Chen, pravdy. Sedadlo v mačičkoese zmenila na posteľ a o chvíľu, plná nezodpovedaných otázok, zaspala.

\begin{center}

*

\end{center}

Ticho, nepäté, nenávistné ticho medzi Morjou a Jegrigsenom nikto neprerušil. Ľudia mali strach, stáli tak ako sochy, maximálne z obranným poľom. Nikto sa nepokúšal odtiaľ dostať. Báli sa Jegrigsena Goona, ktorý si, ale v celej budove zdanlivo všímal len jednu osobu – Morju Tlogenovú. Tá predpokladala, že plánuje, respektíve D plánuje ju uniesť a urobiť z nej... to čo z Jegrigsena. Ak to, o polodémonoch D bola pravda a patrí tam aj Jegrigsen. V duchu sa striasla pri tom pomyslení. Jegrigsen sa jej stále díval do očí, prebodával ju pohľadom. Premýšľala nad svojimi šancami. Zásobu mágie získala z okolia, ale Laser ju vysiľoval, ale rovnako Jegrigsena Solan. Premýšľala čo je pre ňu najlepšie urobiť. Mohla čakať naňho, na to čo urobí on, ale to zrejme bude až po totálnom vyčerpaní energie. Nevedela koľko mágie má ešte v sebe on, či náhodou to, že bol polodémonom túto magickú kapacitu nezväčšilo. Nepredpokladala to, pretože všetko to bolo o genetike, ale predsa... Nemohla špekulovať, dochádzala jej mágia a Jegrigsenovi, aspoň na pohľad, ešte nie.

\begin{center}

*

\end{center}

"$ $Sme takmer tam.“

"$ $Neodhalili nás?“

"$ $Nie... Podľa senzorov nie. Zrušila som navigáciu, teda pilotujem cez hlagenovku. GPS senzory Wymyslenska nás nezachytia ako mačičkoes. Maximálne ako niečo lietajúce."$ $ Sylvia sa už upokojila. Navonok. A mierne aj vnútri. Podarilo sa jej všetci pocity viny nejako potlačiť, uzamknúť v sebe a zas bola takou, aká prišla do Mraninho domu včera... teda predvčerom... To už je tak dávno? Napadlo ju. Bolo okolo polnoci. Premýšľala, čo je teraz s Mraniným domom, jej pracovníkmi... Nie! Nemôžeš na to myslieť! Prikázala si. Teraz treba myslieť na to, čo bude, nie čo bolo. Napadlo jej. Knihu odložila a pokúšala sa normálne rozprávať s Nielou a Loviisou. Nemohla zniesť ticho. Vtedy sa jej uzamknuté myšlienky dokázali dostať na povrch.

\begin{center}

*

\end{center}

Nechal pole kontrolovať okolie okolo domu Oetovej a on sám prechádzal zoznam ľudí v zoznamoch, ktorí mali s Mänchenovou nejakú spojitosť. Oetová, Räkkänová, Mokrá-Plavčíková. To boli prvé položky zoznamu, ktorý bol, ale vcelku krátky. Prvé dve osoby boli s najväčšou pravdepodobnosťou pri Mänchenovej, ale ak nie, boli s ňou a určite sú hrozbou pre bezpečnosť. Tretia mala neznámu polohu. Už štyri dni. Aj ona bola na zozname osôb nebezpečných a teraz aj použiteľných proti Mänchenovej. Pri vyberaní týchto osôb bolo dôležité, aby na ňu dostatočne dobre vplývali, že vydieranie zaberie. Premýšľal, že na zoznamoch, ktoré má, sú len osoby z Wymyslenska. A keďže Mänchenová musela určite pomáhať D a spoločenstvu, oplatili by sa aj osoby odtiaľ. Požiadal si o ne.

Zo spoločenstva M nemali v databáze všetky osoby. Fequel zaklial. Potreboval vedieť spojitosti s Mänchenovou, ale tie sa nezobrazovali. Podal sťažnosť. Databáza musí byť doplnená. Akceptované. Bolo, ale otázne, kedy. Nechal vyhľadávať jej minulosť i pred príchodom do Wymyslenska. "$ $Posily dorazia o minútu, prehľadávame okolie."$ $ Ohlásilo sa veliteľstvo. Fequel si zas prezrel mapu vzdušného priestoru. Objekt mohol byť neviditeľný... Ale tie sa predsa vypátrať dajú.. napadlo mu. Nechal hľadať objekty, začínajúce dráhu letu v blízkosti domu Oetovej, keďže nepredpokladal, že ak by sa nepremiestňovali, tak by odlietali z iného miesta, zvlášť ak vedeli, čoho sa dopustili. A zas, iné vozidlo, ako od Mänchenovej či Oetovej, nemalo dôvody začínať svoju jazdy pri dome Oetovej (ak sa samozrejme nerátali náhodné zastavenia, ktoré, ale služba zarátavala do programu). Vyhľadávanie sa dokončilo. Nejaké teleso naozaj začínalo svoju dráhu nad Oetovej domom. Začal objekt sledovať.



\begin{center}

*

\end{center}

 "$ $O čo sa pokúšaš?"$ $ Z ničoho nič, aspoň pre ľudí okolo nich, sa ho naraz podráždene spýtala. Jegrigsen na mierne usmial, ale neprehovoril. Morja skrývala vyčerpanie. Vedela, že musí Jegrigsena vyprovokovať ku priamemu útoku, nie len čakať, kým jej dôjdu sily.

"$ $Tak čo chceš, Jegrigsen?! Zabiť ma? To nemôžeš. Zničiť ma? Alebo len narobiť čo najväčšiu spúšť, aby bol na teba D patrične hrdý?"$ $ Jegrigsen sa zas pousmial, mlčal a Morji sa zdalo, že tuší, o čo sa ona pokúša. Ostávalo jej už veľmi málo možností, kým jej dôjde mágia, hoci ju čerpala zo vzduchu, a všetky tieto možnosti boli riskantné. Ale musela to skúsiť.

Keď odišla zo spoločenstva, mala pri sebe ešte zbraň.. Má ju v Bratislave. Teraz je až príliš ďaleko. Pomyslela si a ponorila sa do energetického poľa a nasávala energiu. Jegovi sa dívala do očí. Už takmer samu seba presvedčila, že nezaútočí, ale to mohla byť osudová chyba. Nemohla ani na spustiť zrak, aby ju neprekukol. Moment prekvapenia. Zo svojej vonkajšej mysle odstránila myšlienky, ktoré uložila do tej najvnútornejšej, ktorú telepaticky bránila.

"$ $Tajíš niečo, Morja?"$ $ Až vtedy Jegrigsen konečne prehovoril. A ona to využila. I keď mohla čakať že niečo pritom rozbije.

\begin{center}

*

\end{center}

Nebol v kóme. Nechala mu liečiť popáleniny a sama ho sledovala. Bol stále uväznený, ale videla, čo dokáže. Niekoho jej pripomínal... pokúšala sa spomenúť. Niekto od D. Jeho tvár jej behala pred očami... kto? Premýšľala. Jasné! Zrazu ju osvietilo. Jean Parvasîe, nebezpečný, s D. Ten bol s D pred D prekliatím. Od nej ešte nič nežiadal. Deoque síce nemala nič proti obchodom so služobníkmi D, ale niečo proti D samotnému. A teraz chcel jeden z nich zrejme zničiť Londýn, kde bývala aj ona, teda ju... Deoque sa zamračila. Nesmie si do hlavy tlačiť paranoidné výplody, inak zas bude to s ňou zlé... Ale čo keď sú pravda...? Potom sa jej môže stať, že... Prestaň! Zahriakla sa v duchu. Venuj sa práci! Zapla v zariadení, ktoré ho väznilo, niekoľko meracích prístrojov a prezerala si výsledky. Odobrala mu niekoľko vzoriek krvi a nechala ju detekovať. Po chvíli sa dorazil výsledok, ktorý ju vcelku prekvapil. To nebol Jean Parvasîe. I keď mal jeho DNA. Jean nemal dvojča... a to by nevysvetľovalo ten jav, ktorý našla. Musel to byť jeho klon. A nie moc starý...

\begin{center}

*

\end{center}

"$ $Skap!"$ $ Odvetila a všetku energiu čo v sebe mala využila na prelomenie Jegrigsenovej ochrany a jeho odhodilo dozadu. Až vtedy sa ostatní prestali pozerať na ich, spočiatku tichý duel a mierne sa ich chytala panika. Niektorí tiež poslali do Jegrigsena svoje kúzla, ale zväčša len nanajvýš vyčarili obranné pole a zutekali s veľkou panikou. Niektorí volali agentúru, alebo len v chaose vrhali zaklínadlá. Morja, napriek tomu, že Jegrigsen tam bol kvôli nej, neodišla. Stála pár metrov od Jegrigsena, a držala sa steny. Útok jej vzal väčšinu energie, ale rýchlo ju čerpala. Stále to nebolo namáhavejšie než teleport. Zhlboka dýchala a neprestávala sledovať Jegrigsena. Nebol v bezvedomí, pozeral na ňu, mračil sa a naraz sa mu na tvári zjavil úsmev. Morja zneistela. Jegrigsenov úsmev zväčša znamenal, že si je istý svojím víťazstvom. Nevedela aké eso má v zálohe. Nemohla zaváhať. Hľadela naňho a pokúšala sa nájsť pre seba čo najlepšie ďalšie kroky. A vtedy sa Jegrigsen zdvihol a zaútočil zas on. Morja si ledva stihla vytvoriť obranné pole okolo seba a odhodila ju vlna mágie dozadu. Padla na popadané stoličky. Vyskočila, nevnímajúc bolesť, načerpala mágiu zo vzduchu a vyslala ju na Jegrigsena. Ten sa vyhol a odrazil útok. Na toto bola Morja pripravená. Guľu nechala preletieť do jej obranného poľa a vysala z nej mágiu, ktorú poslala na Jegrigsena. Zatiaľ sa jej mágia nemíňala, keďže v ovzduší jej bolo dosť. Len dúfala, že Jegrigsen nevytiahne ľudské zbrane, ako pištole.

Už dlho síce mágiou nebojovala, ale Morji sa po chvíli vrátila prax. Odrážala kopy koncentrovanej mágie, sama ich vytvárala a posielala na Jegrigsena, držala si obranné pole a pokúšala sa prelomiť jeho, pohlcovala mágiu, snažila sa vymyslieť to, čo by mala v nasledujúcej chvíli urobiť a odhadnúť jeho plány. Dúfala, že o chvíľu príde agentúra. A neodsúdi ju.

Odrazilo jeho útok a vrhla naňho Solan. Odpovedal Laserom. Zdalo sa, že sa všetko deje podľa jednej šablóny. Morji sa to nepozdávalo. Čo bol teda Jegov zámer? Sústredila sa na boj, ale mierne sa zamyslela. Vtedy, keď pocítila vlnu si spomenula na to, čo jej bolo dávno prízvukované v agentúre. Sústreď sa na boj. Na protivníka. Nenechaj sa zamyslieť. Cítila, že bola paralyzovaná. A sakra. Pomyslela si.

\begin{center}

*

\end{center}

"$ $Objekt mieri do Tramtárie. Jedná sa o teleso tvaru mačičkoesu. Zrejme zničili GPS a pripojenie na navigáciu. Teleso je monitorované. Zachytené živé objekty vnútri, veľkosti zovero a/alebo ľudí, počtu tri."$ $ Podal Fequel hlásenie.

"$ $Akceptované. Vyšlite sa za objektom. Máte povolenie. Prehliadku Oetovej domu vykoná druhá vyslaná skupina."$ $ Odpovedalo veliteľstvo. Fequel akceptoval. Vošiel do mačičkoesu a nechal ho navigovať na smer letu objektu.

Mám ťa Mänchenová... pomyslel si. I keby som mal porušiť suverenitu tej démonskej Tramtárie, ktorá si aj tak nič iné nezaslúži...

\begin{center}

*

\end{center}

Klon sa prebudil. Deoque odstránila všetky popáleniny, sebe aj jemu. Deoque si bola istá, že to nie je démon, keďže ostal po požití elixíru v rovnakej podobe. Pulz, dýchanie v poriadku. Rovnako skontrolovala, či dokáže rozprávať – potrebovala informácie. Pripojila ho na snímač mozgových vĺn a sama pomocou telepatie skenovala jeho mozog. Čakala kým bude úplne prebudený. Vtedy začne s výsluchom.

Otvoril oči. Nevedel kde je. Pamätal si záblesky svetla a silnú bolesť. A potom už nič.

"$ $Jean... kde je Jean...?"$ $ Pýtal sa na to, nenávidel toho, kto mu odpovedal. Nechcel mu povedať, kde je Jean... zabiť... zničiť... explodovať... Uvedomil si, že leží. Chcel vstať a zničiť to okolo neho. Chytili ho, vedia kde je Jean... musia to povedať... Nevermore! Chcel vykríknuť, ale niečo ho zadržalo. Nahnevalo ho to ešte viac. Mykol sa a chcel všetku mágiu v sebe nechať vybuchnúť, úplne inštinktívne. Nepodarilo sa mu to. Nedokázal to. Ani kvapku mágiu. Mykal sa, ale Deoqueine väznenie bolo účinné. Pokúsil sa zas použiť mágiu, ale zas nič. Len počul Deoquein hlas.

"$ $Ani sa nepohni! Prečo ťa D poslal?!"$ $ Nevermore jej nerozumel. Nikto ho nikdy jazyk neučil a žil len pár dní. Počul slovo D, ale bez kontextu. Netušil kto to je, a už vôbec nevedel Deoque odpovedať. Zavrčal. Videl ako sa osoba pred ním zamračila a povedala.

"$ $Tak čo? Nevieš rozprávať?!"$ $ Zreval svoje meno. "$ $Ticho! Odpovedaj alebo mlč!"$ $ Mračil sa, skrúcal prsty a niečo nezrozumiteľne bľabotal. Deoque si až vtedy všimla sken jeho mozgu. Nebol dostatočne vyvinutý. Na to sa dbalo pri klonovaní na dospelé jedince, aby boli schopní sa učenia. Lenže D mal akési špeciálne požiadavky, ako usúdila podľa skenu mozgu klonu. Mrzuto naň hľadela a povedala si sama pre seba.

"$ $Bože, D, ty si ale debil. Ak nerozumieš neurológii, nezahrávaj sa s ňou...“

\begin{center}

*

\end{center}

"$ $Kde sme?"$ $ Spýtala sa Sylvia, keď videla pred mačičkoesom svetlá. "$ $Hádam nás len..."$ $ Znepokojene sa pozrela na Mranu. Tá pokrútila hlavou.

"$ $Sme pred Tramtáriou a toto je pohraničná hliadka. "$ $Ako to je strážené?“

"$ $Dosť. Dosť na to, aby nás odhalili.“

"$ $Dokážeme sa tadiaľ dostať?“

"$ $Záleží na nás. Ešte nás nespozorovali, tak máme šancu. Okolo lode nechám hlagenovo pole, ktoré dokáže zastierať prítomnosť lode, podľa našej vôle, a potom nechám vyvolať vrstvu Trengoho poľa, na oklamanie strojov a senzorov. Ty musíš vytvárať mágiu. Ja riadim loď a ty máš dostatočnú zásobu mágie. Nemáme iný magický zdroj a Trengoho pole potrebuje stály magický zdroj.“

"$ $Len ja, alebo mám zobudiť...?"$ $ Pozrela sa na Loviisu, ktorá medzičasom zaspala.

"$ $Urob to... potrebujeme veľa mágie...“

"$ $Mám nápad..."$ $ Spomenula si Sylvia na svoju programovaciu mágiu.

"$ $Ja mám..."$ $ Vybrala kufor a zväčšila ho.

"$ $O čo ide? Nesmieme strácať...“

"$ $MPL."$ $ Odvetila a ukázala na guľôčky.

"$ $Nechytaj sa ich. Sú plné mágie, pri dotyku explodujú. Dokážem z nich vytiahnuť mágiu, je jej tam vcelku dosť. A tú použijem a dám do zdroja. Kde je zdroj?“

"$ $Si si istá? Ja viem, že mágiu budeš musieť načerpať, ale aj tak...“

"$ $Nemám nejakú zvlášť veľkú magickú kapacitu. A ak sa raz vyčerpá, stratili sme zdroj. V tých guľôčkach je v každej mnohonásobne väčšia magická kapacita, ako dokážem sama...“

"$ $Pri používaní MPL, ak odčerpáš priveľa mágie, explodujeme. A ty nedokážeš odobrať všetku...“

"$ $Využijem túto vlastnosť pri predávaní do zdroja...“

"$ $Ako? Vybuchne...“

"$ $Poznáš niečo ako vodič? Tak to použijem v mágii, a ak máme vodičov niekoľko, dokážeme preniesť mágiu do zdroja. Fungujú na princípoch rovnováhy mágie...“

"$ $To máš odkiaľ?“

"$ $Kniha, tridsiata siedma kapitola Princípy magickej techniky, strana tristo osemdesiata deviata odsek posledný, magické obvody.“

"$ $Dôverujem ti. Niečo som o tom už počula. Ty to máš tu?“

"$ $Prirodzene."$ $ Sylvia niečo zväčšila a vybrala odtiaľ drôty. "$ $Zapojím to do zdroja na každú guľôčku, podľa toho, koľko mágie potrebujeme.“

"$ $Nenechaj to vybuchnúť. Ale... aký to má výkon... nemôže do zdroja tiecť nejako rýchlo mágia...“

"$ $Použijeme magické odpory. Mám dostatok rezistorov. Aký to má mať výkon?“

Po chvíli počítania a zapájania Sylvia vytvorila magický obvod. Jej naprogramované guľôčky držala vo vzduchu.

"$ $Kde je zdroj?“

"$ $V paneli ovládania vidíš obrazovku ‚Trengoho pole‘ a pod tým je nápis ‚zdroj‘. Tam.“

\begin{center}

*

\end{center}

"$ $Prečo to robíš?"$ $ S námahou zo seba dostala. Pre Jegrigsenovo kúzlo sa nedokázala pohnúť a to platilo i na hlas. Slová sa jej z úst drali veľmi ťažko.

"$ $Nemôžeš..."$ $ Chrčala. Jegrigsen mal na tvári úsmev. Mlčal.

"$ $Nem..."$ $ Morja nedokázala hovoriť, bola pritlačená kúzlom ku stene a paralyzovaná. Nemohla nič robiť. A Jegrigsenova mágia sa tak skoro nevyčerpá...

"$ $Nemôžeš..."$ $ Nedokázala sa ponoriť do magického poľa a prerušiť kúzlo. Jediná možnosť bola využiť stratu jeho pozornosti... a to by sa nestalo... na to si dával pozor...

"$ $Mi..."$ $ Iba keby niekto prišiel... agentúra, hocikto teoreticky, ktorý by vyrušil jeho pozornosť.

"$ $Ublí..."$ $ Nikto. Nič. Cítila tlkot svojho srdca. Celý úrad bol vyprázdnený a niekto už určite zavolal agentúru... Tak prečo nikto nechodí?

"$ $...žiť..."$ $ Žeby spôsobil ilúziu? To by musela pocítiť, alebo...?

"$ $Pri..."$ $ Možno na to ani nepomyslel, a možno je v ilúzii. To by vysvetľovalo nepríchod agentúry.

"$ $...sa..."$ $ Premýšľala o tejto možnosti a naraz si uvedomila, že keby Jegrigsen prečítal jej myšlienky...

"$ $...hal..."$ $ Myseľ paralýza nezasahuje! Spomenula si. Vytvorila obranu pred zásahmi do mysle. Telepatiu paralýza nezasiahla. To jej poskytovalo šancu.

"$ $...si."$ $ Dopovedala konečne a čakala na Jegrigsenovu reakciu. Ten mlčal. Vedela, že ho musí vyviesť z miery, aby sa prestal sústrediť, aby sa zas mohla ponoriť do magického poľa. Ak nevedel, že jeho paralýza nezasiahla telepatiu, mala ešte väčšiu šancu, ale na to sa moc nespoliehala. Rozhodla sa radšej hneď neútočiť na jeho myseľ, pretože ju mal takmer isto chránenú. Nie iba pred ňou. Uvedomovala si, že musí konať, aby bola o krok pred ním. Aby ju nedokázal odhadnúť, ale ona jeho áno.

Spomínala si na to, čo vedela o telepatických útokoch. Všetky myšlienky si pritom chránila. Telepatické útoky fungovali na princípe dobíjania mysle toho druhého, a jeho oslabenia. Skutočný útok mal prísť až potom. Na telepatiu bolo potrebné minimum mágie, to minimum, ktoré fungovalo aj pri paralýze. Len sa nesmie unáhliť. Musí ho prekvapiť. Ukrývala si myšlienky a hľadala vhodnú chvíľu. Nastal relatívny pokoj. Nesmie zas urobiť tú istú chybu... pomyslela si. Sústredila sa. A vtedy pocítila jeho myšlienky v jej mozgu. Sugero? Nie, telepatia. Jej obrana zatiaľ fungovala.

"$ $Akosi si ticho Morja... už si to vzdala?"$ $ Nie, a niky nevzdám, ty debil. Pomyslela si, ale rozhodla sa neodpovedať. Tváriť sa, že prehrala. Že to vzdala... I keď Jeg ju mohol odhaliť...

"$ $Tak prečo mlčíš, Morja?"$ $ Náznak prekvapenia? Možno. Morja stále mlčala a pomedzi sústredenie sa analyzovala správu.

Telepatické správy neboli a nie sú dokonalé. Neobsahujú len správu, poslanú myšlienku, ale i celé rozpoloženie. Telepatiou sa neposiela správa, ale stav mysle posielajúceho a keď myšlienka je najsilnejšia, tak vtedy hlbšie neskúmajúci adresát rozozná len ju. Ale po pozornejšom skúmaní sa dajú analyzovať aj ostatné zbytky, či celé myšlienky, obavy a pocity...

Jegrigsen sa niečoho mierne bál. Jej? Ale to nie je... Never unáhleným záverom. Upozornila sa. Zas pevnejšie upevnila myšlienky v mysli za obranou a sústredila sa naňho. A zas sa pozrela na tú správu. Teda jej okolie.

Bál sa jej. Žeby niečo vedela... bolo to o telepatii. Jej schopnostiach? Čo už také... Premýšľala... A vtedy si spomenula. Keď dostala zásah.

\begin{center}

*

\end{center}

Svetlá hraničnej stráže zasvietili Sylvii do očí, ale ona to ignorovala. Ďalej sledovala magický obvod, aby sa nevymkol spod kontroly.

"$ $Sledujú pohyb objektov na počítači, však?"$ $ Mrana riadila a tak sa nesústredila na to, čo Sylvia hovorila.

"$ $...čo? Riadim, pekla!“

"$ $Či sa do nemôžeš nabúrať. Je to bezpečnejšie. A inak, prečo nemáš program na riadenie...?“

"$ $Lebo som nepredpokladala náš prechod cez Tramtárijskú hranicu! A pekla, nevyrušuj ma...“

"$ $Nemôžeš to napísať...?“

"$ $Teraz?! Šibe ti?“

"$ $Nie. Počkaj... Loviisa!“

Spala. Niečo ju volalo a triaslo ňou. Prudko otvorila oči. Volala na ňu Sylvia... Z okien svietili prudké svetlá. Nezdalo sa je, že sú bezpečne v... Ale prečo ju...?

"$ $Vstaň, dopekla! Poď sem a nehádaj sa!"$ $ Prudko vysypala Sylvia. Mierne rozospatá vstala a prišla hrbiaci sa k Sylvii.

"$ $Čo sa sakra...?!“

"$ $Nehádaj sa a udržuj tento prúd, vidíš? A hlavne ho neprerušuj kým neucítiš dôjdenie mágie, vtedy to nechaj na mňa."$ $ Loviisa a na ňu mierne nechápavo a rozospato pozrela a povedala.

"$ $Ešte raz prosím..."$ $ Sylvia sa zamračila.

"$ $Vidíš ten obvod? Z poslednej guľôčky tam prúdi mágia a tak ďalej. Udržuj ich prosím nad zemou, lebo vybuchneme, to sú tie naprogramované, a k tomu kontroluj či v niektorej nie je nízka hladina mágie. Ak áno, tak ma zavolaj, inak tu vybuchneme."$ $ Myslela to vážne. "$ $Nechceš to radšej..."$ $ Desila ju predstava tej zodpovednosti. "$ $Ja budem riadiť..."$ $ Mrana jej skočila do reči.

"$ $Hej! Teraz je potrebný dôsledný vodič...“

"$ $Viem riadiť dobre. Už som utekala pred D, pred spoločenstvom, cez podzemné cesty. Ver mi. Ty programuj. Nabúraj sa im do systému."$ $ Videla ich nedôverčivé pohľady. "$ $Keby sa niečo dialo, premiestnim sa, keďže to bol môj nápad."$ $ Uznala neochotne.

"$ $A prečo...?“

"$ $To nie! Nikdy!"$ $ Mrana sa zamračila.

"$ $Sylvia, bol to tvoj nápad, ale verím ti. Si dostatočne inteligentná, takže si uvedomuješ čo robíš. A aj to, že máš nás na zodpovednosť, ak sa niečo stane.“

"$ $Beriem to. Loviisa."$ $ Kývla na ňu. Tá si vzdychla.

"$ $Ale ja nemám takú kapacitu mágie ako ty. Dokážem to, ale nemám takú kapacitu.“

"$ $Keby sa čokoľvek dialo, vrátime to do pôvodnej podoby. Jasné? Nepreceňuj sa, lebo tu ide o skutočnosť."$ $ Prikývla. "$ $Tak poďme do toho.“

Neboli ešte nad hranicou, len tesne pred ňou. Sylvia si prezrela mapu svetiel a ich plôch, na ktoré svietili. Hranica bola rovnomerne rozsvietená.

"$ $Mrana..."$ $ Oslovila ju mierne zamračene.

"$ $Komplikácie?“

"$ $Hm... aký je toto mačičkoes?“

"$ $AA... prečo...?“

"$ $Nejdeme cez vesmír?“

\begin{center}

*

\end{center}

Odbila ho. Vrátila mu ho. Odrazila telepatický útok, ktorý na ňu Jegrigsen poslal. Ten ho odrazil späť, ale ona mu ho vrátila. Nevadí, že bola paralyzovaná, jej schopnosť telepatie to neohrozilo. Dala do údere väčšinu svojej sily a sústredila sa na to, aby to Jegrigsen nemohol odraziť. Prehovorila mu do mysle.

"$ $Počuješ ma? Počuješ ma Jegrigsen Goon?!“

Jegrigsena mierne striaslo. Pamätala si na svoj telepatický hlas. Teraz, ak by zaváhal, by mohla zrušiť paralýzu. To nemohol dopustiť, musel ju dopraviť na miesto, kde mal. Musel, to bol rozkaz.

Morjin hlas prerušoval jeho myšlienky. Nedokázal sa sústrediť. Musel... Musel... ale nedokázal. Morja pokračovala.

"$ $Tak už ma konečne počúvaš?! Skap! Zomri, dopekla! Nenávidím ťa, Jeg! Premiestni sa do nejakého močiaru a utop sa! Alebo skoč pod vlak! Zomri! Nenávidím ťa!“

Morja neprestávala telepatizovať. Ešte boli veci čo chcela vedieť. Jegrigsen na chvíľu prestal držať paralýzu a ona unikla. Teraz bola v prevahe ona.

"$ $Odpovedz mi Jegrigsen! Kde je?! Kde je moja dcéra?!“

Nemohol odpovedať. Nemohol. Nemohol zradiť. Mlčal. Musel splniť príkazy. Nemohol sa premiestniť. I keď... Ale to by bolo riskantné. Teraz sa musí sústrediť na to, aby zlomil zas Morju. A dostal ju do domova D. Ale na to musí odraziť jej útok...

Dôvod, prečo ešte neprišla agentúra bol prostý – Jegrigsen vytvoril ilúziu, niečo, čo Morja vylúčila. Ilúzia bola obyčajné zmyslovomagické pole, ktoré bolo zamerané na všetky zmysly, pričom nemenil obraz, len akoby izoloval od seba miesto v jednom čase. Iste, ilúzia sa dá zničiť, ale pri silnom zdroji je to pomerne obtiažne.

Hliadky prišli na úrad o chvíľu po zalarmovaní agentúry. V úrade bola vyššia miera mágie, ako by mala byť, aj u nich. Jegrigsena Goona, hlavný dôvod zalarmovania polície, nikde nevidel. Prirodzene, mohol použiť jednoduché zmyslové kúzlo, ale žiadne podozrivé vibrácie vzduchu nezachytil a to bol dobrý mág. Možno jednoducho zmizol a zobral zo sebou aj tú, s ktorou bojoval, vraj sa jednalo o Morju Tlogenovú. Premýšľal o ďalších možnostiach. Žeby... ilúzia?

Jegrigsen mlčal. Morja sa ovládala, neprepadala slepej nenávisti a hnevu, ale stále útočila. Dokázal ju mierne, teda jej myšlienky, nevnímať, ale späť privodiť paralýzu už nedokázal. Morja si držala hladinu mágie tak, aby nemala v boji jej veľkú stratu. Jegrigsena nedokázala zničiť, na to sa dokázala ubrániť, ona len bojovala o čas. Kým príde agentúra. Niečo jej na tom nesedelo. Agentúra mala zásahový čas pomerne malý, tak kde... je možné, že ich niekto zdržal, alebo prišiel od D niekto iný a... D nemá veľa ľudí, a väčšina z nich nepracuje preňho tak ako napríklad Jean alebo Jegrigsen... Na ňu vyslal Jega, mal si s ňou poradiť a doviesť ju späť, pretože je polodémon, tak sa mohol premiestniť... D má na práci zrejme niečo iné... Tak prečo... Odrazila magický výpad a vtedy jej prišlo na myseľ niečo, o čom nepredpokladala, že... Na zničenie ilúzie musí nájsť zdroj... A ten je väčšinou vysielajúci... Jeg...?

Prišla na to, pomyslel si. A do pekla... Ale zas, mal ju len doviesť do pevnosti. Na to sa jej stačí chytiť...

Skočil. Morja bleskovo zareagovala, uhla, ale Jeg dopadol bez zranenia.

"$ $Nemôžeš mi..."$ $ Začala, prestala telepatizovať a vytvorila si okolo seba obranné pole.

"$ $Ublížiť, ja viem. Ani to nechcem, Morja... O to sa pokúšaš ty..."$ $ Morja si zachovala kamennú tvár. Hľadala zdroj. Jegrigsen rušil jej pole, chcel aby sa jej mohol dotknúť. Ona miestami ustupovala, miestami pole zužovala a posilňovala.

"$ $Morja, prečo nespolupracuješ? Bude to dobre aj pre teba. Poď dobrovoľne.“

"$ $Nikdy, Jeg.“

"$ $Zas ma tak oslovuješ? Pokúšaš sa...“

"$ $Nie! Máš pridlhé meno!“

"$ $Nechajme mená menami, Morja, to nie je dôvod prečo som tu.“

"$ $Odíď Jegrigsen! Odíď!“

"$ $Ani ma nenapadne."$ $ Usmial sa.

Určite. Prečo na to došiel až teraz?! Zaklial, keď uvidel v magickom poli hranicu mágie, používanú na "$ $zmiznutie"$ $ magickej operácie. Nedala sa odhaliť prístrojmi, teda zatiaľ nie. Preto sa na jej odhalenie používalo jednoduché vzhliadnutia do magického poľa. A tak sa na maskovanie používala takzvaná Mágiozmyslová mágia. Prvá prišla s myšlienkou zmyslu mágie, ktorý mali mať len ľudia a zovero schopní postať sa do magického poľa. So schopnosťou cítiť mágiu. Ponoril sa do poľa pokúšal sa zrušiť hranicu, clonu. Bola silná. Ale čo iné sa dalo čakať od Jegrigsena Goona, ak tam vôbec je od D a Wymyslenska len on. Tušil, že s týmto si neporadí.

"$ $Veliteľstvo. Volá Gle'n O\v{}te\c{e}'n. Jednotka AG-alfa. Zameraná mimoriadne silná ilúzia s magickou hranicou. Pošlite posily – je potrebné zrušiť hranicu a pole.“

Uhla sa. Na poslednú chvíľu. Sklonila sa, uskočila. Nepoužívala mágiu, šetrila si ju, kým i Jeg s ňou nezačne. Ten to v najbližšej budúcnosti zjavne nemal v úmysle. Ale nedal jej ani chvíľu, kedy sa mohla ponoriť do mágie a ničiť zdroj. Mal prevahu. Zas ju chcel nechať paralyzovať, ale včas to ucítila, uhla sa kúzlu.

"$ $Nikto ťa nezachráni, Morja Tlogenová... Nikto... tak to vzdaj, Morja...“

"$ $Nikdy. Nikdy, Jegrigsen Goon..."$ $ Jegrigsen sa ani nepohol. Zbadal, že niekto mimo ilúzie poslal po posily. A zmizol.

Morja neodhalila nijaké kúzlo. Jegrigsen sa premiestnil, ani v magickom poli nijaký náraz mágie neodhalila... On sa premiestnil... V ilúzii ostala sama, nevidela v poli zdroj... Tak prečo tá ilúzia stále bola?! Premiestniť ilúziu... čo keby... o tejto možnosti ešte nepočula... Zdesila sa. Čo ak predsa... Vtedy by tu ostala uväznená... a bola by v rukách D... Tejto možnosti sa zrejme bála viac ako smrti... O to viac, keď vedela, čo má D v rukách...

Gle'n O\v{}te\c{e}n čakal na jednotku. Medzitým zbadal istú zmenu v magickom poli. Zdroj z ilúzie zmizol, ale tá stále držala. Nechápal to. Prečo...? Zamrazilo ho, cítil husiu kožu... A zmizol.

Na veliteľstve sa ozval hlas z telepatiónu.

"$ $Hlási sa Gle'n O\v{}te\c{e}n, Jegrigsen Goon zlikvidovaný, zastavte posily.“

Morja hľadala zdroj. Jegrigsena. V ilúzii nebol, to boli isté. Nevidela ho tam, rovnako ako žiadnu zvýšenú hladinu mágie. Rovnako ako žiadnu hranicu... Zdalo sa jej to zvláštne a mala podozrenie... Nesmie sa vzdávať... i keby bola pri D, musí si zachovať chladnú hlavu. Niečo za ňou zašušťalo, on sa vrátil. Prekvapil ju. Chcel sa jej dotknúť, ale ona včas vytvorila ochranné pole, držiac si ho sa ponorila do mágie a začala hľadať zdroj. Aby ho mohla zničiť.

\begin{center}

*

\end{center}

"$ $Sú vo vesmíre hliadky?“

"$ $Podľa doterajších informácií menšie ako tu.“

"$ $Výborne, že ma to hneď nenapadlo. Ako sa u teba mení typ na raketu?“

"$ $Vidíš tú páčku vpravo hore, medzi ukazovateľmi rýchlosti a tlaku? Prepni ju na automatika a zapni mód atmosféra a vzlietaj. Následne sa ti tam vyhodia možnosti a prepni to a daj potvrdiť."$ $ Sylvia nadstavila na "$ $raketa"$ $ a nechala mačičkoes stúpať nahor a obrátila sa ku Loviise.

"$ $Ja to za teba prevezmem... Potom to zas dám tebe..."$ $ Loviisa si vydýchla a sadla si do kresla. Sylvia nechala obvod prúdiť, kým sa nedostali z atmosféry. Potom ho prerušila a len udržovala guľôčky vo vzduchu. Nestratili ešte ani polovicu mágie. Len si všimla že dvom za chvíľu klesne hranica mágie na kritickú hodnotu a tie zas mágiou naplnila za pomoci MPL.

"$ $Kde má byť Loriatar?"$ $ Spýtala sa Mrana, keď na chvíľu upustila oči od hackovania fentenzíjskych pohraničných stráží.

"$ $V strede pralesov Tramtárie, tam kde ľudská noha ani noha zovero, či aj dragony tam nezablúdia. A ten Loriatar, to je časopriestorové pole, nie inopoľom volané, ale ďalšou dobou. I tak pravia povesti Tramtárie. Loriatar podľa nich na prameni Iasue leží, tam je brána doňho, tak vstup do siedmej doby. Legendy o inosvetoch, kapitola ôsma: Loriatar, strana stodeviata, odsek druhý."$ $ Odcitovala Sylvia.

"$ $To hovoria legendy...“

"$ $O tom sme už hovorili. Myslím, že je dosť veľa dôkazov.“

"$ $Ako myslíš. Teraz je pre nás skutočnosťou. Ak sa ukáže že je len legendou, alebo sa nachádza na inom mieste ostaneme len na Tramtárii, tam by malo byť bezpečne."$ $ Sylvia prikývla.

\begin{center}

*

\end{center}

Bol na niekoľko desiatok kilometrov od nich. Zdržalo ich, že znenazdajky zastali a niečo chvíľu riešili. Aspoň tak sa zdalo, keďže sa pár minút nepohli z miesta. Mierili rovno ku Tramtárijským hraniciam. Táto prekliata anarchia si chráni hranice dobre, pomyslel si.

"$ $Ak sa tam dostanú..."$ $ Vedel, že ale nič nesmie vylučovať. Žiadnu možnosť. Zrazu mu zmizli z radarov a rovno z celého monitoringu vzdušného priestoru. Neboli neviditeľní... zmizli... Žiadne teleso ani nespadlo, nedostali poruchu... Údaje o mačičkoese poukazovali na jeho zmiznutie. Áno, pravda, boli neviditeľní, ale... Zmyslové kúzlo sa dalo zrušiť, a ak bolo príliš silné, prístroje sa ním neoklamali... Aké kúzlo dokázalo bez reálnej zmeny objektu oklamať prístroj, bez zásahu do prístroja? Oetová je programátorka, a podľa Fequela činila určite i nejakú nelegálnu činnosť... Ktoré pole dokázalo oklamať prístroje? Spomenul si. Trengoho pole bolo možné zrušiť zvonku dvoma spôsobmi. Zničením zdroja alebo hacknutím systému, keďže išlo o program. Zavolal na veliteľstvo.

"$ $Hlásim sa. Okolnosti nasvedčujú tomu, že okolo mačičkoesu s Mänchenovou je Trengoho pole. Spojte ma s štábom počítačových technológií.“

"$ $Akceptované, Fequel. Spojenie aktivované."$ $ Vtedy sa ozval iný hlas. Spojili ich.

\begin{center}

*

\end{center}

"$ $Kto si? Ešte raz, odpovedz mi!"$ $ Nevermore zavrčal. Deoque sa zas zamračila. Uspala klon, i napriek jeho agresívnym protestom a zas si premietla jeho mozog. Bol na úrovni niekoľkoročného malého dieťaťa. D zrejme odmietol umelé spomienky spolu s vedomosťami, a k tomu to zničenie emočného centra! Zakliala. Klon bol naprogramovaný na zabíjanie, zdalo sa jej, že má v sebe spomienky na mágiu. Nebol mentálne retardovaný, bol schopný sa učiť, ale bude na to musieť byť čas. Rozmýšľala o možnosti umelého doplnenia vedomostí, ale nechcela vedieť, čo to urobí s jeho mozgom. Na začiatok by bolo dobré porozmýšľať o možnostiach aspoň dočasného nahradenia chýbajúcich emócií. Centrum nebolo poškodené mágiou, ale mechanicky zničené. Možnosť nápravy mágiou bola síce možná – cez MPL, ale to by existovala možnosť na výbuch jeho mozgu, a to zatiaľ riskovať nechcela – nemala záujem, aby sa jej niečo stalo s domom, alebo dokonca ňou. Veď, koniec koncov – mohla mu emócie upravovať i inak. Len mu ich bude nielen vyvolávať, ale i sprostredkovávať.

Nebolo to až také jednoduché. Zas ho prebudila, na hlave mu nechala elektródy. Zapla program, aby si nechala potvrdiť svoju domnienku.

\chapter{Quert}

Padala a vstávala. Jegrigsen útočil a jej sa vyčerpávala mágia. Držal ju v šachu. Nedokázala zničiť ilúziu, utiecť... Všetko nasvedčovalo jej blízkemu padnutiu do rúk D. Ak si úplne vyčerpá mágiu, príde o poslednú možnosť. Už raz D ušla, i keď vtedy ju celkom nezajal... Nevedela ako je v pevnosti D a po pravde, ani to nechcela vedieť. Mohli tam byť blokády kúziel, i keď by tým ubližovali hlavne sebe. Jegrigsen ju zajme, to bolo takmer isté, už dlho sa nič nemenilo a mal obrovskú prevahu... jediné, čo by to mohlo zmeniť je zázrak, a tie sú predsa veľmi vzácne. Zajme ju. Budem sa musieť odtiaľ dostať, nepotrebujem, aby mi D urobil to čo Jegovi, alebo niečo horšie... Vie D o tom teleporte? O tom, ako minule s Nielou ušli? D nie je hlúpy, mohol si to domyslieť. Ale na druhej strane, mohla sa tváriť, že má vyčerpanú mágiu... Ale čo keby to prekukol. Nesmie ostať v pazúroch D, musí sa odtiaľ dostať...

\begin{center}

*

\end{center}

Vo vesmíre! Všetko nasvedčovalo, že Mänchenová, Oetová a tá Räkkänová išli do vesmíru! Iste to je niečo s D, alebo rovno zamierili do spoločenstva. Letel, samozrejme, za nimi. Štáb počítačových technológií odhalil ich polohu – z nejakého dôvodu prerušili Trengoho pole, sami od seba. Pasca? Možno. Mohli ich vlákať na tajné rokovanie D, a nechali by ich pobiť... Načo by, ale bolo všetko okolo toho? Ten útek z Felanzie... Áno, ak ich chceli vlákať do pasce, potrebovali získať ich pozornosť... Najskôr sa javili, že sa ich chceli striasť, ale prečo potom...? Zachytili ich? Ich konanie mu nedávalo moc zmysel. Ak bola s D, čo zrejme bola – bola to polodémonka a bola proti Wymyslensku, tak sa ich buď chcela zbaviť, alebo, čo sa mu zdalo pravdepodobnejšie, vlákať ich do pasce. To by vysvetľovalo to vypnuté Trengoho pole, ale prečo ho nemali vypnuté hneď...? Chceli ich zmiasť, aby Wymyslensko neodhalilo, že ich chcú vlákať do pasce... Alebo, tu bola možnosť, ktorú od začiatku zatlačil do úzadia, pretože sa mu zdala až príliš nepravdepodobná... Najskôr sa len zdalo, ešte pred odhalením toho, že to je polodémonka, že je to obyčajná zmätená osoba, ovplyvnená propagandou spoločenstva, neuvedomujúc si, do čoho ide. Tento predpoklad mu ale podkopával fakt, že Mänchenová je polodémonka a mala prepracovaný plán úteku... A každý polodémon je s D, to je predsa fakt, každý polodémon pochádza od D, alebo sa pridal ku D a ku zločinnému, s D spolupracujúcemu, spoločenstvu M. Polodémoni... zberba, ktorú treba zlikvidovať, zničiť. A rovnako ako démona. Žiadny z nich si nezasluhuje žiť. A rovnako tých, čo ich podporujú... Videl ich už on, čo spoločenstvo oslavovali a démona rovnako. A hlavne na Fentenzii. A ako skončili... Bol osobne pri likvidácii niektorých z nich, videl ich programy, jasne potvrdzujúce snahu o likvidáciu Wymyslenska a ovládnutie ich krásnej krajiny D, napokon to, že ich zrušili, bolo dobré i pre nich, nezažijú vládu D, v ktorú tak naivne verili... Že mier a diplomatické vzťahy so spoločenstvom... A všetko na Fentenzii, len pár na pravom Wymyslensku... Ešte začnú snahy o pripojenie Teresova... Toho démonského štátu... Síce, s niektorými krokmi Solema Krutého, brata predchádzajúcej corlovny Wymyslenska, ktorá podľa Fequelovej mienky dokázala v štáte udržať aspoň poriadok, ale jej vláda bola zlá, veľmi zlá... súhlasil, napríklad s kompletným podrobením si Tretenov, utlačiteľského národu z čias pred Megy, ktorý chcel vyhladiť národ Wymyslenský, ale nepodarilo sa mu to. Srdcu, Wymyslenskej národobudiacej, aspoň podľa Fequelovej mienky, organizácii, ktorá proti Tretenskému útlaku bola, za to ďakovať treba... Veľká Jasnovidka Srdca dobre konala, určite i teraz by proti Démonom bola, ich česť národa podporovala... A nie tá Fentenzia sa búri... treba nad ňou prebrať kontrolu... A potom konečne nad spoločenstvom...

Miernou iróniou bolo to, že Jasnovidka Srdca sa celý čas svojho pôsobenia držala Fentenzíjskej filozofie, ktorú vyznávala a celkový koncept srdca – organizácie, brojacej proti tehdajšej Tretenskej nadvláde. Vychádzali z princípu Srd\v{}cie – teda Ja'Sno\v{}vid – oka, alebo Wymyslensky povedané, jasnovidky, hlavy organizácie, dvoch členov interesovaných hlavne vo výskume, jednu morálneho, či skôr filozofického člena organizácie, jedného, ktorý mal byť niečo, ako ochranca a napokon dvoch spájajúcich – mediátorov. Tradícia oka – najvyššieho duchovného vodca Fentenzie sa tak nevedno ako, dostala do krajiny, na kilometre vzdialené od Fentenzie, niekoľko tisíc rokov predtým, ako Fentenzia bola objavená. Wymyslenský patrioti, ktorí považovali Fentenzíjčanov za kolaborantov so spoločenstvom a zločinný druh, a napriek tomu považovali Jasnovidku za národného hrdinu, nevideli jasnú paralelu medzi Fentenziou a Srdcom. Jasnovidka bola predsa zovero – nie nejaká Fentenzíjčanka, či človek...

Ich mačičkoes sa dostal na obežnú dráhu, ale neodlietal preč. Bol stále na približne rovnakú vzdialenosť od nich. Stále si pole zas nezapli... prečo? A prečo si ho vôbec vypli? Nepredpokladali, že ich už dohonil, alebo to bola súčasť pasce? Podarilo sa mu ich oklamať?

\begin{center}

*

\end{center}

"$ $Pozri sa na pole, Mrana. Mám taký čudný pocit, že nás sledujú...“

"$ $To nás Wymyslensko nikdy neprestane...“

"$ $Ale teraz... Nezdá sa mi, že by sme ich úplne striasli... Predsa len, program je možné oklamať vždy a ty to vieš."$ $ Mrana mlčala.

"$ $Loviisa, prosím, skontroluješ tú obrazovku, úplne vzadu... Je tam kozmický priestor podľa údajov z Tramtárijského vesmírneho programu. Zisti, či niečo nie je za nami..."$ $ Loviisa mlčky prešla na druhú časť mačičkoesu a študovala mapu.

"$ $Inak, ak by nás niekto sledoval, tiež môže použiť to pole, Trengoho, či ako sa volá..."$ $ Po chvíli zamyslenia povedala Loviisa.

"$ $Nad tým som premýšľala. Preto vysielam okolo nás a pokúšam sa zachytiť stopy používania Trengoho poľa...“

"$ $A zachytila sa niečo?"$ $ S miernym nepokojom sa spýtala Sylvia.

"$ $Je to tu!"$ $ Povedala, či skôr skríkla Loviisa ukazujúc na obrazovku.

"$ $O čo ide? Nemôžem sa tam pozrieť, riadim!"$ $ Sylvia nemohla zapnúť autopilota, pretože ten bol napojený na globálnu sieť GPS, ktorú kontrolovalo Wymyslensko.

"$ $Niečo je za nami... je to síce v nejakom poli, názov neviem, ale..."$ $ Mrana odtrhla zrak od počítača a obzrela sa.

"$ $Ako to ukazuje tvoj indikátor polí?"$ $ Mrana sa naň zahľadela a zakliala.

"$ $Vyzerá to, že nás našli. Zapni Trengoho pole a potrebujeme sa čo najskôr dostať do Tramtárie. Ako Ďaleko sme od Loriataru?“

"$ $Prameň Iasue je niekoľko sto kilometrov pred nami. Ale odporúčala by som túto vzdialenosť prekonať na Fanase. V Tramtárii, keď už budeme na jej území, tak tam Wymyslensko polezie za možnosťou vojny s Tramtáriou. Síce, väčšina obyvateľstva..."$ $ Začala Sylvia, ale Mrana ju prerušila.

"$ $Teraz politiku nie. Namier mačičkoes do atmosféry Tramtárie. Loviisa... Alebo... no proste zapojte zas obvod na Trengoho pole."$ $ Sylvia prikývla. Otočila mačičkoes na smer Fanasa a mágiou zapojila zas celý obvod a nechala ho udržovať Loviise. Mrana kontrolovala hliadky a tých, ktorí ich prenasledovali. Podľa údajov to bol jeden mačičkoes, vybavený niekoľkými zlepšeniami a Mrana to typovala na Wymyslenskú tajnú políciu.

\begin{center}

*

\end{center}

Niekoľko stoviek kilometrov od nich Fequel zaklial. Nieže sa mu zas stratili z dohľadu prístrojov, ale posledný, preňho známy, smer trasy bola odbočka do Tramtárie... Žeby sa im vtedy pokazil mechanizmus Trengoho poľa, alebo im vypadol doňho magický zdroj a teraz ho obnovili... Alebo ho chceli tým manévrom do Tramtárie oklamať? Premýšľal. Zas zavolal na veliteľstvo. Potreboval štáb počítačových technológií.

\begin{center}

*

\end{center}

Nemala mágiu. Už takmer žiadnu. Nevládala. Jegrigsen ju porážal. Ponorila sa do magického poľa a uvidela veľký chuchvalec mágie. Keby ho načerpala, nestihla by sa brániť, keby nie, došla by jej mágia. Rozhodla sa pre prvú možnosť.

Neodrazila jeho útok, ten ju zasiahol, ale neublížil jej – na to si Jeg dával zvlášť pozor. Predsa prisahal. Chytil ju, ona sa už nepokúšala vytvoriť pole, len dočerpala naplno mágiu a premiestnil sa. Ilúzia prestala existovať. Bola z nej preč. V paláci D. Jeg ju pustil. To bola jej príležitosť. Skôr, než sa stihli D, Jeg a ešte nejaká čarodejnica spamätať, vytiahla Laser a teleportovala sa.

\begin{center}

*

\end{center}

"$ $Stratili sme sa im?"$ $ Mrana neurčito pokrčila plecami.

"$ $Tým si nemôžem byť istá. To sme si mysleli aj predtým a boli nám na stope... ako dlho potrvá kým sa dostaneme do zóny Tramtárie?“

"$ $Pár minút. Máš dobrý mačičkoes, Mrana. Upravený, mám pravdu?“

"$ $Samozrejme. Inak by sme nespravili Trengoho pole."$ $ Tramtária sa približovala. "$ $Sú tu hliadky?“

"$ $Očividne. Ale menšie ako na hraniciach vo Fanase. Predsa len, preprava cez vesmír je palivovo i nárokmi na techniku mačičkoesu náročnejšia..."$ $ Mrana sa zrazu zamračila.

"$ $Pekla... vnikajú nám do Trengoho poľa... a do počítača... ale zatiaľ neprenikli.“

"$ $Keby sme im poslali falošné údaje? Zmeniť údaje o polohe mačičkoesu, pričom tento faktor vypneme pri ovládaní a budem ovládať mačičkoes len ručne...“

"$ $Dokážeš to?“

"$ $Hej. Aspoň myslím."$ $ Fentenzíjska pôda bola na dosah. Na mieste, pod ktorým boli, nebola premávka – boli nad pralesom.

"$ $Mrana, riaď na minútu z mňa... Musím do nášho obvodu dočerpať mágiu...“

Tramtária – na to, že to bola anarchistická, či skôr minianarchistická samospráva, mala veľmi husté hraničné hliadky. To bol vlastne jeden z troch úloh, ktoré robila Tramtárijská vláda. Tá sa starala len o súdy (a vyhosťovanie kriminálnikov), hraničné hliadky (aby sa im do krajiny len tak nikto nedostal) a diplomaciu. Celá Tramtária bola ako štát len kvôli tomu, že videli ako Mega má zálusk na ich ostrov. Vtedy bola vytvorená "$ $vláda“, ktorej jediná úloha bola dosiahnuť aby sa ich územia nezmocnil žiadny iný štát. Zatiaľ sa im darilo. Dokonca bolo v Tramtárii najmenej útokov D v porovnaní so spoločenstvom a Wymyslenskom. Povesť Tramtárie ako anarchistického raja sa vo siedmom storočí po založení Wymyslenska (alebo po Kristovi) tak rozšírila, že tam húfne prichádzali nielen nadšenci anarchie, ale aj zločinci a kriminálnici, v snahe vyhnúť sa trestom. Vtedy "$ $vláda"$ $ Tramtárie zakročila a vytvorila hliadky. Okolo a nad celou krajinou bola hranica, cez ktorú sa dalo prejsť z Tramtárie, ale späť nie. Tramtária mala tak najväčšie a najlepšie hraničné hliadky spomedzi Spoločenstva, Wymyslenska a Tramtárie. Teresovo nikdy do týchto porovnávaní nevťahovali, nielenže z neho nemali informácie, ale bolo de facto len bábkovým štátom D.

\begin{center}

*

\end{center}

"$ $Zachytil si niečo?"$ $ Skúmavo sa pozrel na mapu, poslanú telepatickým dorozumievaním.

"$ $Nie... Nič. Dnes, ako vidno, nemáme veľa práce.“

\begin{center}

*

\end{center}

"$ $Ešte pár kilometrov od pristátia... darí sa nám...“

"$ $Netešme sa predčasne, stále neviem ako je na tom Wymyslensko. Síce má našu falošnú polohu, aspoň dúfam, že sa mi to podarilo, ale...“

"$ $Stratili sa nám?“

"$ $Tak to vyzerá. Ale nezabúdaj, že stále mohli použiť Trengoho pole, alebo niečo podobné."$ $ Sylvia mlčala.

Dostali sa do štandardnej letovej zóny, ktorá už bola pod hraničnými hliadkami. Sylvia sa po niekoľkých hodinách zas usmievala, i keď vedela, že stále im hrozí nebezpečenstvo. Skontrolovala, koľko mágie im ešte v obvode zostáva.

"$ $Dokedy je podľa teba dobré udržiavať pole?"$ $ Spýtala sa Mrany.

"$ $Ešte nejaký čas určite. Nevieme, ako sú na tom tí, čo nás prenasledujú."$ $ Odvetila. Loviisa už získala prax s udržovaním guľôčok o vzduchu a celkovom napájaní obvodu a tak popri tom študovala mapu Tramtárie.

"$ $Od prameňa Iasue sme asi stopäťdesiat kilometrov... Je to v pralese... Ste si istý, že tam ideme mačičkoesom?“

"$ $A čím teda myslíš...?"$ $ Loviise sa zrazu jej otázka zdala nepremyslená, ale aj tak odvetila.

"$ $Je to prales... prales... ničíme...“

"$ $To je možno pravda... bolo by možno lepšie keby sme sa im stratili najskôr niekde bližšie ku prameňu a potom už mačičkoes nepoužívali. Budeme síce zraniteľnejší, ale ak sa pripojíme, aspoň najskôr, ku nejakej skupinke, a predtým ešte sa im stratíme z dohľadu, stratíme sa im. Ľudia a zovero nevysielajú tak výrazné signály ako mačičkoesi.“

"$ $Neznie to ako veľmi zlý nápad... Ale budeme pomalší a unavenejší... A ak sme sa dostali do Tramtárie my, je dosť možné, že sa tu dostane i Wymyslensko.“

"$ $Wymyslensko riskuje vchodom do Tramtárie vojnu. Tramtárijčania sú schopní sa brániť, a to dosť účinne, myslím,"$ $ odvetila.

"$ $Dokedy pôjdeme mačičkoesom?“

"$ $Do poslednej cesty od prameňa. Bude lepšie, ak sa budeme tváriť ako Tramtárijčania, je to menej nápadné.“

Kým nedorazili ku prvej usadlosti mlčali. Nezdalo sa, že ich niekto teraz prenasledoval... Mohol ich oklamať. Alebo oni jeho.

\begin{center}

*

\end{center}

Mierili na Zem... Ako mohol tušiť... Ale načo bol teda ten manéver na Tramtáriu? Dostali si im do počítača, priamo na údaje polohy a oni im v tom ani nezabránili... Bol spokojný so svojou prácou, všetko išlo tak... Nebolo to podozrivé? Predsa len, Oetová bola programátorka a ak bola dobrá natoľko, že dokázala naprogramovať Trengoho pole... Prečo si to nevšimla...? Vždy tu, prirodzene, bola možnosť, že ich systémy boli natoľko dômyselné, že ich nemohli oklamať, ale... áno, Wymyslensko bolo v oblasti technológií silné, ale čo keď... Nikdy nesmie zanedbať žiadnu možnosť... Ak Spoločenstvo získalo náskok, treba ho zastaviť. Ale ak by Oetová vedela, že ich sledujú, prečo to nezastavila? Zrejme išlo o pascu... alebo... Fequel i niečo uvedomil... Vesmírny priestor pred ním bol prázdny od mačičkoesov...

\begin{center}

*

\end{center}

"$ $Tak čo je to tá kniha?"$ $ Spýtala sa ich Pauline po príchode na Quert. Všetci boli dosť unavení a nemali energiu odpovedať.

"$ $Ráno... Lebo pri tom vysvetľovaní zaspím..."$ $ Odvetil Tarny, zívol a o chvíľu zadriemal. Tulienka Deľa rovnako. Pauline sa zaspať nedalo. Bola síce poriadne unavená, ale v ten deň prežila výbuch mágie, popáleniny tretieho stupňa a zas bola miliardy svetelných rokov od zeme... Keď tak nad tým rozmýšľala, od čias, keď sa dozvedela o spoločenstve prežila toho viac ako za svoj celý život... Bola neskorá noc, i na Querte...

\begin{center}

*

\end{center}

"$ $Zachytila si nejaké signály, že by nás niekto sledoval?“

"$ $Ani nie... Vyzerá to tak, že nám naleteli. Ale nemôžeme vedieť, za akú dobu sa spamätajú...“

"$ $Kde si ich poslala?“

"$ $Na Zem. To vcelku odpovedá ich predstave, že chceme zvrhnúť Wymyslenskú vládu...“

"$ $To chceme,"$ $ prerušila ju Sylvia.

"$ $Počkaj, ešte som nedopovedala."$ $ A pokračovala. "$ $A nastoliť vládu Spoločenstva a D... Len dúfam, že to hneď neprekuknú. Mali by sme sa zapojiť do Tramtárijského bežného života skôr, ako na nás príde Wymyslensko...“

"$ $A dostať sa do Loriataru, tam chvíľu pobudnúť, kým nebezpečenstvo pominie..."$ $ "$ $Čo sa stane nikdy...“

"$ $Nemyslím si, že na nás bude Wymyslensko dávať veľké prostriedky, ak zmizneme a nebudeme provokovať. Vtedy budeme môcť bežne žiť v Tramtárii, alebo si zmeniť identitu a vrátiť sa...“

"$ $Ja chcem ísť domov...“

"$ $Neviem ako to bude s tebou... Ja... Ja... vážne sa ospravedlňujem, že som ťa do toho zatiahla, ale nemôžem...“

"$ $Sylvia... Teraz sa hlavne psychicky nezrúť. Nejaký spôsob bude, a teraz nemyslím veď vieš čo."$ $ Sylvia bola očividne zo sebou veľmi nespokojná.

"$ $Ale je to moja vina, je to...“

"$ $Nie je to tvoja vina, Sylvia. Prestaň. Raz revolúcia príde, a raz sa to vyrieši. To kedy, nezáleží len na nás. Ale aj na nás. A tým, že sa budeš obviňovať, nič nevyriešiš.“

"$ $Nateraz sa pokúsim... no, nezrútiť sa... Ale bojím sa o vás... Nechcem aby sa vám...“

"$ $To nechceme ani my, a budeme si dávať pozor...“

"$ $A čo Tulienka Deľa? Ja som...“

"$ $Tulienka Deľa vedela o tom, čo bude u vás v škole. A myslím si, že ťa dostatočne pozná na to, aby sa nevracala.“

"$ $Myslíš?“

"$ $Myslím."$ $ Sylvia sa už upokojila. Začala sa zas zaoberať nie sebou ale plánom.

"$ $Vystupujeme? Teraz sme pred dedinou, prišli sme z nejakého mesta.“

"$ $Prejsť sa po pralese?“

"$ $Hej.“

"$ $Tramtárijské pralesy...“

"$ $Časť z nich je nesprístupnená... A to aj tá, kde chceme ísť. Stratíme sa turistickej skupine...“

"$ $Máš peniaze?“

"$ $Mačičky...“

"$ $Dopekla... z toho budú otázky... nevieš ako...?“

"$ $Nezmeníme ich, nemáme ich mať. Musíme nejako získať Fentenzíjske peniaze...“

"$ $Ty a Loviisa nie ste plnoleté... Je zvláštne ísť takto do pralesa... Väčšinou...“

"$ $Zabúdaš na zmyslovku. Alebo hlagenovku.“

"$ $A čo kamery?“

"$ $Odkedy sú v pralese...?“

"$ $Nemyslím v pralese! Myslela som okolo neho... myslím, že ho chránia...“

"$ $Dobrá pripomienka. Musíme sa pripojiť k turistom. A stratiť sa.“

"$ $Nebude to nebezpečné?“

"$ $To bude. A nielen od tunajších obyvateľov, ale aj Wymyslenska, D a tiež aj tunajších dragonov, ktorí obývajú tento prales.“

"$ $Čo je tu podľa teba najnebezpečnejšie?“

"$ $Hmyz. A hadragony. Na tie si treba dávať veľký pozor. Našťastie, máme možnosť vyčarenia ochrany, takže, niekto ju bude musieť udržovať...“

"$ $Alebo ju naprogramujeme.“

"$ $Si si istá, že toto zvládneš?“

"$ $Musím len nájsť dostatočný zdroj mágie. Potom to pôjde.“

"$ $To sa niekde nájde... Alebo to necháme na ručný zdroj.“

"$ $Ako dlho si myslíš, že nám potrvá cesta?“

"$ $Za deň to prejdeme. Podľa mapky sa turistická cesta, na ktorej sa priblížime ku prameňu na asi päťdesiat kilometrov, vtedy sa stratíme a začiatok tej turistickej je kúsok mačičkoesom. V poslednom meste pred pralesom.“

"$ $Kedy ide najbližšia výprava?“

"$ $Podľa ich stránky zajtra,"$ $ pozrela sa na hodiny. "$ $Čo je o osem hodín.“

"$ $Máš už zarátaný časový posun?“

"$ $Áno.“

"$ $Tak to musíme vyriešiť náš problém s peniazmi a pripojiť sa k nim... ak sa to ešte...“

"$ $Dá,"$ $ odvetila Mrana. "$ $A dokonca prijímajú aj mačičky.“

"$ $Super. O jeden problém menej. Poďme tam.“

\begin{center}

*

\end{center}

Morja sa teleportovala domov. Necítila sa po tej teleportácii dobre. S časovým posunom sa ešte dokázala vysporiadať, ale tá náhla zmena tlaku... Musí sa vyspať... pomyslela si. Nemala ani zlomok mágie. Cez noc sa jej určite doplní, len musí dúfať, že sa jej nič nestane... a po udalostiach posledných dní by si nebola taká istá...

\begin{center}

*

\end{center}

"$ $Čo chceš robiť?“

"$ $Nevieme kde sú, ale aspoň žijú. Zobrali si všetky veci...“

"$ $Išli hľadať knihu... to som si takmer istá... Tarny...“

"$ $Už si to hovorila...“

"$ $Je tam Goonová proroctva... Musíme ich nájsť. Máš predstavu, kde by mohli ísť?“

"$ $Ani najmenšiu. Oni si teraz chceli oddýchnuť. Lenže... Predpokladala by som, že išli niekde, kde je dostatočne bezpečne na spánok.“

"$ $A to je kde?"$ $ Bella mlčala. A rovnako i Arabela a Niela.

"$ $Napadá mi...“

"$ $Hm?“

"$ $Quert...“

\begin{center}

*

\end{center}

Pauline sa už prebudila, ale Tarny s Tulienkou Deľou ešte stále spali. Nechcela ich budiť, ale ona sama už nebola unavená, tak vstala a porozhliadala sa. Zbadala, že pri nich sú tri prekladače, ktoré používali predtým. Ako vidno, premiestnila sa na správne miesto, spoznali ich.

Bolo Quertské poludnie, prišli tam na Quertské ráno. Slnko bolo vysoko, ale nejako zvlášť nepálilo. Rastliny sa plazili po najväčšej osvetlenej pláni. Nevidela žiadnych domácich obyvateľov, ale určite tam boli – veď sa prebudila pri prekladačoch.

Krajina, kde boli, nebola rovinná. Na pravo od nej sa v pozadí vynímali nejaké Quertské kopce. Vyzerali vysoké, ale nezazrela na nich ani kúsok snehu, či znaku zimy. Na ľavo od nich zas videla nerovný terén, zdalo sa jej to, že to bolo poškodené a nie práve prírodne. Pahorok bol čierny a vyzeral, že v nedávnej minulosti horel.

Vzala si prekladač, keby náhodou niekoho z miestnych stretla a rozhodla sa poobzerať sa okolo. Nechcela sa príliš vzdiali – nevedela čo všetko na Querte je, tak dlho tam nebola, a ani nechcela vystrašiť Tarnyho a Tulienku Deľu, ak by sa prebudili a nevideli ju. Prešla okolo nich a išla bližšie ku zvláštnej pôde vľavo. Každú chvíľu sa obzerala na Tulienku Deľu a Tarnyho, kontrolovala či na nich má stále dohľad, a či sa ešte neprebudili, a tiež, či nezazrie nejakých domácich obyvateľov. To, čo videla, sa jej javilo stále rovnaké. Rovnaká čierna zem...

Znenazdajky, z tej čiernej zeme vyšlo niekoľko osôb. Neboli od čiernej ani trochu špinaví, stále vyzerali ako kryštalické bytosti, ktoré Pauline videla, keď bola na Querte prvý raz. Zdalo sa, že si ju všimli. Nevedela z ich výzoru odčítať emócie, ale zdalo sa jej, že niečo nie je v poriadku. O krok ustúpila.

"$ $Ja som..."$ $ Chcela povedať, ale oni nereagovali. Len niečo medzi sebou blikali, a Pauline im vôbec nerozumela. Zrejme prekladač nefungoval, ako mal, alebo len ona ho použila zle. Niečo ju chytilo. Jedna z Quertských rastlín sa jej chytila, omotávala sa okolo nej. Pauline zdesene vykríkla. Podarilo sa jej vymotať nohu a bežala ku Tarnymu a Tulienke Deli. Stále spali.

"$ $Tarny! Tulienka Deľa! Zobuďte sa! Nejaká ich rastlina..."$ $ Stále spali. "$ $Zobuďte sa!"$ $ Triasla nimi. Hádam nemohli tak tvrdo spať...

Nevšímala si to, čo sa odohrávalo za ňou. Kým nimi triasla, niečo ju chytilo zozadu, tá rastlina, čo predtým a ona sa už nemohla pohnúť. Z úst sa jej vydral nezrozumiteľný výkrik, ale nikto ju, okrem Tarnyho a Tulienky Deli predsa nemohol počuť. Oni stále spali! Čo im len urobili? A prečo?

Nemala čas na premýšľanie, už nič nestíhala. Querťania zrejme neboli taký milí , ako sa jej zdalo z prvého dojmu. Zabudla na mágiu. Zabudla na to, že sa dokáže premiestňovať. Rastlina ju omotala, ale nedusila, zdalo sa jej, že ju, aspoň na nejaký čas, chce nechať nažive. Ale, nie pri vedomí. Posledné, či si pamätala, bolo, že rastlina ju pichla...

Nevedela, koľko bola v bezvedomí. V každom prípade nevidela ani Tarnyho, ani Tulienku Deľu. Mohla otvoriť oči, ale nevidela o nič viac, ako keď ich mala zatvorené. Kde sa to dopekla dostali?!

"$ $Tarny! Tulienka Deľa? Kde...?"$ $ Skríkla čo najhlasnejšie, ako dokázala a v napätí čakala. Odvetila jej ozvena. Bola zúfalá. Nechcela sa premiestniť bez nich, a zároveň sa nedokázala premiestniť ku nim, keďže nevedela, kde sú.

Vstala. Či skôr sa o to pokúsila. Niečo ju držalo za ruky a nohy. Rastlina... Trhala sebou. Potrebovala sa odtiaľ dostať, ale ako sa mohla premiestniť, keď...? Veď má mágiu! Úplne na ňu zabudla. Otázne bolo, či sa jej to podarí, keďže jej pokusy čarovať končili zle. Vybavila si magické pole. Pauline, keď sa prvý raz do magického poľa dostala, hneď nečarovala, ju fascinovali obrazy v ňom. Videla pohyby magiónov, ich množstvá... Objekty, videné normálne, boli zachytené len obrysmi, celé sa jej to zdalo, ako program, v ktorom dodávala mágiu do objektov a manipulovala s nimi...

Kým očami videla iba absolútnu tmu, teraz, pri pohľade na magické pole videla svoju situáciu jasnejšie, obrysy prezrádzali viac než tma. Ležala spútaná rastlinou v zvláštnej, kryštalickej miestnosti. Mimo nej boli ďalšie a ďalšie. Medzi nimi sa nachádzala len malá chodba, ktorou práve prechádzalo niekoľko kryštalických bytostí a viedli niekoho... Žeby Tarny alebo Tulienka Deľa? Ďalej sa jej magické pole nedostalo – mala dosah len v hraniciach jej väzenia. Väčšina miestností bola prázdnych, magicky úplne vyšťavených rovnako. Bolo pravdou, že okrem svojej nevidela žiadnu, kde by niekto bol. Čím dlhšie sa na pole pozerala, tým väčšiemu zúfalstvu prepadala. Nikto nepríde... a oni zmizli... Zas sa vrátila do poľa, zbadala chuchvalec mágie, pri nej. Mohla by ho použiť na rozsvietenie, alebo na zničenie tej rastliny... Keby mohla hýbať rukou, pokúsila by sa vytvoriť Solan, ale ruku jej pevne zvierala rastlina. Vzala energiu. Sústredila sa z celých síl a sústredenia na svetlo. Pokúšala sa vytvoriť ho... Nič. Nedokázala to. Takmer sa rozplakala... Kde, do akej kaše, sa to dostali...?

\begin{center}

*

\end{center}

Zívla si. Bola prebudená. Otvorila oči a nechápala tomu. Bolo možné, že ešte spala? Chcela zdvihnúť ruku, aby sa udrela a prebudila sa. Veď musí to byť sen. Sú na Querte, nie v nejakom väzení... Niečo ju pevne držalo. Po pohľade cez magické pole uvidela rastlinu, ktorá ju držala. Toto sa jej musí predsa snívať... Nejaká škaredá nočná mora... Ležala v nejakej škaredej, temnej kobke. A k tomu bola z kryštálu. Zas sa dostala do magického poľa. Malé chumáče magiónov presahovali podľa jej odhadu od oka prirodzenú hodnotu. Quert poznal mágiu. To čo tu videla, sa jej nezdalo, že by táto mágia bola len od Leany a Belly...

Svoju vlastnú zásobu mágie mala naplnenú. Chcela aspoň vidieť čo sa deje. Nie že by si jej oči nezvykli na tmu, ale táto tma bola absolútna, nebolo v nej vidieť úplne nič. Ako zvyčajne sústredila magickú energiu, mala otvorené oči, vedela, že to vyjde...

Nič. Tma, ako predtým. Nemohla tomu uveriť. Áno, boli polia a inopolia, kde sa mágia používať nedala, resp. ju dokázali používať len démoni a polodémoni. Ale toto sa jej nezdalo. V týchto poliach obyčajne nebolo možné magické pole ani zobraziť... Toto sa jej zdalo mimoriadne čudné... Žeby nejaká... clona, alebo niečo také? Počula o niečom, čo blokovalo magióny – nebolo pri zásahu toho možné čarovať... Žeby sa jednalo o to? Ak áno, bez takého množstva mágie, ktoré by sa dokázalo dostať cez blokádu... nemožné. Aspoň o inom spôsobe samozničenia clony Tulienke Deli známa mágofyzika nevedela.

Ak tam teda bola clona, jej možnosti na únik pomocou mágie boli mizerné. Nemohla telepatizovať, ale vedela, aspoň sa jej zdalo, že Querťania nemajú sluchové orgány, a tak z celého hrdla zakričala.

"$ $Tarny! Si tam niekde?!"$ $ Odpoveďou jej bola ozvena. "$ $Pauline! Dopekla, je tam niekto?"$ $ Ticho. Nič. Buď tam bola nejaká zvuková izolácia, kúzlo, alebo tam proste nikto nebol. Zakliala. Uvedomila si, že pri sebe nemá knihy, čo si vzali od Belly a kúpili. Toto ju znepokojilo viac než fakt, že nevie kde je, kde sú ostatní a ako sa z toho plánujú dostať. Dúfala, že s knihami Querťania nemanipulovali. Lebo, keby áno... Nemohla o ne prísť! Minula na ne ak veľa a mnohé boli vzácne... Dopekla! Pomyslela si. Ležala na zemi, ale zakazovala si spanikáriť. Musí nájsť Pauline, alebo Tarnyho, najlepšie oboch a premiestniť sa po knihy a späť na zem...

\begin{center}

*

\end{center}

"$ $Tulienka Deľa... Čo sa to..."$ $ zamumlal zo spánku Tarny a niečo ním trhlo. Otvoril prudko oči. "$ $Tulie, čo sa to...?!"$ $ Nič nevidel. Zažmurkal, ale pred očami sa mu stále zračila hustá tma. Nespal, aspoň bol o tom presvedčený. Niečo sa muselo stať... Posledné, čo si pamätal, bolo, že zaspali na jednej z Quertských lúk. Keďže vtedy, keď zaspali nebol na Querte večer a Quertský deň bol dvakrát tak dlhý, ako ten na Zemi alebo Wymyslensku... Toto nebola noc. Nepredpokladal, že by prespali vyše pozemského dňa... Neboli až tak unavení... a ostatní by ho iste zobudili... Bolo na tom niečo čudné. Na oblohe (ak to obloha bola) nesvietili hviezdy, ani Quertské mesiace. Dalo sa to, pravdaže vysvetliť oblačnosťou, ale to sa mu nezdalo... Rozhodol sa postaviť. Niečo ho držalo. Dopekla! Pomyslel si. Querťania... žeby oni? Nepoznali ich tak dobre, aby im mohli zasa úplne dôverovať... Alebo ich nespoznali... To by bolo možné, nepoznal prístup Querťanov k cudzincom... Z nejakého dôvodu ich uväznili, ale prečo...?

V tme uvidel záblesky, či skôr blikanie. Farby. Reč Querťanov... Možno to boli ďalší väzni... alebo dozorcovia. Nerozumel ich reči, nemal žiadny prekladač. Nemal tušenia kde sú Tulienka Deľa a Pauline... Zrejme v rovnakej situácii ako on. Počul približovanie sa Querťanov a tiež čoraz jasnejšie blikanie. Kroky... Ale oni nepočuli... Mohli ale vedieť o tom, že ľudia počujú... Aj tak... do čoho strašnejšieho by sa mohol dostať... keby tak, má mágiu... mágia, prečo ju nepoužil? Nevedel, či Querťania poznajú mágiu. Zas, bola tu jeho matka a tá Leana... Je pravdou, že nemuseli prezradiť Querťanom nič o mágii – koniec koncov, ani ľuďom bez magického génu mágiu nikto neodhalil, ale... Pamätal si dosť dobre to, že jeho matka upokojovala Dfefarau, keď bolo spomenuté, že Pauline je polodémonka... Ak tam nebola len korelácia, ale i kauzalita... Poznali mágiu, zrejme áno... Nevedel, či ju dokázali používať, ale poznali ju... V každom Wymyslenskom i väzení patriacom spoločenstvu boli blokácie mágie... Ak je toto väzenie a Querťania vedia používať mágiu...

Nemýlil sa. Do magického poľa sa dostať vedel, ale nie mágiu použiť... Querťania boli čoraz bližšie a bližšie... Nepočujú. Pomyslel si a z celej sily začal kričať, len pre overenie či tú ešte nažive...

"$ $Tulienka Deľa!? Pauline?! Ste tam niekto?!“

\begin{center}

*

\end{center}

"$ $Nejaké stopy po Wymyslensku?“

"$ $Žiadne. Nemyslím si, že sme ich dokonale oklamali, to si určite všimli, alebo došli k záveru, že riskovať vojnu s Tramtáriou im za nás nestojí.“

"$ $Sledovať nás budú. Nebudú chcieť aby sme sa vrátili a vyvolávali otázky. To som pokazila ja. Musíme čakať, kým na nás zabudnú."$ $ Pozrela sa na Loviisu, pre ktorú sa to, z hľadiska jej túžby vrátiť sa domov, vyvíjalo zle a dodala. "$ $A teba nejako dostaneme do Wymyslenska. Koniec koncov, prešli sme cez hranice Tramtárie,"$ $ Loviisa prikývla. Dúfala, že Wymyslensko má krátku pamäť.

Dostali sa na zoznam do turistickej skupiny. Ešte mali voľné miesta a Sylvia si nebola istá, či Mrana nezaplatila viac, ako bola cena. Ona a Loviisa zmenili svoj vek a radšej i výzor zmyslovou mágiou. V Tramtárii síce nebola žiadna oficiálna veková hranica plnoletosti, ale napriek tomu vedeli, že nesmú vzbudiť podozrenie. Väčšina ľudí a zovero necestovala po pralesoch v štrnástich. Síce tam bolo to nebezpečenstvo, ktoré bolo veľmi lákavé, ale i unavujúce a nepohodlné. Na cesty po pralesoch vyrážali zväčša zabezpečený dobrodruhovia, amatérsky prírodovedci alebo obyvatelia s mimoriadnou láskou ku nebezpečenstvu v prírode. Ony si vybrali na zapadnutie do davu prvú možnosť, v turistickej skupine bolo najviac presne takých. Na pôvod mačičiek mali vytvorený dôvod dedičstva po nejakom starom rodinnom priateľovi – wymyslenčanovi. Ale nepýtali sa ich. Tramtária síce neprijímala turistov z iných krajín, ale Tramtárijčania niekedy cestovali na Zem či do Fentenzie (alebo ostatných častíc vesmíru), hlavne za vedeckými účelmi, ale pri tomto vycestovaní sa dalo vrátiť (na rozdiel od prostého opustenia Tramtárie). Takto sa ku mačičkám dostať dalo, ale zmeniť ich bolo prakticky nemožné, a tak spolu zo všemožnými lokálnymi menami (Tramtária nemala celoštátnu menu) sa platilo i mačičkami.

Bolo im oznámené, že pred odchodom do pralesa musia prejsť krátkym kurzom prvej pomoci, správania sa v pralese a tiež správania sa v prípade nebezpečenstva. Mal trvať pár dní.

"$ $Mohli sme to čakať,"$ $ povedala Sylvia, po vypočutí si nasledujúceho programu.

"$ $Čo myslíš, ako dlho sa tu zdržíme?"$ $ Spýtala sa Loviisa, dúfajúc, že to bude krátko. Mrana i Sylvia pokrčili plecami.

"$ $Hovorili že kurz bude trvať okolo týždňa. Hádam nás Wymyslensko dovtedy nenájde.“

"$ $Potom sa budeme brániť. Loviisa má Lasermeč a... keby, tak ujdeme na Zem.“

"$ $To sme nemohli hneď?“

"$ $Nie. Prepáč Loviisa, nie. Je to riskantné, riskantnejšie ako toto.“

"$ $Hovorila si, že máš rada riziko.“

"$ $Ty dobre vieš, čo ti Wymyslensko vzalo. Musíme počkať na vhodnú príležitosť.“

"$ $A to bude kedy?!“

"$ $Neviem, prepáš."$ $ Odvetila Sylvia, otočila sa a vykročila od nej. Mrana ju zastavila.

"$ $Sylvia, neopúšťaj sa zas! Teraz ťa potrebujeme a naopak.“

"$ $Tak čo?! Všetci ma...“

"$ $Nebuď paranoidná... zas začínaš.“

"$ $Ja...“

"$ $Sylvia vážne prestaň."$ $ Povedala jej Loviisa.

"$ $A čo...?!“

"$ $Počúvaj ma!"$ $ Loviisa sa jej pozrela do očí. "$ $Možno si presvedčená, že všetko je tvoja vina, možno si presvedčená, že všetko máš pod kontrolou! Ak sa mýlim, prepáč, nevidím ti do hlavy! Ide o to, že ak sa tu zas opustíš, narobíš si sama sebe a tiež nám problémy. A tiež si prestaň namýšľať, že ťa každý nenávidí! Dopekla, je to blbosť. Nie si síce najpríjemnejší človek na svete; neber to zle, ale nie tak, aby ťa každý nenávidel! Uvedomuješ si ale, že tým, že sa tu budeš psychicky rútiť nič nevyriešiš, ale len pokazíš?! Dočerta, chápeš to?!“

"$ $Ale ja som...“

"$ $Sylvia, prestaň s tým. Nikto z nás ťa tu preto neupaľuje. A ľudia okolo teba nevedia kto si. Tak ako by mali... to nedáva logiku.“

"$ $Nepriamo.“

"$ $Takže sa s...“

"$ $Nimi."$ $ Podráždene doplnila Sylvia, aby Loviisa nedajbože nepoužila slovo démonmi.

"$ $...Identifikuješ."$ $ Dokončila.

"$ $Nie, ja nie som...“

"$ $Tak prečo ťa to potom uráža?“

"$ $Vieš prečo?! Lebo keby vedeli kým som, nenávideli by ma. Chceli by ma zlynčovať, zničiť, absolútne... chápeš?! Ja nedokážem... nedokážem... falošný pokrytci. Nenávidím sa, kto som... lebo som...“

"$ $Dostávame sa ku problematiky toho, že sama sa ku tým, ktorí lynčujú démonov pridáš.“

"$ $Nikdy.“

"$ $Takto to dopadá, Sylvia. Preto, aby nespadol tieň podozrenia na teba, budeš tými čo lynčujú.“

"$ $Nie. Nikdy.“

"$ $Už sa nenávidíš.“

"$ $Lebo som veď vieš čo! Vieš aký je postoj ľudí ku mne! Nemôžem sa nenenávidieť! Som...“

"$ $Už úspešne nenávidíš seba a aký je krok ku tomu, aby si začala nenávidieť aj...“

"$ $Nenávidím D, lebo to je celé jeho vina. Nenávidím ľudí a zovero, ktorí zovšeobecňujú, nenávidím seba, lebo som...“

"$ $Je aj niekto, či niečo, čo nenenávidíš?“

"$ $É..."$ $ Premýšľala. "$ $Vás nenenávidím.“

"$ $Aspoň niečo...“

"$ $Ale ty to nechápeš! Ja... ja nemôžem s tým žiť! A nemôžem kvôli tomu umrieť! Chápeš...?!“

"$ $Keby si zomrela, čo by bolo s tvojou revolúciou? Čo by bolo so všetkými tvojimi plánmi?“

"$ $Nestali by sa. Čo už. Ale ja by som sa nemusela trápiť so sebou.“

"$ $Čo by si bola schopná spraviť pre svoju smrť?"$ $ Sylvia sa otočila na päte a obrátila sa zas ku ním. Mračila sa a hľadela Loviise do očí.

"$ $To je citové vydieranie!“

"$ $Nie, to je otázka. Čo si všetko schopná obetovať pre svoju smrť? Zožieraš s tu za ostatných a nedokázala by si byť s tým, že sa niekomu z tvojich priateľov niečo stane. Tak čo by si obetovala? Pre smrť. Obetovala by si ostatných? Alebo čo ti je prednejšie?“

"$ $Je to citové vydieranie!"$ $ Bránila sa Sylvia. Mrana bola zvedavá na Sylviinu odpoveď, pretože ona sama sa jej na to nikdy nespýtala.

"$ $Je normálna dilema. Budeš môcť zomrieť s tým, že si niekoho pre to obetovala?“

"$ $D obetujem vždy.“

"$ $Tu nejde o D, Sylvia, ale o nás a tvojich ďalších priateľov. Čo si ochotná obetovať pre tvoje ciele? Po smrti už predsa nič necítiš. Už nie si. Tak bola by si ochotná zomrieť aj za takú cenu, alebo sa zožieraš nielen pre pocit?“

"$ $Ja..."$ $ Bola nervózna. Nepozorovane cvakala zubami, otáčala sa a vyhýbala sa odpovedi.

"$ $Tak čo? Alebo nedokážeš prežiť to, že by sa niekomu z nás niečo stalo len pre ten pocit?"$ $ Sylvii mlčala. Nevedela odpoveď, ešte nad tým nepremýšľala... Naozaj nedokázala dopustiť aby sa im niečo stalo len preto, že sa cítila neskutočne vinne, alebo pre to, že jej na nich záležalo, i na tom aby sa im nič nestalo...? A nie je to rovnaké, neprelína sa to? Čo bolo dôvodom toho, že jej na nich záležalo? Keby necítila výčitky, netrápila by sa pre nich? Kvôli D by necítila výčitky, ani kvôli Cecílii... Aký bol medzi tým rozdiel? Vo vzťahu ku tej osobe? Ale odtiaľ pochádzali výčitky, či nie? Sylvia neverila na nijakého boha, či vyšší morálny zákon, rovnako ako na nejaké absolútne svedomie, nezávislé od emócií. Je to, že sa obáva toho, že by sa im niečo stalo, len následkom strachu z výčitiek? Alebo...? Nevedela odpovedať. Bála sa svojej odpovedi. Nedokázala odpovedať. Nedokázala počuť svoju odpoveď... aká vlastne je?

"$ $Nevieš? To vážne nevieš?“

"$ $Citovo ma vydieraš!“

"$ $To by znamenalo že by si...“

"$ $Nie! Ja neviem! Nepremýšľala som nad tým!“

"$ $Tak nad tým premýšľaj teraz.“

"$ $Ja neviem! Ale za to, žeby sa mojou vinou niekomu z vás niečo stalo, za to by som zomrela.“

"$ $Ty chceš zomrieť za veľa vecí, Sylvia.“

"$ $Správne. A raz sa to tak aj stane.“

"$ $Stále hovoríš, že proroctvá sú výmysly, Sylvia,"$ $ odvetila jej Mrana.

"$ $Ale zomrieť nemôžem. To je aspoň zatiaľ fakt. Zatiaľ je tu nádej s D.“

"$ $Ako myslíš. Len prosím, zas sa nezrúť, vážne. Teraz nie.“

"$ $Mrana... ja sa bojím. O Tulie... dúfam...“

"$ $Nezačínaj s tým zas. Vieš, že to nedokážeš ovplyvniť. Musíme sa skryť a potom... potom sa vrátime."$ $ Sylvia mlčky prikývla, stále premýšľajúc nad Loviisinou otázkou.

"$ $No... mali sme náročný deň, idem si ľahnúť."$ $ Prerušila Sylvia ticho. "$ $Dobrú noc."$ $ Mrana zrušila zvukovú izoláciu v miestnosti a i ona sa pobrala spať. Keby boli von z hotela, videli by vyjsť slnko.

\begin{center}

*

\end{center}

Na tom celom bola predsa len jedna vec dobrá. Konečne mal prekladať a mohol zistiť, čo od neho chcú a prečo ho (ich) väznia. Možno napokon došlo ku omylu... Ale tých zlých vecí na tom bolo oveľa viac. Nemohol používať mágiu, nevedel, kde sú ostatní, nemal sa ako odtiaľ dostať, zistil, že netuší, čo sa stalo s ich vecami (to bolo to najmenej) a bol pripútaný ku nejakému kreslu. Nič, čo by sa dalo považovať za prijateľnú situáciu. Bytosti medzi sebou niečo blikali. Nerozumel im napriek prekladaču, nevidel ich, otočili sa od neho, aby nemohol poznať ich konverzáciu. Mohli používať mágiu. V poli to videl, ale jeho blokovali. Keby tak aspoň vedel prečo, dopekla! Pomyslel si.

Nevedela ako dlho tam už sedí. Querťania odchádzali, blikali, prichádzali, ale nedozvedel sa prečo tam dopekla je. Rozprávanie by mu zrejme bolo nanič, i keď... nevedel ako tie prekladače fungujú. On nevedel blikať... premýšľal, či ten prekladač bliká, alebo ho majú i oni... Mali ho vtedy, alebo nie?

Rezignoval na pozorovanie okolia. Bol sám, Querťania sa niekde vytratili. Nemohol sa pohnúť. Bol aj vcelku hladný, ale to bolo to najmenej. Prečo ho vytiahli z tej cely a nechali ho tu? Dopekla... Aspoň tu bolo svetlo...

Na stenách kryštalickej miestnosti nebolo nič. Nejaké náhodné zárezy, ale nič podstatné v nich nenašiel. Ani piktopísmo, ani nič také... Alebo tam možno niečo bolo, ale rozoznať to pre mimoquerťana bolo nemožné. Na podlahe miestnosti ležali rastliny, skrútené do klbôčok. Žiadny Querťania.

Magické pole ukazovala mágiu okolo neho. A jeho blokovali... sakra, sakra! Nebol únik. Žiadny. Nebol zúfalý to nie, ale chytala sa ho bezmocnosť. Myslel si, že má plán a takto to dopadne... D získa knihu a oni sa to nikdy nedozvedia... skončia na Querte... Pauline sa mohla premiestniť, ale nevedel, či by to spravila bez nich, a či sa odtiaľ vôbec premiestňovať dalo. Mágiu blokovali, tak prečo by nie premiestňovanie.

Bola ale pravda, že Tarny o žiadnej metóde blokovania premiestňovania nevedel. Určite existovala, ale buď bola iba prirodzená, vstavaná v inopoli, alebo bola nesmierne zložitá, inak by ju iste v spoločenstve používali...

Nepočítal čas. Miestnosť bola stále rovnaká, nič sa nemenilo, žiadny indikátor času. Jeho vnútorné hodiny zlyhali. Mohol tam byť hodiny a možno aj pár dní. Zvláštne bolo to, že smäd a hlad ustúpil – nepamätal si, že by zaspal, alebo jedol. Bolo to proste čudné, i keď nepochyboval o tom, že to je dielo Querťanov. Keby sa tak aspoň ukázali... ale odvtedy, čo tam boli na začiatku sa už neobjavili. Ubiehali minúty a hodiny. Premýšľal, či ich tu proste chcú nechať zomrieť, alebo kde sú.

Zistil, že telepatia funguje. Aspoň niečo. Telepatické pole mal síce veľmi zúžené (v dôsledku blokovania mágie), ale našiel Querťanov. Boli úplne na okraji jeho dosahu. Otázne bolo, či sa oni dokážu dorozumieť telepaticky, alebo skôr, či sa dorozumievajú telepaticky tak ako na Zemi alebo na Fanase. Predpokladal, že nie. Predsa len, telepatiou dali posielať obrazy aj slová (či skôr myšlienky vo forme slova) a Querťania nemali koncept slova, tak ako oni. Querťania sa dorozumievali obrazmi, ale nie tak ako bolo napríklad obrázkové písmo – Querťania v obrazoch (aspoň to tak predpokladal) aj mysleli. Alebo skôr v blikaní.

Telepatiou sa dalo dorozumieť i bez poznania jazyka toho druhého. Ale len pri použití čistej telepatie, teda bez telepatiónov, ktoré s telepatiou pomaly nemali nič spoločné – viac boli ako tradičné telefóny. Telepatiou sa posielali myšlienky, ale telepatické pole rozlišovalo formu myšlienky. Telepatická správa bola myšlienka a táto myšlienka sa posielala, a keď hu adresát dostal, jeho mozog si ju preložil do jeho jazyka. Nevedome. Pri obrazoch to tak úplne nefungovalo. Poslal sa presný obraz, ktorý odosielateľ posielal a mozog adresáta ho ako rovnaký spracoval. To bol jeden z výhod telepatie, keďže sa ňou dali pozerať a odosielať celé filmy. Alebo nevýhodou. Záleží od toho, koho pohľadom sa na to pozriete.

Querťania nemysleli v slovách – nemali sluchové orgány, nemali koncept slova ako vytvoriť. Možno bolo možné, aby sa slová automaticky... to sa mu zdalo nemožné. Veda ohľadne telepatie na zemi sa ešte nemala možnosť stretnúť sa s bytosťami, ktoré nemali slová, aby tento problém vyriešila. Rozhodol sa to skúsiť.

Nič. Neodpovedali mu. Buď sa to nepreložilo, ako predpokladal, alebo ho proste ignorovali.

Zistil že má cez telepatiu poslaných niekoľko filmov. Keďže nič iné nemohol robiť, meral čas cez nich.

Od vtedy, ako začal merať čas cez filmy, uplynulo asi pol Quertského dňa. Teda jeden pozemský. Došli filmy a on bol už unavený, nebavilo ho čakať na Querťanov, ktorí sa očividne rozhodli neprísť. Zaspal.

\begin{center}

*

\end{center}

Oklamali senzory. Mohol to čakať, že ich chcú oklamať... Nakoniec išli do Tramtárie! Do tej zločinnej krajiny, plnej prívržencov a posluhovačov Démona! Do Tramtárie ísť bolo nebezpečné pre porušenie status quo ktorý zabraňoval celofanasskej vojne. Tramtária sa nemieša do vnútorných záležitostí Wymyslenska a naopak. Tento stav bol často predmetom diskusií – prečo by malo Wymyslensko akceptovať ten zločinný štát, ktorý ani štátom nie je? Celý tento stav pochádzal z toho, že Wymyslensko i Tramtária (aspoň jej diplomatický zástupca) si uvedomovali výhody súpera a nikto nechcel riskovať porážku. To, že na Fanase nebol vojna bolo výsledkom toho, že štáty boli približne rovnako silné a vojna by pre nich znamenala zánik. Rovnako to bolo i so spoločenstvom. Keďže Wymyslensko a Spoločenstvo boli ešte aj na inej planéte, vojna samotná by prebiehala komplikovane.

\begin{center}

*

\end{center}

Zobudili ho Querťania. Nebol unavený, konečne sa mohol dozvedieť o čo ide. Alebo ho zas zobudili z dôvodu, z akého ho vytiahli z cely. Nijakého. Nerozoznával Querťanov – väčšina z nich sa mu zdala rovnakých. Mlčali. Ani neblikali. Stáli pred ním. Asi to zas nebude nič, chcú ma len zničiť. Pomyslel si, a vtedy jeden z Querťanov prehovoril, či skôr začal blikať.

"$ $Si mág."$ $ Nebola to otázka. Tarny mlčal. "$ $Používaš mágiu."$ $ Premýšľal či mu tykali zámerne, alebo to bolo v prekladači, alebo niečo ako vykanie a tykanie nemali.

"$ $Čo odo mňa chcete?!"$ $ Neblikali. Prekladač tú správu preložil, ale oni sa správali, že ju nezaregistrovali. "$ $Čo sa deje?“

"$ $Odtiaľto sa už nikto nedostane. Už tu nie je Ölverínska nástupkyňa."$ $ Pochopil, že hovoria o Leane.

"$ $Čo má...“

"$ $Je tu kontaminácia mágie, už dlhé roky. A už všetko je na tebe.“

\begin{center}

*

\end{center}

Konečne vyrazili. Po dňoch príprav, cvičení, nácvikov a školení boli oprávnení vkročiť do pralesa. Sylvia mala celý plán už dávno prebratý s Mranou aj Loviisou.

Wymyslensko ich nehľadalo, aspoň sa to zdalo. Udržiavali zmyslové kúzlo vždy keď neboli sami a zdalo sa, že ich neodhalili.

Sylvia nebola nervózna, to nie. Zakazovala si myslieť na Tulienku Deľu a dúfala, že je v poriadku. Vedela, že keby to začala rozoberať, skončilo by to zle. Namiesto toho vždy, keď mala čas prezerala plány, dopracovávala detaily, prezerala mapy. Loviisa pri sebe nosila pre prípad nebezpečenstva lasermeč a pri najmenšom podozrení vytvárali obranu. Musí to vyjsť, musí to vyjsť... premýšľala. Nesmiem sa zrútiť. Musí to vyjsť. Do Loriataru sa nedostanú... to nie je možné... ale ako sa tam dostanú oni? A čo v Loriatare vôbec je? To Sylvia len tušila, ale nevedela. Žiadne presné dokumentácie o Loriatare neexistovali. Len nech to nie je niečo nebezpečné... pomyslela si.

\chapter{Hlavný súdruh vás víta}

Pochod po pralese bol dosť fyzicky namáhavý, ale po tréningoch pred vstupom do pralesa sa dal zvládnuť. Všetko bolo zo strany organizátora dostatočne premyslené, takže ich žiadne komplikácie nezasiahli. Meškali len pár hodín, kvôli dažďom, ktoré boli ale v týchto končinách dosť časté. Blížil sa deň, kedy mali dosiahnuť najväčšiu blízkosť prameňa Iasue. Vždy keď nešli, nemali prednášku ani žiadny program, kontrolovali plán. Ku miestu, kde sa chceli stratiť boli vzdialený na deň cesty. To, aby si niekto nevšimol, že chýbajú, zabezpečila Mrana. Na jednom z batohov mala akonáhle ju spustia fungovať hlagenovka na princípe toho, že turisti ich nechcú stratiť, teda ich chcú vidieť. Vedela, že nebude bežať večne, ale aspoň na pol dňa batéria stačiť mala.

Nemohla povedať, že sa jej prehliadka džungle nepáčila – mala ju prvý raz v živote možnosť vidieť naživo. I napriek tomu, že celá turistická skupina bola len najbezpečnejšia cesta, ako sa dostať do Loriataru, cesta ju zaujímala. Bola rada, že Loviisa sa zachovala zeleno a nechcela ničiť prales mačičkoesom.

Nepodozrievali ich. Zmyslové kúzlo vytvorila dostatočne silné a okolo Loviisi ho pre istotu i naprogramovala, rovnako ako počas spánku.

Nastal deň D. Loriatar bol najbližšie. Podľa plánu mali sa zneviditeľniť a vtedy zapnúť hlagenovku. Sylvia si celé plánovanie nepripúšťala neúspech. Vedela, že tým, že ich zaviedla do pralesa ich dosť ohrozila, ale nepripúšťala si to. Nemohla.

Čítala mapu.

"$ $Musíme si dať pozor. Nemali by si nás všimnúť, ale mali by sme rátať i s tým, že by sme Loriatar nenašli. I keď to nepovažujem za...“

"$ $Nemôžeš vedieť ši budú chcieť sa odhaliť. Nič o nich nevieme.“

"$ $To je možno pravda, ale ak sa pozrieme na zápisky o Loriatare...“

"$ $Sú to len zápisky.“

"$ $O tom sme už hovorili. V prípade neúspechu sa nejako dostaneme preč. Nedostanem vás do nebezpečenstva.“

"$ $V tom už sme.“

"$ $To som si všimla. Ale nie do smrteľného. Nedopustím to.“

"$ $Dokážeš to?“

"$ $Iba vo veľmi zúfalých prípadoch."$ $ Povedala, pričom slovo ‚zúfalých‘ zdôraznila.

"$ $Kedy teda bude tá vhodná príležitosť na odchod?"$ $ Spýtala sa Loviisa.

"$ $Dnes popoludní. Mám to tam preskúmané, cesta tam je relatívne prístupná..."$ $ Odvetila Mrana.

"$ $Ty si tam bola?!“

"$ $Sylvia, samozrejme, že nie. Ešte sme sa ku tomu bodu nedostali. Poznáš dačo také ako kamery, všakže?"$ $ Sylvia sa upokojila.

"$ $Ešteže tak. Hovoríš cesta je v poriadku?“

"$ $Áno. A to ma prekvapilo. O prameni Iasue sa hovorí...“

"$ $Iasue... ja som vedela, že...“

"$ $Čože?"$ $ Nechápala Loviisa. Mrana tiež nevedela, čo Sylvia hovorí.

"$ $Iasue... niečo o nej som počula, či skôr čítala. Ale keby som si spomenula čo! Myslím, že to do seba zapadá! A dosť dobre! To prečo sa ku jej prameňu nechodí... dopekla, čo to bolo?!“

"$ $Ohrozuje to niečo?“

"$ $To neviem. Niečo je s tým prameňom.“

"$ $Spomeň si. Kde si to čítala?“

"$ $To bola nejaká poézia, alebo žeby legenda...“

"$ $Čo tam bolo?“

"$ $Prameň Iasue... prameň života, či čo to dopekla... prameň smrti... miesto zakrýva skutočnosť. Pri ňom pole, inopole, nedostane sa tam bez ich vôle... či to tam bolo.“

"$ $Pamätáš si celý Tlejúci oheň."$ $ Nechápala Loviisa. "$ $To si si nevedela zapamätať nejakú mizernú legendu.“

"$ $Nepamätám si celý Tlejúci oheň. Pamätám si moje obľúbené časti.“

"$ $Tu nejde o to, čo si z čoho pamätá Sylvia. Tu ide o tú legendu. Si si istá tým, čo si povedala?“

"$ $Nie úplne. Ale myslím, že to je niečo tak ako som hovorila. Prameň Iasue, prameň iný, prameň života a smrti. Pri ňom pole, inopole, nedostane sa bez poľa vôle. Tým poľom, inopoľom bol zrejme myslený Loriatar. A to by vysvetľovalo tú nedostupnosť.“

"$ $Ale máme problém aj my, Sylvia.“

"$ $To teda máme. Budete sa držať pri mne. Keby niečo tak...“

"$ $Ja rozumiem.“

"$ $Aj keď som sama sebe prisahala, že to bolo naposledy.“

"$ $Rozumiem.“

\begin{center}

*

\end{center}

"$ $Žije. Tá, ktorá má podľa proroctva poraziť Démona. A je to Démonka.“

"$ $A napokon ho porazíme my. Ale ak je to démonka... Tým si si istá? Nemôžeme sa dostať do omylu. To nie je možné.“

"$ $Prirodzene.“

"$ $Tak nie je to len polodémonka?“

\begin{center}

*

\end{center}

"$ $Nevšimli si to?“

"$ $Nespozorovali. Žiadna mágia okrem našej. A nemyslím si, že by chceli skrývať svoju mágiu.“

"$ $Nevieš kto sú.“

"$ $Nevedia kto sme my.“

"$ $To nevieš.“

"$ $Musíme byť paranoidní?“

"$ $To, že som paranoidná, to neznamená, že po mne nejdú."$ $ Odvetila zo smiechom Sylvia, ale o chvíľu zvážnela. "$ $Chápeš ako to myslím. Nie je to chorobné, ale opatrnosť nie je nazvyš. Najmä ak sme v strede pralesa, kde už dosť dávno nikto nebol. A teraz ku momentálnej situácii. Mrana, aký je kurz?“

"$ $Vidíš mapu. Stále rovno. Počkaj, ukážem ti to."$ $ Mrana zastala a vysvetľovala cestu. Sylvia prikývla.

Išli relatívne rýchlo. Miestami zastavovali pre kontrolu situácie. Udržovali pre všetkých okrem seba neviditeľnosť, nie tak ako pre možných nálezcov z expedície ako pre dragony, žijúce po celom pralese. Každá mala okolo seba nepretržitú obranu, aby zabránili zraneniu.

Prvú noc sa striedali na hliadkach. Vlhkosť prostredia eliminovali mágiou. Zatiaľ všetko vychádzalo. Až podozrivo dobre. Z päťdesiatich kilometrov prešli po pár dňoch polovicu.

"$ $Je zaujímavé, že komplikácie sa nám zatiaľ vyhýbajú. Žiadne útoky divokej zvery, nikto nás tu nehľadá...čo keď Loriatar chce, aby sme sa tam dostali?“

"$ $A to je problém?“

"$ $Pozri, ak niekto chce aby sme sa niekde dostali, tak logicky nie len tak. Niečo od nás chcú a mám s tým, že odo mňa niekto chce zlé skúsenosti.“

"$ $Hm?"$ $ Nechápala Loviisa. Sylviu o sebe v minulosti nikdy hovoriť nepočula.

"$ $Chcú ťa využiť.“

"$ $Čo sa ti...“

"$ $Nič. Nerieš."$ $ Pokračovali v ceste.

\begin{center}

*

\end{center}

"$ $Čo chceš, Morja? Prečo si sa vrátila?“

"$ $Izabeta, počúvaj ma. Toto je pre dobro spoločenstva a...“

"$ $O tom, čo je pre dobro spoločenstva viem najlepšie sama! A vôbec ty...“

"$ $Ty niečo hovor! A čo Čeria a Rever?!“

"$ $To nie sú...“

"$ $Čo tu zapieraš Izabeta. Každý tu má škandály – ty aj ja.“

"$ $Ja..."$ $ Nenechávala sa v žiadnom prípade vyviesť z miery.

"$ $Ty! Čeria a Rever Thix! Bože, každý to vie Izabeta, ale nikto to neprizná, lebo ty si Izabeta! Tebe to nikto neprezradí nahlas, pretože ty si Tlogenová Izabeta, zakladateľka! A čo som ja? Teraz nič, takže moja vina môže byť verejná!“

"$ $Morja, to čo si...“

"$ $Sa stalo a nie som na to hrdá. Nepopieram to, na rozdiel od teba, ak si si nevšimla!“

"$ $Čo popierať Morja! Ty si sa previnila voči rodine, už do nej nepatríš!“

"$ $Dokedy budete toto dopekla opakovať?! Ja som tu neprišla žiadať o vaše odpustenie!“

"$ $Tak prečo si sa vrátila Morja?! Dopekla!“

"$ $Izabeta, ty vieš prečo. Kde je Niela?“

"$ $S Arabelou.“

"$ $A kde sú?“

\begin{center}

*

\end{center}

"$ $Vidím Iasue..."$ $ Vydýchla Sylvia.

"$ $Zdá sa, že sme pri mieste. Ale...“

"$ $Je to podozrivé."$ $ Dokončila Sylvia.

"$ $Čo je podozrivé? Nie je snáď dobré, že sme tu?“

"$ $Ako som povedala, never tým, čo ti pomáhajú. Dostaneš sa tým len do problémov.“

"$ $Ani vám nemám dôverovať?"$ $ Podpichla ju Loviisa.

"$ $No... nám by si mala, keďže...“

"$ $Sme v strede Tramtárijského pralesa a prenasleduje nás Wymyslenská polícia."$ $ Dodala Loviisa ako by to bola úplná samozrejmosť. "$ $Ale teraz tu sa ťa pýtam na to, čo si povedala. Ak nemám veriť tým čo mi pomáhali...“

"$ $Chcela som povedať ‚Never tým čo ti len tak pomáhajú a nepoznáš ich.‘“

"$ $Teba poznám pár dní.“

"$ $To je možno pravda, ale hádam...“

"$ $Ty dobre vieš, že ty si ma zatiahla do toho všetkého!“

"$ $Loviisa... nie som si istá, či teraz je dobré miesto na to, aby sa Sylvia..."$ $ Zasiahla do toho Mrana.

"$ $Emočne ma nevydieraj! Ty si sa rozhodla ísť so mnou!“

"$ $Mne sa zrútil svet, dopekla! Čo sa stalo tebe?! Stratila som svoju rodinu... všetko... ale ja som na nich zabudla... ja som... Cecília... komu sa teda dopekla rúca svet?!“

"$ $Na to, že sa ti rúca svet to berieš výnimočne dobre."$ $ Sylvia bola podráždená.

"$ $Lebo sa ovládam! Tým, že sa zrútim nič nevyriešim!“

"$ $Tak prečo sa rúcaš?“

"$ $Čo ti je dopekla?! Čo sa stalo tebe?! Myslím, že určite niečo tragické, keď...“

"$ $A to stalo! Presne to! Ako hovoríš!"$ $ Bola mimoriadne nahnevaná. Mrana na seba vzala celé ich obranné pole, pretože Sylvia vyzerala, že sa čo chvíľu zrúti.

"$ $Prestaňte sa dokelu hádať! Uvedomte si kde sme! Dohádate sa potom. Sylvia! Vážne prestaňte, lebo sa niečo zlé stane!“

"$ $Tak povedz čo sa ti..."$ $ Nedopovedala. Sylvia bola vážne nahnevaná. Ustúpila pár krokov dozadu a vystúpila z obrany. Nevšimla si to.

"$ $Sylvia! Vráť sa!"$ $ Mrana jej ústup zaregistrovala, ale už to nestihla. Sylvia vošla do prázdna. Stáli na dosť úzkej, jedinej ceste ku prameňu nezaplavenej Iasue. Prúd bol silný a rieka široká.

"$ $Sylvia!"$ $ Zvreskla Loviisa.

"$ $Som nesmrteľná dopekla!“

Nestratila súdnosť. Okamžite kúzlom zastavila vodu, ktorá do nej narážala a vysušila sa. Po ešte pár kúzlach vystúpila na súš.

"$ $Zabiješ sa!“

"$ $Vieš ako rada!“

"$ $Nie si normálna!“

"$ $To som si všimla!“

"$ $Nehádajte sa dopekla! Teraz na to nie je čas!"$ $ Zakričala ne Mrana.

"$ $Toto oľutuješ."$ $ Precedila cez zuby Sylvia Loviise. "$ $Do mojej minulosti ti nič nie je."$ $ Loviisa sa najskôr pozrela na ňu, potom na Mranu. Tá prikývla.

"$ $O Sylviinej minulosti nevie nikto okrem Sylvie. Ja to rešpektujem. A to radím aj tebe.“

Ďalej išli mlčky.

\begin{center}

*

\end{center}

"$ $Je to ona.“

"$ $V mene štátu sú potrební aj tí druhí.“

"$ $Nečakal som, že sa naučíš za taký krátky čas dôverovať oficiálnej filozofii.“

"$ $To bolo pre ostatných.“

"$ $Nedozvedeli by sa to.“

"$ $Nevieš.“

"$ $Vymazávanie pamäti je dostatočne účinné.“

"$ $Vôbec nie.“

\begin{center}

*

\end{center}

Iasue vyvierala mimoriadne blízko miesta kde stáli.

"$ $Kde je ten Loriatar.“

"$ $Nechce sa otvoriť..."$ $ Sylvia stisla zuby. Potrebovala pokoj. Aspoň na chvíľu pokoj od všetkého. Dúfala, že ho nájde v Loriatare, bájnom meste... Ale čo o ňom vedela? Prečo doňho vkladala nádeje? Vtiahla ich do nebezpečenstva pre ňu! Iba pre ňu! Loriatar bola báj! Mýtus! Obzerala sa, túžiac vidieť bránu do Loriataru, nejaké dvere... Ale všade bola len hustá džungľa a modrá, zvíjajúca sa Iasue.

"$ $Nič tu nie je..."$ $ Sklesnuto priznala. "$ $Loriatar je mýtus. Ako si vravela Mrana. A ja teraz..."$ $ Nadýchla sa a odmlčala sa. Pozerala na rieku. "$ $Keby som nevedela, že to nemá zmysel, skočila by som tam.“

"$ $Naozaj to nemá zmysel."$ $ Chcela prikývnuť, ale vtedy si to uvedomila. Ten hlas nebol Mranin. A nebol ani Loviisin. Otočila sa.

Nad prameňom Iasue sa vznášal svet. Loriatar. Tak predsa... Pretrela si oči, neveriac, že to čo vidí je pravda... dostali sa ku Loriataru... majú jeho bránu, stačí len vojsť...

"$ $Sylvia, čo robíš?"$ $ Loviisa bola z náhleho objavenia sa brány mierne vydesená. Sylvia skôr zaskočená, ale to jej nebránilo, aby išla smerom ku bráne.

"$ $Idem do Loriataru. A možno by bolo dobré, aby ste vošli aj vy.“

"$ $Veľmi správne rozhodnutie."$ $ Ozval sa zas hlas. Pochádzal z brány. V nej stála žena a pozerala sa na nich. Nevychádzala ani krok z brány. Sylvia nepremýšľala prečo. Nebola mladá, ale ani úplne najstaršia. Bola to ale Wymyslenčanka, takže odhadnúť jej vek sa takmer nedalo. Mala čierne vlasy ostrihané nakrátko. Na sebe mala plášť s nejakým zvláštnym erbom, Sylvia rátala že to bol erb Loriataru.

Sylvia zastala. Pár krokov pred sebou mala vstup do Loriataru, ale nevchádzala. Prečo sa naraz objavil? Chce od nich niečo? Mala zlé skúsenosti, ak niekto ju len tak zachránil. Veľmi zlé... Pomaly začínala ľutovať, že v Loriatar verila. Bol tu, ale... Mala pochybnosti.

"$ $Loriatar vás víta, slečna. A vašich priateľov rovnako.“

"$ $Ďakujem za pozvanie,"$ $ chladne odvetila Sylvia.

"$ $Sylvia, prečo nevstupuješ?"$ $ Spýtala sa jej Loviisa.

"$ $Neviem, či im môžeme veriť.“

"$ $To hovoríš teraz?“

"$ $Hej."$ $ Osoba v bráne nejavila známky netrpezlivosti. Sylvia sa pozrela na Mranu a na Loviisu.

"$ $Tak ideme tam?"$ $ Mrana sa obrátila na Sylviu. Tá sa nadýchla a vykročila ku bráne.

"$ $Presne tak.“

Loriatar ju ohromil svojou veľkosťou. Nebol obrovský, to ani nečakala, ale bol viac ako mesto. Odhadovala by, že bol väčší ako Kralovo, hlavné mesto Wymyslenska. Bránu zvnútra nebolo takmer možné rozlíšiť.

Vošli do vysokej budovy, vyzerajúcej veľmi veľkolepo oproti ostatným, ktoré zatiaľ videla. Žena ich viedla. O budove sa nedalo povedať že bola zvlášť luxusne zariadená, ani to, že by bola skromná. Bola veľmi vysoká, ale nie väčšia ako megadom v ktorom vo Wymyslensku bývala. Výťah bol len medzi niekoľkými poschodiami, na niektoré chýbal. Snažila sa zistiť, či v tom nie je zmyslová mágia, ale žena jej pokus o čarovanie pocítila a veľmi škaredo sa na ňu pozrela. Tu niečo nie je v poriadku, pomyslela si.

"$ $Mrana? Nie je tu Hlagenovka?"$ $ Dávala si pozor, aby svoju telepatickú správu dostatočne zabezpečila, i keď, keby tam bolo Hlagenovo pole, nemala by šancu.

"$ $To neviem. Je to vcelku možné. Nesledujem to.“

"$ $Ako môžeš? Nevieme kde sme.“

"$ $Bol to tvoj nápad. Nemala som čas zapnúť indikátor Hlagenovho poľa.“

"$ $Nemáš ho snáď zapnutý stále?“

"$ $Žerie to batériu a v pralese nebolo napájanie.“

"$ $Na koľko batéria ostáva?“

"$ $Dva dni. Nestíham to zapnúť.“

"$ $Dopekla."$ $ Nemohla sa ďalej sústrediť na telepatiu, nechcela vzbudiť ženinu pozornosť. Ešte si na niečo pomyslela a zatelepatizovala Loviise.

"$ $Nenápadne vyber Lasermeč. Vytesni si z mysle väčšinu myšlienok. Neodpovedaj mi. Je možné, že je tu Hlagenovo pole."$ $ Videla ako Loviisa vytiahla Lasermeč. Držala ho v ruke, ale tak, aby ho nebolo možné vidieť. Zdalo sa, že žena si nič nevšimla. Mrana, Sylvia i Loviisa stáli niekoľko krokov za ňou. Pre istotu. Žena ich každú chvíľu kontrolovala a Sylvii sa stále zdalo, že niečo nie je v poriadku.

Na trinástom poschodí vystúpili z výťahu. Poschodie bolo zariadené rovnako priemerne ako to dolné, v ktorom do budovy vstúpili. Žena ich viedla do miestnosti na druhom konci chodby. Od chvíle, keď prešli cez bránu mlčala. Keď došla ku dverám, nezaklopala len vyslovila nejaké slová v nejakom cudzom jazyku. Ani jedna z nich netušila o aký jazyk sa jedná, nieto ešte vedela čo žena hovorila. Sylvia si bola istá, že Tulienka Deľa ten jazyk poznala. Ale tá tam nebola.

Dvere sa otvorili. Miestnosť do ktorej vošli bola zariadená v duchu toho, čo už zatiaľ z budovy videli. Ukázala na kreslá vnútri a prehovorila.

"$ $Sadnite si. Hlavný súdruh prehovorí k vám."$ $ Po tejto vete Sylvii zatelepatizovala Mrana.

"$ $Kde to dopekla sme?"$ $ Sylvii mlčala. Odpoveď nevedela, ale zdalo sa jej, že v peknej kaši.

\begin{center}

*

\end{center}

"$ $To piktopísmo v knihe, Izabeta...“

"$ $Máš spôsob, ako ho prečítať?“

"$ $Ja som ten spôsob.“

"$ $Čože?“

"$ $Ja ho viem prečítať. Tam, vo víre...“

"$ $Nie si spoľahlivá.“

"$ $Nemáš inú možnosť.“

\begin{center}

*

\end{center}

"$ $Hlavný súdruh vás víta, súdružky."$ $ Prehovorila ku nim žena. "$ $Ja som súdružka Mei Tesocová. Loriatar vás víta."$ $ Vo vzduchu pred nimi sa naraz objavila obrazovka. V nej bol muž. Prehovoril ku nim.

"$ $Toto je Loriatar Ha'Blen, štát robotnícky, česť práci, súdružky."$ $ Sylvia došla ku tomu, že sú ešte vo väčšej kaši, ako si myslela pôvodne. Ha'Blen bol Fentenzíjsky výraz pre socialistický štát. "$ $V mene štátu a Ha'Blen musím vyjadriť vďaku vášmu poznaniu o zle imperializmu Fanasy. Budovanie socializmu v San'me't'i Te\v{}hr'en, siedmej dobe."$ $ Sylvia dúfala, že Mrana a hlavne Loviisa sú dostatočne vedomí si situácie, aby nepovedali nejakú blbosť.

"$ $Súdružka Tesocová vám vysvetlí pravidlá budovania siedmej doby. Ale varujem vás, keby ste boli nepriatelia, imperialisti, Loriatar nebude zhovievavý. Česť práci."$ $ Sylvia nedávala najavo nervozitu, strach a napätie. Kde sa to len dostali? Obrazovka sa vypla a prehovorila ku nim Mei Tesocová.

"$ $Vašu spôsobilosť budovania Loriataru posúdi komisia. Budete rozdelení podľa pridelenej pracovnej pozície. Všetky musíte prejsť povinnou vojenskou službou. O jej dĺžke rozhodne komisia. Vaše ubytovanie bude zabezpečené podľa pozície. Komisia vás očakáva."$ $ Pokynula im, aby ju nasledovali a mlčky kráčali späť do výťahu. Sylvia využila chvíľu jej nepozornosti a poslala Mrane a Loviise telepatickú správu.

"$ $Nekontaktujme sa. Mrana, vymysli niečo. Nemôžeme vedieť, či nás nesledujú.“

O Mraninej inteligencii nepochybovala, len Loviisu nepoznala tak dlho, aby si bola istá, že neurobí nejakú hlúposť. Takto, koniec koncov, nepoznala ani seba.

\begin{center}

*

\end{center}

"$ $Morja! Kde si sa dostala?“

"$ $Arabela ti to už predsa povedala?“

"$ $Myslím, že úplne všetko si jej nepovedala.“

"$ $Povedzme, že pár vecí, ktoré mi rada nenechala povedať.“

"$ $Tak."$ $ Prisvedčila Arabela.

"$ $A čo kniha?“

"$ $Nič viac ako som ti posielala správou. Kniha je záhadou, ale akonáhle získame piktopísmo...“

"$ $A prístup ku knihe."$ $ Dodala Bella. "$ $Agentúra ju len tak nedá nikomu z nás, s výnimkou Arabely, a obzvlášť potom, čo sa podarilo tebe, Niela."$ $ Niela mala kamennú tvár a premýšľala.

"$ $Pamätáš si celé piktopísmo?“

"$ $Dúfam. Začala som písať, tu to je."$ $ Morja ukázala na čiastočný slovník najstaršieho piktopísma, ktorého časť zatiaľ stihla napísať. Všetky tri si ho zo záujmom prehliadali. Duch Arabely Tlogenovej po chvíli prehovoril.

"$ $Musím ísť do agentúry, akonáhle to dopíšeš. A musíme vyriešiť tú nepríjemnosť s Goonovou..."$ $ Morja ju prerušila.

"$ $Čo je s mojou dcérou?“

\begin{center}

*

\end{center}

Komisia ich oddelila, akonáhle ku nej dorazili. Mala pred nimi nejakých ľudí, čo sa Sylvii i Mrane zdalo čudné, pretože... ak to neboli priamo z Loriataru... ku prameňu už dávno nikto... obe zahodili tú myšlienku, nezaoberali sa ňou. Súdružka Tesocová stála pri nich a očividne nedovoľovala žiaden slovný kontakt a o telepatický sa radšej nepokúšali. Komisia posudzovala jednotlivo. Najskôr zavolali dovnútra Sylviu. Pri otázke na meno sa nepredstavila svojím priezviskom, desila sa toho, čo by urobili, keby zistili kým je, a po tej nepríjemnosti s Wymyslenskou políciou nechcela riskovať. Verila v inteligenciu Mrany, že urobí to isté, ale u Loviisi sa spoliehala len na to, že pochopila, prečo ona nepoužila svoje meno. Nazvali ju súdružka Kauferová.

"$ $Česť práci súdružka,“

"$ $Česť práci,"$ $ odvetila Sylvia jedinú vec, čo jej nepadla.

"$ $Našou pracovnou náplňou je preveriť vaše znalosti a predpoklady a zadeliť vám prácu a dĺžku vojenského výcviku. Pre budovanie a prosperitu siedmej doby je to nevyhnutné. Klamstvá sú považované za imperialistické sprisahanie s cieľom zničenia siedmej doby."$ $ Siedmej doby... Sylvii sa zdalo, že to už niekedy počula. Pred Loriatarom.

"$ $Pravdivo odpovedzte na otázky v dotazníku..."$ $ Niekedy sa jej zdalo, že to je celá večnosť. Čakala, že ich budú chcieť rozdeliť od seba a tak svoje zameranie špecializovala na počítače s vierou, že Mrana to urobí rovnako. Nasledoval ideologický test, v ktorom si dala poriadne pozor, aby nevzbudila podozrenie. To bolo to posledné, čo chcela. Úprimne dúfala, že Loviisa vie aspoň trochu o socializme.

"$ $Vojenský výcvik?“

"$ $Znalosti v historických zbraniach, čiastočné ovládanie niektorých strelných zbraní."$ $ Neklamala. Toto sa jej mohlo hodiť.

"$ $Spresnite.“

"$ $Šerm a ovládanie všetkých Wymyslenských stupňov v boji v základnom a sekundárnom štúdiu.“

"$ $Viac spresniť.“

"$ $To sú meče, kordy, dýky, luk."$ $ Odvetila.

"$ $Porozumenie. Výsledky dostanete onedlho."$ $ Sylvia prikývla. Nemala čo k tomu dodať. Nebola tak žiadna praktická časť, aby si overili, že hovorí pravdu. Mierne sa bála toho, že sú na to schopný prísť inak. Nenadhodnocovala svoje schopnosti, len isté svoje záujmy zámerne neprezradila. Sadla si na jednu zo stoličiek, ktoré boli v miestnosti, kde sa nachádzala.

Po pár minútach z inej miestnosti vyšiel muž a prikázal jej, aby šla za ním. Vytvorila okolo seba obranu a šla. Nenechali ju čakať na Mranu ani Loviisu, zrejme ich chceli oddeliť... vedeli o vonkajšom svete a mohli by predstavovať hrozbu... na druhej strane, prečo im otvorili bránu? Ak predstavovali hrozbu, prečo... museli mať na to bezpochybne dôvod. Dosť dobrý aby vládnuce kruhy riskovali... je to zas kvôli nej? Nechcela si na to spomínať... nechcela... a desila sa, by sa to stalo zas. Nepripúšťala si to doteraz, ale čo ak... ako sa mohli dozvedieť, že... Wymyslensko? Majú vôbec v Loriatare nejaké spojenie? Podľa Sylviinho úsudku niečo také muselo existovať, inak by nezaregistrovali ich prítomnosť pri bráne. Teoreticky sa mohli dozvedieť kým je, ale to... ak to ešte nevedia musí byť obozretná, nesmie sa nejako prejaviť, najlepšie sa podceniť, aby ju oni podcenili... spriemerniť... v duchu sa zamračila. Nevedela byť priemerná.

Dostala ubytovaciu jednotku. Veci pri sebe mala zmenšené, ale bála sa, že o ne príde, a tak ich nevyťahovala. Prekontrolovala miestnosť. Nenašla žiadne kamery, ale človek nikdy nevie. Ubytovacia jednotka sa skladala z niekoľkých miestností, v ktorých našla posteľ, malú kuchynku, kúpeľňu, kreslo a pred ním televízor. Žiadny počítač. Prvá vec. Žiadne police. Druhá vec. Presne jedna malá skriňa. Tretia vec. Pevná linka s možnosťou vytočenia dvoch čísel – polície a nemocnice. Pekne. Skúsila telepatické pole. Fungovalo. Takže telepatia by bola s maximálnou ostrahou aj možná... ale čo hlagenovka? Indikátor hlagenovky mala len Mrana... Mohla by zakúsiť to riziko... Vlastne... ak tam je hlagenovka, tak potom... Už na tom nezíde. Prinajhoršom... nechcela na to myslieť. Zas porušiť... Rozmýšľala ako je na tom Mrana. A tiež, nadalo ju, vo Wymyslensku ešte nie je plnoletá, tak... možno ju test tak vyhodnotil, alebo tu plnoletosť nemajú... Čo tu vôbec robia neplnoletí, ak tu takí sú? Premýšľala. Nevedela, ako zaradia Loviisu, ona ju tam predsa vzala. Z úvah ju prebudila stena. Zasvietila sa a objavila sa textová správa.

Súdružka Kauferová, dostavte sa na základňu A3B v južnej časti Loriataru. Česť práci.

Sylvia si správu prečítala a premýšľala, kde tá základňa dopekla leží. Potrebujem mapu. Usúdila. Zrejme to je ten výcvik, premýšľala. Mala by som tam ísť.

Všetky svoje veci si uzavrela do čo najbezpečnejšieho magického obalu a vzala ich so sebou. Mala pri sebe kompas. Vybrala ho a hľadala Juh. Išla na tú stranu. Keď sa dostala do nejakého pásu polí, prišlo jej to čudné. Tam, kde sa ulica končila, boli nejaké tabule. Mapy. Sylvia ku nim podišla a zistila, že je v severnej časti Loriataru. Kompas neukazoval sever. Mierne sa zamračila. To je zvláštne. Potom si to nejako vysvetlila tým, že je to inopole a vybrala sa na opačnú stranu Loriataru.

\begin{center}

*

\end{center}

"$ $Ideš do mesta?"$ $ Spýtala sa Chen Rosy. "$ $Môžeš ísť."$ $ Rosa pokrčila plecami. Rozmýšľala nad Cheninou možnosťou. Už dávno nikde nebola a posledné dni strávila nad knihami. Trochu rozptýlenia by sa hodilo.

"$ $Ak to nebudú nákupy."$ $ Odvetila napokon. Nákupy zo srdca neznášala – ak sa, prirodzene netýkali nákupy kníh. Chen sa zasmiala.

"$ $To v žiadnom prípade.“

"$ $Tak sa ide?“

"$ $Hoď niečo na seba a o desať minút buď v chodbe. Nie, knihy neber. Je to kúsok. Pôjdeme peši.“

"$ $Tu je Spoločenstvo?"$ $ Spýtala sa Rosa. "$ $Myslela som si, že bez preukazu...“

"$ $Ten ideme vybaviť.“

\begin{center}

*

\end{center}

Sylvia skočila, urobila salto, dopadla na obe nohy, bežala ku otvoru v plote, spravila kotrmelec a prehupla sa cez dieru. Odrazila sa od steny a schmatla pušku na vrchu plotu. Prebehla cez pole, preskočila cez prekážku, v letku vytiahla pušku a začala strieľať. Zasiahla všetky terče presne do stredu. Zastala. Nebola vôbec zadýchaná, naopak, pokojne dýchala. Položila pušku na miesto a vrátila sa na začiatok. Pozrela sa na svoj čas a potom na súdružku Lerhanovú. Tá pokrútila hlavou. Sylviin pokojný výraz sa zmenil na výraz naštvaný. Podišla bližšie ku svojej nadriadenej a spýtala sa jej:

"$ $Čo odo mňa ešte chcete?! Mám lepší čas a úspešnosť ako deväťdesiat percent vašich dôstojníkov! Čo odo mňa ešte chcete?!"$ $ Súdružka Lerhanová ostala pokojná. A stručná.

"$ $Všetko. Chceme od vás úplne všetko, súdružka M... Kauferová..."$ $ Sylvia dúfala, že to M povedala len omylom. Ale to sa jej zdalo iba ako zbožné želanie. Vysvetľovalo by to ich honbu za tým, aby dosiahla čo najlepšie výsledky. Aby si nevšimli, kým je, podcenila svoje schopnosti, vrátane fyzických.

Démoni a polodémoni majú kvôli väčšej koncentrácii mágie v nich väčšiu magickú kapacitu, ako len ich genetická dispozícia, a ku tomu i lepšie fyzické schopnosti. Tieto by sa samozrejme museli trénovať, aby sa prejavili, ale napriek tomu boli aj prirodzene vysoké. Sylvia o tom niečo tušila, ale nijako zvlášť tomu nedávala význam, len ich trochu podceňovala, odkedy sa istý učiteľ v Kralove nazdával, že by mala ísť na profesionálny šport. Vtedy sa zhoršila, respektíve, prestala sa snažiť podávať čo najlepšie výkony. Až v Loriatare. Predpokladala, že to bude niečo alá agentúra, ale ukázalo sa to omnoho, omnoho náročnejšie.

Vojenský výcvik sa v jej prípade začal tým, čo trénovala teraz. Na rozdiel od ľudí na Zemi (nie v spoločenstve) sa využívala vo výcviku hodne aj mágia. Pri jej správnom použití sa dá lietať (ale len s naozaj obrovskou magickou kapacitou), čo sa využívalo najmä pri pohybových činnostiach využívajúcich beh, skoky a pod. Boj mágiou bol už veľkou samozrejmosťou vo všetkých štátoch a rovnako maskovanie. Vo Wymyslensku bola podmienka na vstup do armády určitá magická kapacita. V Loriatare to tak nebolo, ale určitá magická kapacita bola určite nutná na schopnosť absolvovať výcvik. Loriatar nútil sa zlepšovať, až na hranice možností, ktoré v prípade Sylvie ešte neboli ani načaté. Ani to nemala v úmysle. Loriatar bol dosť dômyselný aj v tom, že zahrať to, to, že niečo je už nezvládnuteľné, bolo prakticky nemožné. Sylvia sa o zmyslovú mágiu nepokúšala – súdružka Lerhanová iste nemala pozíciu veliteľky len tak. Bola pevne rozhodnutá dostať z nej všetko. Po jej prvých výsledkoch donútila Sylviu skúšku opakovať, až kým nedosiahla štandardný výsledok, ako po niekoľkých rokoch tréningu. Sylvia bola preto na seba poriadne napálená a rozhodla sa, že bude čo najhoršia. Najskôr sa jej aj podarilo ostávať na rovnakých výsledkoch, potom nastúpila veliteľka. Loriatar sa pre ňu stal horším a tvrdším než Wymyslensko a nebolo chvíle, kedy nepremýšľala, čo je s ostatnými. Dúfala, že sú v bezpečí, keďže Loviisa nejako zvlášť výnimočná podľa Sylvie nebola a Mranu iste chcú využiť na niečo užitočnejšie ako na armádu. Nestretla sa s nimi od príchodu a dosť jej chýbali. Neodvážila sa poslať Mrane telepatickú správu a Mrana očividne rovnako. Keď jej veľmi chýbali vyhľadávala si ich v telepatickom poli. Boli tam. Našťastie. Jej život sa zmenil na súbeh skúšok, trestov, strachu a nenávisti. Bála sa toho, čo by jej mohli urobiť, keby vedeli kým je a nenávidela sa za to, kým je.

Jej najlepší čas dosiahol hranicu možností bežného človeka s jej magickou kapacitou, ako vypočítala. Nesmie ísť ďalej. Mohli by zistiť jej geneticky danú magickú kapacitu a porovnať ju a zistili by to... nesmie ísť ďalej, nesmie, radšej bude znášať tresty. Radšej rozhnevám čo i len celý Loriatar ako sa odhaliť, pomyslela si zas.

Z jej hnevu ju prebrala telepatická správa. Zrejme ďalšia skúška. Pomyslela si. Po prezretí si správy sa ale potešila. Nebola to súdružka Lerhanová. Mrana sa ozvala. Správa bola dosť nezrozumiteľná. Vyzerala ako text v Teresovčine. Veľmi zlej Teresovčine.

"$ $OU'AG NaD 7L MoSYSeG AWheWS'TeTanQeveQ C‘C VeHeng M1 WaiN V1 UniAle L0 ArtOnUthYNIG'Z ‘V CaT 52 AlBa Hegley C3 Wymelen Y9 ZaiTeZeV'Q'ZeMha B7 ILe'Q'ZhaeW NeMeWymylen SUtheran 43 Wa'Q'thev 10 P.S.: Hlavný súdruh vás víta.“

Chvíľu sa na správu pozerala, potom si ju prepísala a došlo jej to. Zdalo sa jej to jednoduché. Rozlúštila Mranin odkaz a začala premýšľať ako sa má preboha dostať do T6 o 12:00 nasledujúci deň?!

\chapter{Brány Svetov}

"$ $Kam ma to dopekla vediete?!“

"$ $Tam kde treba."$ $ Odblikali stručne bytosti. Pauline bola skutočne vydesená, netušila čo je s jej priateľmi, netušila čo sa môže diať, ani čo by mala robiť. Akosi sa nedala používať mágia, tak na to rezignovala . Jediné, čo ju upokojovalo od paniky, bolo, že v najhoršej situácii sa premiestni. Nevedela však, či sa to bude dať. Z Quertu sa dalo premiestňovať, ale z celého? Nebola spútaná, ale nemohla ísť inam, ako chceli Querťania. Štvalo ju to. V žiadnom prípade nechcela zomrieť. Prečo len dopekla neostala doma? Netušila do čoho sa zapletie, akonáhle s nimi odíde. Už je druhý raz v smrteľnom nebezpečenstve za pár dní! Teda, netušila, ako dlho môže byť na Querte. Vo väzení sa nedali sledovať dni ani noci a nemala absolútne žiadne tušenie, ako dlho tam bola. Keď ju viedli von, bála sa, že svetlo po toľkej temnote by jej mohlo ublížiť, ale očividne nie. Videla normálne, ako človek ktorý uprostred noci zasvieti.

Kráčala po Quertskom kryštalickom chodníku až ku nepriehľadnému múru. Vtedy začala blikať jedna z nich.

"$ $Tu je vchod. Zamorená oblasť. Ste mágovia, ste skúška dekontaminácie."$ $ To bolo všetko. Vyhodili ju za múr. Okamžite pocítila obrovskú dávku mágie. Než stihla čokoľvek urobiť, premiestniť sa, ju taká obrovská dávka mágie omráčila.

Prebrala sa. Myslela som si, že posmrtný život neexistuje. Pomyslela si. Takú dávku mágie predsa nemohla prežiť.

"$ $Neotváraj oči. Odstránim ti z nich mágiu, len počkaj."$ $ Počula hlas Tarnyho. Tak aj on je mŕtvy. Zabolelo ju to.

"$ $Máš šťastie, že si tú dávku prežila. Predsa len, polodémoni majú tuhší korienok ako my, obyčajný zovero."$ $ Nechápala tomu. Prečo hovoril, že prežila, keď zomrela? Je to nejaká zvláštna realita, alebo... naozaj prežila?

"$ $Žijem?“

"$ $Hej, ale len o chlp. Bál som sa, že ťa to zabije. Prečo si si dopekla nevyčarila štít?“

"$ $Aký štít? Myslíš obranu?“

"$ $Nie, štít z mágie, zabraňujúci prejsť mágii zaň.“

"$ $Prvý raz som o ňom počula teraz."$ $ Stále si nebola istá, či ozaj nie je mŕtva, ale nepoznala spôsob, ako na to prísť a tak zatiaľ rátala s tým, že nie je.

"$ $Máš zanedbané vzdelanie. Mohlo ťa to zabiť.“

"$ $Čo! V tomto tu prečudesnom svete som len pár dní.“

"$ $Pár týždňov. Nie pár dní. Báli sme sa s Tulie, že ťa nepustia a nebudeme sa odtiaľto mať ako dostať.“

"$ $Hej! Takže ma berieš len ako dopravný prostriedok!"$ $ Vedomie, že bola na Querte pár týždňov ju zaskočilo len málo, bola na to pripravená. Viac ju štvala Tarnyho mienka o nej.

"$ $To som nepovedal.“

"$ $Ale povedal! Či telepatizoval! Presne to! Báli ste sa, že sa odtiaľto nebudete mať ako dostať. A ja hlúpa som sa z väzenia nepremiestnila, lebo som vás nechcela opustiť. A ty... Mal si o mne väčšiu mienku kým som nebola polodémonka.“

"$ $Chcem ťa upozorniť, že tou si bola od narodenia...“

"$ $Myslím kým si to nevedel! Bože, Tarny, ty si strašný debil!“

"$ $Mám ťa prestať liečiť!? Zomrieš!“

"$ $To zas nie!“

"$ $Ani by som to neurobil.“

"$ $Tak sa mi nevyhrážaj. Ani ma neurážaj.“

"$ $Ja som ťa neurážal. Povedal som fakt. Bez teba sa nemáme ako dostať späť. Ibaže by sme našli to, čo od nás chcú Querťania...“

"$ $Dekontamináciu?“

"$ $O tú im nejde. Je pravda, že toto je čistá mágia a pre Querťanov je ešte nebezpečnejšia ako pre ľudí a zovero, ale im ide o niečo úplne iné. To som si uvedomil hneď.“

"$ $Tebe povedali viac ako mne?“

"$ $Čo ti povedali, keď ťa vytiahli z väzenia?“

"$ $O tom že je tam zamorená oblasť a že my, taj ja a vy sme mágovia a dekontaminovať.“

"$ $Tak pri mne boli oveľa zhovorčivejší.“

"$ $Ako oveľa?“

"$ $No... Ide o to, že milá Leana z Ölverína bola dosť sviňa.“

"$ $Čo?“

"$ $No... ona vedela, že sem je Brána, a...“

"$ $Brána?“

"$ $Myslel som si, že toho o inopoliach vieš dosť. Brána je brána do nejakého poľa alebo inopoľa. A zdá sa mi, že Leana sa na Quert nedostala náhodne. A D o Querte vie, lebo tu už minimálne raz bol. Leana tu niečo chcela, a nezdá sa mi, že by sa o tej bráne dozvedela na Querte. Otázne je, načo by jej tá brána bola. Zrejme to je nejaké dôležité inopole, veľmi dôležité inopole... Toto mi vŕta hlavou. A tiež, prečo Quert? A prečo odišla s nami? Prečo, Pauline?!“

"$ $Vzdala to?“

"$ $Querťania to nevzdali, to je... ako to povedať... čudné. Ak by to ona vzdala... a nepripadala mi ako človek, ktorý sa vzdal. Možno to zakrývala, ale... Proste... niečo tu nesedí... Querťania chceli dekontamináciu, aby sa ku tomu dostali... Ale, potom by to znamenalo, že Leana to už mala, ale kontaminovala, alebo sa to proste kontaminovalo mágiou z toho inopoľa, možno magická nerovnováha... ja neviem. Nenapadá ťa niečo? Ale prečo potom nič nespravili matke, prečo Pauline, dopekla?“

"$ $Tarny, máš priveľkú fantáziu...“

"$ $Nie, Pauline... všetko tomu nasvedčuje, všetko! Sme súčasťou niečoho väčšieho, to je tak... desivé a fascinujúce zároveň. Chápeš to? Ja... vidím to... proste niekto sa tu snaží niečo získať a my sme do toho padli. Brány... ak ich nájdeme a zistíme prečo ich Leana chcela... môžeme... môžeme... to je tak fascinujúce, Pauline... mimochodom, máš už okolo seba štít, ale nepoužívaj mágiu, radšej nie. Môžeš otvoriť oči. Ale nerozprávaj, neviem, či sa táto obrovská koncentrácia mágie neporuší, a tak radšej telepatiu. A to je tak... fascinujúce, tie brány... to...“

Pauline sa pokúšala Tarnyho uzemniť, ale on stále hovoril niečo o Bránach, Leane a veľkých udalostiach, ako ich nazýval. Chcela zmeniť tému.

"$ $Tulienka Deľa tu je?“

"$ $Hej. Už som to spomínal. Momentálne skúma istý magický vír.“

"$ $Takže odtiaľto môžeme vypadnúť. Respektíve, pred tým, ako sme sa do tohto namočili sme chceli niečo s knihou.“

"$ $Toto môže byť dôležitejšie ako kniha osudu! Toto! Bože, uvedomuješ si to! Je to niečo mimoriadne dôležité, celkom iste, kto by bol inak schopný obetovať tisícky rokov svojho života, sakra!“

"$ $Vy nechcete odísť?“

"$ $Prirodzene chceme, ale prioritná je brána. Brána! Toto je... záhadné...“

"$ $Práve to. Nakoniec to bude bezvýznamný svet.“

"$ $Všetko má význam.“

"$ $Nič nemá význam!“

"$ $Ale má!“

"$ $Nie si predsa veriaci!“

"$ $To nie som, ale...“

"$ $Všetko je NIČ, nič, Tarny, chápeš?!“

"$ $Super. A to nič je tak dôležité...“

"$ $Choď do kelu.“

"$ $Žiadny tu nie je.“

"$ $Bože, choď do pekla s tým! Sme na Querte, kde sa nás pokúsili zabiť, D možno už získal knihu, nik na zemi nevie kde sme. To je tak úžasné!“

"$ $Presne tak."$ $ Prisvedčil Tarny. Na rozdiel od Pauline to nemyslel sarkasticky.

"$ $Tarny!“

"$ $Tarny!"$ $ Naraz mu zatelepatizovala aj Pauline aj Tulienka Deľa.

"$ $Tulie! Čo sa stalo?!“

"$ $Mám anomáliu. To ťa bude zaujímať.“

"$ $Bežím. Inak, Pauline je tu.“

"$ $Viem. Všimla som si ju v telepatickom poli.“

"$ $Super. Prídeme obaja."$ $ A potom zatelepatizoval Pauline. "$ $Tulie niečo spozorovala. Poď."$ $ Nevšímal si to, že Pauline sa ho stále pokúšala prehovoriť, aby s tým prestali a premiestnila sa späť na zem. Tarny už bežal za Tulienkou a tak Pauline šla za ním.

Tulienka Deľa stála pri zvláštnom magickom víre. Všetkých troch spoľahlivo chránili štíty. Ukázala naň. Tarny sa nedočkavo, ale opatrne doňho pozrel. To, čo videl predčilo i jeho najtajnejšie predstavy. Bola to brána. Za blanou medzi poliami sa skrýval svet.

Bol vcelku odlišný od ich bežnej reality, vrátane Quertu. Po prvé, bol celý, úplne celý z nejakého pračudesného materiálu, ktorý sa nepodobal na nič, čo keby videli.

"$ $Ideme?“

"$ $Tarny, neblázni! Nevieš, čo nás tam čaká!“

"$ $Práve preto."$ $ Zasmiala sa Tulienka Deľa.

"$ $Ste blázni!“

"$ $Možno."$ $ Odvetí Tarny a pokračuje. "$ $Všetci budeme mať štíty aj ochranu. Nevieme o tamňajšom prostredí nič.“

"$ $Veď práve. Zomrieme. Všetci zomrieme."$ $ Tragicky predpovedala Pauline.

"$ $Až tak zlé to nebude. Nepremiestňujeme sa tam, takže by nás to malo vypľuť pri bráne späť, nie je to tak?“

"$ $Ale je... Ale, čo keď tam budú podmienky, ktoré spôsobia, že sa dostaneš preč od brány?“

"$ $Tak vieš, kde...“

"$ $Ak sa tam nebudeš mať ako dostať?! HA?!“

"$ $Vždy sa dá tam ako dostať. Použijeme obranu a...“

"$ $A keď tam nebude mágia fungovať!“

"$ $Myslíš si, že som taký debil? Uzavrel som mágiu do niekoľkých predmetov a naprogramoval tak, že nám budú vytvárať obranu, kyslíkové pole a iné, životu dôležité podmienky po dobu pár dní.“

"$ $Aj tak! Nejdem zas riskovať život! Toho mám až po krk!“

"$ $Tak si tu ostaň!“

"$ $To nechcem! Nemôžem sa sama vrátiť do sveta ľudí, až príliš sa bojím. Po tomto nie. A magický svet. Nemôžem, Tarny!“

"$ $To riziko to robí... zaujímavším... prečo si potom odišla?“

"$ $Zo sveta ľudí? Lebo som nemohla... sú až príliš odlišní! A vy tiež! To všetci mágovia sú takýto?“

"$ $Myslíš úžasní, dokonalí a geniálni?“

"$ $Nie! Myslím šibnutí, blázni a riskujúci život na každej druhej ulici!“

"$ $Ach tak... ani nie. To len celá agentúra, Tulienka Deľa, ja, Sylvia, naši rodičia a potom kopa ďalších... ale vcelku dosť z nás je čudných ako ty.“

"$ $A ty si normálny?!“

"$ $Presne tak.“

"$ $Choď do pekla, Tarny! Si blázon! Chceš sa zabiť?! Nelez tam, kde nemáš! Nechápem prečo... nemusíme zachraňovať svet, preboha! Sme deti, Tarny! Neviem, či si to uvedomuješ?!“

"$ $Ani nie.“

"$ $Tak by si mal začať! Dopekla Tarny! Máš len štrnásť...“

"$ $V prepočte na ľudské roky.“

"$ $Čo?“

"$ $V celých Wymyslenských len osem.“

"$ $Čo? Zmiatol si ma.“

"$ $No... vývoj zovero je zrýchlený, ale to je na dlhšiu debatu... ideme.“

"$ $To v žiadnom prípade!“

"$ $Tak si tu ostaň!“

"$ $Hej! Myslí, že by sme sa nemali hádať. Aká vážne...“

"$ $Tulienka Deľa, vysvetli Pauline, že...“

"$ $Mne nič netreba vysvetľovať, Tarny! Vy sa idete zabiť! Choďte si tam, dopekla! Budete mi chýbať. Pôjdem na Zem nahlásiť vaše úmrtia tvojej matke, nech sa nenádeja, že prídeš!“

"$ $Tak prosím... Ideme Tulienka Deľa?“

"$ $Pauline, vážne?"$ $ Spýtala sa Tulienka Deľa. Nechcela, aby tam Pauline ostala.

"$ $Presne tak. Nemám záujem zomrieť. Koľko vydrží štít?“

"$ $Asi dvadsaťštyri hodín.“

"$ $Máte hodinu na návrat. Inak odchádzam.“

"$ $Máš hodinky?“

"$ $Budem počítaj!“

"$ $Tak počítaj. My po teba prídeme.“

"$ $Budem čakať!“

"$ $Pauline, si si celkom...“

"$ $Hej! Zbohom Tulienka Deľa! Zbohom Tarny! Budete mi chýbať."$ $ Tulienka Deľa si všimla, že mala slzy v očiach a tak prišla ku nej a objala ju.

"$ $My prídeme. Vrátime sa po teba. O menej než hodinu, nech nemusíš veľa počítať...“

"$ $Choď do kelu."$ $ Povedala cez slzy Pauline. Uvedomovala si, že ju možno vodí naposledy. Potom objala i Tarnyho. Pritom mu, prirodzene, vynadala, aký je on debil, a že dúfa, že sa vrátia.

"$ $Neviem, či si si to všimol, ale mne na tebe záleží, Tarny.“

"$ $Ale ja ti verím. A rovnako mne záleží na tebe. Ako by sme sa bez teba odtiaľto...“

"$ $Tarny... prestaň. Neviem, či si si to už uvedomil, ale týmto Pauline urážaš.“

"$ $Samo...“

"$ $Mne sa zdá, že nie.“

"$ $Nemusíte sa hádať teraz.“

"$ $Nemusí ma znevažovať teraz.“

"$ $Nemusíš ma obviňovať teraz.“

"$ $Môžete s tým prestať?!"$ $ Tulienka Deľa bola rázna.

"$ $Znevažuje ma!“

"$ $Obviňuje ma!“

"$ $Stop!“

"$ $Tulienka Deľa...“

"$ $Tarny! Pauline! Mohli by sme sa porozprávať ako normálny ľudia a zovero?!“

"$ $To nemôžem zaručiť! Stále leziete na smrť.“

"$ $Ani raz nás nedostihla.“

"$ $Raz sa to stane!“

"$ $Ale nie teraz!“

"$ $Stop! Takže, o čo tu ide? Pauline nechce ísť s nami, rešpektujme to. Dáva nám ultimátum, ktoré sa pokúsime splniť. Tarny, svoje poznámky si odpusti a Pauline tiež.“

"$ $To sa ti ľahko povie! Stále ma ponižuje!“

"$ $Bože... ako vy dvaja ste sa spriatelili?“

"$ $Predtým, ako zistil, že som polodémonka bol iný! Teraz ma berie ako by som bola dopravný prostriedok.“

"$ $Keď..."$ $ Začal Tarny, ale Tulienka Deľa ho zastavila.

"$ $Prestaňme s tým. Pauline tu hodinu počká, tu máš hodinky, jedny ešte mám. My máme hodinu na preskúmanie životných podmienok a vrátime sa. Potom, ak zistíme, že je to bezpečné, pôjdeš s nami?“

"$ $Daj mi dôkaz.“

"$ $Telepatické video stačí?“

"$ $Čo to je?“

"$ $Niečo ako telepatická správa, len obraz."$ $ Odvetila Tulienka Deľa a poslala jej ho. Pauline prikývla. Tulienka Deľa ešte raz zopakovala dohodu. I na tú Pauline prikývla. Neochotne, rovnako ako Tarny.

Tak sa Tulienka Deľa s Tarnym vybrali do víru. Pauline mala slzy v očiach, pretože naozaj verila, že sa už nikdy neuvidia. Kývala im, ale oni jej nie. V tvárach mali, že naozaj veria tomu, že sa po ňu vrátia. Pozrela sa na hodinky, ktoré jej dala Tulienka Deľa. Do hodiny sa vrátia, inak budú mŕtvy. Bude to hodina strachu, smútku a čakania. Tak to očakávala.

\begin{center}
*
\end{center}

"$ $Teraz sa môžeš legálne pohybovať po úplne celom spoločenstve. Má rovnaký rozsah ako môj."$ $ Povedala jej Chen. Rosa prikývla. Chen jej povedala len to, čo predpokladala. Preukaz dostala hneď po fotení a zistení základných údajov. Keďže nemala pri sebe žiadny iný preukaz totožnosti, bola tkz. "$ $zaručená"$ $ cez Chen. Emčan za okienkom sa veľmi čudoval pri mene Tlogenová – Goonová. Rosu to nejako zvlášť netrápila, bolo podľa nej málo pravdepodobné, že by bola nejakou príbuznou Izabety Tlogenovej. Zistila pri vybavovaní niekoľko zvláštnych vecí, v ktorých bolo spoločenstvo diametrálne odlišné od Wymyslenska. Napríklad, Chen nebola jej zákonný zástupca – vôbec žiadneho nemala. V spoločenstve M síce ešte nebola plnoletá, ale svojprávna. Takže mohla úrady vybavovať sama, mohla teoreticky aj bývať sama, len nemohla pracovať, len brigádovať a musela sa vzdelávať. To taktiež riešila Chen.

Po úspešnom získaní preukazu sa vybrali na večeru, a po nej, ako Chen Rose sľúbila, išli do verejnej tréningovej miestnosti skúšať boj. Rose sa mierne zdalo, že Chen ju šetrí. Rosa prvý raz v živote držala niečo ako normálny meč, síce odľahčený, ale aj tak bol dosť ťažký. Teoreticky mala nejaké základy boja naštudované, ale len teoreticky a prax sa ukazovala odlišná.

"$ $Musíš sa naučiť pracovať v boji s mágiou. Chápeš? Skočíš, aby si niekoho zasiahla, ale vieš, že len tak fyzicky je to nemožné a tak použiješ mágiu a skočíš. Použiješ vzduch, zhustíš ho a zabráni ti spadnúť, použiješ ho ako palivo. Využi mágiu, zvýš silu úderu, použi zmyslovú mágiu, presvedč súpera, že robíš niečo iné. Ale, on to môže použiť voči tebe. Chápeš? Musíš s tým rátať. Použi mágiu, vyčar štít a zruš súperov. Použi telepatický útok a zablokuj si myseľ voči súperovmu. Ak máš dobrú mágiu, vyhrať v súboji dokážeš, aj keď si úplne krepá. To je dôležité. O tomto je magický boj, o mágii, nie o fyzických zručnostiach, tie len posilňuješ a znásobuješ. Rozumieš tomu, čo som ti povedala?“

"$ $Tak hej. Niečo o používaní mágie som už čítala, aj som si myslela, že to bude tak kompletné. Ale v knihe skôr bolo aj o úderoch a tak...“

"$ $To prirodzene tiež. Musíš sa naučiť aj nejaké ľudské zručnosti, ak nebojuješ len mágiou. Napríklad mieriť, pohybovať sa, dávať pozor na súpera...“

"$ $Pozerať sa do očí, nie na meč. Meč si odľahčiť mágiu, nebyť obmedzovaný...“

"$ $Ale mágiu si šetri na súboj. Každý, aj ty máš obmedzenú magickú kapacitu, teda koľko mágie môžeš udržať za sekundu. Túto nedokážeš rozšíriť. Môžeš len vytvoriť niečo... ale to je dosť zložité. Zatiaľ sa sústreď na boj, ako samotný. Bez mágie. Tú tam zapracujeme neskôr. Je dôležité sa najskôr naučiť bojovať a potom silu znásobiť.“

"$ $Rozumiem, Chen.“

"$ $Tak znovu."$ $ Chen Rosu vždy porazila, ale Rosa sa preto nijako zvlášť netrápila. Nikdy predtým nešermovala, a tak fakt, že Chen párkrát prekvapila bol pre ňu úspech.

Strávili tam vyše tri hodiny, vonku bola už dlho tma. V koncoch času v tréningovej miestnosti sa ku slovu dostala i mágia a bojovali so štítmi. Toto išlo Rose lepšie, štít dokonca niekedy udržala dlhšie než Chen, čo považovala za dosť dobré. I keď, zdalo sa jej, že Chen nebojuje najlepšie ako vie. Vtedy by nemala šancu.

\begin{center}
*
\end{center}

O asi pätnástich minútach sa vynorila z brány Tulienka Deľa a zatelepatizovala Pauline.

"$ $Pauline! Ja tam bezpečne, žijeme, dýchať sa tam dá, proste svet ako na zemi. Aj mágia funguje.“

"$ $Tak skoro?“

"$ $No... Tarny ťa chcel troliť, ale nejako som mu to zatrhla.“

"$ $Tak díky.“

"$ $Tak poď.“

Pauline musela ešte Tulienka Deľa dosť dlho prehovárať, aby konečne vliezla do brány, ale napokon sa jej to podarilo. Pauline vošla do brány, chvíľu sa spolu s Tulienkou Deľou nachádzali v zvláštnej hmote, ktorá im ale nijako nevadilo až ich do vyhodilo do sveta, ktorý už Tulienka Deľa poznala. Čakal ich tam Tarny.

"$ $Trvalo vám to dlhšie, než sme tu doteraz boli. Kde ste sa flákali, ľudia?"$ $ Tulienka Deľa ukázala na Pauline s výrazom, že ona s tým nemá nič spoločné.

"$ $Tu slečne sa nechcelo.“

"$ $Sklapni Tulie.“

"$ $Tulienka Deľa.“

"$ $Tulie, no tak...“

"$ $Sklapni Tarn.“

"$ $Mohli sme tu ešte chvíľu ostať a váš čas na povrchu by sa aj rovnal tomu, ako tu sme.“

"$ $Si drzý, Tarny, ak to ešte nevieš."$ $ Ozvala sa Pauline.

"$ $Môžeme to nechať tak?"$ $ Podráždene sa spýtala Tulienka Deľa. Pauline až vtedy začala vnímať svet, v ktorom sa ocitla. Okolo nich bola vcelku krásna krajina, okolo nich hory a oni stáli na chodníku, ktorý viedol okolo kopca. V strede chodníka bol vír. Bolo cezeň výrazne vidno pole, na ktoré bola Pauline zvyknutá. Živočíchy v inopoli sa neponášali na pozemské, ale neubližovali im, ani od nich neutekali. Ľudí nepoznali. Očividne. Alebo boli skrotené, ale to sa im nezdalo.

"$ $Kde ideme?"$ $ Spýtala sa Tulienka Deľa.

"$ $No... mapu nemáme, ale stojíme na chodníku a tak by som šiel po ňom."$ $ Odvetil Tarny.

"$ $Vy chcete...?“

"$ $Samozrejme. Keď už sme tu...“

"$ $Tarny!“

"$ $A čo si si myslela!“

"$ $Vy ste... príšerní!“

"$ $Vieš Pauline... nemám rád normálny život. Chodíš do školy, niekedy na výlet, ideš na súťaž, vyhráš súťaž, ideš na koncert, chodíš na brigádu, potom s rodičmi na dovolenku, následne na letnú školu, tábor, vrátim sa, oslavujem narodeniny, idem do školy, sú Vianoce... to je tak nudné Pauline. Ja by som sa scvokol. Reálne by som asi dostal depresiu, nezvládol by som to.“

"$ $Nie si normálny. V mojom živote toho bolo ešte menej ako v tvojej predstave normálnosti.“

"$ $To čo si robila celý život? Ako je možné žiť bez smrteľného nebezpečenstva, porušovania zákonov a výletov po svete?“

"$ $Zabúdaš na to, kto...“

"$ $Jegrigsen... chápem. Nemala si príležitosť... ale teraz, teraz...“

"$ $Teraz čo! Tarny, neber to zle, si super, ale... nie každý miluje život aký vedieš ty.“

"$ $Ty si odišla.“

"$ $Vyčítaš mi to?!“

"$ $Vôbec nie!“

"$ $Tak čo robíš! Pekla! Tarny!“

"$ $A čo chceš teda robiť ty?! Máme jedinečnú šancu toto preskúmať, ja to nebudem.. nechcem zahadzovať, to by som si do konca života vyčítal!“

"$ $Preboha! To si budeš, ak tu zomrieme!“

"$ $Neviď všetko čierne! Môžeš sa premiestniť, kľudne to skús, len preboha...“

"$ $Berieš ma ako záťaž, Tarn!“

"$ $Berieš ma ako blázna, Pauline."$ $ Vrátil jej to.

"$ $A čím si!“

"$ $Prestaňte, konečne!"$ $ Zasiahla Tulienka Deľa. "$ $Pauline, dá sa tu premiestňovať?"$ $ Pauline to skúsila a prikývla. Tulienka Deľa pokračovala. "$ $Vydáme so k tej doline vpravo, zásoby jedla a vody nemáme takže uvidíme, koľko dní bude naša cesta trvať. Mapu si budeme priebežne kresliť, aby sme sa nestratili. Pri akejkoľvek zmienke nebezpečenstva aktivujte obranu. Nikto z nás nemá záujem, aby sa druhému a jemu samému niečo vážne stalo. Pri vyššej koncentrácii mágie a automaticky nám zapnú štíty. Ideme?“

\begin{center}
*
\end{center}

"$ $Skočte, súdružka Kauferová.“

"$ $Zabijem sa."$ $ Sylvia bola nervózna, pretože sa blížil čas dohodnutého stretnutia s Mranou, ale stále nevedela, ako sa dostať spod vplyvu súdružky Lerhanovej. Sylvia bola presvedčená, že ak Loriatar vie, akú silu majú Démoni, vedia, že je démonka, alebo polodémonka. Jej nadriadená ju v predchádzajúci deň donútila dať trasu pod desať sekúnd, takže suverénne držala rekord o najlepší čas. Vôbec ju to netešilo. Priemerný čas pri jej magickej kapacite bol pred jej rekordom tridsať sekúnd, pričom boli testovaný len traja. Lerhanová nepoužívala len psychický a fyzický nátlak, ale doslova ju zmanipulovala, aby to urobila. Pod desať sekúnd trasu... Vedeli to. Potrebovala odtiaľ zmiznúť, ale nedalo sa jej. Strážili ju. Premýšľala, čo s ňou chcú. Bolo možné, že nevedia, že existuje spojenie medzi ňou a D. Ak by to bolo tak, bolo by to aspoň v niečom pozitívne. Zato, objavil by sa ďalší problém. To, že ju priali do armády. Najtvrdší výcvik pod dozorom samotnej súdružky Lerhanovej. Nútenie ísť do najnižšieho času. Oddelenie od Mrany a Loviisi. Loriatar ju chcel ovládnuť. Chceli z nej spraviť vojaka. Stroj na zabíjanie. Otroka. Potrebovali ju. A toho sa desila viac, ako prvej možnosti.

"$ $Skočte, súdružka.“

"$ $Nie.“

"$ $Skočte."$ $ Súdružka Lerhanová bola ako vždy pokojná. Sylvia sa jej chladného pokoja desila. Pripadal jej, ako keby jej hovoril, my už všetko vieme a váš boj je márny. My dosiahneme, čo Loriatar chce. Sylvia stále pred priepasťou a odmietala skočiť. Potrebovala odísť, ale nevedela, ako sa dostane zo základne. Výcvik trval zatiaľ len pár dní, ale zdalo sa jej ako celá večnosť. Celý Loriatar jej začínal pripadať horší než Spoločenstvo M a Wymyslensko dokopy, a to bolo čo, aspoň podľa Sylvie, povedať. Chcela odtiaľ ujsť, ale to by nespravila bez Mrany a Loviisi. Premiestniť sa odmietala. Zobrali z nej už všetko, ale Loriatar jej nevezme dôstojnosť. Pokrútila hlavou.

"$ $Neskočím, súdružka Lerhanová.“

"$ $Skočíte, a vy to viete."$ $ Na tvári mala pokojný, sebaistý úsmev, ktorý v Sylvii vyvolával hnev. Zobrali jej všetko čím bola a pokúšajú sa ju využiť, ale nenechá sa..

"$ $Skočte, súdružka Kauferová. Vy chcete skočiť, súdružka. Skočte."$ $ Stále neprejavovala známky netrpezlivosti.

"$ $Odmietam zomrieť. Koľko ľudí a zovero na tejto priepasti skončilo?“

"$ $To nie je pre vás, súdružka Kauferová. Skočíte.“

"$ $Nie som samovrah."$ $ Odvetila Sylvia. Teraz naozaj klamala, pretože o samovraždu sa neúspešne už veľakrát pokúšala.

"$ $Povedzme, že nie. A preto použijete mágiu. Dáte do toho všetko."$ $ Sylvia navonok potlačila zlosť. O to im išlo! Pochopili, že dokáže viac a využiť pud sebazáchovy. Ale ja predsa nemôžem zomrieť... Pomyslela si Sylvia. Ale vtedy by to zistili... musí sa tváriť, nie musí čarovať...

Bol to veľký risk, ale keďže Mrana riskovala tiež, bolo to niečo dôležité.

\begin{center}
*
\end{center} 

Po prvej polhodine mimoriadne tichej cesty sa spýtala Pauline:

"$ $Ako ste vlastne prežili bez jedla v mágii?“

"$ $Mágia. Je to energia a bolo jej tam dosť. E sa rovná m c na druhú."$ $ Odvetil Tarny. Kráčali po jednotvárnom chodníku. Boli v zástupe. Najskôr šiel Tarny, za ním Pauline a napokon Tulienka Deľa.

"$ $Teraz to nie je možné?“

"$ $Nájdi dostatok mágie."$ $ Odvetil Tarny. "$ $Nemáš ho.“

"$ $Dosky."$ $ Odvetila Pauline. Niečo jej napadlo.

"$ $Čo?“

"$ $Dosky s mágiou. Aj tak nám na nič nie sú. Priťahujú len D.“

"$ $Je to sila. Sú ešte ďalšie.“

"$ $Čo to má spoločné. Takto sa budeme musieť vrátiť a sem je zatiaľ bezpečne a nie je tu D.“

"$ $Pauline, čo ti je?“

"$ $Nič, len je tu pokoj a nie je tu D a nikto sa nás nesnaží zabiť. Nechápem, čo tu chceli Querťania.“

"$ $Zatiaľ. Len zatiaľ."$ $ Odvetila Tulienka Deľa a cez ňu Tarny povedal:

"$ $Čo sa ti stalo, Pauline? Je ti niečo, alebo sme ťa nakazili?"$ $ Pauline nechápala.

"$ $Čo?“

"$ $Že si sa nejako po chvíli cesty zmenila.“

"$ $Ak narážaš na našu hádku, ani nie. Je to tu proste zatiaľ istejšie ako na Zemi, kde nikam nepatrím a každý sa nás snaží zabiť.“

"$ $Ako to myslíš, že nikam nepatríš?"$ $ Spýtala sa Tulienka Deľa.

"$ $Nedokázala by som sa vrátiť medzi ľudí... nie po... tomto. Už nedokážem. A ku tomu spoločenstvu... Nie. Nie som s nikým. Ale nepáči sa mi to, na rozdiel od Deoque.“

"$ $Keď skončí D, skončí strach.“

"$ $To bude trvať večnosť.“

"$ $Nemusí.“

"$ $Nemám v pláne s nim bojovať, ak narážaš na to.“

"$ $Si Goonová proroctva.“

"$ $A to kto povedal?“

"$ $Kniha.“

"$ $Akosi jej priveľa veríš? Nikoho zabiť neplánujem.“

"$ $Poraziť."$ $ Opravil ju. "$ $Čo porazí D jeho, mocného, nie najmocnejšieho.“

"$ $Nie najmocnejšieho? To je tam?“

"$ $V knihe. V oficiálnej verzii nie. Zrejme sa oplatí vyvolávať strach.“

"$ $Musíme toto teraz rozoberať?“

"$ $Inak, po takom čase by ste mi konečne povedať o tej knihe, o ktorej sa rozprávali tvoja mama s nimi.“

"$ $Tvojou praprababkou a istou bývalou agentkou."$ $ Spresnil Tarny. "$ $Ale k veci, kniha je... no, sú legendy, že existuje ďalšia, okrem knihy proroctiev. Volá sa Kniha Osudu. Podľa spisov fentenzíjskych sú v nej napísané dejiny, čo sa udeje, deje a čo sa kedy udialo. A je tam napísané všetky alternatívy deja, proste čas, celý čas a nielen zeme. V piktopísme.“

"$ $Musí to byť riadne hrubá kniha.“

"$ $Vraj ani nie. Napriek tomu sa to do nej zmestí, je to jeden z neopticky zmenšených predmetov, rovnako ako Kniha proroctiev. Nevie sa veľa o nej, možno to, že D sa ju snaží nájsť. Ak by vedel piktopísmo knihy a vedel by ju prepísať, alebo pevne zadefinovať... To by všetko zmenilo. Aj jeho. Mohol by všetko zmeniť... proroctvá, vytvoriť zo seba vládcu, zničiť svet... Mohol by prísť na to, ako zadefinovať budúcnosť, alebo, čo je horšie minulosť. Mohol by zničiť realitu a nahradiť ju niečím iným. Tá kniha... vedomosť piktopísma a vlastníctvo knihy stavia Knihu do po pozície takého nástroja moci, že dám i dušu zo to, že ju D hľadá, ak o nej vie. Kniha... Časť agentúry je za to, aby sa hľadala a ochránila, časť je za to, aby sa len D zabraňovalo ju získať, ale o knihe informácie nehľadať, lebo môžu byť zneužité. Problém je taký, že oni nemajú šancu, zistiť čo D robí presne a príde mi to ako dosť veľký risk, takto nechať knihu D.“

"$ $Takže ju chceš nájsť, aby sa nedostala do rúk D a čo potom s ňou robiť?“

"$ $Môžeme sa premiestňovať.“

"$ $To i D.“

"$ $Naproti Spoločenstvu.“

"$ $To má plno vojakov a agentov, ktorí sú stokrát lepší, ako my.“

"$ $Nechajme to teraz tak.“

"$ $Dobre... Nevieme nič iné o knihe, ako to, že by mala existovať?“

"$ $No... Z toho, o čom hovorili mama a Arabela... Vedia o tom, že kniha je na Zemi. O mieste, sa nevie nič. Tá kniha je záhada.“

"$ $Ako ju chceš teda dostať?“

"$ $Výskumom. Čítaním. Hľadaním.“

"$ $Myslíš si, že agentúra to neskúšala?“

"$ $Hm...“

"$ $Čo?“

"$ $Agentúra nemá nás. Teda mňa."$ $ Tulienka Deľa sa zasmiala a Pauline sa naňho pozrela s výrazom, že jeho psychický stav sa raz zlepší.

"$ $Choď dokelu, Tarn.“

"$ $Hm... Keby tu tak nejaký bol.“

"$ $Tak choď do... hm.. po chodníku.“

"$ $To robím.“

"$ $Nehovor.“

Po čase zrealizovali Paulinin návrh a z dosiek získali energiu. Takmer z nej neubudlo. Zasýtili sa a tiež zahasili smäd. Tarny vytvoril i zásoby na ďalšie dni a dosky zatvoril. Bolo by veľmi zlé, keby sa o nich niekto dozvedel. Obaja boli v ľudskej podobe, pre prípad, že by niekoho stretli, i keď, aká bola šanca, že to budú ľudia?

Po hodine cesty si dali prestávku. Dostali sa na druhú stranu kopca, ako stála dolina. Nevideli do nej, bola zahalená hmlou a oblakmi. Po chvíli prestávky putovali ďalej.

Nebolo vidieť na viac, než desať metrov do akéhokoľvek smeru a miestami boli radi, že videli na seba. Tarny sa pokúšal hmlu rozhrnúť, ale zdalo sa mu, že mu bráni mocná mágia, ktorú nedokázala poraziť. Cesta sa zhoršila a terén bol horší. Zostupovali do doliny, ale chodník sa premenil na kamenistý, strmý a klzký. Pokúšali sa ho zlepšiť, ale všetky ich pokusy padali na zvláštnej mágii, ktorá blokovala tú ich. Aspoň vytvorili zábranu, aby nepokĺzli sa a nespadli. Putovalo sa namáhavo, ale nebolo miesto na prestávku. Zdalo sa im, že zostup trvá večnosť. Prvý Tarny sotva videl na zem pod sebou, hmla hustla, a ostatní ho nasledovali, dúfajúc, že vidí lepšie než oni. Chodník bol čoraz strmší a čoraz viac sa im zdalo, že sú obeťou kúziel nejakých tamňajších mágov. Nevedeli nič o čase, hodinky zastali. Stále si hovorili, že keď bude najhoršie, premiestnia sa, lenže hnaní túžbou za tajomstvom, nedokázali určiť najhoršie. Pauline sa putovanie aj zapáčilo, dôverovala Tarnymu a Tulienke Deli, a túra sa jej nezdala smrteľne nebezpečná. Upadla do apatie a už len kráčala za Tarnym.

Všetci boli strašne ustatí a nevládali ísť. Potrebovali spať, lenže nebolo kde. Prehovoril Tarny.

"$ $Nezdá sa vám, že klesáme pridlho? Na vrchol sme vyšli za hodinu a vidí sa mi, že nejako sme zamotaní, kráčame dokola. Nejaké škaredé kúzlo nás drží v slučke."$ $ Tulienka Deľa prikývla, zhlboka dýchajúc – bola veľmi unavená.

"$ $Lenže ako ho prelomiť.“

"$ $Detekujem zdroj. Mágiu máme."$ $ Odvetil Tarny.

"$ $Spadneš a zabiješ sa.“

"$ $Na čo mám vás? Istite ma. Majte pripravené zbrane. Nedopustite, aby som spadol. Ja detekujem mágiu.“

Bola to náročná operácia. Tarny nebol v strede, ale museli sa tak preusporiadať a Pauline by prisahala, že tvorcov toho kúzla asi zavraždí. Obe vytvorili silové pole, ktoré ho udržovalo na chodníku. Popravde, Pauline len udržiavala mágiu.

"$ $Mám to."$ $ Povedal Tarny asi po desiatich minútach v sledovaní magického poľa. "$ $Viem kde je zdroj. Je priamo pod nami. Zrušíme ho. Spoločne. Sám to nedám, potrebujeme všetky magické kapacity, aby sme vzali aspoň kúsok mágie z jednej dosky. Pauline, počúvaj. Toto kúzlo musíš dať, od toho závisíme. Musíme nasmerovať mágiu z dosky do zdroja. To je to žiarivé pod nami, keď sa ponoríš do poľa. Usmerni energiu, vezmi ju, nech prúdi to zdroja a rozbije ho. Všetci naraz. Tulienka Deľa?“

"$ $Viem, čo mám robiť. I keď o robím prvý raz v takomto merítku.“

"$ $Nestraš Pauline.“

"$ $Ďakujem pekne.“

"$ $Nehádajme sa. Na tri. Viete všetci čo?"$ $ Prerušila ju Tulienka Deľa.

"$ $Hej."$ $ Znudene povedal Tarny.

"$ $Povedzme, že."$ $ Povedala Pauline.

"$ $Povedzme. Tak... jeden, dva... tri!“

Zlatistá mágia prúdila do zdroja a trhala ho na kúsky. Použili asi polovicu energie dosky, keď sa zdroj rozpadol. Otvorili oči. Najskôr Tarny a Tulienka Deľa, ktorí trhli Pauline, ktorá bola z toho kúzla mierne mimo.

"$ $Čo?“

"$ $Pozri sa..."$ $ Vydýchol Tarny. Stáli pred bránou mesta, na rovine. Kopec bol preč a pred nimi boli len kamenné múry mesta. Z brány vychádzal voz a na ňom žena, zovero s purpurovou pleťou, v lesklom oblečení, za ňou išlo niekoľko skromne odetých mužov, žien a detí so sklonenou hlavou. Voz bol ťahaný Gle'edonmi, fentenzíjskymi druhmi dragonov, podobných koňom, s nízkou inteligenciou, ale veľkou silou. Žena mala na sebe červené a žlté šaty, odrážajúce lúče slnka, či čo to bol za zdroj svetla a plášť, žiariaci stovkami farieb. Vlasy mala rozpustené, miestami do nich vložené diamanty. Voz bol v podobnom štýle.

Tarny, Tulienka Deľa, ani Pauline netušili, kde by mohli byť, ale niečo im navrávalo, že tá žena je dôležitá. Nevedeli, aký bude mať ku nim prístup, ale boli pripravení okamžite zmiznúť.

Stáli pred corlovne Medizo.

\chapter{Vládkyňa troch tabúľ}

"$ $Brána do Tra'itje je otvorená, mág, tlmočníčka a polodémonka."$ $ Corlovne Medizo sa usmiala a zahľadela sa im do očí. Boli mierne otrasní z jej pohľadu, plného večnosti, smrti, chladu a nádeje. Ale hlavne to bol pohľad vládcu, vedomého si svojej moci.

"$ $Čo hovorí?"$ $ Spýtala sa telepaticky Pauline. Nerozumela jazyku, ktorým hovorili. Tarny jej niečo vložil do ruky.

"$ $Daj si to do ucha. Je to prekladač, bude ti všetko prekladať do angličtiny. A tvoju reč do ich jazyka. Ale, preboha, nehovor Tulie, že som ti ho dal. Daj si ho do ucha."$ $ Poslal jej telepaticky obrázkové inštrukcie. Tulienka Deľa jej celý doterajší príhovor pretlmočila.

"$ $Pani Tra'itje je poctená vašou návštevou,"$ $ predniesol muž. Bol, na rozdiel od zástupu za vozom oblečený v honosných šatách. Vyšiel spoza voza, bol medzi zástupom a corlovne. Nebol zrejme v takom postavení ako corlovne, išiel po svojich, ale nebol na tom tak zle, ale tí v zástupe. Pokračoval. "$ $Bohom požehnaná corlovne Medizo, pani nášho mesta, božia vyslankyňa, vládkyňa sveta tohto predurčená, veštby čítala a vás teraz víta a zve vás. Ste jej ctená návšteva, požehnaní a predpovedaní. Tra'itja je otvorená pre vás, deti veštby. Vstúpte do nášho domu."$ $ Corlovne Medizo spravila nejaké gesto rukou, ihneď pribehli dvaja zo zástupu a postavili ju. Pokynula im a zas sa vrátili na svoje miesto. Celý čas mali hlavu sklonenú. Corlovne stála a stále sa jej na tvári neznačil nijaký záchvev sklamania, prekvapenia, ani inej emócie, okrem vedomosti si svojej sily a pokojného úsmevu.

Tarny, Tulienka Deľa a Pauline sa na seba pozreli.

"$ $Tak čo?"$ $ Spýtala sa Pauline.

"$ $Pôjdeme tam?"$ $ Odvetila Tulienka Deľa otázkou.

"$ $Tá žena..."$ $ zalapala na dychu Pauline. "$ $Ja... kde som ju...?“

"$ $V živote som ju nevidel.“

"$ $Mne sa zdá, že...“

"$ $V spojitosti s Jegrigsenom? Potom musíme byť opatrní.“

"$ $To aj tak."$ $ Namietla Tulienka Deľa, Pauline odpovedala až po nej.

"$ $Nie. Vôbec. Tú tvár som videla... kde to..."$ $ zamýšľala sa a Tarny a Tulienka Deľa začínali byť netrpezliví.

"$ $Kde?“

"$ $Na Querte a u Deoque. Videla som jej tvár, ale až niekedy teraz... to nemá z Jegrigsenom nič spoločné.“

"$ $Tak ideme?"$ $ Zašepkala Tulienka Deľa. 

"$ $Za žiadnu cenu sa nerozdeľujme. Ak by bolo najhoršie, premiestnime sa. Dávajte si pozor. Nechceme skončiť, ako na Querte. Teraz už nie."$ $ Prikývli.

Pani Tra‘itje sa zrejme zdalo, že už dlho čakajú, ale neprejavovala známky netrpezlivosti. Keď si všimla, že sa dodohadovali, pokynula mužovi, aby zas zaujal svoje miesto a sama zas prehovorila.

"$ $Tra'itja vás víta v tento šťastný deň."$ $ Nečakala na ich reakciu, bola presvedčená o svojom úspechu. Naznačila trom zo zástupu, aby prišli ku nim a odviedli ich do mesta. Hlavy mali stále sklonené a na Paulinine otázky, kto sú, neodpovedali, len sa večmi zhrbili a pokúšali sa tváriť, že neexistujú.

"$ $Vyzerajú ako otroci."$ $ Zatelepatizovala im Tulienka Deľa, ktorá nechcela vzbudzovať pozornosť v meste.

"$ $Vcelku máš pravdu."$ $ Odvetil Tarny.

"$ $Čo je to za mesto?"$ $ Vydýchla Pauline, akonáhle vstúpili bránou do opevnenia.

Mesto bolo rozdelené do akoby troch častí. Boli jasne odlíšené, akoby niekto chcel, aby bolo jasné, čo kde patrí.

Prvá časť jasne patrila bohatej vrstve. Nachádzala sa najmä pri bráne samotnej, hneď po vojenskej základni a siahala do stredu mesta. Tam do nej patril zámok, chrám a niekoľko ďalších budov, ktoré ani jeden z nich netušil, na čo slúžia.

Druhá časť bola okolo prvej, obopínala ju a miestami sa prelínali, ale nie nejako výrazne. Z každej budovy na rozhraní sa ešte pomerne zreteľne dalo vyčítať, do ktorej časti patrí. Tvorili ju rôzne domčeky remeselníkov, trh a druhý, o niečo chudobnejší chrám a popravisko. To bolo vidieť už z brány. V momente, keď prišli do mesta, na šibenici niekto visel. Pauline z toho takmer prišlo zle.

Tretia časť bola najzaostalejšia, najšpinavšia a najchudobnejšia. Pripomínal im slum. Videli v nich ľudí a zovero, podobných ako tých, ktorí ich práve viedli mestom. Tvorili ho domčeky zo slamy, máloktoré z dreva a aj na uliciach žili ľudia. Boli oddelení magickým plotom od zvyšku mesta a umiestnení ďaleko od brány, aby ich bežný návštevník len tak nezazrel. Z tohto bolo Pauline horšie, než z popravy.

Ich viedli cez prvú časť až ku zlatom vykladanému objektu nejasného využitia. Vtedy ich otroci, ako ich označili, opustili a prišla ku nim žena.

"$ $Pani si želá, aby ste ostali v tomto objekte Hla'gha. Vy, panie,"$ $ otočila sa ku Pauline a Tulienke Deli. "$ $Vy budete bývať v severnom objekte. A vy..."$ $ otočila sa zas ku Tarnymu. "$ $Ste mág, ako Pani Tra'itje praví, že však?“

"$ $Samozrejme."$ $ Povedal, akoby to bolo niečo úplne samozrejmé, že prečo sa ho to pýta, on je predsa jeden z najlepších v zmyslovej mágii v spoločenstve! Žena sa naňho povýšenecky zahľadela a povedala.

"$ $Na muža ste príliš sebavedomý, nezdá sa vám? Porušujete etiketu a vôľu Tra'itje, ale Pani si vás tu želá, a nebudem odporovať, som jej služobnica. Ale neviem odkiaľ ste prišli, pán...“

"$ $Lietavý."$ $ Odvetil Tarny. Pauline sa mierne uškrnula. Nikdy nepočula Tarnyho sa takto predstaviť.

"$ $Tak pán Lietavý, ste síce mág, ale ste muž, a mali by ste sa podľa toho správať. Je od boha jasné, že muži sú pod ženami, ako veštkyne pravia. Tak si dávajte pozor. Mágov je málo, len preto ste v Hla'gha. Inak by ste tu ani povolenie vstúpiť nemali. Prosím za mnou.“

"$ $Myslela som si, že kedysi boli podriadené ženy, nie muži."$ $ Ozvala sa telepaticky Pauline.

"$ $Na Zemi,"$ $ Odvetila Tulienka Deľa. "$ $V Tramtárii to bolo naopak, vo Wymyslensku nerovnosti neboli.“

"$ $A kde sme teraz?“

"$ $V nejakom inopoli s vývojom na úrovni stredoveku, očividne.“

"$ $Toto je zlé."$ $ Povedal Tarny. "$ $Budeme si musieť dávať pozor.“

"$ $Ty to máš ešte v pohode – si mág a teda máš nejaké zo ženských práv tejto krajine, aspoň podľa jej rozprávania.“

"$ $Za to, čo tu môže byť, to stojí. Ale pamätajte, nestratiť kontakt. V žiadnom prípade.“

"$ $V žiadnom prípade."$ $ Súhlasili Tulienka Deľa a Pauline.

Žena sa ku nim zas otočila a odviedla ich do ich izieb.

\begin{center}
*
\end{center}

"$ $Mohli by už byť späť."$ $ Povedala Morja Belle. Tá sa zamračila a pokrčila plecami.

"$ $Môžu byť kdekoľvek, ak ešte žijú.“

"$ $Môžu sa premiestniť.“

"$ $Existujú zóny, z ktorých je to nemožné.“

"$ $Na Querte ale...“

"$ $Je možné ich vytvoriť. A toho sa bojím. Za ten čas, čo som tak bola... Leana z Ölverína uspokojivo nevysvetlila to, ako a prečo sa dostala na Quert. Bola... zvláštna. A Querťania... pripadalo mi, akoby sa kvôli Leane krotili... teraz, keď tam nie je... môžu čokoľvek.“

\begin{center}
*
\end{center}

"$ $Moc mágov nemajú, už si spravil prieskum, Tarn?“

"$ $Mágie má bežnú koncentráciu. Zatiaľ nič divné. Ale corlovne Medizo, tá je čarodejnica. Dával by som si ňu pozor. Čo zatiaľ viete vy?“

"$ $Tra'itja je vývojovo v magickom stredoveku, muži, ak nie sú mágmi, musia počúvať svoju matku, alebo ženu. Hovoria jazykom, ktorý mi znie ako Fentenzíjčina, zrejme nejaké nárečie, ale je zrozumiteľné. Názov Tra'itja je zrejme kombinácie základnej časti označenia pre trojku, teda Tra‘ vo fentenzíjčine a ‘itja je slovotvorný základ od fentenzíjskeho slova tabuľa, doska, kov. Takže..."$ $ Tarny dokončil vetu za ňu.

"$ $Tra'itja je tri tabule, alebo tri kovy... toto bolo zrejme to, čo hľadala Leana, nie? A možno tri kovy dokážu vytvoriť bránu...“

"$ $Každá dostatočne silná mágia môže byť tvorcom brány,"$ $ ozvala sa Pauline. "$ $Myslím že mágia v jednom kove bohato stačí na bránu. Brána je poväčšine náhodná, ale len v rámci jednej paralely, alebo je vytvorené nové inopole.“

"$ $Potom by ale... ak by si dokázal regulovať mágiu v kovoch, mala by si teoreticky neobmedzenú križovatku... Mohla by si prechádzať medzi svetmi, len by si musela čakať na správnu bránu. Predstavte si to...“

"$ $Nesnívaj, Tarn,"$ $ upozornila ho Pauline. On na to nedbal.

"$ $Snívanie je len ďalší spôsob, ako vzniká pokrok. Sen skutočnosťou a skutočnosť snom. Arabela Tlogenová. Ale toto sme nechceli rozoberať. Zdá sa, že nás ku niečomu potrebujú, a to by som teraz najradšej citoval Sylviine slová, že keď sú k tebe niekde bez dôvodu milí, nie je to bez dôvodu. Asi tak. Ale ujsť zatiaľ nechcem, to v žiadnom prípade. Musíme zistiť, čo je Tra'itja zač. Viete niečo iné, okrem jazykovej analýzy?"$ $ Čakal Tulienku Deľu, ale ozvala sa Pauline.

"$ $Corlovne Medizo je tu na tróne iba pár týždňov. Vraj prišla od Re'igha, mesta na východe. Považujú ju z bohyňu. Alebo skôr bohom požehnanú. Vraj často nevychádza zo svojho paláca, takže to, že vyšla pred mesto je nejaká pocta. A tiež viem niečo o hierarchii mesta. Faktické vládkyne mesta sú čarodejnice a kňažky, tá corlovne je iba nejaký duchovný symbol, ale hovorí si vládkyňa. Mágov tu je málo, ako na Zemi a to je jediný spôsob, ako môže byť muž docenený. Inak je považovaný za pracovnú silu, takže máš šťastie Tarn. Tí, čo boli v zástupe za corlovne Medizo sú príslušníci najspodnejšej vrstvy, ktorí boli vybraní, aby slúžili veľkej corlovne Medizo za jedlo, obydlie a ošatenie. Podľa toho, čo mi jeden muž z nich povedal, je to veľká česť a takmer jediný možný únik z najhoršej chudoby. Jediná ďalšia cesta je schopnosť čarovať.“

"$ $Tak to v ich domovoch musí byť naozaj otrasne, ak toto považujú za česť a šťastie. Sú to otroci. Pokračuj, Pauline.“

"$ $Palác corlovne Medizo je miesto, kde takmer nikto od jej príchodu neprišiel a my sme tam pozvaní.“

"$ $Vieš niečo o tom, čo tu bolo pred Medizo?“

"$ $O tom nehovoria. Pre nich... akoby čas pred... neexistoval...“

"$ $To je dosť podivné,"$ $ ozvala sa Tulienka Deľa. "$ $Musíme preskúmať toto mesto. Bezpodmienečne?“

"$ $Ako to chceš urobiť?"$ $ Pauline mala pochyby. "$ $Nepustia ťa von ani z tejto budovy, nie to z areálu...“

"$ $Zabúdaš na jednu vec,"$ $ prerušil ju Tarny. Pauline si presne vedela predstaviť si, ako sa teraz tvári. "$ $Že majú málo mágov.“

\begin{center}
*
\end{center}

Vo vzduchu udržiavala pevnú stenu, cez ktorú neprenikla žiadna zbraň. Už ju udržiavala asi desať minút, ale stále nepociťovala potrebu ju na chvíľu zrušiť a dočerpať mágiu. Necítila sa veľmi vyčerpaná, len si uvedomovala to málo spánku, ktoré za posledné dni mala. Zastávala názor, že spánok je vhodné čerpať len vtedy, keď nič iné na práci nie je. Takýto pocit už dávno nemala. Ak nečítala, nepočítala, nebojovala a neučila sa, len tak kúzlila. Pocit slobody, ktorý vždy, keď čarovala mala, ju napĺňal šťastím a tak si to bezdôvodné míňanie mágiou užívala. Poväčšine len zdokonaľovala svoju stenu, ako momentálne, ale často sa hrala s magickým ohňom. Pohľad naň a jeho energia ju upokojovali. Dávala si pozor, aby nič nepodpálila a zatiaľ úspešne.

Do bytu vkročila Chen. Pozrela sa na Rosu, ktorá ju kývnutím pozdravila a ďalej sa venovala svojej stene, a tak prešla do svojej kancelárie. O pár minút prišla za Rosou, ktorá stále sústredene sledovala svoju stenu.

"$ $Ako dlho, Rosa?"$ $ Spýtala sa Chen pri pohľade na Rosu a jej stenu. "$ $Ako dlho bez načerpania mágie?"$ $ Rosa sa otriasla, nechala stenu zmiznúť. Na čele sa jej črtali kvapky potu.

"$ $O čom hovoríš?“

"$ $Ako dlho si tu udržiavala tú stenu?“

"$ $Jáj... to myslíš... neviem... myslím, že už dosť dlho. Začínala som byť už unavená a tak som ju pustila, ale myslím, že by som ju dokázala mať aj dlhšie. Ale nie som si istá, či v rovnakej kvalite.“

"$ $Máš veľkú magickú kapacitu.“

"$ $Čo to znamená?“

"$ $Že dokážeš ovládať viac mágie naraz. Laicky povedané. Lepšie je to vysvetlené v knihe Revera Thixa, Magický potenciál. Nájdeš ju v knižnici."$ $ Rosa prikývla a otočila sa smerom ku spomínanej miestnosti. Tento rozhovor sa pre ňu zatiaľ skončil. Začítala sa do knihy a mierne prestala vnímať.

"$ $Rosa?“

"$ $Hm...?“

"$ $Jedla si dnes vôbec?“

"$ $É...“

"$ $Tak v chladničke máš obed.“

"$ $Dík, Chen. Možno, keď dočítam.“

\begin{center}
*
\end{center}

"$ $Bežte ku vchodu, tam sa počkáme. Musíme sa rozdeliť, aby sme ich zmiatli."$ $ Boli síce neviditeľní, ale rozptýlenie mágie v priestore odpovedalo prítomnosti mága. Audienciu u corlovne mali mať až nasledujúci deň, a tak nepredpokladali, žeby ich niekto hľadal. Ale nikdy človek nevie. Všetci boli zneviditeľnení, len Pauline začaroval Tarny, keďže jej zmyslová mágia akosi nešla.

"$ $A pamätaj, Pauline. Keby niečo, štít a utekaj. V žiadnom prípade neruš neviditeľnosť.“

"$ $Nie som blbá.“

"$ $Dúfam, stretneme sa pri bráne. Zatiaľ dovidenia."$ $ Ukončil Tarny telepatické spojenie. Každý sa rozbehli svojím smerom.

Pauline prešla bez povšimnutia dlhou chodbou a zahla doprava. Neboli tam žiadne dvere, ktorých otváraním by mohla vzbudiť pozornosť. Nepamätala si už východ a nechcela riskovať stratenie sa, a tak sa nepremiestnila. Tarny získal plány budovy a ona sa mala dostať cez najbezpečnejšiu nechránenú cestu ku východu. Chodba pred ňou ústila ku schodom. Prebehla cez ne. Dostala sa do veľkej sály. Obzrela sa, aby obhliadla situáciu. To, čo v sále uvidela, ju mierne zdesilo. Bolo v nej okolo sedem brán, pri každej stáli dvaja strážnici. Teda strážkyne. Neboli to ľudia. Ani zovero. Podľa toho, čo Tarny rozprával, sa jej nezdalo, že by to mohli byť víly. Boli na to príliš vysoké. Do tvárí im nevidela, ale ich tieň sa jej nezdal, že by veľmi podobali na ľudské. Nemala poňatia, či sú to čarodejnice, či ju dokážu odčarovať. Možno na nich kúzlo vôbec nepôsobí. Ako na Deoque, na ktorú zmyslové kúzla nepôsobia... Bála sa. Pokúšala sa neprepadať panike, ale nejako sa jej zahmlila myseľ. Nedokázala rozmýšľať, prepočítať svoje možnosti. Zrazu započula zvuk zatvárania dverí. Obrátila sa. Tam, kde boli predtým schody, boli len kamenné dvere. Pokúšala sa ich otvoriť, ale nenašla kľučku. Oprela sa do nich, aby ich odtlačila, ale kamenné dvere nepovolili. Bola nahnevaná sama na seba, že sa takto nechala zavrieť. Keby sa mohla... ale ona sa predsa mohla premiestniť! Úplne na svoju schopnosť zabudla. Pomyslela si na miestnosť, v ktorej začínali. Presne si ju predstavila, do najmenších detailov, aké si pamätala, zatvorila oči...

Nepremiestnila sa. Nešlo to. Prepadala ju panika, vždy si hovorila, že bude môcť uniknúť... Nemohla o to prísť, predsa to je... Pozrela sa na svoje ruky. Bola stále neviditeľná. Čarovať teda očividne mohla. Teda aj telepatizovať. Zatelepatizuje Tarnymu a... Zvyčajný postup. Ponorila sa do magického poľa a prepojila na telepatickú mágiu. Hľadala spojenia. Tarnyho nevidela Nebol ta. Namiesto toho bol blízko nej niekto iný. Neuvedomovala si, že by si mala dávať pozor na to, aby ju neodhalili. Hlasno vydýchla. Niekde blízko bola corlovne Medizo. Čo tam dopekla hľadala?! Na odpovedanie otázku nebolo času. Jej dych bol prihlasný. Strážkyne pri dverách sa začali otáčať. Pauline zahliadla svoje ruky. Zrušili jej neviditeľnosť! Väčšinu ostávajúcu mágiu sa rozhodla použiť na pokus o otvorenie dverí. Zbytkom okolo seba vytvorila štít, ako ju to Deoque učila. Sústredila sa naň, nemala ani najmenší záujem na tom, aby jej ho zrušili, ako neviditeľnosť. Vyslala mágiu ku dverám, aby sa zničili, ale kameň to ani neškrablo. Zakliala. Aj tak už o nej vedeli.

Nevykročili ku nej, zatiaľ ani nevyslali útok a tak využila chvíľu a načerpala trochu mágie na udržovanie štítu. Moc jej to nešlo, rýchlo sa silila, ubúdalo jej mágie a bola nervózna z nečinnosti strážkyň. Pot jej tiekol po čele. Musela štít na chvíľu zložiť, prudko dýchala, nevládala sa ani pohnúť, čerpala pomaly, prirodzene mágiu, ako v nečinnosti. To, že bola bez štítu, to si bleskovo všimli strážkyne. Všetky naraz vyslali zvláštny zelený magický náboj na ňu. Nevedela, že za to vďačí polodémonským reflexom, vtedy to ešte pokladala zázrak, ale v stotine sekundy vyčerila štít a zastavila výboje. Štít ich vstrebal a pod vplyvom mágie sa stal ešte silnejší. Strážkyne na to zareagovali tiež nadpozemsky rýchlo, priam ako keby to boli démonky a polodémonky. Vystúpili, každá asi o krok dopredu, synchronizovane každá vyčarila striebristé svetlo, zamierili na ňu, chvíľu čakali a zrazu, v jednom okamihu, vystrelili.

Pauline čakala, že štít výboje pohltí, ale mýlila sa. Pár centimetrov pred štítom zastali, začali sa spájať a vytvorili akoby kuklu okolo štítu. Pauline sa na okamih dostala do magického videnia a uvedomila si, že tá striebristá kukla, v ktorej sa ocitla jej požiera mágiu. Ostávalo jej mágie len veľmi málo a úplne nezvládala situáciu. Posunula sa o kúsok a kukla sa posunula tiež. Pomaly jej brala všetky sily. Pauline zrušila štít a vtedy sa do nej kukla oprela ešte agresívnejšie. Nedokázala sa jej brániť. Nedokázala tomu vzdorovať. Klesla na kolená a takmer bola v bezvedomí, vedome vnímala len zvláštnu bolesť, ktorú jej kukla spôsobovala. Neuvedomovala si, že jedna brána sa otvoril. Niekto vkročil. A skríkol.

"$ $Stop! Okamžite prestaňte! Nechcete ju úplne vysať! Príde o prirodzenú obranyschopnosť a to nechceme!"$ $ Bolesť náhle ustala. Stále sa cítili tak zle, že nedokázala myslieť, nieto rozanalyzovať situáciu, či vôbec sa dostať do vedomia.

Keby nebola v tú chvíľu v bezvedomí, videla by dosť nahnevanú corlovne Medizo, ktorá ráznym hlasom strážkyniam niečo vyčítala. V sále sa začalo svietiť, a tak, keby Pauline nebola v bezvedomí, uvidela by riadne, kto sú to strážkyne.

Boli vysoké asi ako priemerná ľudská žena. Neboli identické, ale výškovo boli rovnaké. Črty tváre mali ako ľudia, len ostrejšie, ale farba bola diametrálne odlišná. Ich pleť mala zvláštny odtieň zelenej. Pery mali silnomodrú farbu, o ktorej ale nebolo jasné, či je maľovaním. Na lícach, čele, očiach a brade mali podivné tetovania, nie rovnaké, ale podobné. Veľa z nich, ale nie všetky mali na špicatých ušiach pestrofarebné náušnice. Ich vlasy boli prevažne rovnakej dĺžky, spletené vo vrkôčikoch, zakrývajúce uši, okrem končekov a v nich boli zapletené farebné perá, drahokamy a iné zvláštnosti. Niektoré z nich mali vo vlasoch zapletené dýky, úplne miniatúrne alebo jedna dokonca ampulku s nejakou tekutinou. Oblečené boli navlas rovnako – čierne, kožené vesty, hnedé kožené nohavice. Mali drahými kameňmi vykladané opasky, za každým úzky stredný meč, nejaké ampulky z substanciami, dýku a fialový podlhovastý predmet neznámeho využitia, dlhý asi ako pero. Chodili v dlhých čiernych čižmách. Na dlaniach mali čierne rukavice – taktiež z kože. Celý krk a ruky im pokrývali náramky, amulety, tetovania, že bolo ťažké rozoznať, kde sa začína oblečenie a končia ozdoby.

Akonáhle, ako zrušili kuklu okolo Pauline, uklonili sa corlovne Medizo a vrátili sa ku bránam. Medizo si ich po vyhrešení nevšímala a presunula svoju pozornosť ku Pauline. Niečo zavolala a z jednej brány, z rovnakej ako ona, vyšla žena v bielom plášti. Spýtavo sa pozrela na corlovne Medizo, uklonila sa. Medizo jej niečo povedala a žena sa ešte raz uklonila. Prešla ku Pauline. Sklonila sa a začala do nej dávať mágiu. Pauline pomaly otvárala oči, nadobúdala vedomie. Žena, akonáhle mala otvorené oči niečo skontrolovala a obrátila sa ku corlovne Medizo. Tá prikývla. Žena sa uklonila a odišla.

Pauline bola už pri zmysloch, ale nie úplne spracovávala to, čo sa okolo nej dialo. Posledné, čo si pamätala bolo, ako tie strážkyne okolo nej vytvorili kuklu. Predpokladala, že odpadla. Bolelo ju celé telo, najviac asi hlava, ktorá jej stále brnela. Pretrela si oči a podoprela sa, chcela sa postaviť. Vtedy si uvedomila, že stále je v tej prekliatej sieni. Inštinktívne sa chcela premiestniť, ale nedalo sa to. Uvedomila si zrazu, kto stojí pred ňou. Corlovne Medizo. Chcela sa zneviditeľniť. Nevedela ako a jej pokusy o zneviditeľnenie hocičoho skončili katastrofálne. Chcela okolo seba urobiť štít, ale nepodarilo sa jej ho udržať viac ako niekoľko sekúnd. Chcela ujsť, alebo nebolo úniku. Rezignovala. Zvalila sa na podlahu s dúfaním, že to bol iba zlý sen. Nebol. Niečo ňou trhlo, zjavne mágia. Postavilo ju to a držalo jej oči otvorené. Bola zúfalá. Medzitým Medizo prehovorila.

"$ $Naša milá polodémonka, zavítala si tu skorej, ako som zamýšľala. A nemusíš sa pokúšať premiestniť, späť do paláca sa nedostaneš. Len kedy sa ti chcela objavovať sa po tejto sieni."$ $ Pauline premýšľala. Tarny jej hovoril o takom type kúzla... keď mohla byť len v rámci neho a kým nebolo zrušené, nedalo sa z neho dostať... alebo tak nejako... ilúzia...

"$ $Je to ilúzia?"$ $ Typovala. Corlovne Medizo pokrútila hlavou.

"$ $Úplne zle, polodémonka Goonová. Úplne zle."$ $ Pousmiala sa. "$ $Presne naopak."$ $ Kým sa spamätávala a pokúšala vstrebať túto informáciu, pozerala sa na Medizo.

Tá už nebola v tak veľkolepom ošatení ako keď ju videla prvý raz, ale nebolo ani o moc skromnejšie. Tvár mala pomaľovanú zlatou a striebornou farbou v zložitých tvaroch. Vo vlasoch mala zapletené drahokamy. Mala oblečené striebristé šaty, pokryté drahokamami a na krku prívesok, na ktorom bol nejaký znak piktopísma. Na chrbte pravej ruky mala vytetovaný podobný znak, ale nie rovnaký. Na oboch rukách sa jej kopili amulety a prstene.

"$ $Tak... skutočnosť?"$ $ Hádala Pauline, stále otupená na to, aby premýšľala. Corlovne Medizo sa stále veľmi spokojne usmievala, stojac na mieste, si upravovala šaty.

"$ $Presne tak, Goonová – Tlogenová, správne."$ $ Pauline nechápala.

"$ $Tak skutočnosť je potom...“

"$ $Toto. A to, o čom si bola presvedčená, že skutočnosť je, je ilúzia."$ $ Pauline vstrebávala.

"$ $Takže... my sme niekedy vošli do ilúzie a teraz som z nej vyšla... alebo...“

"$ $Správne pravíš, polodémonka, správne hádaš.“

"$ $Kedy...?"$ $ Zmohla sa len na jedinú otázku.

"$ $Akonáhle ste začali klesať ste vošli do dvoch ilúzií. Tú, ktorú mohli prelomiť len mágovia ste prešli, ale tú druhú ste neodhalili. Je silne skrytá, že by ste museli mať neviemaké žriedlo mágie, aby ste ju vôbec objavili, nie zrušili. Celý svet tu sú ruiny bývalého impéria, ktoré som prebrala. Amaz'hany mi v mojom cieli veľmi ochotne pomáhajú."$ $ Ukázala na strážkyne okolo nej.

"$ $Kto ste a prečo...? Ľudia niečo tušia...?“

"$ $Privítali ma. Som pre nich veľká čarodejnica, ktorá vytvorila toto mesto šmahom ruky a kto sa stará o to, či je to len ilúzia. Keby to teda vedeli. A načo búrať ilúziu o mojej kúzelnosti, keď mi to prináša slávu? Keď sa zovero dostane do toho, kde by sa nikdy nedostal, keby nespôsobil paradox a celá slávna Tramtárijská kultúra padla, dostala sa do rúk mužov, čo by mala robiť corlovne? Amaz'hany poznali toto miesto a pomocou mojich síl a ich znalostí sme obnovili mesto. Ľudí sme tu dostali z okolitých dolín, boli roztrúsení a my sme im dali nádej. Konečne niekto, kto si ctil Tramtárijskú kultúru, konečne miesto, kde nebol žiadny pokus o rovnoprávnosť. To by bolo! Ten tu ani nikdy nenastane. Keby muži dostali priestor vo verejnom živote, to by bola katastrofa. Mágov potrebujeme, to je pravda, ale inak... Kto by bol potom lacnou pracovnou silou, kto by hýbal ekonomiku? Konečne svet, v ktorom sa piliere normálnosti za tie tisícky rokov nezrútili..."$ $ Pauline ju pomaly prerušila.

"$ $Nehnevajte sa, corlovne Medizo, ale... prečo mi to preboha hovoríte?! A ako som sa mohla len tak dostať z ilúzie? Tarny hovoril, že...“

"$ $Je to muž, nemá pravdu,"$ $ prerušila ju Medizo. "$ $A prečo ti to hovorím, milá polodémonka... kto hovorí, že sa to niekto okrem teba dozvie? Za väčšinou s týchto dverí sa skrýva niečo, na čo budeš potrebovať i znásobené polodémonské reflexy, aby si prežila..."$ $ Pauline zamrazilo. Dúfala, že Tarny si všimne, že mešká a pomocou mágie v kove nájde realitu a v kročí do nej. Nateraz si nič viac neželala.

\begin{center}
*
\end{center}

Tarnyho vo chvíli, keď premýšľal, kde Pauline môže byť, keď ju nevie telepaticky nájsť ani nenapadla skutočnosť.

"$ $Kde môže byť? Dopekla, vôbec nie je v poli ani na svojej chodbe!“

"$ $Nemali sme ju púšťať samú!"$ $ Vyčítala mu Tulienka Deľa.

"$ $Nie je dieťa a my nie sme jej rodičia. Je to polodémonka, preboha! Môže sa premiestňovať!“

"$ $A čo ak toto bola pasca?! Vošli sme do jamy levovej, Tarn. Dosť zamaskovanej, aby sme si ju nevšimli a toto je koniec. Je to polodémonka a oni to vedia. Uvedomuješ si, prečo sa Syl skrýva?! Polodémonov sa ľudia a zovero boja a zároveň túžia po ich schopnostiach! Musíme ju nájsť!“

"$ $Myslíš si, že...?“

"$ $Hej! Celkom iste! Toto celé bola pasca! Neviem čo tu chceli robiť Querťania, ak chceli toto, ale od nás niečo chcú a teraz sme stratili Pauline. Potrebujeme ju nájsť!“

"$ $Tak poďme! Odkiaľ mala prísť?“

Prebehli chodbu a dostali sa na miesto, kde sa rozdelili. Žiadne stopy.

"$ $Neboli tu žiadne odbočky, nemala sa kde stratiť. Jedine, ak by ju uniesli, alebo ak by...“

"$ $Čo Tarny?“

"$ $Ilúzia. Mohla do nej vojsť. Nevie až tak dobre čarovať, ani by ma to neprekvapila. Teda tá ilúzia musí tu niekde byť.“

"$ $Nájdeš ju?“

"$ $Ilúziu? Alebo Pauline?“

"$ $Obe. Nájdi zdroj, teda, ak tu je."$ $ Tarny sa ponoril do magického poľa a hľadal.

"$ $Nič tu nie je. Respektíve nič nevidím. Buď použili clonu, alebo...“

"$ $Skús nájsť tú clonu. Máš pri sebe zásobu mágie, tak ju nejako usmerni.“

"$ $S tým mi budeš musieť pomôcť. Vytvor okolo nás nejakú silnú neviditeľnosť a ja zatiaľ vyberiem dosku."$ $ Tulienka Deľa prikývla.

\begin{center}
*
\end{center}

"$ $Ale ja neviem poriadne čarovať... ja..."$ $ Bránila sa Pauline. Corlovne Medizo len mávla rukami.

"$ $To sa rýchlo napraví. Amaz'ha\v{}ja!"$ $ Zvolala. Jedna z brán sa otvorila a vyšla z nej Amaz'hana. Podobná ostatným, takmer na nerozoznanie. Uklonila sa a spýtavo sa zahľadela na corlovne.

"$ $Túto polodémonku zober ku vám a daj jej základný výcvik. Je to polodémonka, takže ju nešetri. Pre našu vec."$ $ Amaz'ha\v{}ja sa zas uklonila.

"$ $Ako si želáš, ahma‘Medizo. Vie našu reč?“

"$ $Rozpráva ňou,"$ $ odvetila jej corlovne.

"$ $Aj tou našou? Starou vílčinou v jej krajine."$ $ Corlovne Medizo sa zamračila a obrátila na Pauline.

"$ $Ako si na tom s jazykmi, Goonová? Jazykom Amaz'han rozumieš?"$ $ Pauline pokrútila hlavou.

"$ $Tak na ňu hovorte len mojím jazykom."$ $ Prikázala. Amaz'hana prikývla.

"$ $Aký titul má?“

"$ $Je to síce polodémonka, ale človek a len azma‘."$ $ Odvetila Medizo. Amaz'hana sa otočila na Pauline a prehovorila na ňu ostrým, rozkazujúcim tónom.

"$ $Tak azma‘polodémonka, ideš so mnou. Máš priveľkú cenu, aby si len tak ostala tu."$ $ Pauline sa ani nehla. Bola prikovaná, vydesená, bolo to na ňu za tých pár dní jednoducho až príliš veľa, čo nedokázala a miestami nechcela pochopiť. A teraz ju tá zelená žena volala a vládkyňa mesta sa jej vyhrážala smrťou.

"$ $Azma'Goonová!"$ $ Zvolala netrpezlivým tónom. "$ $Nebudem to viackrát opakovať a v tvojej vlastnej bezpečnosti je, aby si sa podvolila!"$ $ To Pauline ešte viac zamrazilo na mieste.

"$ $Bude zavolané ghe're\v{}un! Azma'človek chce ísť so mnou!"$ $ Vyhrážala sa jej niečím, čomu Pauline úplne nechápala, ale z jej tónu hlasu vycítila, že ide o niečo desivé, ale aj tak sa nehla. Amaz'hana zdvihla ruky, spoza opaska niečo vytiahla a už sa chystala niečo vyčarovať, ale Medizo ju zastavila šmahnutím ruky.

"$ $Prestaň, Amaz'ha\v{}ja! Ani z nej nevysávajte mágiu. Je to človek a potrebujeme ju aj s mágiou! Kontroluj sa! Ak odmieta ísť, existujú aj iné metódy."$ $ Pauline stuhla. Hádali sa celkom iste o nej a síce nevedela, čo chcela Amaz'ha\v{}ja urobiť, celá sa z toho roztriasla.

"$ $Tak ideš, Goonová?"$ $ Spýtala sa, či skôr prikázala autoritatívne corlovne Medizo. Pauline sa nedokázala od strachu hnúť. "$ $Polodémonka! Trasieš sa tu ako decko! Amaz'ha\v{}ja, pre teba ahma'Amaz'ha\v{}ja ťa dopraví do ich centrály. Tam sa vyrieši ten problém s čarovaním a bojom. Nič ti tak nespravia, tam nie. Si priveľmi cenná, aby sme ti zatiaľ ubližovali, polodémonka. Tak nestoj tam ako teľa a nasleduj Amaz'ha\v{}ju!"$ $ Pauline sa malými krôčikmi presúvala ku Amaz'ha\v{}ji a jediné čo chcela, aby sa táto nočná mora skončila a oni boli späť u Tarnyho doma. Nič sa ale nespĺňalo.

Bola vedená Amaz'ha\v{}jou. Za bránou sa otvorili tmavé chodby, ktoré sa jej všetky zdali rovnaké. Najskôr boli prázdne, po nejakom čase sa začali objavovať aj Amaz'hany. Keď sa objavili aj svetlá, miestnosti vôbec neboli také honosné, ako v meste. Boli vlastne nejakým cvičiskom. Okolo Pauline a Amaz'ha\v{}je bolo veľa Amaz'han, ale len ženy. Pauline si nespomínala, že by vôbec počas celej cesty videla nejakého muža od Amaz'han. Zato Amaz'hanských žien tam bolo kopa. Rôzneho veku, ale všetky vyzerali pomerne strojovo, rovnako ako tie na stráži. Nedokázala by ich od seba rozlíšiť. Líšili sa maximálne vekom, niektoré výškou a výzbrojou. Nemali dokonca ani rôznu farbu vlasou či kože. Všetky boli zelené s bielymi vlasmi. Keď kráčajúc rozmýšľala, či to docielili klonovaním uvidela nejakú úplne malú Amaz'hanu, ktorej koža nemala taký výrazný odtieň zelenej, ako ostatné Amaz'hany a jej vlasy boli dokonca čierne. Niekde ju viedla ďalšia Amaz'hana. Možno sa nevydarila, napadlo Pauline. Síce sa to pokúšali zakrývať, Pauline vzbudila pozornosť. Nebola vôbec ani podobného farebného spektra ako Amaz'hany a nemala na sebe žiadne tetovania, či amulety. A nemala špicaté, dlhé uši.

Zastali pred rozdvojkou chodieb. Amaz'hana sa na ňu obrátila a dlhšie si ju premeriavala až napokon jej povedala.

"$ $Azma'Goonová, nie si optimálnej farby. Príliš vzbudzuješ pozornosť. To azma'polodémonka nevie ani základné zmyslové kúzla?!“

"$ $Neviem!"$ $ Odsekla Pauline. "$ $A chcela by som niečo vedieť!"$ $ Vyhlásila skôr ako si uvedomila, čo robí.

"$ $Azma'Goonová nemá právo vedieť viac, ako ahma'Medizo a ahma'Amaz'ha\v{}gha dovolia. Ale azma'Goonová, je ti dovolené pýtať sa. Nesmú to byť však otázky provokačné, ani nikoho z nás či nebodaj Ahma'Treta'ji urážajúce."$ $ Pauline netušila, kto je to ahma'Treta'ji, ale rozhodla sa svoju možnosť využiť, kým môže, ale dávať si pozor.

"$ $Takže, chcela by som sa spýtať..."$ $ Amaz'ha\v{}ja ju však prerušila.

"$ $Zabúdaš na oslovenie a titul. Pre teba som ahma'Amaz'ha\v{}ja a pýtaš sa mňa.“

"$ $Tak nech ahma'Amaz'ha\v{}ja odpustí."$ $ Mierne zdesene sa ospravedlnila Pauline, napodobňujúc pritom používanie oslovení Amaz'ha\v{}ju. "$ $Ahma'Amaz'ha\v{}ja, chcela by som sa spýtať na pár vecí. Takže, prečo tu sú samé ženy? Mali ste nejakú epidémiu? A ako ste dosiahli to, že ste vo farbe pleti a vlasov takmer navlas rovnaké? A kde to ideme?“

"$ $A čo si azma'Goonová myslela, že vznešené Amaz'hany by dali Amaz'titul nejakému podradnému tvorovi, akými aghe‘muži sú? Veď i ľudia ich živia len na ťažkú prácu a pokračovanie rodu, na nič iné súci nie sú, tie stvorenia bez mozgu, preto sme požehnaný národ a napredujeme rýchlejšie než ostatní, lebo veľká Ahma'Treta'ji bola veľmi milostivá a nášmu národu aghe‘mužov netreba. Fyzicky sme lepšie vybavené a na uchovanie rodu stačí vílska genetická manipulácia alebo partenogenéza. Ahma'Medizo ich v Tra'itji živí len pre nedostatok mágov, inak je to len lacná pracovná sila, ktorá rozum nemá, lebo tak boli stvorení, preto sme najdrahocennejšie, ako ahma'Kniha tvrdí a tvrdiť bude. Len teraz začínajú sa ozývať hnutia spochybňujúce tento fakt a chcejúce rovnoprávnosť. Azma'Goonová si toto musí pamätať, inak nebude spasená a Ahma'Treta'ji ju potrestá."$ $ Pauline došla ku tomu, že radšej bude držať jazyk za zubami, ako by si ich mala nahnevať, nech si myslí čokoľvek. Tak prikývla a Amaz'ha\v{}ja pokračovala. "$ $Ahma'Treta'ji tvrdí, že výzor je niečo, čo treba obetovať len jej a ona nám dá naše umenie a znalosti. Preto jej obetujeme náš prirodzený výzor a stávame sa jej služobnicami pre slávu Ahma'Treta'ji. V ahma‘Knihe nám zvestovala, že: ‚Každá moja služobnica sa vzdáva svojej podoby pre slávu moju a spasenia. Muži tejto slávy nikdy nebudú hodní, preto čujte, Amaz'hany, pre vaše spasenie a moju slávu čujte môj hlas a moje rady. Farba spásy je smaragdovoolivovozelená, a preto takú farbu bude mať vaša koža. Farba slávy je biela, a preto nech takú farbu majú vaše vlasy. Nech poznajú i neveriaci vaše božstvo, preto choďte a slávu pre mňa i seba hlásajte.‘. To nám povedala Ahma'Treta'ji v ahma'Knihe. Preto každá z nás bola ako dieťa zasvätená Ahma'Treta'ji a vykúpaná v ahma'prameni Ahma'Treta'ji, ktorý nám dal podobu. To je zákon Amaz'han."$ $ Pauline nechcela sa hádať s Amaz'ha\v{}jou ohľadne existencie ahma'Treta‘ji a tak zas len prikývla, aby sa Amaz'ha\v{}ja dostala ku tomu, kde idú. A tá konečne začala:

"$ $A teraz ide azma'Goonová na miesto, kde sa naučí znalostiam boja, mágie a viery. Príde o chvíľu azam'Amaz'ha'r\v{}a, ktorá vedie výcvik Amaz'han. O tom, že tu je azma'Goonová je informovaná. Len pre Ahma'Treta'ji, nech azma'polodémonka urobí niečo s tým, ako vyzerá. Veď je polodémonka, dokáže sa premieňať, tak nech niečo robí. Bolo by to zneuctenie Ahma'Treta'ji, keby bola takto!"$ $ Až vtedy Pauline napadlo, že sa môže premieňať. Doteraz túto svoju schopnosť využila len u Deoque a tak na ňu úplne zabudla. Rozmýšľala, na ktorú sa premeniť, aby tak náhodou niekoho neurazila, keďže nemala ani tušenie, ako by tá dotyčná zareagovala. Síce sa jej všetky zdali rovnaké, určite sa vedeli rozlíšiť. Tak sa premenila na nejakú menšiu Amaz'hanu, ktorá bola približne rovnako vysoká, ako ona. Po premene sa na ňu Amaz'ha\v{}ja zahľadela, chvíľu akoby čítala znaky, ktoré mala na tvári a potom sa mierne zamračila.

"$ $To je v pohode, zatiaľ sa azma'polodémonka nezmenila na nikoho významného, ale prečo sa azma'Goonovej nezmenili šaty?"$ $ Pauline si vtedy uvedomila, že je stále v tom istom, čo jej dali v meste, nie vo výzbroji Amaz'han. Netušila prečo. Myslela si, že pri premene sa zmenia aj šaty, ale keď tak nad tým rozmýšľala, aké šaty by potom mala mať? Ona predsa nekontrolovala svoje oblečenie, len seba a teda... Nebol dôvod na zmenu oblečenia. Tak pokrčila plecami.

"$ $Sth'agh\v{}hel!"$ $ Zakliala Amaz'ha\v{}ja nejakou nadávkou, ktorú Pauline nepoznala. To skutočne nevieš premeniť si šaty mágiou?! Azma'Goonová, čakala som aspoň trochu viac.“

"$ $Nikdy som to neskúšala."$ $ Priznala Pauline. "$ $Ale môžem...“

"$ $Tak rýchlo! Azam'Amaz'ha'r\v{}a tu môže byť hocikedy a azma'Goonová nechce zneuctiť Ahma'Treta'ji!"$ $ Pauline neskúšala protestovať. Toto síce ešte nikdy nerobila, ale zas, nemohlo to byť až tak komplikované. Možno ako zmena veľkosti. Ponorila sa do magického poľa a predstavila si, ako by jej ošatenie malo vyzerať. Jeho veľkosť, farbu... Otvorila oči. Zmenilo sa! Podarilo sa jej kúzlo. Bola zatiaľ so sebou, aspoň na tento okamih spokojná, keďže sa jej podarilo kúzlo na prvý pokus. Jej spokojnosť však nezdieľala Amaz'ha\v{}ja. Tá sa sústredila na niečo iné. Stáli tam už trochu dlhšiu dobu, akoby Pauline čakala a tak sa rozhodla, že podrobnejšie preskúma, ako vlastne vyzerá.

Kožu i vlasy mala rovnaké, ako Amaz'hany. Nebola ešte tak vysoká ako Amaz'ha\v{}ja. Vlasy mala zapletené do stoviek vrkočov. Boli kratšie, ako tie jej. Stále si zvykala na to, že akonáhle sa pozrela si na ruky, zdobili ich tetovania. Amulety, prívesky a iné ozdoby sa nereplikovali. Pochopiteľne. Oblečenie si vyčarovala priamo na mieru a musela povedať, že jej sedelo viac ako to, čo dostala v Tra'itji. Dalo sa v tom každopádne lepšie pohybovať. Topánky to trochu kazili.

\begin{center}
*
\end{center}

Tarny sa sústredil. Veľa mágie, tak ako vládal, použil na ničenie skrýš, ktoré zakrývali zhluk mágie okolo zdroja ilúzie.

"$ $Je tu strašná kopa mágie! Neviem, prečo na túto ilúziu sa tak usilovali, ale mám neblahé tušenie, že tu nejde len o samotnú ilúziu. Tá mágia zakrýva niečo iné, ako len to čo sme si mysleli, až sa to mierne desím zničiť. Lepšie by aj možno bolo vstúpiť do tej ilúzie, a neviem...“

"$ $Aké to tam je?“

"$ $Príliš zvláštne. Je to príliš zamotané, ale vidím tu východ. Sám tú ilúziu nezničím, ani s tebou. Je potrebný aspoň jeden človek. Ale zato, dokážeme sa do nej dostať a zas z nej. Netuším čo v nej bude, možno východ niekde inde. Táto ilúzia sa nám totiž správa ako nejaké vnorené inopole, či ako by to Pauline nazvala.“

"$ $Inopole v inopoli?“

"$ $Ono sa to správa ako inopole, ale je to ručné kúzlo, teda to nie je inopole, ale ilúzia. Inopole sa nedá spraviť kúzlom, je naň potreba až priveľa mágie, ktorej dávka je smrteľná. A táto ilúzia je síce vytvorená veľkou mágiou, až by som typoval na ďalší kov. Problém je taký, že absolútne neviem určiť, na ktorej strane ilúzie sme.“

"$ $Čože? Myslíš, že my môžeme byť v ilúzii a...“

"$ $Ono možné je to vždy, ale väčšinou rátame z takou prirodzenou realitou. Ale teraz... ten predpoklad sa mi nezdá správny. Toto mesto je čudné."$ $ Tulienka Deľa premýšľala.

"$ $Pauline hovorila, že jej niekto tvrdil, že celé mesto postavila corlovne Medizo za pár hodín mágiou a preto je to veľká čarodejnica. Pred pár mesiacmi tu vraj žiadne mesto nebolo... To mesto na to nevyzerá. To len dáva dôveryhodnosť teórii, že sme v jednej veľkej ilúzii.“

"$ $Ale... tú ilúziu nejde spraviť len s jedným zovero. To je nad rámec našich fyziologických možností. Telo zovero proste neprijme istú hranicu mágie... Zničenie kúzla, to je niečo iné, ale vytvorenie...“

"$ $Nemusela byť predsa sama. Čo ty vieš, kto skutočne ovláda toto mesto. Môže tu byť kľudne nejaký druh, s ktorým sme sa ešte nestretli. Je to predsa nové inopole a je z Quertu. Je podľa mňa aj dosť nepravdepodobné, že v inopoli, do ktorého sa dostaneme bude niečo humanoidné. Ale mali by sme prestať toto rozoberať, ale pomôcť Pauline. Nevieme čo od nej chcú."$ $ Tarny prisvedčil.

"$ $Ideme dovnútra?“

"$ $Máme inú možnosť, Tarn?“

"$ $Nie.“

"$ $Tak vidíš."$ $ Tarny mierne zamračene a nespokojne prikývol a zas sa začal sústrediť.

\begin{center}
*
\end{center}

Amaz'ha'r\v{}a prišla minút po konci Paulininých otázok. Bola takmer na nerozlíšenie od ostatných Amaz'han. Chvíľu sa niečo rozprávala s Amaz'ha\v{}jou v ich reči, ktorú Pauline nerozumela – jej prekladač ju neprekladal. Bolo to zrejme aj niečo o nej, keďže na ňu počas rozhovoru ukazovali, ale rečou ktorej prekladač nerozumel. Zrejme to bolo niečo, čo nemala vedieť, alebo proste – neuvedomili si, že nehovorí ich rečou. V tej chvíli si strašne želala mať schopnosť Tulienky Deli naučiť sa jazyk obyčajným vedomím, že existuje.

Po chvíli sa prestali rozprávať a Amaz'ha\v{}ja prehovorila ku Pauline.

"$ $Azma‘polodémonka, pre Ahma'Treta'ji budeš teraz žiť a konať. Kým sa tvoja úloha neskončí ti budú Amaz'hany sestrami. Preto teraz ideš so mnou sa učiť na slávu Ahma'Treta'ji. Vyučujeme tri základné znalosti, ktorých znalá musí byť aj azma'Goonová. Boj a jeho znalosti musí byť znalá i atna'Amaz'hana a aj ty. Mágii sa nám nevyrovnajú, preto i azma'polodémonka sa jej bude učiť. A napokon, pre slávu Ahma'Treta'ji, ktorá nám nás dala a našu slávu sa budeš vzdelávať vo vzývaní a poznaní Ahma'Treta'ji, aby bola azma'Goonová spasená. Ale dopočula som sa, že nevie azma'polodémonka náš jazyk, od Ahma'Treta'ji daný Amaz'hanam. Jazyk Amaz'han je svätý a len ním môže azma'polodémonka hovoriť v chráme Ahma'Treta'ji. Preto kúzlo uži a uč sa jazyk od Ahma'Treta'ji."$ $ Pauline bola zmätená. Netušila ako sa má cez kúzlo naučiť jazyk. Keby tam tak bola Tuliena Deľa...

"$ $No... ako?"$ $ Rozhodla sa, že sa radšej vyhradí voči tejto alternatíve, nech to znamená čokoľvek. Amaz'ha'r\v{}a sa na ňu zahľadela, ale keďže Pauline nepoznala neverbálnu komunikáciu Amaz'han, nevedela, čo jej naznačuje. Po chvíli aj prehovorila.

"$ $Azma'Goonová! Ako azma'polodémonka nemôže vedieť učenie sa cez kúzla? Je to dar od samotnej Ahma'Treta'ji, jej požehnanie, inak ako by sme mohli mať naše vedomosti, bez tohto daru?! Toto všetko komplikuje, budeme sa musieť modliť a prosiť Ahma'Treta'ji o riešenie!"$ $ Otočila sa zas ku Amaz'ha\v{}ji a niečo rozoberali, pritom Pauline zaregistrovala, že obe dosť veľa krát počas rozhovoru zakliali tak, ako predtým Amaz'ha\v{}ja. Ich plán mierne nevychádzal, aspoň tak hádala Pauline.

"$ $Týchto pár dní, kým nebude azma'Goonová aspoň trochu ovládať náš jazyk sa nebude učiť s ostatnými, ale samostatne, pričom modlitby sa bude musieť naučiť."$ $ Oznámila jej Amaz'ha'r\v{}a a dodala, kým Pauline niečo stihla namietnuť. "$ $A teraz pokračuj so mnou.“

Zabočili do jednej z chodieb a pokračovali komplexom. keď boli pri priehľadnom vchode do obrovskej sály, ozdobenej sochami a obrazmi, asi najkrajšie vyzerajúcej miestnosti, ktorú tam Pauline dovtedy videla. Bola plná Amaz'han, poväčšine kľačiacimi pred nejakou zo sôch. Pauline si stihla domyslieť význam tohto miesta, keď Amaz'ha'r\v{}a prehovorila.

"$ $Toto je chrám Ahma'Treta'ji. Ahma'Treta'ji nám toto miesto dala, pre jej slávu nech žije a pokračuje. O tomto sa Azma'Goonová viac dozvie asi ešte dnes. Do chrámu ale nemôže vstúpiť, to len Amaz'hana, ktorá už prešla druhým obradom zasvätenia. Pre ostatné Amaz'hany a pre ahma'Medizo je tu menší chrám Ahma'Treta'ji. Ten používajú aj tie, ktoré neprešli obradom zasvätenia, teda aj azma'polodémonka.“

Po jej krátkom monológu zas pokračovali. Prešli širokou chodbou hemžiacou sa menšími Amaz'hanami, zrejme deťmi. Nevľúdna atmosféra chodieb ostávala. Prešli okolo niekoľkých hál v ktorých mladé Amaz'hany cvičili boj. Tentoraz sa Amaz'ha'r\v{}a neunúvala hovoriť. Nebolo to zrejme zasvätené ahma'Treta'ji, ale pri tom, čo všetko mala ahma'Treta'ji dokázať, považovala Pauline za pravdepodobné, že ahma'Treta'ji je zasvätený celý komplex. Keď už jej boli zasvätené aj ony samotné.

Amaz'ha'r\v{}a už viac časti komplexu nekomentovala a tak si Pauline domýšľala využitie. Chrám cestou nevidela.

Zastali v jednej z miestností. Nevyzerala tak, že by sa v nej bojovalo a bola prázdna.

"$ $Tu príde azam‘Amaz'he'ria, ktorá vyučuje jazyk. Pre úctu ku Ahma'Treta'ji a pre potrebu prvej modlitby sa musí azma'Goonová naučiť základy nášho jazyka ešte dnes."$ $ Sotva dopovedala, vošla spomínaná Amaz'hana a začala lekciu bez akéhokoľvek odkladu. Pauline nestíhala vnímať. Najradšej by odtiaľ zmizla. Najradšej by bola zas so svojimi priateľmi. Keby ju tak našli...

\begin{center}
*
\end{center}

Zjavili sa na schodoch.

"$ $Sme mimo ilúzie, alebo v ilúzii?"$ $ Šepla Tulienka Deľa. Tarny pokrčil plecami. Prebehol pohľadom magické pole – mágie tam bolo požehnane, vírila sa, bola všade okolo nich, ale najväčšia koncentrácia bola niekde pred nimi.

"$ $Tu zmizla?"$ $ Spýtala sa Tulienka Deľa na jeho názor a pritom použila zmyslovú mágiu, aby ju nikto okrem Tarnyho a prípadne Pauline nemohol počuť. Tarny kúzlo posilnil.

"$ $Netuším, ale zdá sa mi to dosť možné. Zrejme nečakali, že máme zálohu.“

"$ $Nehovor o tom. Vždy môžeme byť odpočúvaní a tvoje zmyslové kúzla nie sú všemocné."$ $ Tarny by sa aj hádal, ale nebol na to vhodný čas.

"$ $Pokračujeme?“

"$ $Ale veľmi opatrne."$ $ Tarny kontroloval každú chvíľu magické pole. Koncentrácia mágie sa blížila. Vedel, že mohla obraz poškodzovať mágia, ktorú nosil pri sebe, ale tá bola tým, že bola v tabuli znížená, rovnako ako jej viditeľnosť.

"$ $Skontroloval si telepatické pole? Ak je Pauline niekde poblízku, videl by si ju.“

"$ $Že mi to skôr nenapadlo... Skúsim..."$ $ Tarny dal do prehľadávania väčšinu mágie, čo v tej chvíli nepoužíval, aby mohol vidieť čo najväčšiu oblasť. Tam, kde mala byť tá kopa mágie nebola ani duša, ale poblízku bol nejaký objekt. I jedným smerom od objektu, iným, ako tým, kde bola mágia, sa kopili postupne čoraz viac a viac objektov. Ani jeden nebol Tarnym identifikovaný.

"$ $Tulienka Deľa..."$ $ Oslovil ju.

"$ $Čo je? Vidíš ju?"$ $ Pokrútil hlavou a malá iskierka nádeje v Tulienke Deli zhasla. Ale Tarny pokračoval.

"$ $Myslíš si, že keď sa Pauline ako polodémonka zmení, redefinuje tak svoju telepatickú stopu? Lebo tá je naviazaná na konkrétnu DNA, alebo...? A démoni a polodémoni prepisujú premieňaním väčšinu svojej DNA, okrem pár informácii a mozgu. Tomu procesu nikdy nepochopím. Alebo je telepatická stopa závislá od neurológie a... Od čoho to presne je? Tam musí byť niečo, čo blokuje premenu démona na démona, inak by bol démonský gén replikovaný a...“

"$ $Sylvia by ťa asi zavraždila, keby ťa teraz počula."$ $ Prerušila ho Tulienka Deľa.

"$ $Na tom nezíde, toto je teraz úvaha dôležitá kvôli Pauline... Lebo ak je zmenená...“

"$ $Nemáš to ako zistiť, Tarn. Pokračujme radšej, udržujme obranu a hádam aj štíty. Uvidíme, Tarn. Uvidíme. Ako presne telepatické pole vyzerá, nechcem si míňať mágiu, budeme ju potrebovať."$ $ Tarny prikývol a pustil sa do opisu. Tulienka Deľa počúvala, ale po krátkom čase ho prerušila.

"$ $Kto je ten telepatický objekt, ktorý je pred nami?“

"$ $Aká je šanca, že ho poznáme? Ak som s objektom ešte nemal telepatické spojenie, tak ho telepatické pole registruje len ako objekt, to je nejaké základné nadstavenie telepatického poľa, ktoré by sme asi mohli predefinovať len kompletným prepísaním DNA ľudí aj zovero a neviem ešte koľkým rasám...“

"$ $Ale nie je to nezistiteľné, Tarn?“

"$ $Nie je, ale tak, ak je to mág, zistí, že sme tu. A to nás pripraví o výhodu,"$ $ odvetil.

"$ $Bez toho to nejde?“

"$ $Čisto teoreticky áno, ale musel by si použiť zmyslovku asi tridsiateho stupňa a tú hádam nedáva nikto.“

"$ $Zmyslovka má stupňov iba päť, nie?“

"$ $A ja dávam štyri. Ten piaty je viac mágia s mágiou, ako to preboha nazvať. Ale proste, nedávam ju a dosť ma to štve.“

"$ $A ja neviem maďarčinu."$ $ Oznámila mu Tulienka Deľa.

"$ $Lenže to nás teraz neobmedzuje!“

"$ $Chcela som ti povedať, že nemáš riešiť, čo nevieš, ale zachraňovať Pauline, ty neskutočný trt! A neber to ako urážku, dohádame sa potom. Sprav nás neviditeľnými.“

"$ $Počkaj, dočerpám mágiu, chcem ju mať na maxime. Ak uniesli Pauline, niečo po nej chcú a neočakávam, že ju dostaneme dobrovoľne.“

"$ $Logicky. Posilním obranu.“

"$ $Desím sa, kvôli tej mágii. Urob nám obom štít, radšej."$ $ Tulienka Deľa prikývla. Pokračovali ďalej.

"$ $Tá sieň je už pred nami. Aspoň na sieň sa to podobá.“

"$ $Ako je na tom mágia?“

"$ $Je to miesto, kde je objekt. A...“

"$ $Áno, Tarn?"$ $ Znepokojilo ju, že Tarny náhle zbledol. To sa nestávalo.

"$ $Že som si to skôr nevšimol... Objektov je tam viac, len sú sústredné v jednom, akoby... Toto je niečo nové, s čím som sa ešte nestretol. Ako keď písala Ló, že pokračovanie boja je v podcenení nepriateľa, či čo to...“

"$ $Písala o budúcnosti magických vojen."$ $ Opravila ho Tulienka Deľa, ale Tarnyho slová je znepokojili. Približne pochopila, čo Tarny myslel.

"$ $A o tom, že zrejme bude vytvorené kúzlo, ktoré ťa presvedčí o vlastnej presile a slabej sile nepriateľa."$ $ Tulienka Deľa sa naňho pozrela. Jej pohľad opätoval. Bol v ňom strach, zdesenie, údiv a hlavne bezradnosť. Niečo veľmi, v Tarnyho prípade, zriedkavé. Pokračoval. "$ $Je ich viac, ako som si myslel. Zvládajú zmyslovku piateho stupňa. Držia Pauline. Zrejme."$ $ To "$ $zrejme"$ $ bolo viac nádejné, ako by malo byť. Tulienka Deľa sa naňho nerozhodne zadívala.

"$ $Nemôžeme to vzdať. My sme ju do toho zatiahli.“

"$ $Toto je horšie, než sme čomukoľvek čelili. Neviem, kto sú, ale silní...“

"$ $Je to stredovek.“

"$ $Nemôžeš vedieť. Možno to je konšpirácia, ktorá drží ľudí v neistote. Možno celá corlovne Medizo neexistuje... Nie je to zmyslovka, ale... ja neviem... čo keď celá zem je...“

"$ $Nefilozofuj teraz, nepomôže nám to!"$ $ Zahriakla ho, ale sama premýšľala, čo by to mohlo byť. "$ $Môže to byť čokoľvek, asi sme sa s tým ešte nestretli... Nepísala o niečom takom Ló?"$ $ Pokrútil hlavou. "$ $Tak teda objavujeme novú rasu,"$ $ oznámila mu tónom, akoby im a Pauline nehrozilo absolútne žiadne nebezpečenstvo.

"$ $Berieš to priveľmi dobre.“

"$ $Nemôžeme sa zničiť, inak nás nezabije nepriateľ, ale naša vlastná bezmocnosť. To povedala Lukózia, keď stála tvárou pred Tretenským vojskom, ktoré bolo niekoľkokrát väčšie než oni. A vyhrala. Vidíš, musíme si veriť.“

"$ $Sme dvaja a ich je..."$ $ počítal. "$ $Dosť veľa,"$ $ dokončil. "$ $Treba mi priveľa mágie, aby som ich rozlíšil, ale je ich minimálne osem.“

"$ $To je len štvornásobná presila. Musíme ich dať.“

"$ $Priveľmi si veríš.“

"$ $A ty sa hneď vzdávaš. Každý nepriateľ má slabiny, nielen silné stránky, Leana hovorí. Je dôležité si všimnúť všetky. A zároveň si uvedomiť svoje. Rozmýšľaj. Vytvorili stredovekú krajinu, alebo je proste toto ilúzia. Sú lepší v mágii než my, teda sa nemôžeme pustiť do otvoreného magického boja. Sú v presile, ale nemusia nás čakať, ale môžu, takže s týmto nemôžeme úplne rátať, navyše vedia o nás, ak vedeli o Pauline. Ďalej, nemusíme používať len magický štýl boja, nesmieme zabúdať aj na iné. Akú máme výzbroj, Tarn? Ostalo nám vôbec niečo z Quertu?“

"$ $Z Quertu mám skryté len knihy a to v tom apartmáne, čo sme dostali. Ale, ako dar sme dostávali nejakú jednoduchú tunajšiu výzbroj, ak si pamätáš a mám pri sebe dva ich odľahčené meče, do ktorých som stihol zapracovať nejaký požierač mágie, ak sa jej meč dotkne, respektíve uchovávať, ako kov. Nemá ale veľmi veľkú kapacitu, maximálne okolo sto kilomagión... Máš ty pri sebe svoj?“

"$ $To nie, ale stihla som získať asi tri ampulky ich ochranného nápoja z rastliny vyvolávajúcej halucinácie a v spojení s mágiou spôsobuje niečo na štýl zmyslovky.“

"$ $Vezmi si jeden meč, ja beriem ampulku. Škoda, že tu nie je Sylvia, tá vie z nás šermovať najlepšie a navyše je ľaváčka..."$ $ Tulienka Deľa súhlasila, ale dodala.

"$ $Ale my nie sme úplne na zahodenie. V šerme by sme nereprezentovali, ale až tak zlí nie sme.“

"$ $Sylvii sa nevyrovnáš.“

"$ $Sylvia je hlavne iný prípad. A musí mať ideálnu náladu, aby sa ju nedalo poraziť. Ale nie o Sylvii, Pauline musíme pomôcť. Čo ešte môže byť ich nevýhoda, alebo výhoda?“

"$ $Môžeme ich zmiasť."$ $ Uvažoval Tarny. "$ $Keby som použil zmyslovku a pripravil moment prekvapenia... To je na zmyslovke tá najväčšia zbraň, keď ju nepriateľ nečaká. Nakukneme neviditeľní a potom sa zmeníme na nich, respektíve ostaneme neviditeľní a keby, zmeníme sa na nich. Odhalia nás.“

"$ $Potrebujeme tam zapadnúť, aby sme sa dozvedeli, kde je.“

"$ $Ten plán má dosť veľa chýb.“

"$ $Iný nemáme."$ $ Tarny sa zahryzol po pery a pozrel na Tulienku Deľu. Dlho spolu mlčky premýšľali a hľadeli si do očí, netušiac, či to, čo chcú podniknúť, nebude ich smrťou.

"$ $Toto je možno náš koniec, Tarn, uvedomuješ si to?“

"$ $Je to príliš skoro...“

"$ $Ale musíme ju zachrániť... ak sú tak silní mágovia...“

"$ $Ja viem."$ $ Smutne prikývol. Nikdy si nepredstavoval, že tak blízko konca bude tak skoro. "$ $Sylvia bude sklamaná.“

"$ $Sylvia sa to nemá ako dozvedieť. Aj tak sme sa nevracali príliš dlho. Je mi jej ľúto. Aj Pauline.“

"$ $Nemôžeme to vzdať, Tulienka Deľa, teraz nie.“

"$ $Teraz nám nepomôžu výroky.“

"$ $Len my sami."$ $ Doplnil ju. Chytili sa za ruky a vykročili dopredu.

\begin{center}
*
\end{center}

"$ $Ahma'Medizo."$ $ Povedala Amaz'ha\v{}ja a uklonila sa.

"$ $Počúvam, Amaz'ha\v{}ja.“

"$ $Máme málo času, ahma'Medizo, primálo. Kým sa dozvedia... Ahma'Treta'ji nás ochraňuj...“

"$ $Musíme počkať, inak to nevyjde. Žiadna z Amaz'han nemá démonské reflexy...“

"$ $Amaz'hany sú obdarené, ahma'Medizo... a čo je ona? Je to iba človek...“

"$ $Polodémonka, Amaz'ha\v{}ja, na to nezabúdajte.“

"$ $Jeden prekliaty gén."$ $ Trhla sebou a všetky jej amulety sa roztancovali.

"$ $Nemôžeme sa unáhliť, nemôžeme a Amaz'hany sa už mnohé obetovali...“

"$ $Tak pôjdu ďalšie! Pre blaho a slávu Ahma'Treta'ji s radosťou položia svoj život!“

"$ $Obetí bolo až príliš veľa. Ani jedna z tvojich najlepších Amaz'han nevyhrala ten boj. Boli príliš pomalé.“

"$ $A tá azma'polodémonka má byť hádam čo? Videla ju ahma'Medizo vôbec?! Ani Ahma'Treta'ji nemôže uctiť, lebo nevie náš jazyk. Ani zmyslové kúzla nevie používať!“

"$ $Démoni a polodémoni majú mimoriadne zrýchlené reflexy. Keď sú...“

"$ $Toľko času nemáme, ahma'Medizo. Musíme sa modliť tu Ahma'Treta'ji, aby sme vybrali tú správnu z Amaz'han..."$ $ Medizo nesúhlasila.

"$ $Treba byť trpezliví a modliť sa, Amaz'ha\v{}ja. Musíme vytrvať."$ $ Amaz'hana nespokojne triasla všetkými svojimi amuletmi a to ich bolo veľa.

"$ $Ahma'Medizo, Ahma'Treta'ji mi káže vám dôverovať, ale čo ste videli, ahma'Medizo, prosím vás. Viem, že vôľa Ahma'Treta'ji je všemocná a musíme ju počúvať, ale zaprisahám vás..."$ $ Corlovne vzhliadla na Amaz'hany po obvode miestnosti.

"$ $A tieto tvoje sestry? Im vadiť nebude, že tu pochybuješ o vôli Ahma'Treta'ji?“

"$ $Sestry kruhu sú príliš pevne zakorenené v poli Tef'eh, aby nás vnímali. Sú to strážkyne.“

"$ $A si si istá, že chceš pohybovať o slovách Ahma'Treta'ji?“

"$ $Robím všetko pre jej a našu slávu, ó Ahma'Treta'ji, ale bojím sa, ahma'Medizo. Vari to nechápete? Hrozí nám niečo zlé a bojím sa, veľmi sa bojím, že nestihneme moc načerpať dovtedy.“

"$ $Ahma'Treta'ji dosiaľ vždy stála pri vás.“

"$ $My tu konáme pre jej slávu a moc, ale pri všetkých nebesách, zaprisahávam vás, Medizo, teraz ako azam‘, aspoň prezraďte čo v tom má tlmočníčka a mág... Je to muž predsa a..."$ $ Medizo neušlo, že Amaz'ha\v{}ja zmenila oslovenie z ahma‘, teda veľmi váženého, na azam‘, čiže rovnaká pozícia a raz ju dokonca nechala bez oslovenia. To sa jej ešte u Amaz'han nestalo. Zostala však pokojná a žiadne pochybnosti nevyjadrila.

"$ $Azam'Amaz'ha\v{}ja,"$ $ oslovila ju rovnako, ako ona ju. Amaz'hany používali oslovenie s azam‘ len vtedy, ak bola dotyčná rovnakého postavenie ako ona, alebo ak to bolo niečo, pri čom si do dotyčnej vkladal preveľkú dôveru. Ahma'Treta'ji bola vždy Ahma‘. Medizo pokračovala. "$ $Pri Ahma'Treta'ji, to čo ti teraz hovorím, je proroctvo a tie sa plnia. V čase, keď som bola ešte staršou, vtedy som ho vyslovila a teraz je vo veľkej knihe, ktorú vlastnia ľudia...“

\begin{center}
*
\end{center}

Tarny nakukol do miestnosti. Nebola tam, aspoň sa tak nezdalo, žiadna brána. Prítmie. V duchu zaklial. Prítmie nemohol zrušiť zmyslovkou, ak nevidel presne, čo tam je. Obrysy rozlišoval. Stálo tam, po obvode, rovnomerne rozmiestnených, veľa strážkyň. Na to nakuknutie ich nepočítal a zaujalo ho niečo iné. Stred, či skôr koniec miestnosti bol podsvietený a jasne sa tam črtali dve postavy. Videl na nich. Jedna bola zvláštna, podľa rysov, asi rovnaká rasa, ako strážkyne a druhú spoznal. Bola to corlovne Medizo. Tarny opísal Tulienke Deli, čo videl.

"$ $Rozprávajú sa,"$ $ zrazu zašepkal Tarny.

"$ $Nahlas."$ $ Uvedomila si Tulienka Deľa. To jej prišlo dosť podozrivé.

"$ $Počúvaj, zosilním to."$ $ To chcela aj navrhnúť, lebo pre hádku by im ušla pointa rozhovoru. Natiahla uši, ale veľmi nemusela. Tarnyho kúzlo bolo vydarené.

"$ $....knihe, ktorú vlastnia ľudia, počúvaj, azam'Amaz'ha\v{}ja, lebo už nebudeš mať možno šancu, pri Ahma'Treta'ji..."$ $ Spoznali hlas corlovne Medizo. Zmyslovka v tom nebola. Očividne ich nečakali. Ale o tom nepremýšľali, lebo Medizo pokračovala. Melodicky a spevavo, prednášala proroctvo.

"$ $Pre troch bude bitka,

Dvaja ju začnú,

Jeden ukončí,

Nič sa tým neskončí,

Pre moc,

Je to prvým stretnutím,

Pri zdroji moci,

Strhne sa boj,

Pri pekle samom,

Bude tam boj,

Démonka, mág a tlmočníčka,

V meste postavenom z mágie,

Prvé, kde vládkyňa zasadne,

Mág a tlmočníčka držia spolu,

Bude to skúška,

Bude to zmena,

A určite len jedna ďalej poletí...“

Tarnymu aj Tulienke Deli napadlo v tej chvíli, čoho boli svedkami, ale nebol čas na rozmýšľanie, a tak si svoje myšlienky nechávali pre seba. Počuli, ako tá druhá, tá zelená žena s bielymi vlasmi prehovorila.

"$ $Pri Ahma'Treta'ji a pri všetkých nebesách... to predpovedá boj... on nastane, ako to Ahma'Treta'ji cez ahma'Amaz'ke'u povedala... Tie slová, sú strašne podobné... A ak sa to práve plní tak..."$ $ Tarny a Tulienka sa na seba pozreli a v očiach sa im zračil strach, že by si v proroctve ženy všimli to, čo oni. Tarny len upevnil zmyslovú mágiu a už ani nemusel počuť, čo hovorila corlovne Medizo. Vedel to.

\begin{center}
*
\end{center}

"$ $Pri Ahma'Treta'ji, nech je k nám milostivá,"$ $ vydýchla Amaz'hana. Pokľakla pred sochou Ahma'Treta'ji a hlboko sa poklonila. Jej biele vlasy, zapletené vo vrkočoch skĺzli na zem a zaprášili sa od popola, v ktorom kľačala. Ruiny starého chrámu boli všade okolo, socha sa akýmsi zázrakom nezrútila. Ale i na nej sa podpísal zub času. Bývalý monument ríše Amaz'han sa pomaly rozpadal. Celé mesto horelo. A kňažka Ahma'Treta'ji sa modlila. Desiatky Amaz'han padli. Ony, ctihodné a Ahma'Treta'ji požehnané... nemohli skončiť... Oheň horel a za sebou nechával spálenisko. To čo ostalo z ich ríše.

"$ $Ahma'Treta'ji zachovaj svoje dcéry, nenechaj zhynúť Amaz'hanam, nech položili život za teba..."$ $ Ešte viac sa sklonila. Celé jej vlasy boli od popola, tiahnuceho sa celým chrámom, v ktorom práve kľačala.

"$ $Ahma'Treta'ji, pri všetkých padlých, ktoré sú na nebesách, pri tvojej sláve, Ahma'Treta'ji..."$ $ Zaklínala. Oheň ani zďaleka ešte nedorazil do chrámu, ale už cítila jeho horúčavu. Cítila smrť, ktorú niesol v sebe.

Kňažka bola v úbohom stave. Na hrdú Amaz'hanu veľmi zlom. Oblečenie mala roztrhané, ohorené a postriekané krvou. Koho krvou, to si presne nebola istá. Topánky mala roztrhané, rovnako obhorené a deravé. Niekoľko amuletov jej spadlo a medzi zostávajúcimi a tetovaním bolo zreteľne vidno zjazvené ruky. Na tvári a krku mala ešte stále zaschnutú krv, ktorá bola pokrytá popolom, ako sa klaňala Ahma'Treta'ji. Veľká jazva sa jej tiahla spod pravého oka, cez celú tvár a končila nad ľavým uchom. Z pôvodnej, bielej farby sa jej vlasy zmenili na popolavé. Veľa vrkočov sa rozplietlo a v popole pred ňou boli popadané drahokamy a ozdoby. Nevšímala si ich a ďalej vzývala Ahma'Treta'ji v nádeji na záchranu.

Nahmatala svoj nôž. Nevypadol jej, stále ho mala za opaskom. Vytiahla ho. Uprela oči na sochu a potom ich zatvorila. Sklonila hlavu a začala odriekať.

"$ $Ahma'Treta'ji, pri všetkých padlých, nech Amaz'hany vytrvajú, nech prežije náš slávny rod, pre slávu teba, Ahma'Treta'ji!"$ $ Hovorila ešte veľa slov a pomaly dvíhala na seba nôž. Práve vtedy, keď sa chystala vykonať obetu svojej bohyni, vbehla do chrámu mladá Amaz'hana a prerušila to.

"$ $Pri Ahma'Treta'ji, tu ste, ahma'Amaz'ke'u! Hľadala vás ahma'Amaz'le\v{}gha!"$ $ Kňažka pustila nôž a otočila sa ku Amaz'hane.

"$ $Ahma'Treta'ji nás ochraňuj, pri všetkých padlých... Čo sa stalo, azma'Amaz'ha\v{}ja?!"$ $ Hľadela jej do očí a pritom naslepo hmatala po noži, ktorý upustila.

"$ $Vaše modlitby neboli márne, ahma'Amaz'ke'u! Ahma'Treta'ji nám dala pomoc a vyviazli sme. Bol to ale krvavý boj, ale prežili sme ho, na slávu Ahma'Treta'ji, stála pri nás. Len sme sa obávali, že ste zahynuli, ahma'Amaz'ke'u."$ $ Kňažky vydýchla od úľavy.

Uklonila sa soche.

"$ $Tak predsa, Ahma'Treta'ji, nedovolila si nám zahynúť... Ďakujeme ti a vzývame ťa, všetky Amaz'hany, za ktoré dnes vďačíme..."$ $ Vstala, ani nepozbierala svoj šperky a vyšla za Amaz'ha\v{}jou. Bola stále dosť zranená, ale na to nedbala. Mesto ale stále horelo.

\begin{center}
*
\end{center}

Amaz'hany po obvode bez povelu vyšli zo svojich pozícií a na ich miesta prišli ďalšie. Na prehliadnutie ich nebolo času, aj keď ich už obaja zreteľne videli. Bez mihnutia oka im zrušili celé zneviditeľnenie. Tarny nahlas zaklial a vytiahol meč. Tulienka Deľa takisto. Stáli proti asi päťnásobnej presile.

Tarny, ani Tulienka Deľa si uvedomovali aj z chabých skúseností, ktoré s ich protivníkmi, teda skôr protivníčkami mali, že nemajú šancu v otvorenom magickom boji. A rovnako ani v otvorenom, akomkoľvek boji, keď stáli pred viacerými. Chodba nebola tak široká, aby sa na ňu na šírku z kúziel zmestili viac ako tri útočiace ženy a to im nedávalo síce výhodu, ale zas to znižovalo ich ohromnú nevýhodu. Museli si ale dávať pozor, aby ich neobkľúčili pevne držať svoju pozíciu. Amaz'hany zatiaľ nevytiahli svoje meče, ale všetky tri vpredu vyčarili niečo striebristé. Chceli ísť do magického útoku, čo im ale Tarny nemal v úmysle dovoliť. Posilnil ich štít a pozeral sa len do magického poľa. Ich protivníčky bolo ľahké rozoznať cez koncentráciu mágie v nich. Tentoraz nepoužívali zmyslovku, alebo nezistili, že ich Tarny odhalil.

Tarny ani Tulienka Deľa netušili, čo za kúzlo chcú ženy urobiť. Bolo to zrejme niečo nové, čo na zemi ani na Fanase ešte nepoužívali.

"$ $Kde je Pauline?!"$ $ Spýtala sa Tulienka Deľa pozerajúc im do očí, čakajúc ich reakciu. Povedala to podvedome v jazyku, v ktorom sa rozprávali corlovne a tá druhá. Nebol to jazyk používaný v meste. Ani Vílčina, ani Fentenzíjčina. Nespoznávala ho. Pre túto otázku nepoužila zmyslovku, aby ho nepočul len Tarny.

Ženy neodpovedali. Zastali pred nim a takmer strojovo kúzlili striebristé niečo. Z poza nich sa ozvali nadávky a preklínanie. Hlas nepatril corlovne, ale tej žene. Preklínala pri nejakej Ahma'Treta'ji. A ešte nejaké slovo, ktoré nepoznala, ale podľa tónu hlasu si domyslela jeho význam.

Tulienka Deľa, vedomá si presily sa rozhodla pokúsiť sa o jeden plán, ktorý nenavrhovala. Tarny bol príliš zažratý do prezerania magickej mapy a prípravy na samotný boj, ktorý už vlastne začal, aby si všimol, čo chystá. A navyše, na hádku nebol vhodný čas.

"$ $Corlovne Medizo!"$ $ Zvolala. "$ $Neviem, prečo na nás útočíte, a čo od nás chcete, ale prišli sme v mieri, opakujem v mieri."$ $ Povedal v ich reči. Dúfala, že nejako nezohyzdila gramatiku, či slovosled a hlavne výslovnosť.

Celý jej pokus bol založený na jej osobnej skúsenosti, že rôzne rasy, či národy majú lepší vzťah ku cudzincom, keď ovládajú ich jazyk. Dúfala, že to platí i pre tie zelené ženy s bielymi vlasmi.

Amaz'hany ale, nanešťastie pre nich, považujú za úplne samozrejmé, že cudzinci vedia ich jazyk. Skôr sa pozastavia nad jeho nevedomosťou, ako vedomosťou. Corlovne Medizo sa pozrela na Amaz'ha\v{}ju a tá potom preriekla. Kúzla však ženy nezrušili.

"$ $Čo tu chcete?! A hlavne čo tu chce ten aghe'Tany, či ako...“

"$ $Tarny."$ $ Opravila ju. Netušil, čo znamená to ‚aghe‘‘ pred jeho menom, ale podľa tónu niečo maximálne opovrhnutiahodné. Jej poznámku si žena nevšímala.

"$ $Je to aghe'muž! Nemôže vstúpiť do chrámu Ahma'Treta'ji! Ahma'Medizo!"$ $ Tulienka Deľa došla na to, že majú na rovnosť pohlaví rovnaký názor, ako tí v meste.

"$ $Tarn! Zmeň sa na ženu, prosím ťa."$ $ Tarny si úplne neuvedomil, o čom hovorila Tulienka Deľa a tá žena, bol príliš zaujatý sledovaním mágie a čakaním, kedy chcú prejsť do útoku.

"$ $Čo?!“

"$ $Že tu nenávidia mužov, očividne! Tak použi zmyslovku!"$ $ Zvreskla v náhodnom jazyku o ktorom vedela, že ho Tarny vie. Nemala záujem na tom, aby im rozumeli. Tarny zaklial použil ju.

"$ $Kde je Pauline?!"$ $ Vykríkol Tarny. Neuvedomil si plán Tulienky Deli. Amaz'hana zaregistrovala a niečo vykríkla. V jazyku, ktorému Tarny nerozumel. Tulienka Deľa však áno. Počula, čo povedala Amaz'hana, a skôr než Tarny niečo stihol spraviť, odstrčila ho dozadu a vyčarila štít. Dostatočne skoro.

Amaz'hany prestali vytvárať strieborné gule, ale poslali ich na nich. Narazili do štítu a požierali ho. Tulienka Deľa všetku svoju mágiu dávala doňho.

"$ $Tarn! Okamžite vyčar štít, ja ho už neudržím!"$ $ Tarny je v okamihu vystriedal.

"$ $Čo to je za prekliate kúzlo?!“

"$ $Niečo s požieraním mágie! Nedovoľ, aby si ten štít vypol! Zabilo by nás to, alebo niečo horšie...“

"$ $Nepoužijeme dosku?“

"$ $Je v nej síce veľa mágie, ale nemôžeme ju len tak míňať, síce z nej len zanedbateľne ubudlo. Štít je v pohode kúzlo.“

"$ $Nie, keď ho súperovo kúzlo ničí! Nebudem vedieť udržiavať zmyslovku a to sa mi ešte nestalo!“

"$ $Počkaj...“

"$ $Čo robíš?“

"$ $Kódim!"$ $ Odvetila stručne Tulienka Deľa. Vystriedala ho.

"$ $Čo?!“

"$ $Teraz máš na sebe kúzlo, ktoré čerpá z ich mágie, teda ich oslabuje a ty sa môžeš celou svojou mágiou sústrediť na štít.“

"$ $Hm..."$ $ Tarny sa ponoril do magického poľa a na niečo sa veľmi sústredil. Keď chcela Tulienka Deľa vystriedať, práve vtedy, vykríkol vcelku víťazným tónom.

"$ $Mám to, Tulienka Deľa!“

"$ $Čo máš?! Poď ma vystriedať!“

"$ $Magický kód ich kúzla a zároveň som dekódoval ich telepatické pole a prenikol do neho.“

"$ $A zistil si niečo?!“

"$ $Bežme, Tulienka Deľa, bež!"$ $ Prebral udržiavanie štítu, schytil ju za ruku a začal bežať tam, odkiaľ prišli. Tým smerom, lebo z poľa sa vyjsť nechystal.

"$ $Čo je Tarn?!"$ $ Spýtala sa ho, keď chytila dych. Ich útek sa začal príliš rýchlo a ani nevedela prečo a pred kým vlastne utekajú. A nevedela v prvých sekundách lapiť dych.

"$ $Zachytil som ich telepatické pole."$ $ Chvatne začal, obzerajúc sa. Amaz'hany išli za nimi. "$ $A uvedomil som si, že proti Laseru a Solanu štít neplatí a nás je príliš málo..."$ $ Tulienka Deľa sa obzerala tiež, ale keď nevidela žiadne červené, či žlté záblesky a nepočula zaklínadlá, bežala ďalej, ale aj tak o trochu rýchlejšie.

"$ $To pole bolo tak slabé?"$ $ Zaujímalo ju medzi dychom.

"$ $Nie, ale to kúzlo ich oslabovalo. Dosť, tak sa ich telepatická správa dala prečítať. Prenikla. Ale mimo toho, štruktúra a druh toho poľa je tak zvláštny, úplne iný ako v prípade zovero a ľudí. Neviem ako, ale to pole je tak zvláštne prepojené, ale zdá sa mi, že skôr umelo, ako prirodzene, ale nie som si istý."$ $ Tulienka Deľa ho prerušila.

"$ $Tarn! Kde to sme? Tadiaľto sme nešli, lebo sme boli v realite, ale teraz v ilúzii...“

"$ $Toto sa mi nejako nezdá, ako ilúzia, ale...“

"$ $Tak uvažujme, že to ilúzia je!“

"$ $Dobre..."$ $ Bežali ďalej, ani jeden nevedel smer a cieľ, len čo najďalej od Amaz'han.

"$ $Kde sme?!“

"$ $Ja neviem! Ako by som to mal vedieť.“

"$ $Ja viem, kde sme Tarn! V kaši! V priveľkej kaši, aj na nás!"$ $ Obzrela sa a spomalila. "$ $Musíme zastať. Nevieme kde sme, a či nelezieme do ešte väčšej jamy levovej.“

"$ $A čo chceš robiť?"$ $ Pokračovali trochu voľnejším tempom, sily s dodávali mágiou a pravidelne sa obzerali za ich prenasledovateľkami.

"$ $Máme mágiu. Dokážeš vytvoriť ilúziu?“

"$ $Prekuknú ju.“

"$ $Ak si si mohol všimnúť ich fungovanie, sami od seba len tak neútočili. Bolo to ako popisuje Ló, že spojené vedomie, Har'ga. Bude im to chvíľu trvať.“

"$ $Budiž. Smeruj tok mágie, chceme silný základ a zároveň aj kryt. Zmyslovka štvrtého stupňa."$ $ Tarny sa ponoril do mágie a hľadal miesto, čo najrýchlejšie ako mohol, kde by založil ilúziu. Zastali, lebo sa mu zle za behu sústredilo. Nevedel, aký mali náskok, ale nečakal, že veľký. Pohľad do magického poľa ho prekvapil. Okolo nich nebola len jedna veľká realita, alebo ilúzia, z ktorej prišli, ale i veľa ilúzií, medzi ktorými existovali prieniky, boli premostené bránami ako inopolia. Keby neboli v smrteľnom nebezpečenstve, rád by tú magickú techniku preskúmal. Neboli ani zakryté žiadnym kúzlom, bolo ich zreteľne vidieť. Nečakali ich, bolo to u nich bežné, alebo to bola pasca. Nemal čas premýšľať, vytvoril skopírovanú ilúziu, obmedzenú na veľkosť chodby a preniesol ich do nej. Akonáhle v nej boli, za chvíľu vykúzlil i to, aby ich ilúziu neboli len tak vidieť a chcel povedať Tulienke Deli o tom, čo videl, ale ona si to všimla tiež.

"$ $Videl si tie ilúzie? Vyzeralo to niečo ako iluzórne inopolia, umelo vytvorené inopolia. Majú zďaleka lepšiu magickú technológiu ako my."$ $ Obaja dočerpávali mágiu. Z kovu ani po toľkých odčerpaniach mágia neubúdala.

"$ $Samozrejme, že videl. Keby sme mali viac času... Bolo by dobré sa im stratiť vo viac než jednej ilúzii. Problém je, že musíme zachrániť Pauline. Možno už ani..."$ $ Tulienka Deľa na jeho pochmúrne úvahy rázne, ale nie úplne presvedčivo, pokrútila hlavou.

"$ $My sme sa tam dostali nechcene. Ona očividne chcene. Aj si počul o čom hovorili corlovne a tá... tá z nich."$ $ Tarny neurčito prikývol.

"$ $Štaglója, prečo sme to tu chceli preskúmať?! Ono tu je tak veľa realít a zdá sa mi, že táto civilizácia... netuším kto sú, ale...“

"$ $Ja to beriem rovnako, Tarn."$ $ Tarny chvíľu premýšľal, obaja boli v tichu. Nedostali ich ešte, ale mohli tušiť, že o pár minút, ak nie okamihov, budú nahratí.

"$ $Čo ak tu dôjdu a použijú Solan, alebo Laser. Chcú sa nás zbaviť.“

"$ $Je tu veľa realít.“

"$ $To je príliš odvážne, Tulienka Deľa a toto je... musím to povedať, i keď veľmi nerád, až príliš zlé.“

"$ $Do ďalšej reality im neujdeme. To by prekukli. Alebo tá, čo ich riadi. Druhý raz rovnaký trik nepoužijeme.“

"$ $Pauline sa aspoň dokáže zamaskovať. My nie... Zmyslovku odhalia...“

"$ $Ak ju budú očakávať."$ $ Pokrútila hlavou.

"$ $Muselo by to byť fakt silné kúzlo, aby ho nebolo len tak vidno.“

"$ $Kov."$ $ Odvetila. "$ $Ak by sme vytvorili niečo, cez čo by permanentne prúdila mágia...“

"$ $Prúd mágie je ľahké odhaliť.“

"$ $Zmyslovka. Pri štvrtom stupni je už treba odhadovať a hľadať. Nenájdeš ju len tak.“

"$ $To je fakt. A musíme ju nájsť. Nielen sa skrývať. Keby sme mali nejaké hlagenove pole... Najskôr by sme sa mali zneviditeľniť a odpozerať ich zvyky.“

"$ $Ich jazyk asi zvládam, keďže som rozhovoru tej jednej a Medizo rozumela..."$ $ Odvetila Tulienka Deľa. "$ $Ale nevieme takmer nič o ich kultúre. To by bola chyba, keby sme sa hneď zmenili za nich.“

"$ $Súhlasím. Musíme sa dostať do nejakej ich ilúzie. Ale pred tým sa zneviditeľníme a použijeme zmyslovku štvrtého stupňa."$ $ Tulienka Deľa prikývla. Vykúzlili všetko tak, ako chceli, ani ich zatiaľ ich prenasledovateľky nedostali.

"$ $Pripravená?"$ $ Prikývla. Preberala si pritom v hlave ich jazyk. Nevedela presne čo to za jazyk je, ale hlavné bolo, že ho ovládala. To ich zachránilo vtedy, že príkazu tej jednej porozumela. Pozrela sa na Tarnyho. Jej kúzlo, ktorým jeho výzor zmyslovou mágiou premenila na dievča urobila v chvate a tak to bolo kúzlo na všetkých, vrátane jej.

"$ $Tarny?“

"$ $Hm...“

"$ $Ako dievča vyzeráš aj dobre. Mohol by si si to nechať aj dlhšie.“

"$ $Choď do kelu, Tulie.“

\begin{center}
*
\end{center}

V žiadnu Ahma'Treta'ji Pauline neverila, ale radšej to nedávala najavo. Pár slov modlitby, ktorým takmer nerozumela, do nej natlačili v tom zhone za pár minút. Ale z jazyka ako takého si nepamätala nič. Spolu s Amaz'ha'r\v{}ou a ďalšími dvoma mladými Amaz'hanami niečo odriekali. Z ich reči porozumela len – tak slovu Ahma'Treta'ji.

Malý chrám bol vyzdobený zlatom a inými drahými kovmi. Veľký – hlavný chrám bol podľa krátkeho náhľadu vyzdobený ešte viac. Ani v jednom nevidela žiadne miesto na sedenie – u Amaz'han neboli žiadne dlhé a skupinové obrady, a pri všetkých ich obradoch aj tak poväčšine kľačali pred nejakou tou sochou ahma'Treta'ji.

Sochu Ahma'Treta'ji zobrazovali ako vysokú, jasnomodrú a čiernovlasú ženu, rysmi sa podobajúcu Amaz'hanam. Vlasy mala na každej soche v zložitom účese, v zrovnaní s ňou mali Amaz'hany zjav niekoho, kto sa svoje vlasy nestaral. Bola celá v zlate a striebre, plášť jej pokrývali diamanty a strieborné brnenie bolo zas ozdobené rôznymi motívmi, zdobiacimi i telo Amaz'han. V rukách držala meč. Jeho vzhľad takisto nezaprel vkus Amaz'han pre drahokamy a drahé kovy. Rukoväť bola bohato ozdobená malými diamantmi a po čepeli zas boli tie isté obrazce, ktoré si na seba Amaz'hany tetovali. Nad druhou rukou jej doslova visela vo vzduchu guľa, ktorá mala predstavovať kúzlo. Amaz'hany zjavne začarovali guľu, aby sa znášala a rovnako boli okolo nej vo vzduchu lúče, ktoré mali vyvolávať obraz kúzliacej Amaz'hany. Pauline musela uznať, že sa jej socha páčila. Ako socha. Keby ju videla v nejakej galérii, považovala by ju za mimoriadne umelecké dielo. Takto boli sochy Ahma'Treta'ji okolo nej a prestala si ich všímať. Namiesto toho iba zatvorila oči, tvárila sa, že sa modlí, ale v skutočnosti iba dúfala, kedy sa to skončí.

\begin{center}
*
\end{center}

"$ $Necítite mágiu, ahma'Amaz'ke'u?"$ $ Spýtala sa starej kňažky pokorne mladá novicka. Bola to, presnejšie povedané, ešte nezasvätená kňažka, ktorá pomedzi svoje povinnosti v chráme pomáhala Amaz'ke'u a slúžila v armáde. Nezasvätené kňažky u Amaz'han tvorili veľkú časť populácie, ale právoplatnými kňažkami Ahma'Treta'ji sa stal len malý zlomok z nich a velekňažka len jedna. Už dlho bola velekňažka Amaz'ke'u, že si už takmer nikto inú nepamätal – bola to takisto jedna z najstarších Amaz'han a určite najváženejšia. Bolo ich pár desiatok, ktoré si pamätali osobne krvavé víťazstvo Amaz'han, ktoré takmer znamenalo zánik ich ríše, ktoré pre ne, podľa verzie všade prezentovanej, vymodlila Amaz'ke'u.

"$ $Tá je všade okolo, amza'Amaz'g\v{}hle'a,"$ $ odvetila Amaz'ke'u, ale predsa len postrehla úzkosť v jej hlase, vycítila, že niečo nie je v poriadku.

"$ $Je to tu všade vo vzduchu, visí to tu ako hmla jedu, ktorá môže kedykoľvek padnúť na nás a otráviť..."$ $ Amaz'ke'u mlčky prikývla. Poznala ten pocit, z dôb, kedy ešte nebola velekňažkou a keď ríša Amaz'han sa rútila.

\begin{center}
*
\end{center}

Amaz'le\v{}gha zomierala. Ako velekňažka Ahma'Treta'ji prežila desiatky rokov a napokon takmer zlyhala – minimálne osobne. Bola presvedčená, že to nebola ona, kto vymodlil ich víťazstvo a dúfala, že tá dotyčná ešte stále žije. Verila však, že keby zomrela, Ahma'Treta'ji ju už hostí v nebesách. A o chvíľu sa aj ona za svojou bohyňou poberie... Ale ešte predtým...

Dala si zavolať Amaz'ke'u. Videla, ako uteká z mesta, ale bola presvedčená, že nie zo zbabelosti. Poznala ju dosť dobre, ale v ohrození života... Na druhú stranu, Amaz'ke'u bola jedna z najlepších kňažiek...

Nevedela, či vôbec žije, ale hlavne, či vôbec stihne prísť. Rýchlo jej ubúdali sily, jej pôvodne čisto biele vlasy boli postriekané krvou, tá presakovala i cez nohavice. Dostala škaredú ranu do stehna a nikto ju nevedel opraviť kúzlami – bola to rana spôsobená agresívnou mágiou a tá nešla len tak liečiť. Laser ju zasiahol na rukách, rany si síce rýchlo zahojila, ale aj tak boleli a tratila príliš veľa krvi. Nepriateľ ich zasiahol príliš prudko a bol vo významnej presile, že takmer prehrali. Z posledných síl si vydobyli aspoň ruiny ich krásneho mesta. Sochy veľkej Ahma'Treta'ji boli popadané a nikto nemal síl ich zdvihnúť.

"$ $Ahma‘Amaz'le\v{}gha!"$ $ Zvolala Amaz'ke'u, akonáhle ju zbadala. Nebo, to výkrik radostný, ani nahnevaný. Amaz'ke'u zvolala z úľavy, že velekňažka stále žila, aj keď na pokraji síl.

"$ $Vyhrali sme... ahma'Amaz'ke‘u... Ahma'Treta'ji nás zachránila... Buď... ahma'Ke'u..."$ $ Zatvorila oči. Amaz'ke'u čakala, kým ešte niečo povie, ale potom si uvedomila, že velekňažka je mŕtva. A nestihla ustanoviť... To sa ešte predsa...Premýšľala, ale vtom si uvedomila, ako ju Amaz'le\v{}gha oslovila. Keď velekňažka ju oslovila ahma‘ znamenalo to, že... Prudko sa nadýchla. Stala sa velekňažkou.

\begin{center}
*
\end{center}

"$ $Viete, ahma'Medizo, čo sa stalo, keď... Ahma'Amaz'ke'u sa stala velekňažkou a my sme sa museli utiahnuť sem. Ak ten prekliaty mág nájde to, ako funguje mesto, pri Ahma'Treta'ji, nájdu nás, Medizo. Ríša Amaz'han je...“

"$ $Amaz'ha\v{}ja! Ja na to dobre myslím a vie o tom už aj ahma'Amaz'ke'u. Upozornili sme ju. Novicky chystajú obranu a...“

"$ $Polodémonka je stále nevyužiteľná. Máme málo času, ale Ahma'Treta'ji musela vidieť, čo robí, keď nám ho dala.“

"$ $Ahma'Treta'ji nás ochráni."$ $ Odvetila Medizo pokojne. Sama nebola veriaca v Ahma'Treta'ji ale v záujme mieru a spolupráce s Amaz'hanami ňou navonok bola.

"$ $Ahma'Treta'ji tvoríme slávu a chrámy tu a ona nás odmení. Musíme zachovať impérium. Už nás raz ochránila a... Nemôžeme sa spoliehať len na to, že nám ahma'Amaz'ke'u vymodlí ďalšie víťazstvo.“

"$ $Neveríte svojej velekňažke?“

"$ $Pri Ahma'Treta'ji! Pochybujete o mojej viere, ahma'Medizo?! Ahma'Treta'ji nás požehnala a dala nám z jej vôle požehnanie a vyvolenosť, ale mi jej musíme slúžiť a prinášať obety a bojím sa, bojím sa, že toto je až príliš. Nech ma Ahma'Treta'ji netrestá, ak pochybujem o jej veľkosti, ale... oni to vedia... vedeli to už vtedy.“

"$ $Ale my budeme mať Gl'estuvele'i... Od Ahma'Treta'ji...“

"$ $Pri Ahma'Treta'ji, nevyslovujte to meno nahlas!"$ $ Zdesila sa Amaz'ha\v{}ja. "$ $Mág a tlmočníčka boli predzvesťou!“

"$ $Zmobilizujte svoje sily, Amaz'ha\v{}ja. A ja vyzbrojím mesto. Pripravme sa na útok."$ $ Amaz'hana slabo prikývla, zmietaná vlastnými pochybnosťami a strachom, miešajúcimi sa s vierou a dôverou ku Ahma'Treta'ji. "$ $A nájdeme ich."$ $ Pokračovala corlovne. "$ $Mága aj tlmočníčku. Oni si prišli po polodémonku. Ale aj ich potrebujeme."$ $ Amaz'hana mlčala, ale po chvíli ticha prehovorila s pohľadom upretým do prázdna.“

"$ $Unikli strážkyniam, pri Ahma'Treta'ji, oni unikli nám... Sth'agh\v{}hel... toto...“

"$ $Mali pri sebe mágiu. Musíte ich dostať, Amaz'ha\v{}ja, spolieham sa na vás. Ja musím ísť do mesta, pripraviť ho na vojnu."$ $ Amaz'hana prikývla, uklonila sa a rýchlym krokom mierila do chrámu.“

\begin{center}
*
\end{center}

"$ $Amah'Amaz'o'en, prišli ste sa pomodliť za naše víťazstvo?"$ $ Tarny vážne prikývol. Tulienka Deľa tam niekde bola ale nevedel ju rozlíšiť v záplave Amaz'han. Neodvážili sa nechať zmyslové kúzlo len na Amaz'hany, pretože tak existovala väčšia šanca na odhalenie. Akonáhle videl nejaké netypické magické pole snažil sa nájsť Pauline, ale nedarilo sa mu. Bola polodémonka a to činilo ich cieľ za veľmi ťažko dosiahnuteľný. Namiesto narúšania mágie sa tak súhlasil s Tulieiným spôsobom hľadania – infiltrovania sa do ich kultúry a spoliehanie sa na to, že je pre nich Pauline veľmi dôležitá. Mal problém s tým, že nevedel ich jazyk a svoj univerzálny prekladač dal Pauline. Tak sa držal pri Tulienke Deli a snažil sa niečo z jazyka pochytiť, i keď vedel, že nemá šancu hovoriť jazykom, ako keby bol jeho materinský – čo nevysvetliteľne dokázala Tulie.

Tá po asi polhodine pozorovania a pol hodine kopírovania Amaz'han vcelku obstojne dokázala viesť rozhovor bez toho, aby mali Amaz'hany akékoľvek viditeľné podozrenie. Považoval to za viac zázračné než samotnú mágiu, keďže ako mu stihla narýchlo vysvetliť, používali veľa oslovení, ktoré by ich pri nesprávnom použití mohli hneď odhaliť. Stále sa jej síce nepodarilo odhaliť a pochopiť význam niektorých zvykov, napríklad tetovania, ich rovnakého výzoru, či úplne celú hierarchiu. Ale pokročila tak, že dokázala nájsť pre Tarnyho dostatočnú výhovorku, keďže netušil jazyk.

Amaz'hany mali niekoľko spoločenských skupín a medzi najvýznamnejšie patrili kňažky a po nich bojovníčky, pričom tieto skupiny mali významný prienik – takmer všetky novicky neskôr odchádzali do armády. Medzi kňažky tiež patrili takzvané Amah‘, ktoré obetovali pre Ahma'Treta'ji svoj sluch a jazyk, aby im bohyňa posilnila ostatné zmysly a dorozumievali sa takmer výlučne telepaticky. Tarny si jednoducho mohol zo svojej Amaz'hanej odčarovať jazyk a bol už Amah'Amaz'o'en. To meno mu vybrala Tulie tak, aby nebolo ani príliš na smiech, ani príliš vznešené. Ona sama sa v ríši Amaz'han volala Amaz'hu'm a bola novicka.

Jedna z vecí čo si Tulienka Deľa všímala boli mená. Slovo Amaz'hana v ich jazyku znamenalo požehnaná žena a slovo Amaz‘, teda požehnaná si kládli aj pred každé meno. Obrad pomenovania bol jedným z najdôležitejších obradov v ich tradícii. Práve pri ňom strácali Amaz'hany svoju prirodzenú podobu a stali sa zasvätenými bohyni Ahma'Treta'ji. Vtedy tiež pred ich meno začali dávať Amaz‘. Samotné mená Amaz'han boli zloženinami ich bežných slov, rovnako ako vo Fentenzíjskej či vílskej tradícii. I jazyk mali mierne podobný.

Obrady, ktoré Amaz'hany vykonávali boli poväčšine všedné, ale raz za čas prišiel jeden z veľkých obradov. Medzi tie patril aj obrad pomenovania. Ďalším z nich bol tiež obrad zasvätenia, pri ktorom sa mladé Amaz'hany stávali novickami. Týmto si prešli takmer všetky Amaz'hany. Vtedy prestali byť pre bežnú populáciu azma‘ – žiačka. Obradom pomazania prechádzali len tie Amaz'hany, ktoré sa stávali kňažkami. Kňažky boli vo väčšine prípadov aj vojačky. Tým, že sa stali kňažkami získali bežné Amaz'hany príponu ahma‘. Posledný obrad, ktorý, ako sa Tulienka Deľa dozvedela, sa konal naposledy tak dávno, že si to pamätajú iba najstaršie Amaz'hany, bolo to na konci nejakej veľkej vojny. O nej počula už za tú chvíľu, čo u Amaz'han bola, viac než jej prišlo normálne. Amaz'hany sa o nej rozprávali medzi sebou a všade vo vzduchu bol cítiť nepokoj. Nezdalo sa jej to normálne, ale úplne nebola zasvätená do ich kultúry. V chrámoch sa tvorili hlúčiky a bol z nich cítiť strach, napriek tomu, že v každom slove vyzdvihovali veľkosť ich bohyne a to, aké sú ony požehnané.

Nenápadne sa pokúšala dostať reč na povrch, ale Amaz'hany o ňom vraveli s pohŕdaním. O návštevníkoch – polodémonke, mágovi a tlmočníčke nepadlo ani slovo, ale Tulienka Deľa sa nevzdávala. Na Tarnyho chcela mať stále dohľad a Tarny na ňu, ale po chvíli sa prestali rozoznávať a nadobro sa rozdelili.

Tulienka Deľa sa dostala do spoločnosti pár noviciek, ktoré kráčali do väčšieho z chrámov. Viedla ich kňažka nerozoznateľného veku, s tetovaniami po celých rukách a tvári. To, že to je kňažka poznala iba z úcty, ktorú jej ostatné novicky preukazovali.

Význam tetovaní bol pre ňu záhadou. Keď sa na ne pozrela pozornejšie, nenašla dva úplne rovnaké, ale celkovo sa jej zlievali do seba a nevedela by ich rozlíšiť. Sama sebe a Tarnymu vymyslela tetovanie podobné iným zo skupiny, ktorou sa mali stať. Nevedela v nich nájsť žiadny vzor ani pravidlo, ani symboliku. Napriek tomu si nemyslela, že sú čisto náhodné – na základe vtedajšej túžby ich nositeľa. Zásadovosť a poriadok Amaz'han jej nedovolil myslieť si to.

\begin{center}
*
\end{center}

Pohreb Amaz'le\v{}ghy prebehol podľa starodávnych tradícií. Amaz'ke'u ako nová velekňažka zapálila telo a popol sa rozsypal po celom chráme. Chrám bol stále zničený z vojny a stále sa z neho dymilo. Amaz'hany vždy pochovávali mŕtvych v deň ich smrti, takže žiadne opravy sa nestihli vykonať. Bol to chrám, kde Amaz'ke'u vymodlila ich víťazstvo.

Amaz'hany sa modlievali na popole svojich predchodkýň. Tie sa tak dostali ku Ahma'Treta'ji a tá ich požehnala v poslednom obrade. V to verili.

Po obrade sa velekňažka odobrala modliť k druhému chrámu, tomu, poblízku ktorého Amaz'le\v{}gha skonala. Pomedzi modlitbami sa stále pokúšala vyrovnať s poslednými udalosťami, ktoré zbehli tak rýchlo... stala sa velekňažkou, Amaz'le\v{}gha zomrela, s veľkými stratami zvíťazili a ostalo im polozhorené, polozbúrané mesto, veľa zranených a mŕtvych. Bola vďačná Ahma'Treta'ji za ich víťazstvo, ale popravde, desila ju všetka zodpovednosť ktorú na seba musela teraz zobrať. Ich nepriatelia sa isto vrátia a oni musia vybudovať mesto. Obnoviť sily. Obnoviť ríšu. Pozdvihnúť slávu Ahma'Treta'ji. Vztýčiť jej sochy po celom svete a stať sa tým, na čo ich predurčila. A namiesto toho im teraz zachraňovala holé životy, svoju velekňažku si nechala zomrieť, Ahma‘Treta'ji, a na jej miesto si posadila mňa, prečo, Ahma'Treta'ji? Pomyslela si. Bola to česť, stať sa velekňažkou, ale všetky, úplne všetky boli na to nejakú dobu pripravované, učili sa učenie ahma'Knihy a...

Celú noc sa modlila nová velekňažka v chráme, kým vyšla medzi Amaz'hany. Boli zničené, popálené, zranené, ale verili v jej požehnanie. V silu Ahma'Treta'ji.

Mesto sa obnovovalo príliš pomaly. Obnova neprichádzala a síce sa Amaz'hany modlili ako veľmi mohli, pracovali ako stíhali, ale starý chrám stále nebol obnovený. Amaz'ke'u klesla pred sochou Ahma'Treta'ji a ako každý večer.

"$ $Ahma'Treta'ji, pomôž nám, tvojím velebiteľkám,"$ $ ticho začala. Pozdvihla zrak. Socha Ahma'Treta'ji tam stále bola. Neporušená, nepohnutá. Amaz'ke'u však verila, že ju bohyňa počuje. Začala zas. "$ $Ahma'Treta'ji, prosím ťa, pre všetko čo mám, pre všetko, prosím, Ahma'Treta'ji, tvoja sláva nech je naším budovaní, pri všetkej moci tvojej, vráť nám silu, nech vrátime ušlé životy nás, Amaz'han, pre tvoju všetku slávu, pre všetko, čo si nám udelia. Krvou nepriateľa pokropíme tvoje oltáre, len nám daj silu, Ahma'Treta'ji, tvoja velekňažka ťa prosí, na navrátenie sily Amaz'hanam. Dala si nám víťazstvo, keď sme strácali nádej, ale nie vieru v teba a ty si nás odmenila, tak odmeň nás aj teraz, Ahma'Treta'ji."$ $ Ešte dlho kľačala Amaz'ke'u pred sochou Ahma'Treta'ji. Začali už aj svietiť mesiace a hviezdy a len tie osvetľovali kľačiacu Amaz'hanu.

Napokon vstala. Oprášila si vlasy sivé od popola, ten si striasla z rúk. Nebolo vidieť takmer ani na krok, len sčasti osvetľovalo chabé osvetlenie sochu Ahma'Treta'ji. Prešla pár krokov ku východu a vtedy sa otočila. Len tak, bez dôvodu, zo zvláštnej intuície. A vtedy zrazu uvidela prudký záblesk svetla vzchádzajúci zo zeme. Najskôr znehybnela, vydesilo ju to. Svetlo zmizlo. Mohla odísť a predstierať, že sa jej to zdalo, že to nič nebolo. Ale na to ju to príliš vydesilo a nebola Amaz'hanou, ktorá by len tak utiekla. Bola predsa velekňažka. Podišla bližšie. V tme bolo všetko rovnaké a tak išla len po pamäti, kde záblesk bol.

Vedela, prirodzene, že sa mohlo jednať o niečo nebezpečné. Respektíve, mohlo, nebol dobrý výraz. Bolo pravdepodobné, že sa jedná pascu, vzhľadom ku množstvu nepriateľov, ktoré Amaz'hany mali.

Keď bola na mieste, kde vedela, že približne tam mal záblesk byť, ucítila obrovský prúd mágie. Ostala stáť na mieste, vybudovala si magický štít a postupovala ďalej. Meč netasila – voči mágii bol zbytočný.

Išla bližšie, už cítila zdroj veľmi dobre, aby identifikovala presne jeho polohu. Ten záblesk pochádzal z mágie. Pri veľkom množstve mágie sa pravdepodobnosť takýchto javov zväčšovala, ale musela byť obrovská mágia aby bola pravdepodobnosť reálna.

A teraz bola pri zdroji. Mágia prekypovala a ona cítila ako prúdi. Mala štít, ten ju chránil. Jedna vec jej ale nešla do hlavy, mimo iných. Prečo mágiu nebolo cítiť v celom chráme? Čo ju drží? Odkedy? A ako?

To jej nedalo len tak odísť. Musela ísť úplne ku zdroju. Musí predsa...

Zosilnila štít. Mágia vytvárala kratšie či dlhšie svetelné záblesky. Tie jej osvetlili dieru. V zemi bol tunel, z ktorého ústila mágia. Predrala sa do neho a vošla.

"$ $Ďakujeme ti, Ahma'Treta'ji,"$ $ hlesla.

\begin{center}
*
\end{center}

"$ $Pohotovo!"$ $ Vykríkla Amaz'hana, ktorá ju učila bojovať. Pauline sa však otočila neskoro a schytala tvrdou palicou do boku.

"$ $Rýchlejšie!"$ $ Zas ju okríkla Amaz'hana. Tentoraz zaútočila z úplne inej strany ako obvykle a Pauline zostala stáť, hľadajúc jej zbraň, kým nebola zas zasiahnutá.

"$ $Pri Ahma'Treta'ji! Hovorím odraz! Uhni sa! Nechceš sa nechať zasiahnuť!"$ $ Paulin nebola rýchla. Nikdy. Zbraň sa jej zdala príliš ťažká a brzdila ju. Amaz'hany boli do toho navyše nadľudsky rýchle. Ona nestihla ani zaregistrovať, na ktorú stranu útočí jej protivníčka. Začalo jej chýbať modlenie. Síce v ňom nevidela zmysel, tam ju nikto aspoň neudieral.

"$ $Načo máš tú zbraň? Nech azma'polodémonka začne využívať svoju palicu."$ $ "$ $Je ťažká."$ $ Hlesla Pauline skôr, než sa zastavila. Na tvári Amaz'hany sa objavil neidentifikovateľný výraz.

"$ $Azma'Goonová je polodémonka. A čarodejnica. Načo sú tu magióny?!"$ $ Pauline bola bezmocná. Netušila ako. Zväčšiť – zmenšiť, to by vedela, ale hmotnosť ostávala stále rovnaká. A nepotrebovala mať zbraň inej veľkosti. Ostala bezmocne stáť a pokrčila plecami.

\begin{center}
*
\end{center}

"$ $Ahma'Medizo!"$ $ Prihnala sa rýchlo Amaz'ha\v{}ja ku corlovne, prichádzajúcej z mesta. "$ $Corlovne! Pri Ahma'Treta'ji, ešteže ste už tu!“

"$ $Čo ahma‘velekňažka? Boli ste?“

"$ $Ahma'Amaz'ke'u sa modlí a zvoláva armádu. Na mieste, kde je..., tak sa ešte nikto neodvážil vstúpiť, zapečatili sme to. Pri Ahma'Treta'ji...“

"$ $A čo polodémonka?“

"$ $Je nanič. Je horšia než všetka naša armáda, nevie sa ovládať. Ani náš jazyk neovláda, úctu ku Ahma'Treta'ji nevyznáva. Prečo práve ona, ahma'corlovne...“

"$ $Viete to a pri Ahma'Treta'ji... nemôžu zvíťaziť.“

"$ $Sú príliš blízko.“

"$ $Je to pár dní cesty a ak ich zdržíme v ilúzii...“

"$ $Bude to koniec, ahma'Medizo...“

"$ $Bude to víťazstvo. Prečo by dala Ahma'Treta'ji svojmu ľudu striebro, keby ich víťazstvo by nepožehnala. Mesto už barikády buduje, na boj sa chystá.“

"$ $Sú tam aghe'muži, a tí, čo mágiou nie sú požehnaní...“

"$ $Zomrú. Práve preto sú prvá línia, Amaz'ha\v{}ja.“

"$ $Zdržia ich. Aspoň dúfajme. Nech Ahma'Treta'ji žehná...“

\begin{center}
*
\end{center}

Corlovne Medizo, zvyčajne odetá v honosnom odeve, ktorú jej poddaný vídavali len v šatách, teraz stála pred mestom ako veliteľka armády. Jej korunu vystriedala prilbica a namiesto šiat bola v bojovom obleku ušitom podľa vzoru Amaz'han a v rukách držala meč. Mesto bolo pred vojnou.

Stála na okraji vyvýšeného nádvoria, pod ktorým stáli jej vojaci, mágovia a radoví občania. Pripravovala sa na prejav. Bola vládkyňa. A toto bolo jej mesto.

"$ $Mesto Tra'itja chcú napadnúť nepriatelia, ktorí ale nikdy nepočuli o sláve a moci nášho ľudu, chcú nám vziať naše mesto, ale my ho nedáme! Deti tabule a občania mesta nevydajú ho do rúk nepriateľa, ktorý nás chce zotročiť a zničiť. On ešte nevie o našej sile, on ešte nevie o našej sláve, ktorú mu ukážeme. Kto chce nám vziaľ naše múry a naše bohatstvo? Kto chce nás zotročiť a získať nadvládu? Kto? Ten, kto ešte netuší o moci nášho mesta a odboji našich vojakov. Oni chcú prebúrať dieru do nášho mesta, ale my prebúrame dieru do ich šíkov a zoberieme im ich túžby o nadvláde. Mesto troch tabúľ nepokľakne pred dobyvateľom, ale on pokľakne pred nami. On útočí, ale my mu to vrátime. Mesto troch tabúľ nezhynie!“

\chapter{Veľká zbraň}

To čo spozorovala považovala najskôr za prelud. A potom uverila, že jej to zoslala Ahma'Treta'ji.

Mágia sa správala väčšinou poslušne, ale obrovských množstvách bolo jej konanie nepredvídateľné a zvláštne. Vytvárala svetlá, postavy, svety... Amaz'hany si mysleli, že ju skrotili, ale nebolo to nič viac, než ilúzia. Ozajstnú mágiu nikto neovládal, ani Amaz'hany, ani ich nepriatelia. Ozajstná mágia žila svojím vlastným životom. A pod troskami z ríše Amaz'han vytvorila vlastný svet.

Lenivo sa hýbala po chodbách podzemných siení, rýchla prechádzali chumáče z jedného konca na druhý, čarovali svetielka, vytvárali obrazce. Amaz'ke'u sa musela silno sústrediť na udržanie štítu, lebo mágia do nej narážala, chcela vyplniť priestor tesne okolo nej, čo by ale znamenalo smrteľné následky. A zomrieť Amaz'ke'u ešte nechcela a Ahma'Treta'ji to tiež určite nechcela, inak by ju neustanovila velekňažkou a nekázala jej tento zázrak.

Postupovala chodbou dopredu, kým nenarazila na sieň. Tá jej takmer vyrazila dych. Okrúhla miestnosť plná mágie s ôsmimi vchodmi do ďalších chodieb. Amaz'ke'u sa dlho nerozhodovala, išla rovno dopredu. Napriek tomu, že v posledných týždňoch takmer nespala, bola teraz čulá a nič ju nepresvedčilo na to, aby o svojom objave išla povedať Amaz'hanam a prestala sa prechádzať chodbami vytesanými do skaly mágiou.

Spleť chodieb stále pokračovala a neprestávala, ale Amaz'ke'u nemala pocit, žeby sa strácala. Jej štít slabol a slabol a ona to nevnímala...

\begin{center}
*
\end{center}

"$ $Ahma'Amaz'ke'u, pri Ahma'Treta'ji!"$ $ Velekňažka sa otočila.

"$ $Polodémonka? Kde je, Amaz'ha\v{}ja?"$ $ Obrátila sa Velekňažka ku nej.

"$ $Nie je, ahma'Amaz'ke'u, je ako obyčajný mág z jej rodu, ani svižnosť Amaz'han nie je u nej? Prečo dôverujete, ahma'Amaz'ke'u ahma'Medizo? Je len ona azam'zovero, nie Amaz'hana, jej Ahma'Treta'ji žehná. Prečo ahma'corlovne sa teší dôvere ahma'Amaz'ke'u?“

"$ $Znepokojujú ma vaše pochyby a to, že tie pochyby máte vy, majú ich i Amaz'hany. Kam sa podela tvoja viera? Samotná Ahma'Treta'ji zoslala Medizo, aby pomohla k sláve nášmu národu. Keď velekňažkou prvý raz zvali ma, keď prvá ríša bola v ruinách a nám neostávalo síl. Keď táto sieň, kde obe stojíme prvý raz uzrela môj zrak. Veď či nespomínaš si, keď Ahma'Treta'ji nám zoslala striebro, pre slávu a moc, jej. Či až taká malá je tvoja viera? Čo sa stalo s vyvolenými Ahma'Treta'ji?“

"$ $A či neverím, pri Ahma'Treta'ji! Ahma'Treta'ji nech potvrdí moje slová a moju vieru? Či ja som vás, ahma'Amaz'ke'u nevyhľadala, keď Ahta'ahma'Amaz'le\v{}gha prichádzala ku Ahma'Treta'ji?“

"$ $A či to je dôkaz tvojej viery? Činy minulosti sú síce chvályhodné, ale o čom vypovedajú? Keď sa Amaz'hana potrebuje o minulosť opierať, o čom to svedčí? O čom praví taká viera?“

"$ $Strach nosí pochybnosti, ahma'Amaz'ke'u. Strach nosí myšlienky, ale viera od odháňa. Ahma'Treta'ji zachraňuje nás a my jej slávu vzdávame, lež aj naša sila musí vyniknúť, to je na Jej slávu a na jej meno. A naša sila, tá pochádza od Ahma'Treta'ji aj od nás a viera v falošné modly, ktoré Ahma'Treta'ji neveria v, prečo im dôverujeme, prečo veríme vládkyni?“

"$ $Amaz'ha\v{}ja nebola tam, kde slávu získala táto sieň, kde sa našla tabuľa a kde sme zas povstali v ríši. Či poznáš minulosť a či poznáš, kým je amha'Medizo? Či niekedy tvoje pochybnosti nahlodala pravda, alebo len reči šíriace sa medzi nepravovernými?"$ $ Amaz'ke'u bola pokojná. Roky strávené bytím velekňažkou ju naučili trpezlivosti. Amaz'hany síce nemali vo zvyku pochybovať o viere alebo o svojich nadriadených, ale kroky ako spolupráca s Medizo alebo čakanie na Gl'estuvele'i vniesli pochybnosti do zväčša poslušných Amaz'han. Amaz'ha\v{}ja nebola výnimka a to ona bola pri tom...

\begin{center}
*
\end{center}

"$ $Tak prebudili ste sa, amha'Amaz'ke'u."$ $ Niekto ku nej hovoril, napadlo jej ako prvé. Oslovil ma amha‘, napadlo jej podruhé. Vyskočila na rovné nohy, nasadila štít, vytasila meč. Osoba, ktorá ju oslovila, ju udržala na mieste, nemohla ísť dopredu.

"$ $Pri Ahma'Treta'ji..."$ $ pošepla. Bola vo funkcii velekňažky, nemohla zomrieť... a bola tak blízko. Uvedomila si, že má po rukách jazvy, ktoré tam predtým nebola. Žiadna panika, pomyslela si a pomyslela na Ahma'Treta'ji s prosbou o pomoc. Mágia bola všade okolo.

Oslovila ma. Povedala si. Vie moje meno. Ale ja neviem, kto to je. Zdvihla na ňu svoj zrak.

Nie je to Amaz'hana. Napadlo jej. Nemohla byť Amaz'hanou už len pre jej výzor, ktorý odporoval Amaz'hanským bojovým, či slávnostným tradíciám. A jej pleť mala diametrálne inú farbu – purpurovú a vlasy zlatej farby jej padali po chrbte. Mala na sebe dlhý čierny plášť a odetá bola v gle'edonej koži. Za opaskom mala meč, o ktorom Amaz'ke'u vedela, že ho nosia V'eil'ei, zvané tiež víly.

"$ $Koho mi tu zoslala Ahma'Treta'ji?"$ $ Spýtala sa viac do vetra, než osoby, ale tá aj tak odpovedala.

"$ $Amha'Amaz'ke'u, Ahma'Treta'ji zachránila svoju velekňažku z rodu Amaz'han, keď amha‘Amaz'ke'u prestal pôsobiť štít. Ja, corlovne Medizo, titul nášho rodu, staršia, čo má pre rod Amaz'han posolstvo od Ahma'Treta'ji.“

"$ $Len velekňažka a kňažky skrze ňu môžu s Ahma'Treta'ji hovoriť. Amha'Medizo ani Amaz‘ nie je, či ako k nej mohla Ahma'Treta'ji prehovoriť?“

"$ $Ahma'Treta'ji naviedla ma na správnu cestu, čo odovzdať Amaz'hanam treba, pre ich osud a záchranu im mágia mesto vytesala, ale to pre vašu slávu, velekňažka, to ešte čaká v útrobách tejto pevnosti. To, čím vás Ahma'Treta'ji požehnáva.“

"$ $Tak prečo svojej velekňažke, čo víťazstvo pre ľud Amaz'han vymodlila a po tej, čo ju Ahma'Treta'ji teraz požehnáva mi požehnanie velekňažské odovzdala, prečo by nedala svojmu požehnanému ľudu svoje dary. Veď ty, čo meč V'eil'ei nosíš, ten je z rodu, ktorý aghe‘ uznáva a im práva dáva, ktoré svoje sestry nepovýšil, ako ahma'Kniha káže, a tvoj ľud? Ktorý je tvoj ľud, jediný požehnaný ľud od Ahma'Treta'ji sú Amaz'hany, to vraví ahma'Kniha od Ahma'Treta'ji a tá sa nemýli.“

"$ $Veď ja Medizo som staršia iba z našich, prešla som časy a dostala som sa sem, kým odídem na pár vekov. Lebo mne veľká kniha proroctiev, čo Ahma'Treta'ji káže, predpovedala byť vládkyňou trikrát a nájsť veľkú zbraň a pozdvihnúť cez ňu požehnané. A kto by iný bol požehnanými, ak nie vy, Amaz'hany. Preto Ahma'Treta'ji vám dáva striebro, druhý kov zo zásobami mágie ale nie len ten, ale i veľkú zbraň, o ktorej sa hovorí.“

"$ $Veľkú zbraň ahma'Kniha nezmieňuje, ani len vašu knihu. Kto má vedieť pravdu? Našu ríšu ešte prednedávnom zachvacovali boje, v ktorých skonala i moja predchodkyňa, veľká Amaz'le\v{}gha.“

"$ $Nám, nepožehnaným nie je súdené byť vo veľkej priazni Ahma'Treta'ji, ale ona i nám dala knihu, ‘bo chce svoje velebenie a svoju slávu. Veľká kniha proroctiev obsahuje slová veľkého poznania, ale nie až také, ako ahma'Kniha. Pri Slovách a sláve Ahma'Treta'ji, ona predpovedá, ona nás navádza, ona zachraňuje, ona požehnáva. A môj národ sa vzoprel jej učeniu a prial aghe‘. Preto musíme byť zajedno, amha'velekňažka a veľkú zbraň získať.“

"$ $A či neklameš, či nezavádzaš? Či nezachránila si ma len na obetu inému bohu, aby som ostala pre Ahma'Treta'ji stratená? O veľkej zbrani nehovorí nikde ahma'Kniha.“

"$ $Či Gl'estuvele'i nie je vo vašich poznatkoch? Ja som ju našla a tu je a rastie, je pre vás. Len dovediem vás ku nemu, len to je moja úloha a striebro vám odovzdať, lebo to sa tvorí pri Gl'estuvele'i, čo veľký je zdroj."$ $ Vtedy Amaz'ke'u pochopila čo je to veľká zbraň. Gl'estuvele'i bola legenda, i ahma'Kniha ju spomínala, hovorila o Gl'estuvele'i, mocnej palici od Ahma'Treta'ji. Ale všetci ju považovali za božskú zbraň. Za niečo, čo má len bohyňa.

"$ $A či Gl'estuvele'i naozaj je tu? Veď len Ahma'Treta'ji ho má, ona je jeho vlastníčkou.“

"$ $Ahma'Treta'ji vie o vašich osudoch a vie, že ho budete potrebovať. Či nedôveruješ?“

"$ $To nie, veľká je moja dôvera v múdrosť, silu a moc Ahma'Treta'ji. Nepokúšaj, amha'Medizo. Moja viera je silná a pevná, nenaštrbuje ju nič. Len Ahma'Treta'ji sa nikdy nespomína s Gl'estuvele'i inak ako jej zbraňou.“

"$ $Jej požehnané Gl'estuvele'i potrebujú. Preto som tu, amha'velekňažka. Môj čas sa kráti a vrátim sa niekedy v čase, či skôr, či neskôr. V piktopísme v zlatom chráme, kde hodujú deti smrti, tam boli napísané slová. Tak tam, kde leží Gl'estuvele'i, tam vedú chodby tieto, tam ma zaviedla Ahma'Treta'ji, tam som poznala silu Gl'estuvele'i, ktorá lež nie je mne určená. Je určená jej požehnaným, jej ľudu. Mne predpovedala byť vládkyňou, ale komu, Amha'velekňažka? Nie vám, požehnaným, lež prostému ľudu, lebo vám vládne len Ahma'Treta'ji a vám náleží Gl'estuvele'i po práve i po písme. Keď sa dostaneme ku Gl'estuvele'i, ovládnu Amaz'hany i striebro a vybudujú ríšu. A keď príde útok, vtedy porazia nepriateľa silou veľkej zbrane. Tak pravia proroctvá.“

I tak šla velekňažka za corlovne Medizo. Prešli ešte dlhé chodby, kým prišli ku sieni, z ktorej šľahali záblesky. Obe zosilnili svoj štít a prišli priamo pred ten zázrak.

Gl'estuvele'i bolo zahalené v opare mágie.

"$ $Môžem ho vziať?"$ $ Spýtala sa velekňažka, očarená ako dieťa, zabúdajúc na to, že je to ich právo a ich dedičstvo.

"$ $To je koreň všetkého, amha'velekňažka. Tak pravia proroctvá od Ahma'Treta'ji. Gl'estuvele'i pre vás získa požehnané ľudské dievča. Polodémonka príde s dvomi a získa pre vás veľkú zbraň. To je proroctvo. Nikto iný sa ho nedokáže dotknúť v opare mágie. To je slovo Ahma'Treta'ji.“

"$ $My sme jej požehnaný ľud.“

"$ $Požehnané Amaz'han už majú striebro, alebo o chvíľu mať budú. Z nich vybudujú ríšu, lebo tak pravia kniha proroctiev od samotnej Ahma'Treta'ji. Lebo ja, Medizo oddelím zdroj, a ten bude váš.“

"$ $Svoj život dávate Ahma'Treta'ji?“

"$ $Len svoj čas, lebo vír čaká. Tam je veľa miest a veľa rokov, ale hádam sa ešte vidíme, keď budete pozdvihovať Gl'estuvele'i ku nebu, veľkú zbraň vlastniť a spoločne Ahma'Treta'ji velebiť, pre jej slávu a moc dobývať a aghe‘ dáme miesto, kde majú byť. Tak nech sa stane velekňažka. Vybudujte svoju ríšu v skale mágie a získajte veľkú zbraň.“

"$ $Ak hovorí amha'corlovne, že je smrteľné sa priblížiť ku veľkej zbrani. Tak ako získate striebro?“

"$ $Striebro vytvorí energiu na bránu a ja sa dostanem, ak bude Ahma'Treta'ji milostivá, preč. A odtiaľ získam pre nás ríšu, nech Ahma'Treta'ji mám svoju moc a slávu.“

"$ $Nech vás Ahma'Treta'ji sprevádza na vašich cestách."$ $ Povedala jej velekňažka.

Corlovne Medizo vošla do mágie. Gl'estuvele'i sa ani nedotkla, ale mágie okolo nej jej rýchlo požierala štít. Velekňažka klesla na kolená a začala sa modliť. A Medizo už nemala štít, ale nezhorela, ako sa stávalo pri takýchto nehodách, ale niečo pevne schytila a hodila to Amaz'ke'u, ktorá to so svojimi Amaz'hanskými reflexami chytila. Medizo zmizla. A velekňažka mala zrazu obrovské zásoby mágie v jednom kuse striebra, pred sebou mýtické Gl'estuvele'i a ríšu vyhĺbenú do kameňa, pre ne – požehnaný ľud.

Napriek tomu, že sa všetko zdalo ako úspech, neprestávala sa Amaz'ke'u modlievať ku Ahma'Treta'ji, nech jej vyjasní niektoré Medizine slová. Hovorila o tom, že budú mať Gl'estuvele'i, ale aj o tom, že ich nepriatelia zas prídu...

\begin{center}
*
\end{center}

A ten čas prichádzal. Velekňažka o tom vedela. Tušila to už dlhšie, ale keď sa vrátila corlovne... Vtedy začínala byť hrozba útoku oveľa aktuálnejšia a bližšia, než kedykoľvek. V tichosti na ňu Amaz'hany pripravovala. Pomaly zvyšovala čas na výcvik a modlitby, upevňovala ilúzie, ktoré ich dole chránili. Samozrejme, Amaz'hany tušili, že sa niečo deje, ale nahlas pochybnosti nikto nevyslovil. Niektoré zo starších, ako Amaz'ha\v{}ja o hrozbe vedeli, ale boli viazaný mlčanlivosťou, pri Ahma'Treta'ji.

A teraz, keď proroctvo o ktorom corlovne Medizo hovorila... Nestrácala svoju vieru, to nie, modlila sa ku Ahma'Treta'ji, ale bitky sa bála. Vojny sa bála. Polodémonka mala pre nich získať Gl'estuvele'i, ale bola ako obyčajný človek, menej než zovero. Nevedela poriadne mágiu, tak akú šancu... Presne toto hovorila aj Amaz'ha\v{}ja... To bolo zlé... Zas sa v duchu obrátila ku bohyni.

Teraz potrebovala pomoc, ale bola sama. Rovnako ako vtedy, keď sa len stala velekňažkou. Ale teraz už nebola neskúsená mladá kňažka, teraz bola...

Pokora u Amaz'han bola len pri Ahma'Treta'ji, alebo pri velekňažke. Amaz'hany boli požehnanými a preto sa neskláňali, len pred ich bohyňou. Preto ani velekňažka sa nešla ponižovať. Musí existovať predsa východisko... pomyslela si. Vždy existuje...

V proroctve sa objavili ešte dvaja... Tlmočníčka a mág... mág... aghe'... pomyslela si. Prečo aghe', Ahma'Treta'ji, prečo aghe'? Ahma'Treta'ji predsa dala moc im, nie aghe', tak prečo by práve jeden z tých...? Prečo by tí, ktorých považovali od vzniku Amaz'han za menejcenných mali by byť kľúčom ku víťazstvu? Prečo, Ahma'Treta'ji? Pýtala sa svojej bohyne. Vedela, že Ahma'Treta'ji musela poznať zmysel toho proroctva, ale prečo, prečo? Tá otázka ju kvárila dlhšie, ale nikdy nebola tak aktuálna.

Velekňažka vedela, že ľudia ani zovero, dokonca ani víly nezdieľajú ich odpor ku aghe'. Tých pár kultúr, ktoré si túto zdravú tradíciu zachovali zabili ich vlastní aghe', ktorých si nedržali dostatočne na uzde. Aj Medizo bola jedna z toho národa, lež ona tradície zachovávala a aghe' rovnako menejcennými pokladala, ako to Ahma'Treta'ji za správne brala. I oni, ľudia z ktorých aj polodémonka bola, aj oni nepočúvali Ahma'Treta'ji v jej prikázaniach a slovách o aghe', a preto cestovala s aghe', čo i len s jedným. Medizo za jej rozhodnutie dávať mágom – aghe' rovnocenné postavenie kritizovala. Ona tvrdila, že je to nutné zlo, ale čo Ahma'Treta'ji? Spolupráca z aghe' bola zrada a teraz to žiadala aj od nej? Nemohla zradiť svoju bohyňu, bola predsa velekňažka, ale na druhej strane, nemala ju poslúchnuť? Amaz'ke'u bola plná pochybností. Čo by mala robiť? Ako každá Amaz'hana bola proti akejkoľvek spolupráci s aghe' a... teraz sa v ríši jeden objavil a podľa proroctiev mal byť článkom k víťazstvu...

Musí ho nájsť. Oboch votrelcov. Aj tlmočníčku aj aghe'. Zavolala Amaz'ha\v{}ju.

"$ $Pošlite jednotku na nájdenie votrelcov."$ $ Povedala velekňažka. Amaz'ha\v{}ja sa uklonila.

"$ $Ahma'Amaz'ke'u, hlásim, že jednotka, ktorá stráži vchod bola vyslaná na hľadanie.“

"$ $Pošlite špeciálnu jednotku. Strážkyne nech strážia vchod. Blíži sa vojna, Amaz'ha\v{}ja a vy to viete. Musíme Ahma'Treta'ji priniesť slávu a česť. Ako je na tom polodémonka?“

"$ $Nevie nič. Nevie čarovať! Nevie pracovať so zbraňou, je..."$ $ velekňažka premýšľala.

"$ $Ste si, Amaz'ha\v{}ja istá, že práve ona je polodémonka?“

"$ $Mení sa. Bez akejkoľvek spotreby mágie. Je to polodémonka alebo démonka, ale Démon je len jeden a to je už medzi ľuďmi a zovero.“

"$ $Izolovaní úplne nie sme. Veľká zbraň je záležistosť všetkých."$ $ Pri zmienke o veľkej zbrani stíšila hlas. "$ $Čo ešte o nej?“

"$ $Nekomunikuje. Sú u nej jasné známky apatie, ani jedna z našich lektoriek mágie z nej nevie dostať nič..."$ $ Velekňažka premýšľala. Ani jedna z ich čarodejníc...

"$ $Preto aghe'mág, Amaz'ha\v{}ja! Preto proroctvo o ňom hovorí. My nebudeme spolupracovať, to by bola zrada učenia ahma'Knihy a Ahma'Treta'ji. Polodémonka je človek a Ahma'Treta'ji nectiaca, preto ona len prostriedkom je k našej a Ahma'Treta'ji sláve. Polodémonka je príliš neverec, aby sme u akceptovali a ona so spolupráce s aghe' bude mať len hriech a Ahma'Treta'ji by ju k sebe nevzala a my si roky nepošpiníme. Tak Ahma'Treta'ji chce a pre jej slávu my spravíme."$ $ Amaz'ha\v{}ja prikývla a uklonila sa.

"$ $Tak, ahma'Amaz'ke'u, čo mám spraviť?“

"$ $Nájdite votrelcov. Toho aghe' pošlite k polodémonke a predtým mu povedzte čo má spraviť, vyhrážkami šetriť nemusíte, len pre teraz, nech sa mu nič nestane, lebo aghe' je tentoraz prostriedok.“

"$ $A čo tlmočníčka? O nej ste nezmienili sa, ahma'Amaz'ke'u."$ $ Tá v jej plánoch nebola, ale Amaz'ke'u nepochybovala, že aj jej miesto je.

"$ $Tá... Ahma'Treta'ji vie... priveďte ju... ku Stene."$ $ Prikázala. Amaz'ha\v{}ja sa poklonila a išla splniť príkazy.

\begin{center}
*
\end{center}

"$ $Všetky novicky do malého chrámu! Všetky Amaz'hany, čujte, Ahma'Amaz'ke'u ku vám prehovorí, lebo naši nepriatelia idú. Lež Ahma'Treta'ji nám žehná a my jej slávu dobijeme! Do chrámov! Ahma'Amaz'ke'u k vám prehovorí!"$ $ Tulienka Deľa sa obrátila ku ostatným Amaz'hanam. Tie okamžite po začutí oznamu sa obrátili na smer malý chrám, kde podľa oznamu mali byť. Tie, ktoré vyzerali už predtým nervózne začali šepkať. Nebolo to príliš nahlas, ale aj tak Tulienka Deľa dokázala počuť, že vzývajú Ahma'Treta'ji.

Amaz'ke'u bola niekde vzadu v chráme a hovorila s jednou vojačkou. Vyzerala na významnú Amaz'hanu, keďže ju ostatné Amaz'hany – vojačky volali s predponou ahma‘. Tulienka Deľa niekoľkokrát započula slovo Medizo, ale hovorili potichu a nerozumela ich jazyku natoľko aby pochopila tému rozhovoru a nie už z diaľky.

Rozmýšľala kde je Tarny. Ten sa jej stratil a nechcela sa pokúšať obísť telepatickú sieť Amaz'han. Vedela, že to bolo nad jej možnosti. Nebola nejako zvlášť schopná v telepatickej mágii, aby sa im vyrovnala.

Do chrámu sa zmestila prekvapivo veľa Amaz'han. Spolu ich bolo oveľa viac, než Tulienka Deľa očakávala a nevedela, akým kúzlom sa všetky kňažky a novicky napratali do malého chrámu. Ten raz, čo tam Tulienka Deľa predtým bola, sa jej zdal vcelku malý. Nevedela, či boli otvorené nejaké ilúzie, alebo sa do jednej dostali, alebo či chrám mal nejaké zatvorené miestnosti, alebo bol nejakým zvláštnym kúzlom rozšírený. Prvú možnosť nevylučovala, hoci sa jej zdala nepravdepodobná, Wymyslenská mágofyzika ešte nedošla na spôsob akým prepojiť ilúziu a realitu jednotným prechodom s viditeľnou bránou bez projekcie, ale Amaz'hany mohli byť oveľa ďalej.

Atmosféru chrámu vypĺňal nepokoj. Amaz'ke'u stále hovorila s tou istou Amaz'hanou a ostatné Amaz'hany čakali. Síce sa nikdy Tulienke Deli nezdali netrpezlivé, ani neusporiadané, ale teraz takú nervóznu masu pripomínali. Tulienka Deľa tušila niečo zlé. A to, že práve vtedy tam prišli oni sa jej prestávalo zdať ako náhoda...

\begin{center}
*
\end{center}

Tulienka Deľa dúfala, že úloha, do ktorej ho dosadila by ho mala ochrániť. Bola síce málo nápadná, ale uvedomoval si, že je veľmi ľahko odhaliteľná. Nepoznal zvyklosti Amaz'han a nemal talent Tulie na zakamuflovanie sa prakticky kdekoľvek.

Zvolávali ich. Do chrámov. Zatiaľ nepocítil náznak toho, že by vedeli, že ich infiltrovali, ale každé zdanie môže klamať. Veľmi ľutoval, že nevie zmyslovku piateho stupňa. Veľmi. Vyššia štvrtého stupňa dovoľovala zakryť svoju vlastnú magickú stopu ale slabo a navyše odhaľovanie cudzej magickej stopy bolo oveľa sťaženejšie. Radšej sa nechcel pliesť do magickej stopy Amaz'han. Stačilo, že mal pri sebe kov.

Trval na tom, aby si ho vzala Tulie, ale ona nesúhlasila. Áno, bol lepší mág ako ona, ale schopnosť kamuflovať sa bola v ich momentálnej situácii oveľa dôležitejšia.

\begin{center}
*
\end{center}

"$ $Aghe'!"$ $ vojačka udržovala Tarnyho v bezpečnej vzdialenosti od nej, aby nemohol utiecť a zároveň aby sa ona, ako Amaz'hana nedotkla aghe'.

Tarny sa cítil mizerne. Nevedel, kde je Tulie. Nevedel jazyk a svoj prekladač dal Pauline, ktorú mali zachrániť a namiesto toho upadli do zajatia aj oni. Nemal šancu na útek, Amaz'hany zrušili všetky zmyslové kúzla kategórie jeden a okolo neho ich bolo priveľa.

Ich mágia bola prisilná a mali presilu.

Prekvapilo ho, že ho nechali nažive, v moment odhalenia sa už lúčil zo životom. Napriek tomu ho len odviedli. Medená tabuľa, ktorú prezieravo zmenšil bola stále uňho, ale pochyboval ako dlho tak tomu bude. Neprehľadávali ho, ale rátal s tím, už rátal som všetkým. Stále nevedel, či bude žiť, alebo toto bolo len odvedenie na popravu. Podľa stručných Tulieiných inštrukcií aj z chatrných útržkov toho, čo odpozoroval bol nie len ako zovero, ale aj ako muž tvor nižšej kategórie, rovnako ako tomu bolo v meste. Len tu sa nič nemenilo tím, že bol mág.

Tarny už zažil veľa vecí, ale všetko bolo priveľmi plytké, aby tomu venoval pozornosť v hlbšom zamyslení. Všetky tie časy, kedy nelegálne lietali po Zemi a Fanase na ukradnutom mačičkoese a utekali pred políciou a D, keď nebrali akékoľvek riziko vážne, keď si nikto z nich neuvedomil, čo vlastne robia. Boli trojica ľahkovážnych tínedžerov, pre ktorých nebezpečenstvo nič neznamenalo, z ktorého sa vždy dostali a domov sa vracali s pocitom rebelantstva, ktoré nebolo ničím iným, než ľahkomyseľným zahrávaním sa zo životom. Mali pocit, že sú bohovia toho momentu, keď ich nemohol dostať ani jeden štát, ktorého zákony porušili, ani jeden nepriateľ, ktorý mohol po nich naťahovať prsty. Boli elita, inteligenciou sa všetci traja vyrovnali agentom, ale chýbala im rozvážnosť na to, aby nimi sa mohli stať. Posledné roky ich zo svojho sídla brzdievala Keria Oetová, jedna z tých, ktorý sa vo Wymyslensku snažili zvrhnúť Cecíliu relatívne mierovou cestou. Ale ona nebola tam, ona vždy len Sylviu upozorňovala, ale slová sú len slová. Sylvia ju brala niekedy vážne, ale kto bude myslieť na starosti, keď oni boli tí, ktorím bolo predurčené stať sa budúcnosťou? Ich vedomosti a moc im dodávali sebavedomie, ktoré ich vzájomne posúvali vpred, ale zároveň posilňovalo ich ľahkovážnosť. Každé nebezpečenstvo vždy zvládli, tak prečo nie zas? Ich plány boli veľké, ale nikdy si nikto nepomyslel, ako ľahko sa môžu zrútiť. Až doteraz.

Zastali. Vojačka ho doviedla ku ďalšej Amaz'hane. Stále ale ostávala, neodchádzala.

Amaz'ha\v{}ja, ku ktorej ho priviedli, dostala správu o odhalení okamžite a nečakala. Bolo to po dlhej dobe, čo mala prehovoriť s aghe', ak sa jej to vôbec niekedy stalo. Kontaktu s nimi sa vyhýbala, ako každá Amaz'hana.

Sumarizovala si informácie, ktoré mala. Bol mágom a ušli spolu s tou tlmočníčkou strážkyniam. Zrejme to bola jej zásluha, ale napriek tomu, že bol aghe', nepodceňovala ho.

"$ $Prehľadať!"$ $ Prikázala. "$ $Je to aghe', ale či pri Ahma'Treta'ji nič neskrýva?"$ $ Zdvihla ruku. Pozrela sa do magického pohľadu a na čo jej uprel zrak bola nevídaná koncentrácie mágie pri Tarnym. Amaz'hany prakticky v mágii žili, ich domovy ňou boli vyhĺbené a mali dávku imunity voči nej, ktorú víly, zovero a tobôž ľudia nemali ako získať. Dávka, ktorú mal pri sebe mág bola priveľká i na ne, nie to na zovero. Musel byť chránený.

Ani štít by také množstvo mágie neudržal, to vedela, nie tak koncentrované. Napadlo jej jediné vysvetlenie.

Všetky rasy, ktoré poznali mágiu mali mýty o troch kovoch, v ktorých bola nazhromaždená mágia väčšia, než kto kedy spotreboval.

Aj Amaz'hany poväčšine verili, že je to legenda, i keď Ahma'Treta'ji in v ahma'Knihe zvestovala ich pravdivosť. Vieru im navrátilo až to, keď Amaz'ke'u našla striebornú tabuľu a s ňou i nové mesto Amaz'han. A ak sa jej intuícia nemýlila, tak Ahma'Treta'ji buď k nám milostivá, pomyslela si. Bolo treba zakročiť čo najrýchlejšie.

Telepaticky sa spojila s vojačkou a vydala pokyny. Znehybnili ho. Amaz'ha\v{}ja sa sústredila na mágiu.

Tarny pochopil, že ho odhalili.

\begin{center}
*
\end{center}

"$ $Tlmočníčka..."$ $ Hovorila nahlas velekňažka, prechádzajúc sa pred Tulienkou Deľou, pripútanou ku čiernej stene. Hovorila v ich jazyku, ale Tulienka Deľa jej rozumela.

Stena bolo miesto, kde bola naposledy na výsluchu 'ro\v{}gha, zbavená titulu Amaz', pre jej zločin. Bolo to dávnejšie, už pred pár rokmi. Amaz'hany nepáchali zločiny vo veľkej miere, 'ro\v{}gha bola tmavá výnimka na očiach Ahma'Treta'ji.

Nielenže nadviazala kontakt s aghe', ale chcela preňho odísť z ich ríše. Samozrejme – bola odhalená a následne verejne pranierovaná a obetovaná Ahma'Treta'ji. Amaz'hany nevykonávali krvavé obety, zločinci z ich vlastných radov boli výnimky. Stena bola miesto výsluchu a tlmočníčka mala čo hovoriť.

"$ $Kde je Pauline, ahma'Amaz'ke'u?"$ $ Spýtala sa Tulienka Deľa. Vedela, že pošmyknutie môže znamenať smrť, ale rovnako tá môže byť už predurčená. Nevedela, čo s ňou majú v pláne, ale ona chcela pred pravdepodobnou popravou sa dozvedieť aspoň niečo. A počas jej krátkeho pobytu medzi Amaz'hanami sa naučila základy jazyka a tiež pár vecí, ktoré sa jej mohli hodiť, ako napríklad to, že Amaz'hany si veľmi zakladajú na svojom jazyku a hovoriť iným pre nich znamená zníženie sa, ku ktorému neradi prichádzajú. Tiež si všimla, že si veľmi zakladajú na diplomacii a svojej viere a tak Tulienka Deľa sa rozhodla aspoň pokúsiť dostať sa zo situácie, v akej sa ocitla diplomaticky. Oveľa viac si robila starosti o Tarnyho. Už len preto, že ho Amaz'hany považovali za menejcennú bytosť a ona ho poznala dostatočne..,

"$ $Či tlmočníčka vie jazyk Amaz'han, Ahma'Treta'ji požehnaný a darovaný?“

"$ $Ahma'Treta'ji požehnáva, a požehnáva aj iných, než Amaz'hany, ak toto prehlásiť si dovoliť môžem, ahma'Amaz'ke'u. Lebo či je ahma'corlovne Amaz'hana? A či nie je požehnaná Ahma'Treta'ji, ahma'Amaz'ke'u?"$ $ Amaz'hany zvykli hovoriť dlhé prejavy, plné vzývania bohyne a viacnásobného oslovovania adresáta reči.

"$ $Ahma'Treta'ji žehná, ale slávu jej vzdávať musíme, pre naše požehnanie, tlmočníčka. A či slávu Ahma'Treta'ji žehnáte tým, že s aghe' putujete? Či tým si dar Ahma'Treta'ji zasluhujete? Ahma'Treta'ji dáva, ale len jej služobníkom, tlmočníčka, ktorá dar reči máte, čo Ahma'Treta'ji oceňuje, že náš požehnaný jazyk od Ahma'Treta'ji daný ste uctili a nenútite nedobrovoľne Amaz'hany na jazyk iný jazyk ako ten požehnaný brať, ale či toto nie je nulované, alebo úplne zanedbateľné tým, že nedávate aghe' správne miesto a do pozície rovnocennej, či Ahma'Treta'ji nedopusť, povýšenej. A či to na slávu Ahma'Treta'ji slúži?“

"$ $Mágom je a zlom menším to je. Nie sme požehnaní, ako Amaz'hany, Ahma'Treta'ji len tie požehnáva večným a najväčším požehnaním. Ale nie sme všetci mágovia a nech Ahma'Treta'ji odpustí mi moje slová, aghe' sú potrebný na prežitie. Je mágom a mágom veľkým, ako len aghe' môže byť."$ $ Tulienka Deľa premýšľala. Nemohla uraziť kultúru Amaz'han – za to by putovala na popravu, ale nemohla odpraviť ani Tarnyho. Akonáhle by prehlásila, že je dôležitý pre ich výpravu, znamenalo by to urazene Amaz'han a ich diskriminujúcej filozofie, ale zároveň nemohla Tarnyho odpraviť. Musela ich presvedčiť o jeho dôležitosti, aby ich pustili. A rovnako Pauline. Ale Tarny bol hlavný dôvod.

"$ $Nech Ahma'Treta'ji odpustí mi,"$ $ pokračovala opatrne. "$ $Že teraz aghe' obhajujem, že teraz jej učeniu protirečím,"$ $ veľmi premýšľala ako sa z toho vyvliecť. "$ $Ale, i ahma'corlovne mágom dáva miesto a nejakú moc aghe', ak sú to mágovia, nech Ahma'Treta'ji odpúšťa mi, za to, že aghe' obhajujem, lebo ako ona káže, ako ahma'Kniha káže, 'hana je požehnaná, nie aghe' nadovšetko, tak ako Ahma'Treta'ji hovorí, to preto Amaz'hany sú Amaz', lebo aghe' u nich nie je,"$ $ dúfala, že použila dobrú gramatiku, lebo bol to jazyk pre ňu nový a tak používala vílsku gramatiku, na ktorú sa najviac podobal. Snažila sa používať čo najviac pasáži z posvätnej knihy Amaz'han, ktoré si pamätala.

"$ $Amaz'hany sú požehnané od Ahma'Treta'ji, tá ich stvorila za jedno a zbavila ich aghe', čo čo vzdávate vďaky Ahma'Treta'ji, ale čo tí, ktorý nie sú takto požehnaný?"$ $ Dúfala, že toto si velekňažka nevyloží tak, že sa chce zbaviť Tarnyho. "$ $My, zovero a ľudia žijeme ako nás Ahma'Treta'ji stvorila, aghe' aj my. A aj aghe' sú mágovia, tak ich Ahma'Treta'ji stvorila, nie len nás a preto ich potrebujeme, nech Ahma'Treta'ji odpúšťa mi moje slová. Mág je požehnaný, ako len aghe' môže."$ $ Dúfala, že si to veľmi nepokazila.

Ahma'Treta'ji chvíľu premýšľala, ale nie pridlho. Ahma'Knihu predsa poznala, i to čo Ahma'Treta'ji kázala. Podľa viacerých by mala aghe' i tlmočníčku zabiť – vniknutie do chrámu a vydávanie sa za Amaz'hany bolo rúhaním. Hlboko v jej vnútri bola však zvedavosť. Stretla sa s niekým, koho predtým nevidela. Amaz'hany necestovali, ich život bol rutina. Medizo bola prvá zovero ktorú kedy stretla. Dovtedy sa s tvormi, inými ako ony stretávala len ako s nepriateľmi a vtedy si aj tak nevideli do tvárí, nehľadeli si do očí. Boli pre druhého zverou, ktorú bolo treba zabiť, lebo stáli na opačnej strane barikády. Ale Ahma'Treta'ji bola milostivá, ale...

Opovrhovala, rovnako ako všetky Amaz'hany ostatnými rasami, ktoré považovala za menejcenné, ale nemohla si pomôcť – bola zvedavá. Jej zvedavosť bola veľká a bojovala vnútri nej s jej zbožnosťou. Ahma'Treta'ji obdarila ich, nie zovero ani ľudí. Zvedavosť nebola veľmi vítaná, a Amaz'ke'u tomu rozumela, akceptovala to a snažila sa s tím súhlasiť, ale keď bola velekňažka a mala šancu...

Stala sa velekňažkou veľmi mladá, mala len polovicu veku Amaz'le\v{}ghy, keď usadla tá na miesto velekňažky Amaz'han. A možno bol jej problém. Váženosť Amaz'hany prichádzala nielen s postavením, ale i s vekom. Mladé Amaz'hany si vždy uctili staršiu Amaz'hanu i rovnakého postavenia formulou amze', teda staršia. Mladšie Amaz'hany boli náchylnejšie na neposlušnosť, ale to sa trestalo v zárodku. A ona bola tak mladá, keď dosadla na miesto velekňažky a fakticky vládkyne Amaz'han. To, že dôverovala Medizo bol jej prešľap, ktorý niekedy nepovažovala za správny, ale bol z toho úžitok. Dôverovala corlovne, lebo ju zachránila a lebo vedela o Ahma'Treta'ji, ctila si Ahma'Knihu a... bola zvedavá. A stále bola, nielen vtedy. Nikdy sa nezbavila mladej Amaz'hany v nej... To bola chyba. To čo robila bola chyba. Keby bola správna Amaz'hana dala by ich popraviť a... namiesto toho tam teraz debatovala s tlmočníčkou a právach aghe'. Bolo to smiešne. Aghe' nemali mať žiadne práva, ich existencia ich odsudzovala na bytie podradnou skupinou, nezasluhovali si nič. A táto zovero s ktorou sa rozpráva sa cez Ahma'Knihu pokúšala obhájiť...

Amaz'hany nepripúšťali viaceré výklady ahma'Knihy. Dovoľovali len ortodoxný výklad ktorý považoval knihu za slovo Ahma'Treta'ji, ktoré platí naveky. Iné pokusy o vykladanie knihy sa pokladali za rúhanie a boli prísne trestané. Amaz'ke'u súhlasila s trestom za rúhanie a aj so súčasným výkladom knihy. Bola najvyššia autorita Amaz'han a bola poverená aj výkladom, ale jeho zmena bola neprípustná aj na jej postavenie. Jedine Ahma'Treta'ji má právo zmeniť učenie Amaz'han a tá sa im pridlho sama neohlásila. Áno, dávala im znamenia, ale jej slovo bolo naposledy dané v ahma'Knihe a tá bola ak stará... A táto zovero teraz hovorila, že Ahma'Treta'ji stvorila i ich požehnaných, nielen jej požehnané. A dokonca aghe'! Toto by Amaz'ha\v{}ja i akákoľvek iná Amaz'hana považovala za rúhanie. Lebo rúhaním to bolo a ničím iným. A predsa ju mladá zovero fascinovala, i keď by si to nikdy nepriznala.

"$ $Ahma'Treta'ji dovolila len jeden výklad písma, tak prečo teraz hovoríte ako jej kňažka, čo má výklad ahma'Knihy? Či vám bolo dané toto právo? Len velekňažka má právo vykladať knihu? O čo sa vy, tlmočníčka pokúšate?"$ $ Tulienka Deľa v duchu zakliala. Nemala prechádzať na taký tenký ľad a teraz urazila velekňažku. Hovorila len na základe toto, čo počula na tej jedinej bohoslužbe na ktorej sa v prevleku Amaz'hany zúčastnila a jej znalosti neboli ani trochu dostatočné. Pokúšala sa argumentovať knihou, ale to jej teraz mohlo prísť osudným. Velekňažka vyzerala pobúrená jej opovážlivosťou, ale tiež si ale povšimla, že neodpovedala hneď, ale vyzerala, že premýšľa. Akoby si nebol istá.

Tulienka Deľa sa musela brániť. Nemohla dovoliť smrť Tarnyho, lebo by si to nikdy neodpustila, a rovnako nemohla dopustiť aby sa niečo stalo jej a rovnako Pauline. Ich životy však boli príliš nezávislé a zároveň závislé na nej a bála sa, že už nemôže nič urobiť.

Musela odpovedať. Ale už nechcela pôsobiť tak, ako predtým. Predtým bol jej cieľ presvedčiť velekňažku o tom, nech ich pustí tak, že ju ako cudzinka ohromí znalosťami knihy, ale to sa nestalo. Namiesto toho sa dostala do veľkej kaše a hrozilo jej každý moment obvinenie z rúhanie a už po krátkom čase medzi Amaz'hanami došla na to, že ich viera a tradície sú pre ne príliš dôležité a ak chcela prežiť, nemohla ich spochybňovať.

"$ $Nech mi Ahma'Treta'ji odpustí moju trúfalosť a nech mi odpustí i ahma'velekňažka. Ja som iba zovero a nepatrí mi požehnanie Amaz'han ani vaša múdrosť, ahma'Amaz'ke'u. Nechcela som vás, ahma'Amaz'ke'u., ani veľkú Ahma'Treta'ji uraziť, ani rúhania sa dopustiť."$ $ Vydýchla. Vedela, že musí ešte niečo povedať, niečo čím sa ospravedlní, ale začínala mať z velekňažky pocit, že toto je len hra pred popravou.

"$ $Priznávam, že som sa pokúšala o vysvetlenie toho, prečo aghe‘ s ktorým som prišla je väčší mág než ja a nech Ahma'Treta'ji mi odpustí, dopustila som sa pritom chyby, lebo som zle si vyložila slová Ahma'Treta'ji, ktoré nemám právo vykladať. Nech Ahma'Treta'ji zabudne na môj neuvážený čin a nech dá mi, ako pokorne žiadam, vysvetlenie."$ $ S napätím čakala na odpoveď. Bála sa, to nemohla poprieť. Bála sa ako nikdy predtým.

\begin{center}
*
\end{center}

"$ $Tarny!"$ $ Vykríkla Pauline, zabúdajúc na to, kde je. Chcela ísť ku nemu, ale Amaz'hana vyčarila vzduchovú stenu, na ktorú narazila. I Tarny zdvihol hlavu, keď ju uvidel.

Stratil už všetku nádej, stratil kov aj Tulienku Deľu a teraz aspoň Pauline bola nažive. Ak sa mu nezdalo. Ponoril sa na okamih do magického poľa a zdalo sa mu, že zdanie neklame a aspoň ona bola živá. Ako veľmi v dobrom stave bolo otázne.

Jedna z Amaz'han ktoré ho viedli niečo povedala v ich reči, ktorej nerozumel. Tá, ktorá bola tam predtým s Pauline niečo odpovedala, tiež v ich reči.

Potom sa ešte chvíľu rozprávali, keď sa prehovorili po Fentenzíjsky na Tarnyho. Hovorili s veľkým odporom, videl to na nich.

"$ $Nech Ahma'Treta'ji odpustí mi, že hovorím na aghe‘. A nech aghe‘ počúvne toho, koho počúvať má, s vôle Ahma'Treta'ji. Ahma'velekňažka káže, aby si polodémonku naučil boju s mágiou a nech čo najskôr má znalosti, pre veľkosť Ahma'Treta'ji. Nech vie s mágiou i so zbraňou zachádzať a to, čo ako polodémonka dokáže v boji využívať, lebo taká je vôľa ahma'Amaz'ke'u a Ahma'Treta'ji. A ak aghe‘ nebude spolupracovať s vôľou Ahma'Treta'ji bude potrestaný, ako ahma'velekňažka určí. Ak polodémonka nebude spolupracovať, bude potrestaná rovnako, nech je veľká Ahma'Treta'ji."$ $ Dokončila svoj prejav Amaz'hana.

Tarny sa pozrel na Pauline a obaja pochopili. Nie sú síce v bezprostrednom ohrození života najbližší čas, ale to nič nemení na tom, že sú v poriadnej kaši.

Tarny usúdil, že počúvnuť Amaz'hany je asi najrozumnejšie čo možno spraviť. Prikývol. Ale nevedel, čo má robiť. Amaz'hany tam stáli a pozorovali ich a on premýšľal čo by mal spraviť. Mal by sa tváriť, že učí Pauline, alebo by mal čakať...? Opýtať sa ich netrúfol, zdalo sa mu, že pre Amaz'hany je veľmi ponižujúce už len to, že musia byť v jeho prítomnosti a nahnevať tie, ktoré mali kontrolu nad jeho životom a smrťou naozaj nechcel.

A Amaz'hany tam stáli a vyzerali veľmi nespokojne, i keď ich výrazy nevedel veľmi dobre čítať, keďže nepoznal ich kultúru a predsa len, bol to iný druh. Aj ľudia a zovero mali v niečom inú neverbálnu komunikáciu a Amaz'hany boli...

Zdalo sa mu, že sú viac nahnevané ako predtým, i keď nebol si istý z ich kamenných tvárí.

Napokon sa rozhodol osloviť Pauline.

"$ $Pauline? Rozumela si, čo chcú?"$ $ Nepýtal sa vo fentenzíjčine, ale zároveň bol ostražitý. Nevedel do akej miery by mu dovolili stratiť nad nimi kontrolu.

"$ $Tarny? Oni ťa...“

"$ $Teraz nie. Pýtam sa ťa, či si rozumela čo od teba chcú.“

"$ $Ja...“

"$ $Áno?“

"$ $Chcú od teba aby si ma niečo učil, Tarny? Prečo?“

"$ $Nerob zo seba hlúpu, Pauline. Chcú nás využiť, lebo teraz jediné čo máme sú naše životy a oni môžu... Ak chceme prežiť, musíme ich počúvnuť, inak..."$ $ Nedopovedal to. Jednak preto, že sa dalo domyslieť, čo chcel povedať a dvak, Amaz'hany naňho niečo vykríkli a prerušili ho. V ich reči. Nerozumel im, ale už len z tónu pocítil, že sú asi pobúrené.

"$ $Na posvätnej pôde Amaz'han hovorí aghe nesvätým jazykom! Nech nám Ahma'Treta'ji odpustí a nech aghe' prestane znesväcovať našu pôdu viac než len svojou prítomnosťou, nesvätého jazyka používaním. Nech Ahma'Treta'ji stojí nad nami, nech ďalej nás požehnáva, ale aghe' nechala žiť len preto, aby na jej slávu bola vzdelaná bola polodémonka v mágii a boji, aby na slávu Ahma'Treta'ji a Amaz'han slúžila. Tak nech koná aghe', inak bude hnev ahma'Amaz'ke'u a Ahma'Treta'ji veľký."$ $ Tarny pochopil.

\begin{center}
*
\end{center}

"$ $Ahma'Medizo, ste tam? Volá vás ahma'Amaz'sie\v{}t'en, ako vojna sa blíži, nech nás Ahma'Treta'ji požehnáva, potrebujeme vás."$ $ Amaz'ha\v{}ja klopala na dvere, ktoré oddeľovali hrad od komnát corlovne. Nikto neodpovedal. Zaklopala ešte raz. Nikto. V duchu zakliala. Potrebovali corlovne na poradu.

Nesúhlasila s tým. Corlovne bola síce niekým, koho oslovovala ahma', ale jej lojalita pochádzala len zo slov velekňažky, ktorú si nadovšetko vážila. Neodvažovala sa o nej pochybovať, ale aj tak jej prišlo to, že sú od mesta závislé, proti zámeru Ahma'Treta'ji mať hrdý a silný národ. Ale bolo to na velekňažke. To Amaz'ke'u postavila základ obrany ich ríše na meste, ktoré malo tvoriť prvú líniu.

O Amaz'hanach obyvatelia nevedeli a malo to tak ostať. Amaz'hany do mesta nechodili, bolo to pod ich úroveň a nechceli sa prezradiť, ale na to existovala mágia. Teraz tam Amaz'ha\v{}ja šla len kvôli Medizo s ktorou na chvíľu stratili kontakt.

Rozhodla sa ju nájsť pomocou telepatického poľa, čo nemohla urobiť v ríši Amaz'han, keďže sa tie dva svety nachádzali v inej realite. V komnatách nebola, tam nebolo ani živej duše. Prechádzala v poli hrad, ale ten bol takmer prázdny... až na...

Počula za sebou zdesený výkrik. Otočila sa a zbadala ľudskú ženu. Podľa tých málo informácií ktoré o meste mala nevyzerala na otroka ani na čarodejnicu. Skôr to bola služobná, ktorú nepovolali do boja. Vyzerala mladá, aj na ľudí.

Zareagovala rýchlo, ako vycvičená Amaz'hana. Nevyzerala na nikoho dôležitého a smrť sa dá počas boja vysvetliť dostatočne dobre, volila teda zabitie.

Vyslala laser do ženy, či skôr dievčaťa, ktoré sa ani len neuhlo. Laser ho zasiahol a ono padlo na zem. Amaz'hana neprejavila žiadne emócie, bolo to iba zabitie podradnej rasy. Chcela ešte zničiť telo, ale čo sa stalo vtedy, to vôbec nečakala.

Na mieste kde dievča zasiahol laser sa začala tiahnuť čiara, na ktorej sa telo začala rozpoľovať niečo z neho vychádzať. Oblečenie dievčaťa sa na mieste rozpadlo a ostalo tam nahé telo, ktoré sa na ľudské prestávalo čoraz viac podobať.

Na miestach, kadiaľ sa tiahla čierna, spálená čiara začala koža černieť a vráskavieť a rozpadávať sa. A z nej začalo niečo vychádzať. Napokon už ostali len zbytky toho, čo niekedy bolo dievčaťom a z jeho pôvodného miesta niečo liezlo...

To niečo vyzeralo odporne, ako nič, čo dovtedy Amaz'ha\v{}ja videla, a to vrátane toho, čo zažila za predchádzajúcej vojny o ich ríšu.

Bolo to niečo čierne, odlupovala sa z toho koža a vydávalo to zvuky podobné syčaniu. Amaz'ha\v{}jin prvý reflex bolo do toho strieľať. Chcela sa do toho zbaviť. Lenže akonáhle sa stvorenie dotklo lasera, ten ním len prešiel a zdalo sa, že sa pohlo dopredu rýchlejšie.

"$ $Šegach!"$ $ Zakliala Amaz'ha\v{}ja, ale ešte nespanikárila, bola predsa Amaz'hana. Vytvorila okolo seba vzduchovú stenu a sklopila laser. Namiesto toho vyslala na stvorenie strieborný magický prúd.

Ten obalil stvorenie ale to išlo stále dopredu nemalo štít ani nič také ani sa nezdal jeho vplyv. Amaz'ha\v{}ja vyčarila sebe štít a obranu, lebo sa zdalo, že stvorenie nemá zábrany. Sústredila sa na stenu, ktorá ju oddeľovala od stvorenia.

To vyzeralo, že mu jej kúzla nijako neubližujú a striebornú farbu kúzla prestávala vidieť. Vyslala naňho oheň, ale ani ten ho nezastavil a mala pocit, že čím viac kúziel vysiela, tým je stvorenie silnejšie.

"$ $Pri Ahma'Treta'ji!"$ $ Zvreskla, keď pocítila silu stvorenia, keď narazilo na jej magickú stenu. Keď sa na moment prešla do magického pohľadu videla ako to niečo nenásytne pohlcuje mágiu. Vstrebávalo ju. Malo podobné správanie ako strieborný vstrebávač mágie. Pochopila, že mágiou nemá šancu vyhrať.

"$ $Ahma'Treta'ji pomáhaj..."$ $ zašepkala a vytiahla meč.

Obranu aj štít okolo seba nechala a udržovala ich, iné kúzla už neskúšala. Zrušila i stenu a vrhla sa na tvora. Na prvý raz mu odsekla kus tela, ale to nie aby prestalo útočiť, ale začalo si žiť svoj vlastný život. Pri útoku sa ku tomu priblížila príliš blízko a kusy tvora sa doslova prilepili na jej štít. Cítila ako sa z nej vyciciavajú mágiu, ale ona stále držala. Modlila sa ku Ahma'Treta'ji.

Pochopila, že zbraň nemá zmysel a tvor prežíva. Odhodila je, nemala čas na odkladanie. Vybrala si bleskurýchle z opasku do jednej ruky fialovú paličku – magický zosilňovač a do druhej schmatla ampulku s bledomodrou tekutinou. Tú odzátkovala a jej obsah vystrekol na kusy tvora, na zem a po stenách.

Kde sa tekutina dotkla povrchu rozhorela sa. Okolo Amaz'hany sa vytvorila horiaca čiara. Drevené časti paláca sa chytili rýchlo a oheň sa rozširoval. Len tvory to netrápilo a boli prisaté na jej štít a vyciciavali mágiu z Amaz'ha\v{}je.

"$ $Ahma'Treta'ji! Pomáhaj nám!"$ $ Zvreskla, keď tvor sa dostal už veľmi blízko jej. Horel. Ale očividne mu to neprekážalo. Zosilňovačom mierila na štít a sústredila sa naň. Niečo ďalšie vtom pocítila vzadu na štíte.

Obzrela sa, a to ďalšie tvory, rovnaké ako tie, ktoré ju napadli. A Amaz'ha\v{}ja pochopila že ofenzíva nemá zmysel.

A šancu na útek už premeškala.

Jediné, čo jej v živote ostávalo bolo upozornenie Amaz'han.

"$ $Nech ma Ahma'Treta'ji po smrti k sebe vezme..."$ $ zašepkala. A namierila všetku mágiu čo ešte mala do zosilňovača.

A jej zosilňovač mágie explodoval. A Amaz'ha\v{}ja s ním. Rovnako ako časť hradu.

\begin{center}
*
\end{center}

"$ $Pri Ahma'Treta'ji, stoj pri nás..."$ $ zašepkala Amaz'sie\v{}t'en. Na porade zrazu nastalo ticho. To čo práve počuli bolo varovanie prvý výstrel. Amaz'hana, ktorá ho vyslala už nebola nažive. Prvá obeť vojny.

"$ $Nikto nevyjde von z inopoľa. Urýchlená mobilizácia. Amaz'ha\v{}ja je mŕtva, nech ju Ahma'Treta'ji hostí na nebesách. Potrebujeme veľkú zbraň, nech je Ahma'Treta'ji nad nami. Opakuje sa útok a Ahma'Treta'ji musí získať naše víťazstvo, ako my nosíme jej slávu. Amaz'hany, vojna sa začala."$ $ Amaz'sie\v{}t'en nehovorila dlho, bol stav núdze. Zomrela v bitke prvá Amaz'hana odvtedy, ako obnovili ríšu a táto už nesmela padnúť. Ahma'Treta'ji ich predsa nemohla naveky chrániť, jej požehnaný ľud ony boli, ale ony mali šíriť jej slávu.

Ich armáda už čiastočne mobilizovaná bola, ale predsa len nečakali, že rýchlosť útoku bude tak vysoká.

Jednotky Amaz'han za chvíľu už boli rozostavené po celej ríši, v chrámoch boli kňažky, niektoré sa modlili a iné boli povolané bojovať za svoju bohyňu a ríšu.

Nikto však nevedel, akému nepriateľovi budú čeliť, ale bolo im jasné, že čokoľvek, čo dokáže zabiť Amaz'hanu bude silný nepriateľ, alebo výrazná presila. A nebude ich nepriateľom robiť problém zničiť mesto, takže sa na corlovne Medizo nemohli spoľahnúť. Amaz'sie\v{}t'en už vlastne rátala s tým, že corlovne je mŕtva, alebo sa ňou za chvíľu stane.

\begin{center}
*
\end{center}

Pauline a Tarny poskočili do štádia, že Pauline dokázala sama vytvoriť štít, ale len veľmi slabý a Amaz'hany pozorujúce ich to považovali za zlé.

"$ $Pauline, nie, zle. Musíš vytvoriť magickú stenu, ale nie tak, aby ju mágia oslabovala. Ach nie! Ty chceš ju prinútiť prijímať narážajúcu mágiou tak, aby ti štít posilnila. Bez toho ti mágia štít zničí.“

"$ $Ale ja to neviem!“

"$ $Toto nie je hra, Pauline. Toto musíš vedieť, lebo to rozhodne o tvojom živote a tvojej smrti.“

"$ $Super. Si lepší mág než ja! Prečo ja?!"$ $ Jej štít nikdy nevydržal dostatočne dlho.

"$ $Si polodémonka, Pauline. Polodémoni a démoni majú vyvinutejšie reflexy, mágiou a vôbec všetko lepšie. Len sa učíš príliš krátko. Ale si polodémonka, ty to zvládneš.“

"$ $Ale ako? Ja to neviem!"$ $ Chcela svoj štít úplne zrušiť, mala pocit, že každú chvíľu rezignuj na všetko, ale vždy ju zastavil Tarnyho pohľad. Aj on sa bál.

"$ $Pauline, ešte raz. Predstav si ten štít, skús to úplne inak. Donúť tie magióny sa ku štítu pripájať a replikovať funkciu magiónov v štíte. Musíš to urobiť. Musíš. Toto nie je hra."$ $ Pauline mala slzy na krajíčku a bola zúfalá. "$ $Nevzdaj sa Pauline. Prosím."$ $ Tarny ju o to poprosil, čo často nerobil a Pauline mala pocit že je zúfalý aj on. Chcelo sa jej z toho zrútiť sa, ale vedela, že nemôže.

\begin{center}
*
\end{center}

"$ $Vy nemáte právo odísť, tlmočníčka. Vy ste vošli do našej ríše a zneuctili ste ju, čo je urážka Ahma'Treta'ji a nech hovoríte čokoľvek, neverím vám, ako Ahma'Treta'ji k nám prehovorila v ahma'Knihe, že kto prestúpi jej vôľu z akejkoľvek pohnútky, ten musí zomrieť, tak hovorí sa v ahma'Knihe, tak hovorí Ahma'Treta'ji. A vy, čo ste chrám Ahma'Treta'ji zneuctili, dokonca aj s aghe', aké to rúhanie! Vy, si myslíte, že Ahma'Treta'ji bude k vám milostivá a odpustí vám? Za rúhanie a Ahma'Knihu ste pokúsili vykladať, hnev Ahma'Treta'ji na vás! A ten aghe', to je len samostatný podiel na hriechu, tlmočníčka, tak to chce Ahma'Treta'ji. Aghe' a aj vy, tlmočníčka, tak Ahma'Treta'ji to chce, vy musíte zomrieť. Len svoju úlohu dokončíte a už spočiniete na smrti a nebude to Ahma'Treta'ji, kto vás bude hostiť v nebesách. Lež jej hnev vás stihne, tak hovorí ahma'Kniha. Na túto stenu sa pripojí i aghe' a polodémonka a vy skonáte pre vôľu Ahma'Treta'ji a vôľu ahma'Knihy, tak Ahma'Treta'ji hovorí a tak si zasluhujete."$ $ Keď dohovorila, chvíľu sa odmlčala a Tulienka Deľa mala pocit, že sa začína triasť. Amaz'hana nad ňou a nimi vyniesla rozsudok smrti. Nikdy predtým nebola tak blízko smrti ako práve vtedy.

Jej nádeje vypršali. Velekňažka bola ako ostatné Amaz'hany – ortodoxná, tvrdá a nekompromisná. Nehovorila nič, vedela, že to nemá zmysel. Mohli už len čakať na smrť. Takto mali skončiť. A nikto z ich sveta sa nikdy nič nedozvie. Začal sa jej pred očami premietať jej život, postupne si uvedomovala, čo mala stratiť a bola čím viac v panike v čakaní na sud, ktorý nemala ako zastaviť.

\begin{center}
*
\end{center}

Osoba prekročila mŕtvolu a išla smerom ku hradbám, z ktorých ostávali trosky, útočiaca armáda postupovala rýchlo a devastačne. Stačilo pár hodín z hradieb bola zrúcanina. Pešiaci mesta Tra'itja sa nezmohli na nič, maximálne na chvíľkové spomalenie nepriateľa, ale ich sila bola príliš slabá na to, čo útočilo. Z pred pár hodinami živých obyvateľov Tra'itje ostávali kostry a niekedy ani to. To, čo útočilo neboli ľudia, zovero, víly ani Amaz'hany.

Prešla okolo kostí roztrúsených pred bývalou bránou do mesta. Zdalo sa, že ju bitka prebiehajúca pár metrov od nej nezaujíma a neohrozuje.

Vošla na územie mesta. Okolo nej bolo spálenisko.

Jej kroky smerovali ku zadnej časti mesta, kam sa boje ešte nedostali. Vojaci Tra'itje sa pokúšali nepriateľa zadržať, ale nedarilo sa im.

Palác pre hostí a mágov bol ešte nedotknutý. Tam smerovali jej kroky, tam kráčala.

 Brána do paláca stále vyzerala ako predchádzajúci deň, keď Tra'itja bola mestom, zlatom vykladané dvere boli zatvorené a stenu zdobili fresky. Osoba bez rozmýšľania zdvihla ruku a dvere sa začali otvárať, pomaly. Okolo postavy sa akoby zastavil čas, nevnímala zvuky bitky, ktorá zúrila len pár desiatok metrov od nej.

Vkročila do paláca. Bez dlhého rozmýšľania, akoby mesto už dávne celé prešla, poznala celé jeho zákutia a bola si úplne istá čo robí, pár krát mávla rukou a ovládla magióny vo svojom okolí. Len chvíľu si ich podmaňovala a zmizla. V paláci jej viac nebolo.

\begin{center}
*
\end{center}

"$ $Ahma'Amaz'ke'u! Nech Ahma'Treta'ji pri nás stojí! Potrebujeme vás, ahma'Amaz'ke'u!"$ $ Amaz'je\v{}sha, mladá kňažka, dostala úlohu vyhľadať velekňažku. Mesto bolo napadnuté, jedna z Amaz'han už zabitá a velekňažka bola pri zovero, votrelcovi a namiesto popravy sa v mesta Amaz'han zdržiavali i ďalší dvaja votrelci, medzi nimi aj aghe' - akéto znesvätenie miesta posvätného! Síce si velekňažku vážili, lebo rúhanie bol akýkoľvek odpor ale vrchné veliteľky armády začali považovať aj velekňažkine rozhodnutia za rúhanie a mimoriadne nebezpečné pre ich mesto a ríšu. Preto povolať velekňažku a požiadať ju o začiatok boja považovali za najlepšie, ak sa rúhať nechceli.

"$ $Amaz'li'eda!"$ $ Zvolala jedna z veliteliek na mladú kňažku. Tá pribehla a uklonila sa.

"$ $Áno, ahma'Amaz'lisa\v{}del, nech pri nás Ahma'Treta'ji stojí?“

"$ $Choďte do cvičebne Me'zsa'del a spýtajte sa na výcvik troch. Podajte mi hlásenie, čo najskôr, nech vás Ahma'Treta'ji ochraňuje.“

Ostali tam už len najvyššie postavené Amaz'hany, veliteľky a najstaršie kňažky. Čakali na velekňažku.

"$ $Musíme získať veľkú zbraň. Tá je naša nádej, tak nech Ahma'Treta'ji pomáha nám, získať pre ňu slávu. Slávu a moc pre našu bohyňu. A ak sa ukáže polodémonka ako neschopná...“

"$ $Tak zlyhali sme, nech Ahma'Treta'ji nedopustí to. Lebo potom nastal omyl a ako my môžeme s mýliť, tak nás Ahma'Treta'ji skúša. Ak nezíska zbraň polodémonka tak potom sa mýli Medizo, tá zovero ktorá sa k nám votrela, nech Ahma'Treta'ji odpustí mi ,ak neverím jej posolstvám. Ak Medizo zlyhala, potom musíme my. Akonáhle príde tá kňažka, ktorú sme vyslali pre pokrok a donesie zlé správy tak...“

"$ $Polodémonku môžme zabiť, aghe' predhodiť ako štít na nepriateľa a rovnako tlmočníčku, nech Ahma'Treta'ji prijme krvavú ich obetu.“

"$ $Ak by sa predsa len stala, že pravdu mala Medizo, musí byť polodémonka pripravená čo najskôr. Nemali by sme až tak pochybovať od velekňažkných slovách.“

"$ $Nech nás Ahma'Treta'ji nezatratí, ale ahma'velekňažka pustila do našej a aj jej ríše aghe' a čo je trestuhodnejšie, čo tvrdí ahma'Kniha? Pre slávu Ahma'Treta'ji a pre náš domov a našu ríšu musíme tento boj vyhrať.“

"$ $O čo ide nášmu nepriateľovi, Amaz'hany? Máme pri sebe veľkú zbraň a ona je naša záchrana, ale čo keď nepriateľ vie kde je veľká zbraň, to že u nás je ukrytá? Medizo to vedela a kto to vie ďalší? Aký je zdroj je poznania? Ako sa objavila v našej ríši a kde zmizla? Kto je Medizo?“

"$ $Amaz'lisa\v{}del, nie je neskoro na tieto reči? Či nie je čas na bitku? Kedy sa k nám nepriateľ priblíži?“

"$ $Neviem o našom nepriateľovi nič, Amaz'kl'e'en! Len Ahma'Treta'ji vie o ich úmysloch. Ale musí to byť veľký nepriateľ, alebo veľké množstvo, keď to zabilo Amaz'ha\v{}ju. Ahma'Treta'ji hovorí že poznať nepriateľa je ako vyhrať vojnu a na tú poslednú sme neboli pripravený. Ak im ide o pád našej ríše, nepoznajú striebro a meď a máme výhodu. Ak vedia o zbrani nesmieme im dovoliť dostať sa ku nej a musím eju získať čo najskôr, nech Ahma'Treta'ji dá nám silu.“

"$ $Musíme hlavne konať rýchlo. Aká je situácia na fronte?“

"$ $Mesta sa rýchlo rúca. Netrvá dlho a prídu až ku nám.“

"$ $Kto útočí? Kto je to? Žije Medizo?“

"$ $Je medzi generálmi a áno.“

"$ $A kto teda útočí? To je dôležité."$ $ Odpovedajúca Amaz'hana vydýchla.

"$ $To nevedno. Nikto o ničom takom ešte nepočul.“

"$ $Opíšte.“

"$ $Takmer nikto zo stretu s nimi neprežil. Ale...“

"$ $Ale?“

"$ $To o čom sa hovorí, hovorí, nech Ahma'Treta'ji chráni nás, o čiernych, nedefinovateľných bytostiach ktoré slepo bežia za mágiou, vyliezajúce z mŕtvych tiel."$ $ Vtedy takmer všetky Amaz'hany zbledli, ako im to ich zelená koža dovoľovala.

"$ $Nech Ahma'Treta'ji spôsobí, že to nie je pravda. Nech zachráni nás ak to pravda je, lebo inak nás ani striebro, ani meď nezachráni, lebo náš nepriateľ je niečo čo je zákernejšie než predtým, Ahma'Treta'ji pomáhaj nám, ak nás napadli deti smrti...“

\begin{center}
*
\end{center}

Objavila sa v nedefinovateľnom inopoli, jednej z ilúzii, prepletajúcich sa a tvoriacich ríšu Amaz'han. Jednoducho prešla cez inopolia a ilúzie do skutočnej ríše, priamo na jednu z rušných chodieb. Usporiadané Amaz'hany, povolané do boja, vidiac votrelca hneď strieľali. Zasiahli ju laserom do tváre a ona nemala šancu. Padla na zem. Amaz'hany ju chceli úplne zničiť, ale nestihli odpratať jej telo. Smrtiaca rana na nej sa otvorila.

To, že boli Amaz'hany boli napadnuté vo svojom vlastnom sídle oznámi výkrik z jednej z nich.

Prišli deti smrti.

\begin{center}
*
\end{center}

"$ $Polodémonka!"$ $ Skríkla Amaz'hana. "$ $Aghe'!“. Pauline sa bála. Tarny rovnako. Vliezli do jamy levovej a on nevidel možnosť úniku. Triasla sa. Tak blízka možnosť konca a niečoho, čo nedokázala prekonať ju desilo až na smrť. Amaz'hany sa ich stránili, akoby prenášali mor, či niečo podobne nákazlivé. "$ $Prichádza vaša chvíľa. Tak chce Ahma'Treta'ji, aby naše víťazstvo priniesli ste. Taká je jej vôľa.“

Amaz'hana vyzerala hlavne nervózne – aj ona sa bála. Bola mladá a síce svojej bohyni deklarovala poslušnosť a službu, aj ona sa bála smrti a tá bola tak blízko. Veď len kúsok od nich zomierali Amaz'hany.

Z Amaz'han, čo žili, si len málo pamätalo na poslednú bitku a to hlavne tie z vedenia armády a najvyššie kňažky. Mladé Amaz'hany boli síce vychovávané v poslušnosť Ahma'Treta'ji, ale smrť bola predsa len niečo s čím sa nestretli – popravy neboli až tak časté a oni trávili svoj život tvrdou prípravou a učením k ich nadradenosti. Neboli pripravené na nikoho lepšieho, a napriek tvrdému výcviku bol pre ne šok poznať niekoho lepšieho.

Velenie armády sa celý čas pripravovalo na veľkú bitku, tie čo si pamätali poslednú vedeli o ich nepriateľoch, ale pýcha je predsa len ťažká prekážka. Boli trénované, ale na také straty neboli pripravené a ostražitosť nemali kde predtým získať a výcvik nestačil. Deti smrti boli niečo neprebádané – niečo pochádzajúce zo starých zákutí mágie, mágie ktorú nepoznal nik, mágie tak nebezpečnej, že sa jej všetci mágofyzici vyhýbali a všetky jej plody boli len zdrojom mýtov a legiend.

Deti smrti sa dlho považovali za niečo absurdné, nemožné existencie, zápisy o nich nemali ani vílske, ani fentenzíjske pramene, ktoré ukrývali väčšinu neprebádaných kútov mágofyziky. Len pár slov o nich bolo zapletených do mýtov a legiend a takmer nikto neveril v ich existenciu.

Väčšine bytostí priveľká mágie škodí a pud sebazáchovy ich od priveľkých zdrojov mágie odháňal. Deti smrti boli výnimka. Tie mágia priťahovala to bolo zhruba jediné čo sa o nich vedelo.

V svete sa takmer neobjavovali. Na ich existenciu sa zabudlo a skončili ako starý príbeh, vyskytujúci sa len hororoch.

\begin{center}
*
\end{center}

Tulienka Deľa ostala sama. Bola pripútaná a nemohla sa hnúť a bola zviazaná mágiou, takže nemohla ani len čarovať. Bola odsúdená na smrť, ale velekňažka sa očividne nenamáhala ju popraviť, bola stále nažive, a namiesto toho odišla do bitky. Najskôr si myslela že získala šancu na útek, ale čoskoro pochopila, že jej väzenie bolo starostlivo navrhnuté, aby minimalizovalo šancu uniknúť. Čoraz viac mala pocit, že ju tam velekňažka nechala ako návnadu pre nepriateľa.

Nevedela, čo má robiť, strach zo smrti, ju načisto ochromil a triasla sa. Stena ju studila a z pút ju začínali bolieť ruky, ale nevedela sa z nich vymaniť.

Nevládala. Ostala visieť na stene, odchádzali jej nohy a mala pocit, že každú chvíľu bude mať ruky vykrútené.

Z diaľky počula buchot bitky. Tušila, že počuje svoju smrť.

\begin{center}
*
\end{center}

Pauline zostala stáť. Civela na výjav pred sebou.

Vošli do najtajnejších zákutí ríše Amaz'han. Tam, kde sa velekňažka vtedy, v dávnych dobách, stretla s Medizo, tam odkiaľ vyvierala mágia tvoriaca tunely a polia ich sveta.

Napriek tomu, že bola na smrť vystrašená, prekvapilo ju to natoľko, že zastala. Ich sprievod, pozostávajúci z niekoľkých vysoko postavených Amaz'han a velekňažky ju schytil a potiahol ešte o pár krokov dopredu. A vtedy uvidela finálny cieľ ich cesty.

To striebro horelo. Plameň z mágie, vychádzajúci z dosky horel už veky a mágia stále by ňom nedochádzala. Bola životne nebezpečná. To už vedela aj Pauline.

A uprostred magického plameňa sa vznášalo niečo tak nádherné, že i Tarny zabudol na svoj paralyzujúci strach. Spoznal to, čo bolo v ohni. Poznal Gl'estuvele'i. Ale vždy videl na si trochu myslel, že je to legenda. A teraz ho videl na vlastné oči.


\chapter{V tieni striebornej žiary}

"$ $Vezmite to azma'polodémonka. Je to vaša povinnosť od Ahma'Treta'ji, ktorá vládne nám, pre jej slávu, pre slávu nášho ľudu."$ $ Keby nebola Pauline prikovaná strachom k zemi, niečo by im odsekla, ale nemohla sa ani hnúť. Velekňažka zakliala.

"$ $Tak pohne sa azma', alebo nie? Inak Ahma'Treta'ji bude hnevu plná a jej hnev zakúsiť nechce žiadna Amaz'hana, ani polodémonka, ani človek, ani zovero."$ $ Pauline mala ale pocit, že teraz už nedokáže pokaziť nič. Nemohla sa pohnúť, triasla sa a hľadela na to krásne, čo malo byť jej záhubou.

Amaz'hany vyzerali veľmi nahnevané, a keby Tarny neprerušil tú veľmi napätú atmosféru, asi by Pauline na tom bola oveľa horšie.

"$ $Môžem jej dať posledné pokyny?"$ $ Spýtal sa Amaz'han. Tie s nevôľou prikývli, vidiac že to inak nepôjde, ale ponížené tým, že by mal nejaký aghe' dokázať to, čo oni nie. A tak Tarny sa obrátil na Pauline. Vo fentenzíjčine, ktorú jej prekladal prekladač.

"$ $Počúvaj ma Pauline. Toto je dôležité. Musíš si zosilniť štít a nepúšťať ho. Nie obranu, štít. Potom ho musíš len držať. Počúvaj mágii. Ona je tým, kým si ty. Staň sa ňou. Chyť Gl'estuvele'i a..."$ $ poslednú vedu narýchlo dokončil v angličtine, dúfajúc, že Amaz'hany nerozumejú.

"$ $Presekni mi putá a bež."$ $ Po tejto vete sa Amaz'hany naňho škaredo pozreli. On sa začal ospravedlňovať, dúfajúc, že Pauline pochopí, že v skutočnosti to čo povedal platí.

"$ $Ušlo mi slovo, prepáčte."$ $ Tváril sa akoby to všetko bola jeho zlá výslovnosť. Potom sa ešte raz pozrel na Pauline.

"$ $Dáš to.“

Dá sa hádam triasla ešte viac. Nebola si istá, či Tarny tú vetu povedal naschvál, ale ak áno, tak...

"$ $Tak..."$ $ Zdalo sa že ešte chvíľu a sama velekňažka ju k tej zbrani hodí.

Obzrela sa. Pozrela sa na Tarnyho. Videl ako sa chveje. Aj chápal tou že by najradšej utiekla preč.

Akoby chápal, čo cíti. Že sa nemôže ani hnúť, lebo ju strach prikoval k zemi. Že tak veľmi chce ujsť. A jeho oči jej niečo hovorili. Že bude môcť. Len nech ho zoberie. Gl'estuvele'i.

Sústredila sa na svoj štít tak veľmi ako to šlo. Tak ako jej hovoril Tarny. Postupne ho zvyšovala a snažila sa nemyslieť na to, čo malo ísť potom.

Vošla do magického poľa. Bola síce paralyzovaná strachom, ale cítila, že sa jej do krvi dostáva adrenalín. Cítila mágiu. Ako nikdy predtým. Videla okolo seba zhluky magiónov. A tam, kde bola ich koncentrácia najvyššia, tam v samom strede, tam sa vznášalo Gl'estuvele'i. Cítila ako sa jej štít zmenšuje smerom k nej. Adrenalín ja zaplavil. Pomaly nevedela čo robí, akoby intuitívne, s rovnakou ľahkosťou akou rozprávala o inopoliach. Magióny sa tlačili do dosky ktorá sa vznášala v strede magického víru.

Schmatla Gl'estuvele'i a s ľahkosťou sa na mágii vzniesla a pretrhla Tarnymu magické putá. Ten ju schmatol, ona upustila Gl'estuvele'i a bežali preč. Strieborná doska pulzovala a zdalo sa, že smeruje k výbuchu.

Nevnímala. Bola voľná. Možno. Na pár sekúnd. Zaplavil ju adrenalín, že necítila bolesť. Mala spálené ruky od veľkej zbrane. Len tie. Boli to ešte relatívne malé zranenia na to, aké mohla pri takom množstve mágie utŕžiť.

Bežali. Tarny ju držal za ruku, Pauline samotná bola v priveľkom šoku na orientovanie sa kde chce ísť. Len bežala s ním, nerozmýšľajúc nad ničím. Popravde, ani on poriadne nevedel kde ide. Len volal meno Tulienky Deli.

\begin{center}
*
\end{center}

Počula ako niekto volá jej meno. Najskôr sa jej to zdalo ako halucinácia, ale keď ho počula dvakrát, nepochybovala. Síce hlas nebol hlasný, ale spoznala v ňom Tarnyho. Je nažive! Uvedomila si.

"$ $Tarny!"$ $ Zvrieskla. Bolo to takmer okamžite. Nerozmýšľala. To by zaberalo priveľa času.

"$ $Tarny! Pomôž mi!"$ $ Potrebovala ho. Ona sama sa nedokázala poriadne pohnúť a mykanie jej spôsobovalo len bolesť.

\begin{center}
*
\end{center}

Tarny ňou mykol a odbočili. Priamo za Tulienkyným Deliným hlasom.

"$ $Tulie!"$ $ Podlaha pod nohami sa im rúcala. Boli v spleti chodieb a z jednej z nich vychádzal hlas ich kamarátky.

"$ $Tarny! Pauline!"$ $ Zas počuli jej hlas. A bežali za ním. Pauline mala pocit, že ju ide roztrhnúť, šialene ju boleli nohy, pichalo jej v boku, nedokázala sa nadýchnuť. Mala pocit, že čo chvíľu zastane a už sa nepohne. Jediné čo ju držalo na nohách bola mágia, ktorú do nej Tarny posielal, aby sa dokázala vôbec pohnúť.

Keď našli Tulienku Deľu, Pauline sa oprela o stenu a ťažko dýchala. Jej pľúca horeli a mala pocit, že každú chvíľu odpadne.

Tarnymu netrvalo dlho, kým zrušil putá Tulienke Deli a obaja sa obrátili na Pauline, ktorá sa nevládne opierala o stenu a prudko dýchala. Tulienka Deľa natiahla k nej ruku, z ktorej začalo prúdiť svetlo. To ju zdvihlo a kĺzala sa za nimi, bezvládna, ako handrová bábika.

\begin{center}
*
\end{center}

"$ $Máme Gl'estuvele'i!"$ $ Vykríkla Velekňažka. "$ $Ale pri Ahma'Treta'ji, oklamali nás, aghe‘, nech nás Ahma'Treta'ji netrestá! Však pozrite sa na strieborný kov! Jeho žiara je taká, ako nikdy predtým a pulzuje, pri Ahma'Treta'ji, však on vybuchne! Bežte preč, pri Ahma'Treta'ji! Potrebujeme ísť preč z ilúzie!“

"$ $Ahma'Amaz'ke'u! Ste si istá, pri Ahma'Treta'ji?“

"$ $Pozrite sa zrakom svojim pozorne, pri Ahma'Treta'ji, nech nás zachráni. Necháme nepriateľa vybuchnúť a my sa zachránime, áno, pri Ahma'Treta'ji, len sa treba dostať z inopoľa!“

"$ $Lež čo je realita, ahma'Amaz'ke'u?“

Amaz'ke'u schmatla Gl'estuvele'i, vzhliadla k svojej bohyni a poprosila ju o silu. Z rúk jej začali žiariť belasé prúdy svetla, vytvárajúce štít okolo každej jednej z nich. A Amaz'ke'u vytiahla medený kov, ktorý získali od toho aghe', čo ich zradil. A ten začal kmitať, hýbať sa smerom ku striebornému kovu.

"$ $Necháme meď vybuchnúť. Ako pascu. Nepriateľa láka mágia. Bežte, pri Ahma'Treta'ji, bežte!“

\begin{center}
*
\end{center}

"$ $Potrebujeme sa dostať z inopolí, Tarn! Mám pocit, že toto je celé ilúzia a prestávam vnímať, čo je realita a čo je nadstavané inopole.“

"$ $Nemôže sa premiestniť Pauline?“

"$ $Tu nie! Potrebujeme sa dostať do mesta. Alebo úplne von. Toto je strašná spleť!“

"$ $Vieš čo? Sprav veľkú obranu. Ja sa nás pokúsim dostať z tejto splete.“

Keď sa ponoril do magického pohľadu, videl strašný chaos, magióny poletovali sem a tam v náhodných hlúčikoch. Brány do ďalších svetov sa prekrývali a on nevedel, ktorý je ten skutočný a ktoré sú len ilúzie.

Celá ríša Amaz'han bola postavená na umelo postavených inopoliach, teda ilúziách. Neexistovala žiadna mapa a pre Amaz'hany bolo otázkou bezpečnosti ovládanie inopolí. Mechanizmus nebol väčšine známy, ani do magického spoločenstva na Fanase a zemi sa ešte nedostal, ale tak boli inopolia a polia, s výnimkou popredných mágofyzikov otázka legiend.

"$ $Potrebujeme sa dostať k mechanizmu. Alebo ísť náhodne.“

"$ $Kde je zdroj?“

Tarny si vzdychol.

"$ $Mám pocit, že je to striebro. A to ide vybuchnúť a tým pádom toto zruší.“

"$ $A čo s nami?“

"$ $Neviem! Musíme nájsť východ! Priprav sa!“

Premiestnil ich do jednej z ilúzií. Chrám. Hneď do ďalšej. Hala z ich príchodu. Plná nastúpených Amaz'han. Rýchlo preč. Prázdne inopole. Dochádzala mu energia a mágia.

"$ $Počkajte."$ $ Doberal mágiu. Krátil sa im čas.

Ďalším miestom bolo cvičisko. Tiež sa odtiaľ rýchlo dostal.

"$ $Sme v meste.“

"$ $V tom čo z neho ostalo."$ $ Vydýchla Tulienka Deľa.

\begin{center}
*
\end{center}

"$ $Koľko času ešte majú?“

"$ $Do hodiny. Už začínam strácať trpezlivosť.“

"$ $Deti smrti našli zdroj.“

"$ $Výborne. Takže môj odhad bol dobrý.“

"$ $Prežije Gl'estuvele'i?“

"$ $Je to legendárny predmet. Prežije. Hlavne aby nám s ním tie zelené potvory neušli. Optimálne to chce ďalšie jednotky.“

"$ $Už nemá...“

"$ $Stále nie ste poriadne zaučená. My máme všetko. Len stačí povedať.“

\begin{center}
*
\end{center}

"$ $Ako sa dostaneme z mesta, Tarn?“

"$ $Niečo mi tu nehrá Tulienka Deľa. Nie som si istý, kto túto skazu spôsobil, ale v každom prípade, bolo to len za pár hodín a možno ani to. Chceme sa niekde mimo z otvorenej ulice.“

"$ $Čo najrýchlejšie."$ $ Tarny prešiel mihom oka magický pohľad na mesto. A celé mu to prišlo absolútne neprirodzené, takú magickú skazu ešte nevidel. Mesto zasiahli doslova výbuchy mágie a to v takej miere...

"$ $Mám pocit, že tu útočilo niečo, čo nepoznám. Alebo ja neviem. V každom prípade, musíme odtiaľto zmiznúť. Štíty. Absolútne odmagizovanie. Sú tu výbuchy mágie v prázdnom, kedysi ľudnatom meste. Zobuď Pauline. Nedovolím si použiť mágiu. Ak ma intuícia neklame, tak tu niečo magické vybuchuje.“

Tulienka Deľa postavila Pauline na nohy. Tá vzápätí otvorila oči a zmätene sa okolo seba pozerala.

"$ $Rýchlo. Nie je čas na vysvetľovanie. Skrátene, ak neujdeme, asi vybuchneme, alebo niečo horšie. Takže bež za nami."$ $ Schmatla ju za ruku a Pauline sa ani nestihla spamätať a už mechanicky bežala za nimi.

"$ $Kde ideme Tarn?“

"$ $Tretia časť mesta. Mesto chudobných. Je to tam najmenej dotknuté, útok sa sústredil na palác a časť, kde sa išlo do ríše tých zelených.“

"$ $Si si istý?“

"$ $Myslíš si, že som si vôbec ešte niečím istý?"$ $ Tarny bol, napriek všetkému nebezpečenstvu, strachu a smrti ktorá ho obklopovala, rád, že Tulienku Deľu počuje. Ich cesty sa rozdelili príliš rýchlo a nevedeli, či sa vôbec niekedy ešte uvidia.

Mesto tretej vrstvy bolo úplne iné než prepych, ktorý videli najskôr. Boli síce najmenej zničené, ale to neznamenalo, že ho netvorili ruiny.

Chatrče obyvateľov ani chatrče nepripomínali, boli stavané zo slamy a hliny, niektoré, ktoré vyzerali lepšie v rámci možností, mali v konštrukcii chatrče viditeľne použité drevo. Dvere nemali a strechy tvorili len upletené kusy trávy, diery v stenách mali zrejme značiť okná. V niektorých boli diery od magických výbuchov, niektoré stáli ako-tak už len z definície.

"$ $Dobre."$ $ Zastal Tarny. "$ $Zdá sa, že všetci zahynuli.“

"$ $Koľko máme času?"$ $ Spýtala sa Tulienka Deľa. Pauline sa pokúšala spracovávať, čo sa vlastne deje.

"$ $Nemám poňatia. A netuším ani to, ako sa stadiaľto dostaneme.“

"$ $Ideme preč z mesta.“

"$ $To je možno jednoduché, ale... Počkať, Pauline, môžeš sa premiestniť?“

"$ $Čo...? Možno? Netuším.“

"$ $Tak to skús preboha."$ $ Pauline začala myslieť na miesto blízko nich, tak ako vždy keď sa premiestňovala, ale nič.

"$ $Nefunguje to.“

"$ $Sakra. Dobre, máme o plán menej. Naša hlavná priorita, dostať sa z mesta. Optimálne bez strát. Chcelo by to mapu. Lebo toto, toto je hotový labyrint. A to nie je všetko. Potrebujeme sa dostať z ilúzie, kým vybuchne. Takže..."$ $ Odmlčal sa a pozrel sa na Tulienku Deľu. Pauline cítila nebezpečenstvo vo vzduchu, ale stále sa nespamätala z toho, čo všetko sa udialo za posledné minúty a tak len počúvala.

"$ $Ako vyzerá magické pole?“

"$ $Mišmaš. Chaos. Nič nevidno. Výbuchy mágie. Niečo to tu výrazne poškodilo.“

"$ $Dobre, to viem. Ďalej, videli sme mapu. Ja som v knižnici niečo predtým čítala. Len si spomenúť.“

"$ $Mapa bola len paláca, kde sme boli.“

"$ $Ty si pňak. Dobre, alternatíva dva. V popise mesta bolo uvedené, že podrobná mapa sa nachádza v severnom chráme. Ak sa tam dostaneme načas a nie je zničený...“

"$ $Tak to môže byť naša jediná nádej, chápem. Len naozaj netuším kde v tom meste vlastne sme.“

"$ $Ale vieš kde je sever.“

"$ $Dobre."$ $ Prikývol. Nemal na výber. Nevidel inú cestu k záchrane.

Cesta viedla cez chatrčovú časť mesta, tam kde boli. Na ceste boli kamene, jamy a museli si dávať pozor pod nohy. Nikde ani živej duše. Všetko bolo mŕtve a v diaľke počuli výbuchy. Mesto sa im zrazu zdalo oveľa väčšie než prvý raz a ulice im pripadali nekonečné a všetky rovnaké, rovnako chudobné a zúfalé.

Po čase sa cesta zmenila na normálne ulice. Lemovali ich mŕtvoly, alebo skôr ich časti, vzdialené od seba na desiatky metrov. Všetky telá boli spálené, ulicami sa niesol smrad spáleného mäsa. Pauline by bola aj znechutené, ale nevládala vnímať.

Keď videli chrám, Tarny spomalil. Nazrel krátko do magického poľa a potom prikývol, že asi môžu pokračovať.

Chrám bol relatívne zachovaný. Jedna jeho stena sa zrútila a namiesto nej boli telá, akoby magicky naskladané tak, že nepadli dovnútra. Veľká socha vnútri vyzerala neporušená. Boli už príliš blízko keď si Tulienka Deľa všimla že vnútri niekto je. Chcela sa už otočiť, ale vtom sa postava obrátila a v čiernom plášti, s dvoma tabuľami v rukách na nich hľadela corlovne Medizo. Dvere chrámu sa razom zabuchli, až ostali prekvapený, nevšimli si predtým, že nejaké mal. Tarny inštinktívne chcel vyčariť obranu a štít, ale na okolí proste mágia nebola. Boli v ďalšej ilúzii. Potichu zaklial a neskryl svoj strach. Tulienka Deľa sa ho chytila. Aj ona mala smrti a strachu už okolo seba dosť. Vtom corlovne Medizo prehovorila.

"$ $Výborne. Výborne."$ $ Jej hlas sa ozýval po celom chráme. Pokračovala. "$ $Žiaden strach, toto je úplne iná ilúzia, ktorá nezávisí od striebra."$ $ Obrátila sa na chvíľu od nich, položila na oltár tabule a vzala odtiaľ palicu. Gl'estuvele'i.

"$ $Takže naše Amaz'hany boli presvedčené, že verím v Ahma'Treta'ji, aké pozoruhodné, že? A pritom na to stačilo, aby som slečne velekňažke zachránila život. Nuž milosrdenstvo sa občas vyplatí, že?"$ $ Skúmavo si prehliadala Gl'estuvele'i. Tarny prestal mať paralyzujúci strach a pozorne ju počúval a pritom premýšľal. Vymanil sa Tulienke Deli. Pauline bola stále mimo.

"$ $Pričom poslúžili ako dokonalá návnada. Síce, ich ríša mi bola vcelku sympatická. Škoda ich. Ale keď už tie deti smrti chceli mágiu..."$ $ Tarny si až teraz uvedomil, odkiaľ boli záhadné výbuchy mágie. Deti smrti videl raz v živote a to bola situácia, keď sa im tesne vyhli a Pauline takmer zomrela. Teraz mu ani nezišli na um.

"$ $Ani neviem kto ich poslal, ale na tom nezíde. Gl'estuvele'i je von z mágie a teraz je mi vydláždená cesta k tomu naplniť proroctvo."$ $ Pauline ju vôbec nepočúvala a ani sa tak netvárila. Bola rada, že sa mala o čo oprieť a tak sa malátne opierala o stenu chrámu.

"$ $Počuli ste polodémonka?"$ $ Pauline sa strhla a vyzerala veľmi nechápavo.

"$ $O jej budúcnosti sa vraví a ona ani nepočuje, tak tomuto vy hovoríte slušnosť... Polodémonka!"$ $ Vykríkla na ňu. Pauline vyzerala vystrašene a chytila sa Tulienky Deli. Triasla sa.

"$ $Takže, keď už ste všetci veľmi pozorní, pokračujme. O chvíľu nebude žiadne z inopolí vytvorených Amaz'hanami existovať, nastane totálny kolaps a aj ich bohyňa existuje..."$ $ Hovorila veľmi výsmešne. "$ $Tak nech ich zachráni."$ $ Usmiala sa. Tým spôsobom, ukazujúcim, že to je ona, ktorá je pánom situácie.

"$ $Aj vy aj Amaz'hany ste mi výborne poslúžili, ale vaša úloha ešte neskončila."$ $ Odmlčala sa a chvíľu sa na nich pozerala a sledovala ich reakcie.

"$ $Mesto Tra'itja nebolo mestom troch tabúľ nazvané len tak. Tri tabule existujú a dve čo chvíľu zaniknú a stanú sa záhubou. A tretia... Vlastne, nepotrebovala by som ich, ale... Ak sa chceme držať symboliky, tak chcem všetky tri. A vtedy príde vládkyňa zas. Pekne podľa symboliky a legiend. Ako hovorí Oko, všetko sa opakuje. Aj vládkyňa príde tak ako som povedala."$ $ Tarny ju prerušil.

"$ $Takže vy nás chcete využiť na niečo, čo vlastne nepotrebujete, len vám to zapadá do vašej rozprávkovej cesty tu vláde, ku ktorej sa viete ale dopracovať aj inak. Alebo sa mýlim?“

"$ $V podstate nie. Ale toto je hra. Celý život je hra v ktorej každý nakoniec zomrie, ale pokiaľ je účastníkom tejto hry snaží sa výhru. Ale výhra neexistuje, dá sa k nej len priblížiť. A aké suché by to bolo, keby len tak. Je to hra a tá potrebuje svoju estetiku. Tu nejde o cieľ, ale o cestu k nemu. Keď máš príbeh a vieš koniec, stále ho chceš čítať, počúvať? Lebo tu nejde o výsledok, ale o cestu pán Lietavý. A vy ste súčasť chodníku.“

"$ $Kto vám zaručí, že vaša rozprávka sa stane skutočnosťou?“

"$ $Kontakty. Hovoríš o mikro-príbehoch, každý je nahraditeľný. Ale v makroskopickom pohľade všetko vyjde ako má. No a že sa vzoprie jeden, dvaja, systém sa nevzoprie. A systém je dobre nadizajnovaný.“

"$ $Tak prečo dávate úlohy nám, keď každý je nahraditeľný?“

"$ $Ja vám nedávam úlohy, ja informujem. Ako hovorím, aj keby ste nechceli budete v situácii, keď to, čo chcem bude pre vás to najvýhodnejšie. Tak tak. Systém je dobre nadizajnovaný.“

"$ $A čo od nás chcete? A prečo nám to hovoríte?“

"$ $Pochybnosti, Lietavý, pochybnosti sú mocná vec. A možno od vás nič nechcem. A vlastne áno, spomenula som si. Alebo?“

"$ $Zahráva sa s nami."$ $ Pošepla Tulienka Deľa Tarnymu.

"$ $Nepotrebujete vedieť čo máte robiť, lebo to aj tak urobíte."$ $ Zasmiala sa.

"$ $A teraz, škoda rečí. Nech je k vám vír milosrdný."$ $ Keď dopovedala a nestihli si ani uvedomiť, čo robí, začali z tabúľ vyletúvať magióny. Až vytvorili bránu, ktorá ich pohltila.

\begin{center}
*
\end{center}

"$ $Výborne.“

"$ $Ako sa dalo čakať.“

"$ $Máme Gl'estuvele'i.“

"$ $A čo sa iné dalo čakať, Leana?“

"$ $To bolo len spokojné konštatovanie.“

"$ $Len aby.“

\chapter{Príbuzný Alchymista}

"$ $Keria!"$ $ Sylvia sa síce pokúšala ovládať, ale keď uvidela Keriu a Loviisu, pribehla k nim.

"$ $Nie tak zhurta, Sylvia. Toto je informácia, ku ktorej som sa náhodou dostala a mám pocit, že si tu nelegálne. Optimálne sa o tomto Loriatar nesmie dozvedieť. Menej optimálne tomu neprikladať vážnosť. Spravila som okolo nás blokádu a teda by nemali nič z toho čo hovoríme rozoznať. To telepatické pole im posiela niečo úplne iné. A na nás je zmyslovka.“

"$ $Výborne."$ $ Prikývla Sylvia. Loviisa sa nesmelo rozhliadala.

"$ $A čo teda? Rýchlo. Ak nechceme vzbudiť podozrenie.“

"$ $Nasledujte ma."$ $ Mrana kráčala k budove, ktorá mala byť laboratóriom.

"$ $Čo si našla, Keria?“

"$ $Počkaj. Čím menej o tom budeme hovoriť, tým lepšie. Však vieš."$ $ Sylvia prikývla.

Prešli ulicu a mierili do bielej vysokej budovy. Tá mala na sebe nápis "$ $Chemické výskumné centrum"$ $ a Sylvii nenapadalo čo by stálo za Mraninu pozornosť.

"$ $Ako sa chceš dostať dovnútra?“

"$ $Jednoducho."$ $ Podala im papiere. Sylvia si ich letmo prezrela.

"$ $Sme pozvané?"$ $ Mrana na ňu žmurkla.

"$ $Hej. Našťastie korupcia funguje všade.“

"$ $O čo vlastne ide?"$ $ Odvážila sa spýtať Loviisa, ktorá nevedela poriadne o čo ide. Jediné, čo jej Mrana povedala, bolo, že sa musia s niekým stretnúť. A že si má vziať Lasermeč. Tak to asi bolo niečo vážne.

"$ $Uvidíš. Tu na ulici to nie je dobré rozoberať.“

"$ $A tam?"$ $ Mierila Loviisa pohľadom na tú budovu.

"$ $Máš pravdu, je to vládna budova."$ $ Prikývla a spýtavo sa pozrela na Mranu.

"$ $Máš pocit, že by som chcela nám niečo zlé? Teda áno, vždy ma mohli nadopovať drogami a prikázať mi zabiť ťa, ale no tak. Neboj sa. Máme kontakt. Ktorému dôverujem."$ $ Sylvii sa to moc nezdalo. A nie úplne páčila. Na tom neznámom, nepriateľskom mieste radšej nedôverovala nikomu.

"$ $Loviisa?"$ $ Oslovila ju Mrana.

"$ $Áno.“

"$ $Maj v ruke Lasermeč. Teda nie viditeľne. Len ho drž a buď pripravená ho použiť, keď ti poviem, alebo keď uvidíš, že je vhodná chvíľa. Lebo Murphyho zákony. A myslím, že výnimky sme už všetky vyčerpali."$ $ Loviisa bez slova prikývla a nasucho preglgla. Už by si na tie ohrozenia života mohla zvyknúť.

Budova nebola oplotená. Chodník a vlastne aj celé okolie bolo dláždené betónom a popri okraji ulice viseli bilboardy o budovaní socializmu. Žiadne reklamy. Len trochu iný vizuálny smog.

Vstup do budovy vyzeral stroho. Biela fasáda, biele steny, len nad vchodom sa týčil veľký, červený nápis "$ $Budujeme Loriatar“.

Vnútri budova nebola až tak zaostalá, ale ničím nepripomínala sídlo Hlavného Súdruha. Steny boli z nejakého zvláštneho druhu dreva, na stenách viseli socialistické plagáty a hľadela na nich nevľúdne zazerajúca zovero.

"$ $Nie ste pracovníci."$ $ Oznámila im.

"$ $Máme pozvanie. Loran Hellie, stredisko kvantovej chémie a mágochémie, kancelária 5L."$ $ Mrana vyzerala úplne kľudne. Podala jej papiere s pozvánkou. Vrátnička si ich nedôverčivo prezerala, ale napokon prikývla a pustila ich.

"$ $Nie je tu Hlagenovo pole. Výborne."$ $ Oznámila im Mrana telepaticky.

"$ $Ako to vieš?"$ $ Spýtala sa jej Loviisa.

"$ $Myslíš si, že nie som pripravená? Moje prístroje ho nezachytili.“

"$ $Výborne. Kde teraz?“

"$ $Tadiaľto."$ $ Pri rozcestníku chodieb odbočili do mágochémie. Až zastali pred miestnosťou z menom 5L. Mrana zaklopala.

Po chvíli im otvoril muž, zovero, ktorý Sylvii niekoho strašne pripomínal. Keď uvidel Mranu vydýchol si.

"$ $Poďte dovnútra. Rýchlo."$ $ Telepatizoval im. Keď vošli, náhlivo zatvoril dvere a zamkol.

"$ $Poďte za mnou."$ $ Oznámil im. Loviisa stále zvierala Lasermeč, ale šli za ním.

Zaviedol ich do izby, ktorá vyzerala, že v nej býva.

"$ $Tak. Už môžeme rozprávať. Toto je ilúzia, takže, nás nenájdu. Vždy ju tvorím nanovo. Aby som sa predstavil, Keria, vy už viete, ale vás dve. Som Tanery Lietavý, otec Tarryho Lietavého. Moja manželka bola Čeria Lietavá, dcéra Arabely Tlogenovej. Do Loriataru som musel odísť. Moje dôvody sú na inokedy. Teraz ide o to, že vy chcete odísť."$ $ Mrana prikývla. "$ $Viem čo je Loriatar zač a vaše rozhodnutie je to najlepšie, čo môžete spraviť. A teda, kým ste, ak to nie je tajné?"$ $ Sylvia sa nedôverčivo pozrela na Mranu. Tá prikývla a tak sa predstavila.

"$ $Sylvia Mänchenová. Inak poznám vášho vnuka, Tarnyho Lietavého.“

"$ $To je zaujímavé počuť. Nemal som správy o rodine odvtedy ako som odišiel."$ $ Tanery nevyzeral, ako keby mal z rozšírenia svojej rodiny nejaké emócie. Alebo ich len nechcel dávať najavo. Prišiel čas na Loviisu.

"$ $Loviisa Mänchenová."$ $ Prikývol, ale meno nepoznal. Nemal odkiaľ.

"$ $Vy nám chcete pomôcť? Prečo?"$ $ Sylvia mu nedôverovala.

"$ $To je na dlhšie. Ale skrátene, nemáte tu byť. Loriatar nie je pre vás. Pre vás zvlášť, slečna."$ $ Sylvia sa podvedome roztriasla. Nevedela, čo vedel, ani čo mu Mrana povedala, ale z toho domnelého Taneryho jej naskočila husia koža. Pokračoval. "$ $Loriatar je zložitá štruktúra a vy musíte odísť. Ešte nie je neskoro. Ste tu síce v relatívnom bezpečí, ale nie nadlho. Toto je miesto, kde ak ostanete pridlho, už sa odtiaľto nedostanete. Loriatar je systém a vy nie ste jeho súčasťou, ale Hlavný Súdruh to nechápe. Keď ste sa tu dostali, keď som sa tu dostal ja... História sa bude opakovať a siedma doba nie je pre vás. Pre nikoho z nás, ale naše rozhodnutia si musíme niesť. Vy ešte máte šancu. A ja vám chcem pomôcť.“

"$ $Prečo to robíte? A prečo pre nás?“

"$ $Je viac dôvodov. Po prvé, robím to aj pre moju rodinu. Napriek tomu, že som ju pred rokmi a možno i desaťročiami opustil, stále si myslím, že chcem aby niečo vedeli. A dúfam, že im to oznámite, čo vám potom poviem, aby ste im povedali. Druhý dôvod je, že viem, že tu nepatríte. Patríte do šiestej doby a pre vás, ešte nie je neskoro. A po tretie, v šiestej dobe sa niečo chystá. A chystá sa to už od doby ako Izabeta Tlogenová opustila brány Loriataru. A vy ste toho súčasťou.“

"$ $Čo?"$ $ Sylvia nevydržala. "$ $Izabeta Tlogenová bola tu...?"$ $ Pre Sylviu aj Loviisu to bol šok, Mrana o tom očividne už vedela. Sylvii sa jej nevedomosť vôbec nepozdávala.

"$ $Takže to nie je známa informácia. Ako chce Loriatar tak sa koná. Pravdou je, že Izabeta Tlogenová, pred rokmi tu strávila pár týždňov, a odtiaľto pochádza aj Čeria a Alfred Botes, Izabetine prvé deti.“

"$ $Čo?"$ $ Loviisa nechápala. V spoločenstve nebolo slušné klebetiť o Izabete a tak nevedela, že má aj ďalšie deti, a vlastne ju to predtým absolútne nezaujímalo. Teda tak si pamätala.

"$ $Kto je otcom sa nevie. Ale je fakt, že keď sa Izabeta vrátila z Loriataru, vrátila sa s dvoma svojimi deťmi. Ale oficiálne to nikdy jej deti neboli.“

"$ $Nemohli by sme prejsť k plánom?"$ $ Spýtala sa ho trochu nervózna Mrana.

"$ $Máte pravdu, každá sekunda je drahá. Takže, s Keriou som sa rozprával a je potrebné, aby ste odteraz až do vášho odchodu nevyšli z tejto ilúzie. Vyžaduje to síce dôveru, ale bez nej to zrejme nedokážeme. Potrebujete získať kľúče od Loriatarskej brány a dostať sa k nej. Je však v sídle Hlavného Súdruha. Takže, moje podmienky sú, že ak vás zatknú, ja s týmto nič nemám. Nie som v pozícii, aby som z Loriataru odišiel a potrebujem tu prežiť. To, že vám pomáham je pre mňa dosť nebezpečné. Rozumiete?"$ $ Prikývli.

"$ $Ďalej, ako dostaneme kľúče. Je známe, že ich má Mei. Je potrebné, aby ste ich získali. Či už silou, alebo lsťou. Mám plány a rozvrh Mei Tesocovej, vďaka tu Kerii.  Takže, nezabúdajme, posledný krok je brána. Ak si ešte pamätáte váš príchod, tak vchod je tam. Je to viac menej v útrobách sídla hlavného súdruha a tak si už musíte poradiť sami. Ja vám pomôžem len čiastočne. Takže, dôležité veci k organizácii. Odo mňa máte niekoľko substancií. Dúfam, že ale nebudú potrebné. A ešte niečo."$ $ Tanery vytiahol knižku, pripomínajúcu denník a v ňom list. "$ $Toto prosím, odovzdajte mojej rodine. Prosím. Spolieham sa na vás."$ $ Mrana prikývla. Sylvia ju nasledovala a tak Loviisa rovnako, síce vôbec netušila, kto je jeho rodina. Zavládla chvíľa ticha, ktorú prerušila Sylvia.

"$ $Tanery, je ešte, čo od nás chcete? Alebo prejdeme k plánu?“

"$ $Kľúče. Prvý krok."$ $ Zhlboka sa nadýchol.

"$ $Ako?“

"$ $Mei Tesocová je možno hlboko oddaná režimu, ale nie je dokonalá.  Ste tri. Nie ste bežní zblúdilci, to viete. Treba vystihnúť správnu chvíľu. A tá, tá podľa Kerie nadíde večer. Ak teda Mei nezmenila svoje plány z posledných dní. Vtedy by sa mala nachádzať v severnej časti sídla a...“

\begin{center}
*
\end{center}

"$ $Myslela ste si, že ovládnete Loriatar?"$ $ Mei sa naklonila k Sylvii, paralyzovanej mágiou, a výsmešne sa usmiala. Tak istá si svojou výhrou pokračovala.

"$ $Ale v žiadnom prípade. Loriatar vyhrá. Siedma doba príde a podmaní si svet. Toto bolo napísane vyslovené.“

"$ $Pozoruhodné."$ $ Ozvalo sa za ňou. Otočila sa. A tam stála rovnako Sylvia. Mei sa pozrela späť, ale tam už nebola. Prepla do magického pohľadu.

"$ $Takže hráme sa so zmyslovou mágiou... uznávam, že dobre, ale..."$ $ Sylvia sa rozplynula a Mei sa posmešne usmiala. Jej úsmev ale zmrzol.

"$ $Možno až príliš."$ $ Ozvala sa zas Sylvia. Mei sa otočila a vybrala zbraň.

"$ $Ach, Mei Tesocová."$ $ Ozvalo a spoza nej. Ďalšia Sylvia. Stáli a medzi nimi  bola ona. Vytiahla svoju druhú zbraň a mierila na obe.

"$ $Ilúziu..."$ $ Opakovali obe ilúzie.

"$ $Nemôžete..."$ $ Opakovali a postupne, po každom slove sa ich počet zdvojnásoboval.

"$ $Zničiť..."$ $ Pri poslednom slove dostala Mei dávku Solanu do krku.

Neomráčilo ju to ale. Síce mierne vyviedlo z miery, ale aj tak sa spamätala vytvorila laser, ktorým opätovala Solan. Ilúzie sa rozplynuli a mierila na ňu Loviisa Lasermečom a za ňou stála Mrana, tvoriaca okolo nich obranu.

"$ $Kľúče od brány."$ $ Povedala jej Sylvia. Mei stuhla. Sylvia jej priložila zbraň ku spánku.

"$ $Teraz hneď. A nechám vás žiť.“

"$ $Nie.“

"$ $Naozaj?“

"$ $Áno."$ $ Mei Tesocová ale skôr než dopovedala chytila Sylvii ruku, skrútila ju, až Sylvia takmer zjojkla od bolesti. Zbraň ale nepustila. Lež za to, že Mei prestala vysielať Laser, do nej zasiahol prúd Solanu z Loviisinho Lasermeča.

Sylvia sa jej vykrútila, ale Mei stihla okolo seba spraviť obranu. Na chrbte mala prepálený plášť od Solanu, ktorý jej spôosobil tiež ľahké popáleniny. Sylvia jej zas priložila zbraň k hlave.

"$ $Takže kľúča od Loriatru.“

"$ $Zabudni!"$ $ Mei sa jej prudko vytrhla, ze Sylvia napriek jej reflexom ju nestihla zastaviť  Mei vybehla cez jedny z dverí.

Sylvia sa rozbehla za ňou. Nedávala do toho všetko. Ani teraz to nedokázala. Nechcela.

Mei bežala po chodbe, zatočila do prava. Sylvia sa na ňu pokúsila vyslať kúzlo, lež Mei ho šikovne odrazila a namierila po schodisku hore. Sylvia pochopila, že ide za Hlavným Súdruhom.

"$ $Mrana!"$ $ Zvolala telepaticky, ale neprestávala nasledovať Mei. Tá jej unikala, unikala obybne a šikovne, rýchlo a svižne, netrafila ju ani jedna jej strela, ani jeden jej pokus o zničenie ochrany nebol úspešný.

A vtom, Mei sa zvrtla a otočila sa na päte.

"$ $Tak, milá moja. Myslím, že sme osameli."$ $ Sylvia stuhla a obzrela sa. Nič sa jej na prvý pohľad nezdalo iné, ale pri prepnutí pohľadu pochopila o čo Mei išlo. Dostala ju do ilúzie. Tu sa Mrana nedostane. Ale možno ani hlavný súdruh. Ak tam už nie je.

Sylvia nemala až také znalosti z teórie o iluzórnej mágii. Síce čoto čítala, ale iluzórna mágia bola oblasť, ktorú nemala až tak preskúmanú. Bola to síce súčasť jednej z jej obľúbených častí mágie – programátorskej mágie, ale ilúzie boli zvláštna vec aj na tú. A teraz sa v jednej ocitla a cítila sa bezmocná. To nebola len mágia, ona bola v nej, v tomto divnom útvare.

Ilúzia nebola inopole, ale niečo iné, špecifické. Obraz súčastnosti zakotvený v mágii, miestosť, z ktorej únik bol skrytý hlboko vnútri samotrnho magického poľa. Z toho čosi mála, čo Sylvia vedela o ilúziach len vedela, že sa nikto sa tam ľahko nedostane. V tej chvíli chcela, aby bol pri nej Tarny. Ten vedel o ilúziách oveľa viac…

Mei sa na ňu škodradostne usmievala, stojac pred ňou, s víťazstvom na tvári.

"$ $A myslím, že hľadáš toto..."$ $ Spopod plášťa vytiahla akýsi kľúč. A Sylvia pochopila od čoho je, i keď ho nidky nevidela… Bol taký ako ho Tanery popisoval… Prečo ho neskúšala od nej ho násilím získať, keď ho mala tak nablízku...

"$ $Ale nedostaneš ho. Takýto vzácny úlovok si Loriatar len tak nenechá ujsť.“

"$ $To si len myslíte."$ $ Drzo odvrkla Sylvia. Potrebovala sa tváriť nad vecou, hoci sa v skutočnosti začínala báť.

"$ $Nechať si ujsť polodémonku….? Prosím ťa a navyše takúto… druhý najlepší čas dievča, si nemysli.“

"$ $Prvý."$ $ Odvrkla. Vnútri sa triasla, že ju už spolovice odhalili – ale zároveň si vychýchla, lebo povedal to "$ $polo"$ $ pred tým prekliatym slovom… Mala byť pozornejšia, mala…

"$ $Nemysli si dievča, že si až tak jedinečná."$ $ Škodradostne sa zasmiala Mei a vtom sa zmenila na Hlavného Súdruha. Sylvia sa rýchlo prepla do magického pohľadu, a nebola to zmyslová mágia, nič nevidela…

"$ $Mei, nie..."$ $ Rozklepala sa po celom tele, nabehli jej zimomriavky, nemohla si pomôct…

 "$ $Nemusíš ma volať menom drahej Tesocovej… Som tu sám, osobne… Mala by si to brať ako poctu, dievča..."$ $ Sylvia bola naozaj vydesená. Zažila už všeličo, ale toto nečakala. Proti žiadnemu polodémonovi ešte nebojovala, ani o tom nevedela, a po tomto odhalení jej nabehla triaška.

"$ $Drahá Mei bola maska, skutočná Tesocová sa tam zabáva s tvojimi priateľkami… A ja som tu prišiel za tebou..."$ $ Načiahol sa, ale Sylvia uskočila, vytvorila si okolo seba obranu a štít.

"$ $Také pekné dievča, a takto uskakovať. Nepatrí sa to, nie mne…"$ $ Sylvii nabehli ďalšie zimomriavky. Nevedela úplne, čo mal za lubom, ale začínala sa jej v mysli črtať teória.

"$ $Neopovažuj sa..."$ $ Zašepkala s odporom.

"$ $No, no, no. Drahá Izabeta možno vykĺzla, ale teba lantárť nebudem. Myslím, dievčatko, že si dostatočne inteligentná, aby si súhlasila dobrovoľne...“

"$ $Nedotkneš sa ma."$ $ Premáhajúc triašku, s odporom cedila slová.

"$ $Och, nejaká nedotklivá. Aj Izabeta taká spočiatku bola, a aká bola poslušná..."$ $ Sylvii sa začínali spájať súvislosti. Izabeta odtiaľto prišla s dvoma deťmi… keby bola emčanka, zrejme by jej táto možnosť ani len nenapadla, ale ona si prežila už príliš veľa.

"$ $Nikto ma nebude zneužívať. Nie som tvoj otrok!“

"$ $Aká zrazu papuľnatá… Aj tak ti to je na nič platné..."$ $ Ešte raz sa k nej načiahol a Sylvia naňho vyslala laser.

"$ $Nedokneš sa ma. Lebo ťa zabijem."$ $ Špekala.

"$ $Drahá Izabeta sa tiež vyhražovala…"$ $ Zasmial sa. "$ $Nie je to na nič platné!“

"$ $Zabijem ťa."$ $ Šepkala Sylvia.

"$ $Drahá polodémonka, toto,"$ $ vytiahol z kabáta kľúče. "$ $Nikdy nedostaneš. Nič iné ti neostáva."$ $ Zasmial sa ešte raz, jeho pocit moci bol zreteľný a Sylvia vedela, že posledné čo by potrebovala by bolo prepadnúť panike. Potrebovala kľúč. A potrebovala sa odtiaľ dostať. To druhé by dokázala, ale získať kľúč? Na to sa musí dostať k nemu. A to on zneužije. Je polodémon, takže má rovnocenného súpera.

"$ $Zabijem ťa."$ $ Oznamovala mu. "$ $Zabijem a spálim.“

"$ $Zlé dievča! Také…"$ $ Sylvia zrazu zrušila obranu.

"$ $Ja som vedel, že..."$ $ Priblížil sa k nej, schmatol ju za vestu. A Sylvia splanula. V chvíli, keď bol on v šoku, schmatla kľúč a vyšla z inopoľa.

A stále horela. A Hlavný súdruh tiež.

\begin{center}
*
\end{center}

"$ $Sylvia!"$ $ Skríkla Loviisa, keď ju uvidela.

"$ $Preboha! Horíš!“

"$ $V pohode! Mám toto."$ $ Sylvia ukázala kľúč.

"$ $Bežme, bežme k bráne!“

"$ $Sylvia, nechceš uhasiť?“

"$ $Mrana prosím."$ $ Mrana pokrútila hlavou a držala sa radšej ďalej od Sylvie. Tá mala popáleniny, ktoré jej ale rýchlo mizli.

"$ $Kde je brána?“

"$ $Ak si pamätám, tak doprava na tretej ulici.“

Brána bola zo zlata. Doslova žiarila. Jej žiaru si po príchode nevšimli, zvnútra bola úplne iná. Bola zatvorená, vyzerala ako osamelá brána, bez akéhokoľvek významu, len tak tam bola, ležiaca v poli. Keby to bolo múzeum, dalo by ta to považovať za moderné umenie.

Bola zdobená ornamentami a mala hore nápis zvláštnym písmom hore "$ $Loriatar“. V jej strede bola zavesená zámka. Bola zo zvláštneho matriálu a okolo nej lietala mágia.

"$ $A prečo odtiaľto!"$ $ Sylvia vytiahla kľúč a vtedy okolo jej krku preletel nôž.

"$ $Tak  rýchlo sa nevzdáme!"$ $ Mei vytiahla ďalší.

Sylvia sa pohotovo vyzbrojila obranou a zakričala Loviise ktorá od šoku zastala.

"$ $Vyťahuj, vyťahuj lasermeč!"$ $ Loviisa chvíľu nehudne stála, ale ked okoo nich prefrčal ďalší nôž, prebrala sa.

"$ $Kde je?"$ $ Nemala ho ani v jednom vrecku.



"$ $Loviisa! Spadol ti!"$ $ Mrana len sa obzrela. Lasermač ležal na zemi, meter od začiatku ich obrany. Vtedy na drobnú figúrku upriamila zram Mei.

Mrana nečakala. Zvrtla sa, vyslala do Mei laser a načiahla sa po meč. Chytila ho, ale do ruky ju zasiahli, z ramena jej tiekla krv.

Podala lasermeč Loviise.

Tá zdesene ho vzala do ruky a vyslala po nich solan.

"$ $Loviisa! Laser! Laser!"$ $ Loviisa zavrela Solan a otvorila Laser. Nevedela čo robí. Počúvala tie dve a mierila Laser na Mei. Tú to prinútilo prestať strieľať a aby ju Laser nezasiahol tak vysielal Laser.

Sylvia, vidiac, že Mei stratila kus svojej pozornosti sa briblížila k zámku a vložila doňho kľúč. Brána sa ešte viac rozžiarila a sálala z nej mágia.  Odsúvala ju od zámku, akby na ňu mágia pršala, horela, jej plamene ju oblizovali, ale ona, napriek bolesti a napriek mágii krútila kľúčom. Keď sa brána pohla, počula Mranin výkrik.

Bola zasiahnutá do lýtka. Príštila krv, aj keď sa snažila zranenie si vyliečiť.

Hlavný súdruh bol spať.

Oheň ho zasiahol, prišiel o prá vlasov, ale keďže nebol magický, neublížil mu fatálne.

"$ $Nikdy mi neutečiete! Nikdy!"$ $ Kričal na nich a vysielal na nich mágiu. Sylvia netušila z čoho, ale ich obraňu bompardovali magické prápory, magické prúdy a ona citila, že obrana padá. Brána ešte nebola úplne otvorená, ale otvárala sa, otvárala. A čím viac sa otvárala, tým väčší zmätok spôsobovala.

Začala horieť. Či ju zapálila Sylvia, či ona sama, nevedno, ale horela. Mágiou? Sylvia vytvorila štít, udržiavala ho sama a čerpala z mágie brány, ale bola zoslabnutá. Mrana jej nemohla pomôcť, bola zranená, ale stále na nohách.

Jej štít ich pred ohňom chránil, ale Sylvia si uvedomovala, že nedokáže ho držať večne.

Loviisa namierila lasermeč na súdruha, lež ten sa premiestnil, uskakoval využíval svoje schopnosti a pritom sa rehotal.

"$ $Loriatar vás nepustí! Loriatar vás nepustí!“

Po ohni prišiel vietor. A brána? Jej otvorenie už bolo na prejdenie človeka. Vir ich metal na strany. Nevidela nič. Ani v magickom pohľade, tam všetko zaslepovala žiara a mágia pochádzajúca z brány.

"$ $Mraaaanaaaaaa!"$ $ Zvreskla. "$ $Mraaanaaaa!"$ $ Snažila sa zistiť kde je, ale vietor, tornádo, magické tornádo ich vzal do víru.

Išla okolo brány. Videla ju, tam sa tornádo rútilo. Prešla by ňou. Má ísť? Nemá ísť? Vír striedali záblesky. Mágie, Solanu, Laseru a Mágie. Veľa mágie.

Príde ku bráne ešte niekedy? A prídu tam ony? Vietor jej nenechával dlhé rozhodnutie. Chytila sa brány.

Vietor ju ťahal ďalej, mágia ju ťahala ďalej. Vietor jej zrušil štít, a ona sa len držala. Bála sa. Nie o seba. Bála sa o Mranu, bála sa o Loviisu. Veď Keria… Keria bola zranená…

"$ $Keeeeriaaaa!"$ $ Zakričala. Začínali sa jej slzy tisnúť dom očí… čo keď…

Brána bola otvorená. Bola zčasti v Loriatare, zčasti von. Divný pocit, ale ignorovala ho, resp… nebol tak silný, aby si ho vôbec uvedomovala.

Kričala do vetra, až kým sa nezačala zadúšať slzami.

"$ $Keeeriaaa! Mraaaanaaa! Loviiiiiisa!“

Keď videla niekoho letieť okolo, snažila sa ho zachytiť, možno sa jej to podarilo, ale to nevedela, ona v tom momente nevedela nič.

"$ $Sylvia! Sylvia!"$ $ Zrazu počula hlas. Loviisa! Aspoň!

"$ $Sylvia, pusti sa, poď z Loriataru! Syyylviaaa!“

Pustila sa. Pustila sa. Aj tak už nevládala, už jej hlas zlyhal.

Na zemi ležala Mrana. Na ramene bola ohavne popálená, cez pravú nohu jej viedla krvavá stopa…

"$ $Sylvia… Sylvia… vďaka bohu…“

"$ $Mrana! Mrana! Si zranená!“

"$ $Myslím… myslím že zomieram..."$ $ Hovorila prerušovane, ale neblúznila, nezdalo sa tak.

"$ $Nie! Nie…."$ $ Sylvia začala hystericky plakať.

"$ $Ty nemôžeš zomrieť, nemôžeš… nieee Mraanaaa… musí sa ti dať nejako pomôcť.“

"$ $Hlavný súdruh ma zasiahol… mágiou… Toto nevyliečiš… Syyyll..."$ $ Sylvia sa pozrela do magického pohľadu. To nemôže byť proste pravda, nemôže! Nieeee nieee! Mrana ju nesmie opustiť… je to jej vina… nieeeeee…

Mágia mala vo veľkých množstvách devastačné účinky a nič iné ako čas na ich opravu zatiaľ nevzniklo. Ďalšie magióny by len katalyzovali deštrukciu.

Zranenia mágiou, čistou veľkou dávkou mágie sa nedali liečiť inak než čakaním kým sa magióny umúdria. Lenže tentoraz zasiahli dôležité životné funkcie a bez potrebných nemagických medicínskych zázrakov nič nepomôže.

"$ $Sylvia..."$ $ Snažila sa Mrana hovoriť. "$ $Zachráňte sa... dostaňte sa odtiaľto... prosím...“

"$ $Mranaaaaaa..."$ $ Plakala Sylvia.

"$ $Prosím... Je tu ešte Loviisa... Sylvia prosím..."$ $ Mlela z posledného. Tep sa pomaly zastavil.

Sylviine oči boli plné sĺz – Loviisa bola v šoku, triasla sa.

Sylvia zrazu vstala, chytila Loviisu a... premiestnila sa.



\begin{center}
*
\end{center}



"$ $Kde sme sa to ocitli, Ježišmária!"$ $ Predniesla in Tulienka po obhliadke okolo seba. Dvere, dvere a dvere.

"$ $V chodbe. Ale až tak jednoduché to nebude.“

"$ $A ja mám pocit, že nás niekto vtrhol do divnej puzzle hry.“

"$ $Ale toto je divné. Chodíme z kelu do kelu, z času do času, zo sveta do sveta. Akože, čosraz viac to začína pripomínať takú krvilačnú rozprávku. Akoby sa s nami niekto hral...“

"$ $Konšpiračná teória?“

"$ $Kľudne. Celé je to divné.“

"$ $Meh."$ $ Ozvala sa Tulie a zas sa rozhliadala po okolí.

"$ $Ak mágia vytvorila toto miesto... Dvere? Prečo dvere? Prečo nie vchod do jaskyne, vír, dvere, prečo preboha dvere?“

"$ $Hlagenovka? To by možno šlo.“

"$ $Alebo ilúzia. Človek nidky nevie.“

"$ $No ale čo teraz?“

"$ $Dostať sa odtiaľto. To už začne byť o chvíľu taká až nútene klasická veta. Och bože.“

"$ $Skontrolujem mágiu."$ $ Iniciatívy sa chytil Tarny. "$ $Meh. Meh. Ja neviem.“

"$ $Vidíš niečo zvláštne?“

"$ $Veď práve.“

"$ $Tie steny chodby...“

"$ $Čo sa deje?"$ $ Spýtala sa Pauline.

"$ $Mágie neukazuje, že je niečo za dverami.“

"$ $Možno to je portál.“

"$ $To je možné. Skúsime?“

"$ $Preboha."$ $ Okomentovala Pauline.

"$ $Tak sa nikde nedostaneme, Paulina!“

"$ $Dobre, dobre... Tak poďme. Ech.“

"$ $Všetci sme vyčerpaní..."$ $ Pauline prevrátila očami. Síce nezomierali, nekrvácali, boli v relatívnom bezpečí, ale cítila sa štvane a to, po čom túžila, bola posteľ, spánok a svet, kde nemusela z každej idyly utekať preč.

Prvé dvere, ktoré otvorili boli tie napravo od nich, ničím sa nelíšiace od ostatných. Biele a lesklé s tmavou kľučkou, všety z rovnakej sériovej výroby. O krok ďalej, nedalo sa rozlíšiť medzi nimi.

Tarny položil ruku na kľučku, dívajúc sa na ostatných, či tomu prikývnu. Tulienka Deľa súhlasila, Pauline len naprádzno preglgla.

Dvere nezaškrípali, nevydali priam žiadny zvuk


\end{document}
